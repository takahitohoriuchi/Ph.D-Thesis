論文題目:
\vspace{0.5cm}
競技と生活を一体と捉え身体を問い直す身体知の学びの実践
〜身体部位や情景的体感への志向性を促す「表情」感得ツールの制作を通して〜
\vspace{1cm}

要旨(和文):
\vspace{0.5cm}

身体知の学びの現場に存在する「個人固有性と普遍性との境界」にまつわる難しさは、実践的な学びを阻む見過ごせない要因であるが、従来の身体知研究(スポーツ科学や運動学習研究)はその点を必ずしも問題視してこなかった。その問題の重要性を初めて指摘した研究のひとつに『「こつ」と「スランプ」の研究 身体知の認知科学』(諏訪,2016)がある。

(諏訪, 2016)によれば難しさは少なくとも二つあるという。第一は、身体運動には理想的なフォーム(正解)が存在するのかという問題に関わる。身体は物理法則下で動くので、物理法則に照らして効率的な運動を行う理想像(普遍的な原則)はあるはずである。しかしながら、身体は個人固有性(たとえば股関節・肩甲骨の柔らかさ、筋力)を有し、それに応じて理想的フォームにはそれなりに幅があるのも事実である。普遍性と個人固有性の境界を論じる(その切り分けを行う)ことは難しい。

第二の難しさは、身体の個人固有性に応じた理想的フォームを探り当てたとしても、そのフォームを自身の身体で実現するために意識を向けるべき身体部位や変数(「入力変数」と称する)は個人によって様々であることである。誰にでも普遍的に成り立つ「意識の向け方」との境界は、学びの初期段階にある本人にはわからないし、選手とは異なる特性を有する身体をもつコーチからみても学び手の意識の固有性はやっかいな問題である。

学び手は少なくともこの二つの問題を解決すべく、理論や他者からのアドバイスを参考にしながらも、身体運動の試行と身体感覚の微妙な差異の感得を繰り返し、理想的なフォームを探りつつ、同時に自身にとっての入力変数は何かを探らねばならない。その過程においては、自身の意識・身体感覚・身体部位の動きなどをことばで表現しながら「身体のありよう・ありかたを問い直す」という「からだメタ認知」の手法が有効である(諏訪, 2016)。

本論文における研究1では、十種競技の競技者であった私自身の「走り」の学びを対象にし、日々からだメタ認知の手法を用いて実践とことば化を繰り返し、自身の身体にどのような学びが生じるかを探究する一人称研究をおこなった。

私は自らのままならない身体と向き合い、全身各部位に思考を巡らせ、各部位の動きが醸し出す繊細な身体感覚にも意識を向けながらフォームを創り変えつづけた。たとえば「腕で粘性ある空気を掘る」という独自のことばを編み出しながら、ゆったりと大きく、しかし力強く走るフォームを試してみたりした。身体の動きを可視化するために百均製LEDをもちいたDIY的なツールを実装し、自身のフォームを「手触る」ようにして問い直したりもした。

学びの道のりは険しく、あちら立てばこちら立たずの繰り返しであり、膝の怪我にも見舞われた。大きなフォームを意識しすぎるあまり、接地時に膝関節が過剰に屈伸するフォームになってしまったことが原因かもしれないとも気づいた。

リハビリを続けるなかで日常生活の「歩き方」に根本的な問題があると考え、しだいに私は、「立つ・歩く」という日常的行為を正すことも練習である、つまり「練習は競技場だけにあるのではなく、生活全般の身体運用を考えることが練習である」と考えるようになった。競技だけでなく生活上のあらゆる身体知がひとつの身体に「輻輳」するという当たり前の現実を鑑みれば、この新しい気づきは至極当然である。

このように私の学びが競技文脈から生活全体の文脈へと拡張した様を、研究1では学びの「野生化」であると論じた。研究1における学びの過程は、私の身体・性格・生活を巡る個別具体的なものごとを内包する「物語論文」として執筆し、読者の身体知の学びを触発する点で意義があると評価された。

研究1の試行錯誤をとおして私は、自らのフォームを主体的に創り変えることは可能であること、創り変えに成功した暁には当初は思いも寄らなかった境地に達すること、そしてその学びの道のりは険しいことを、身をもって実感することができた。

同時に、学びのプロセスには、ことばだけでは迫りきれない微妙な、動いている身体が醸し出す「表情」(生々しい全体性(ゲシュタルト))を感得することが肝要だという仮説を生み出すに至った。「表情」とは、顔のこと(だけ)ではなく、研究対象であるひとやものの動き(自身も含む)の全体性(ゲシュタルト)に宿る、生々しくありありとした如実の姿を指す概念である(廣松渉1989)。「表情」を感得することが、ややもすると疎かにしがちになる身体感覚への志向性を保つことや、気に留めていなかった身体部位どうしの関係性に意識を向けることなど、自身の身体を問い直すことに有効であると考えた。これが研究2の着想となった。

研究2は「表情」という概念を主導原理にして行うことになった。「表情」は情緒/感情や意図/意志といったものごとが豊かに、しかし未分化にこもる姿と捉えることもできる。当初の私が目指していた「ゆったり」した大きな走りのフォームをいま観ると、接地時に「ドッコイショ」といった気合いや何かを持ち上げる意志を内包するような「表情」が感じられ、最終的な私の理想像(「スイーッ」と気持ちよく滑空するような「表情」)とは大いに異なることは明らかである。

そこで研究2では「からだメタ認知」ということば化だけでは見逃しがちの「表情」の感得をアスリートたちに促すためのwebアプリ『HJ-Playground』(HJP)を制作した。HJPは動いている身体の「表情」の感得を以下のような仕組みで促す。アプリは、ユーザ本人の身体運動に由来する点群(モーションキャプチャ等であらかじめ、任意部位群の時系列三次元位置データを撮影しておく)を描き出す。ユーザーは、それら点群間に多様な補助線を引いて自由に抽象的図形(「表情図形」と称する)を創ったり、それが醸し出す「表情」をオノマトペで命名したり、からだメタ認知的なことばの記述をしたりする。

ストリートダンサーと三段跳選手にHJPを使ってもらいながら、彼らが「自身の身体を問い直す」実践過程を観察した。彼らは、各々、多様な表情図形を自ら作図し、それと戯れながら、表情図形のなかに、ダンスや三段跳びの運動そのものだけではなく生活上の情景にも関わるような「表情」を感得した。ダンサーは、自身の動作の「揺れ動き」を見出したり、対象運動を実現するための意識の仕方を掴んだり、部位間の主従を転換させると動きのレパートリーを増やせるという一段メタな仮説を形成した。三段跳選手は、目指すべき力強い接地のビジョンを思い描いたり、スポーツ業界ではよく耳にする「脱力」と三段跳熟練者の特徴として語られる「腕で舵を取る」という、なかなか掴みにくい二つのものごとを紐づけて一挙に理解を深めたり、助走一歩目の踏込み方を新たに開拓したりした。HJPがユーザーに「表情」の感得を促し、それを基に自らの身体を問いなおすきっかけを与えることができたと考えている。

本研究の意義は、身体知の学び手が身体知を輻輳した知と捉え、競技と生活を一体にして自らの身体を問いなおすことの重要性を実践的に示したこと、そして、ことば化だけでは必ずしも気づき得ない身体の動きの全体性(ゲシュタルト)が醸し出す「表情」に着眼して、「表情」の感得を促すツールを制作してその有効性を確認できたことにある。


\vspace{2cm}

\textbf{キーワード: \\身体知の学び, 運動学習, 身体で問う, 生きる, 意味生成プロセス, 環世界, 状況, 「表情」, 「野」, デザイン, 一人称研究}

\vspace{2cm}
慶應義塾大学大学院 政策・メディア研究科 後期博士課程 堀内 隆仁