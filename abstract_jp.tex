論文題目:
\vspace{0.5cm}
身体知の学びとしての運動学習の一人称研究
-「アスリートとして生きる」ということの物語と、動いている身体の「表情」の感得を促すツールのデザインをとおして-
\vspace{1cm}

要旨(和文):
\vspace{0.5cm}

本研究では、運動学習を、生きるなかで主体的に意味をつくってゆく「身体知の学び」としてとらえ、2種類の一人称研究による実践をとおしてその意味生成プロセスを記述した。
本論文は、構成的という態度をなるべく大事にしながら書き綴った。

アスリートやダンサーらの運動学習の現場をみてみよう。学習者らは、みずから身体を動かしたり、自他の運動を観察したりしながら、「ああでもないこうでもない」と試行錯誤的に、
主体的に問いを展開する実践的プロセスをとおして、自身に固有な(独自的な)「意味」をつくっている。
この試行錯誤的プロセスは、身体知の学びである。

知の科学の歴史をみてみると(序論\ref{chapter:embodiedwisdom}章)、
心と身体を切り離し、心に知を求める認知主義(情報処理モデル)が主流であった。
情報処理モデルでは、日常の現実生活において混沌とした状況から意味を創りだすひとの知をとらえることができない。
身体性認知の考え方は、認知主義に対抗し、知には身体も必要であり、脳・身体・環境の全体が相互作用するシステムによっていかに知が成り立つかを探究する(DSA)。
近年は脳神経科学をも巻き込むかたちで、脳-身体-環境全体からなる「認知作用」の機序や、それによってどう「認知内容」が生まれてくるか、というむきが強く、
認知内容そのものは、必ずしも主題的には扱われていない。

「運動学習研究群」に絞ってみると(序論\ref{chapter:motorlearning}章)、
運動学習を量的モデルで説明しようとするアプローチは、スポーツ科学をふくめ、情報処理アプローチであれ、身体性認知のアプローチであれ、
多くは「expert-novice間」や「何かの実践の前-後」間を比較して、学習の時間を止めて「差」を示すものが多く、
意味づくりプロセスを陽には扱わない。
質的に運動学習にアプローチする研究群もある。
現象学的なアプローチであるスポーツ運動学や、意味の側面に迫ってはいるものの、体系的で普遍的な学習プロセスのありようを説明しようとするむきも強く、個人固有な
状況論的アプローチは、意味生成プロセスは扱うにしろ、他者との関係性をふくんだ状況に力点があるという点で、「本人からみえる状況」に力点をおく本研究とは異なる。

そこで本論文ではまず実践に先立って、序論\ref{chapter:embodiedwisdom}章で「身体知の学び」という考えかたを呈示した。
運動学習者がする「問う」行為は、「頭のなか」だけで進行するプロセスではない。
「問う」ということがすでに身体的な認知作用であり、
「問う」と相互限定するかたちで「問い」という認知内容が生じているのである。
相互限定は、互いに収束させあうだけでなく、むしろ、もつれあい・ずらしあう関係でもありうることに着目した。
このように問うことを、本研究では「身体で問う」と呼んだ。
これの理論的背景には、木村敏の論じる主体と世界との関わりの一般原理--フッサールやヴァイツゼカーやユクスキュルの思想を統合したもの--がある。
また、ユクスキュルの環世界論を参照して、運動学習における認知は、意味づくりの様相は「個人固有的」でありうるという考えにつなげた。
身体で問うこと自体が、主体性の萌芽であり、身体知の学びプロセスの最小単位的プロセスであり、
身体知の学びとは、身体で問うことによって、主体的に環世界を創り変えてゆく営みだとした。

序論\ref{chapter:mokutekiandhouhou}章では、本研究の目的と態度と方法論を論じた。
本研究は全体として「構成的」という態度を大事にする。
認知は構成的であり、構成的な認知を探究するためには、構成的な手法が必要であること、
そして、研究という営みもそもそも構成的な営みであることを指摘した。
本論文は、構成的という態度を大事にして、通常の科学論文では省かれてしまうような研究者の試行錯誤そのものについても、
著者が必要と思えば(平たく言えばストーリーテリングのようなかたちで)説明するように書いている。
本研究が採る方法論は、諏訪の提唱する認知的手法:からだメタ認知である。
からだメタ認知は身体で問う具体的手法であり、
みずからの思考・知覚・行動、すなわち「自分からみた自身と世界との関係性」について、
それがあいまいな違和感的なものであっても、はっきりした問題意識的なものであっても、積極的にことばにしてみて書きつづり問うてみる。
ことばにしてみることで、連想や推論など、自身とことばとの予期せぬインタラクションを起こし、自らの認知を変容させることになる。
つまり、身体知の意味づくりのプロセスを記述する方法として適したデータ記録方法でありながら、同時に、身体で問うことをうながし、構成のループを駆動する実践手法でもある。
本研究での「ことば」の位置づけは、ものごとを正確に記述したり他者に伝えたりというよりも、本源的に自身の身体との共創を起こす媒体である。

第一部研究について述べる。
\ref{chapter:monogatari}章では、私自身の身体知の学び、すなわち、私がひとりのアスリートとして生活と競技を分けずに「走り」を学ぶ(アスリートとして生きる)様を、物語として描き出した。
私は、自らのままならない身体と付き合い、体感に傾聴しながら、問題意識を醸成し、自分にとって納得できる動きの意味を試行錯誤的に探った。
怪我や生活上の出来事をきっかけとして、百均製LEDをもちいたトラッキングをDIY的に実施することで、自身の身体と動きを手触るようにして問うてみたり、
日常生活における、立つ・歩く動きをスキルとみなして根本的な再構築を図ったりした。
そして、日常生活で自身を取り巻く、競技に一見関係ないモノをツールへと転用しながら、
それらを通して、よりよい身体運用スキル、そして根本的な身体のあり方を問うことにすら試みた。
このようにして、私は数々の問題意識を醸成してきた。
\ref{chapter:hashirinohenka}章では、これらの努力の結果として私の走フォームにいかなる変化が生じたかを考察した。
\ref{chapter:yaseika}章では、私が自覚的にアスリートとして生きようとする態度を「学びの野生化」と命名し、その意義を論じた。
\ref{chapter:monogatarinoigi}章では、物語が他者にもたらす意義を議論した。

第一部研究を終えた私は、自身の身体を触発するさまざまな「トイ(おもちゃ)」づくりをしてみたり、
ある種「悟った」ようになった思惟を、「身体知輻輳性」と命名しながら論じてみたりしていた。
そうした問題意識のなかで私は、第二部研究の着想となる哲学概念「表情」に出会った。
そのことを第二部\ref{sec:tetsugakujousei}節で論じた。

第二部研究では、動いている身体の「表情」の感得をうながす運動学習支援webアプリ『HJ-Playground』を制作し、アプリをもちいた身体知の学びの実践をおこなった。
\ref{sec:hyojo}節では、哲学概念「表情」がどういう概念なのかを確認した。
私たち人間は「表情」の満ち満ちた環世界を生きており、
「表情」とは、視覚が主題になるような現象でありながら、行為の契機や情動や感情の契機をも孕みもつかたちで、生々しく立ち現れてくるものである。
そのことから、みずからの環世界を主体的に創り変えながら生き・学ぶ運動学習者にとっては、
とくに動いている身体が醸し出す「表情」は、意味の源たりうる現象であり、「表情」感得できるのが良いという仮説を立てた。
それをもとに、
\ref{chapter:fortoolmaking}章では、
身体知の学びとしての運動学習支援研究を概観したのち、
動いている身体の「表情」に近しいものに迫っているプロジェクト(研究や作品)を調べて
アプリ制作のヒントを探り、
「運動をもとにして、素朴で抽象的で図形的な見た目を表現する」ことが、「表情」の感得をうながすひとつの方法であるという着想を示した。

そうして本研究では、webアプリ『HJ-Playground』を制作し、\ref{chapter:hj-playground}章にてアプリの仕様を説明した。
本アプリは、あらかじめ計測したユーザ自身/他者の運動データ(各部位の三次元時系列位置情報)を、画面内の三次元空間に動く点群として描き、
ユーザに、それら点どうしのあいだに線分や円などの補助線を描きくわえ「表情図形」を作図することを促す。
ユーザには、作図した表情図形を鑑賞しながら、感得している「表情」をオノマトペで命名し、そのさなかで生まれる問いをからだメタ認知で内省記述することを促す。
これらは保存したり、再鑑賞することができる。
これらによって、ひとつの身体運動にさまざまな「表情」を豊かに感得することをうながす。

\ref{chapter:jissenresult}章では、アプリをもちいた学びの実践について記述した。
対象者のストリートダンサーAは、「oldman」というAの専門とは別ジャンル基本動作を対象に、それが醸し出す「表情」をアプリで探った。
身体内部のあるひとつの仮想的四面体が2通りの「表情」として感じられることを発見したり、
点同士左から右へ一筆書きでつなげることで、風に吹かれて移動するような「表情」を得たり、
両肩をむすんだ線の動きに「鉛筆を指先で軽くつまんで振るような情景」の「表情」をみてとって、予想外に不安定な揺れ方をしているのに気付いたりした。
別の対象者の三段跳選手Bは、一本ゲタ対人運動という不思議な動きを対象に、それが醸し出す「表情」をアプリで探った。
肩と手と肘の3点をむすび、「三角形が相手を突く」という攻撃性ある「表情」をみてとり、自分が到達すべき三段跳の力強い接地のビジョンをみたり、
円をつかった図形を作図することで「自身と他者とのあいだの空間が泡立つ」ような「表情」をみてとることで、それまで三段跳選手である自分が重要視していなかった「脱力」という深い概念について、その意味するところを自分なりに納得したり、
ゲタと足を線でむすびそれを延長してみることで「足首で剣を打ち込む」ような「表情」をみて、三段跳びの助走の一歩目の新しい踏み出し方を発想したりした。
このように本アプリの実践をとおして、対象者らは、アプリで遊んで、自らの身体運動をもとに主体的に表情図形を作図することで、ひとつの身体運動にさまざまな表情を感得し、
自身の運動のしかたを問いなおしていた。

\ref{chapter:analysis}章では、実践者らが表情図形の作図をとおして生み出した問いを分析した。
表情図形が、実践者らの問い立てをうながすパタンには、少なくとも2種類あり、
ひとつは、表情図形をみずからの身体に「仮想的図形」として召喚してそこに身体感覚を呼び起こすパタン、
ふたつめは、抽象的な表情図形にたいし、元の身体運動とは異なる日常生活のドメインの情景(できごとやシーン)に見立てる、メタファ的パタンであった。
ふたつめのパタンは、現役の運動学習者であるAとBよりも、すでに現役を引退しているCとBのほうが顕著にみられた。
また、ふたつめのパタンでは、どういうドメインの情景がメタファのソースとしてもちいられているのか、そのバラエティを調べた。
元々は身体運動をもとにした抽象的図形にもかかわらず、実に多様なドメインが、ソースとしてもちいられていた。

結論部では、以上の流れをおさらいしたのち、
身体知の学びの概念モデルを、本研究結果を踏まえて再構成した。
「身体で問うこと(認知作用=ノエシス)と問い(認知内容=ノエマ)との相互限定関係」は、
「主体的に生き・学んできた環世界」と「日常生活全般の野」との「界面」として生じる。
「表情」豊かに感得されるとき/「意味」が含蓄あるものとして醸成されているとき、
それは、野で出くわした対象が、環世界の多ドメインのものごとの輻輳として(輻輳に照らされて)、
多義的・多重的に、立ち現れてくる。
そして、そのように立ち現れてくるよう学習者は身構えることができる、として、本研究のまとめとした。

\vspace{2cm}

\textbf{キーワード: \\身体知の学び, 運動学習, 身体で問う, 生きる, 意味生成プロセス, 環世界, 状況, 「表情」, 「野」, デザイン, 一人称研究}

\vspace{2cm}
慶應義塾大学大学院 政策・メディア研究科 後期博士課程 堀内 隆仁