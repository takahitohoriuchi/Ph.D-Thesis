\clearpage

Title:
Practices of learning embodied wisdom by reconfiguring the way the body should and could be: 
Through designing a toolkit that facilitates perception of “hyojo” and encourages intentionality toward body parts and bodily sensations 

Abstract:

The difficulties of the boundary between universal principles and individual uniqueness in fields of embodied wisdom learning is one of the factors that could be hinderance to sound learning. 
However, previous studies on embodied wisdom in the field of sports science and motor learning have not necessarily addressed the issue.

Suwa (2016) has pointed out that there are at least two difficulties.
The first concerns the question of whether or not there is a single“correct form”of bodily movement that serves as the universal standard.
Bodies move under the laws of physics, and thus certain forms may be more efficient than others. On the other hand, the body of each athlete has its own individual characteristics― such as joint flexibility or muscle strength―, and therefore that makes it difficult to define a universally ideal form. In other words, ideal forms for athletes exist within a certain flexible range. 
Drawing a clear boundary between universality and individual uniqueness is inherently difficult.

The second difficulty lies in the fact that even if the learner has come to understand his or her ideal form that matches his or her characteristics, what body parts and/or variables need to be attended to (we call them “input variables”) for the purpose of bringing the form to realization is dependent on each person; there is no universally applicable way of directing attention. 
The learner in the beginning stage of learning does not know how to do it.
For the learner’s coach, too, since the coach’s body differs from that of the learner, the uniqueness of the learner’s input variables becomes a wicked problem.

Learners of embodied wisdom need to tackle these two challenges, exploring both their own ideal form and the relevant input variables for themselves.
This process involves being engaged in repeated trial-and-error: attempting bodily movements, perceiving subtle differences in bodily sensations, and referring to theories and advice from others. 
During the learning process, the method of embodied meta-cognition―a method of questioning one's own bodily being by verbalizing what one thinks, feels, and how one’s body moves―is found to be effective empirically (Suwa, 2016).

In Research 1 of this dissertation, I conducted a first-person study on how I have learned to run as a decathlete. 
Using embodied meta-cognition, I repeatedly practiced and verbalized my experience, exploring how learning emerged within my own body.

I faced my uncooperative body and kept reshaping my running form, directing my thoughts toward each part of the body and paying attention to the delicate bodily sensations that arose from their movements.
For example, I devised my own unique phrase―"digging into viscous air with my arms"―and experimented with a form that felt slow and wide, yet powerful.
I also implemented a DIY-style tool using LED lights from a 100-yen shop to visualize my bodily movements and questioned my form as if “touching it with my own hands.”

The journey of learning was not smooth; I often found myself in a trade-off situation where improving one aspect would compromise another. Eventually, I also suffered a knee injury.
Reflecting back, I realized it might have been caused by an excessive focus on making my form bigger, which led to overly deep knee flexion upon ground contact.

While undergoing rehabilitation, I came to suspect that the root of the issue lies in my everyday way of walking.
Gradually, I began to see that correcting how I “stand” and “walk” in daily life is also a significant form of training.
That is, training does not exist only in the athletic stadium, but also in reconsidering the bodily use throughout the daily life.
This thought felt entirely natural, given the obvious actual circumstances that all kinds of embodied wisdom—whether in sport or everyday life—“converge” into a single body.

Through these experiences of learning, I regard the expansion of my learning process—from the context of athletic performance to that of everyday life—as a form of the “going sauvage” of learning in Research 1.

The learning processes I experienced in Research 1 was written in the form of a “narrative paper,” which describes the concrete and particular aspects of my body, personality, and daily life. It has been evaluated as meaningful in the sense that it can give inspiration to readers to reflect on and deepen their own embodied learning.

Through the trial and error of Research 1, I have come to deeply realize that it is indeed possible to transform one’s own form through proactive effort, that such transformation can lead to states or ways of being that I have not necessarily imagined at the outset, and that the path of such learning is inevitably steep and demanding.

At the same time, I arrived at a hypothesis: that perceiving the “hyojo”—the subtle, emergent gestalt of a body in motion that cannot be fully captured by words—is essential to the learning process. Here, “hyojo” does not simply refer to the face, but rather to the vivid, palpable presence—the felt gestalt—of the motion of people or things, including oneself (Hiromatsu, 1989). 
I came to believe that sensing such “hyojo” helps maintain intentionality toward bodily sensations which are often neglected, and prompts reflection on unnoticed relationships between different body parts. This hypothesis became the foundation for Research 2.

Research 2 was conducted under the guiding principle of the concept of “hyojo”. In this context, “hyojo” refers to a form in which emotions, intentions, or will are richly present yet remain undifferentiated. 
When I now look back at the expansive and “relaxed” running form I initially aimed to achieve, I perceive in it an “hyojo” that contains a kind of effortful will—something like the intention to lift or push against resistance, as if shouting “dokkoi-sho!” at the moment of ground contact.
This is clearly different from the final ideal I came to pursue: an “hyojo” that feels like gliding smoothly and effortlessly through the air, almost like a quiet “sweeee.”

Based on this realization, Research 2 set out to create a web application—HJ-Playground (HJP)—designed to help athletes perceive the “hyojo” of the body in motion, which is often difficult to notice through verbalization and embodied meta-cognition alone. HJP fosters the perception of bodily “hyojo” through the following mechanism: the application visualizes a point cloud generated from the user’s own movements (captured beforehand via motion capture, as time-series 3D positional data of selected body parts).
Users can freely draw various auxiliary lines between these points to create abstract geometrical shapes, referred to as hyojo figures.
They are then encouraged to name the “hyojo” they perceive from these figures using onomatopoeia, and to describe their experience in words through an embodied meta-cognitive lens.

I observed how a street dancer and a triple jumper engaged with HJP to reflect on their own bodily being. Each participant actively created various hyojo figures using the app, and by playfully interacting with them, they came to perceive hyojos that not only reflected the physical movements of dance or triple jump, but also evoked scenes from their everyday lives connected to those movements.
The dancer discovered the quality of wavering in his own movements, grasped new ways of directing attention during performance, and formed a higher-level hypothesis: that by switching the hierarchy between body parts (which leads, which follows), one could expand their movement repertoire. 
The triple jumper, on the other hand, envisioned a strong and forceful landing form, and linked two elusive concepts often heard in sports—“relaxation” and the idea that “arms steer the motion,” a trait commonly attributed to skilled triple jumpers—into a deeper, unified understanding. 
The triple jumper also discovered a new approach to initiating his run-up. 
These findings suggest that HJP successfully fostered the perception of “hyojo”, and that this perceptual shift triggered critical reflection on their own bodily being and movement.

The significance of this study lies in two main contributions.
First, it has practically demonstrated the significance of reconceptualizing embodied wisdom as convergent wisdom—where wisdom from both athletic performance and everyday life flows together—and of exploring for and reconfiguring the ways the bodies should and could be by regarding athletic performance and bodily usages in everyday life as a whole.
Secondly, through focusing on “hyojo", the dynamic and gestalt-like quality of bodily movement that often escapes verbal articulation, this study has devised a novel tool that facilitates the perception of such “hyojos”, and has confirmed its effectiveness in supporting learners in exploring for and reconfiguring the ways his or her own body should and could be.

\vspace{2cm}


\textbf{Keywords: \\Learning Embodied Wisdom, Motor Learning, Questioning with the Body, Living, Meaning-Making Process, Umwelt, Situation, Hyojo, the Wild, Design, Research from The First-Person's Viewpoint}

\vspace{2cm}

Graduate School of Media and Governance, Keio University  

Doctoral Program

Takato Horiuchi
