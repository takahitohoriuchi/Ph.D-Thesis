\clearpage

Title:
A Study of Embodied Wisdom in Motor Learning from the First-Person Viewpoint:
A Story of “Living as an Athlete” and Designing a Tool for Perceiving the “Hyojo” of the Body in Motion

Abstract:

This research conceptualizes motor learning as a form of “learning embodied wisdom”—a process in which meaning is actively generated through lived experience—and explores this through two types of first-person inquiry. The author wrote this thesis with a constructive stance, with careful attention to the process by which meaning emerges.

In the real-world settings of athletes and dancers, learners move their bodies, observe both self and others, and engage in trial and error, continuously raising questions through these practical actions. Through such embodied questioning, they generate meanings that are unique and personal. This trial-and-error process constitutes the learning of embodied wisdom.

Historically, in the field of cognitive science (Chapter~\ref{chapter:embodiedwisdom}), knowledge has been primarily located in the mind, with body and mind treated separately under cognitivist frameworks such as the information-processing model. However, such models struggle to account for the creation of meaning amid the messy and dynamic situations of everyday life. In contrast, embodied cognitive science challenges cognitivism, seeking to understand how knowledge arises through the dynamic interactions of brain, body, and environment—a view known as the dynamic systems approach (DSA). While recent research has increasingly focused on the mechanisms of cognition as a system comprising brain, body, and environment, the *content* of cognition—its meaning—has not always been the primary focus.

Focusing specifically on the research field of motor learning (Chapter~\ref{chapter:motorlearning}), many studies—whether grounded in information-processing or embodied approaches—seek to quantify learning through comparisons (e.g., expert vs. novice, pre- vs. post-training). Such approaches often “freeze” the flow of time to highlight differences and rarely address the process of meaning-making. Although there are qualitative approaches, such as phenomenological studies in “bewegungslehre”, they tend to aim for generalized models of learning and do not fully capture the unique and situated nature of individual learning experiences. Similarly, situational approaches that emphasize the social and contextual dynamics of learning differ from this study, which centers on “the situation as seen from within” the learner.

In response, this dissertation proposes a new perspective in Chapter~\ref{chapter:embodiedwisdom}—that of “learning embodied wisdom.” The act of questioning, central to motor learning, is not merely a mental process. Rather, it is an embodied cognitive act, where the “question” and the “act of questioning” mutually constrain each other. This mutual constraint can involve not only convergence but also entanglement and divergence. I refer to this phenomenon as “questioning with the body.” The theoretical background draws on Bin Kimura’s philosophy of subject-world relationship, which integrates ideas from Husserl, von Weizsäcker, and Uexküll. Through Uexküll’s theory of Umwelt, I argue that cognition in motor learning entails the creation of meaning that is personal and situated. “Questioning with the body” represents a seed of subjectivity and serves as the minimal unit of learning embodied wisdom, understood as the ongoing re-creation of one’s Umwelt through “questioning with the body”.

Chapter~\ref{chapter:mokutekiandhouhou} elaborates on the research purpose, stance, and methodology. The research embraces a “constructive” stance throughout: cognition is constructive, and investigating it requires constructive methods. Research itself is a constructive act, and thus I have written the thesis to include even my own trial and error as a researcher—often omitted from conventional scientific writing—when I deemed it meaningful, employing a form of narrative storytelling. The methodology adopted here is Suwa’s cognitive method called “embodied meta-cognition,” a technique of questioning with the body. It involves verbalizing one’s vague or clear awareness of the relationships among perception, action, and thought—i.e., how one experiences oneself in relation to the world. Through this process, unexpected interactions between self and words arise, transforming one’s cognition. Thus, embodied meta-cognition serves both as a means of data collection for describing the process of meaning-making and as a practical method to foster embodied inquiry and drive the constructive loop of learning. In this research, language is not merely a tool for accurate description or communication, but a medium for co-creation with the one’s body itself.

The first part of this study presents my own process of learning embodied wisdom—my life and training as an athlete—in narrative form (Chapter~\ref{chapter:monogatari}). I engaged in trial and error through listening to my body, grappling with its limitations, and cultivating questions based on bodily sensations, in search of movements that felt meaningful to me. Injuries and daily life events led me to create DIY tracking tools using inexpensive LED devices to explore my body in a tactile way. I also restructured basic movements like standing and walking as “skills,” reexamining them from the ground up. Everyday objects that seemed unrelated to sports became tools through which I explored better bodily techniques and a deeper way of being. These experiences gave rise to a multitude of questions. Chapter~\ref{chapter:hashirinohenka} analyzes changes in my running form resulting from these efforts. Chapter~\ref{chapter:yaseika} discusses the attitude of “learning in the wild,” which I adopted consciously as an athlete. Chapter~\ref{chapter:monogatarinoigi} explores the significance of narrative for others.

After completing the first part, I began creating various “toys” to engage with my body and reflected on insights I had “attained,” coining the concept of “Embodied Confluence” to articulate them. This led me to encounter the philosophical notion of “Hyojo,” which became the inspiration for the second part (Section~\ref{sec:tetsugakujousei}).

The second part describes the design and practical application of a web-based tool, “HJ-Playground,” which supports motor learning by fostering the perception of “Hyojo” in the body in motion. Section~\ref{sec:hyojo} discusses the philosophical concept of Hyojo. Humans live in an Umwelt rich with Hyojo—a phenomenon that, while primarily visual, is infused with affective and motivational cues, emerging vividly and dynamically. Based on this, I hypothesized that for motor learners, perceiving the Hyojo of their moving body could serve as a source of meaning and thus be preferable. Chapter~\ref{chapter:fortoolmaking} surveys prior research and projects that resonate with the notion of Hyojo and proposes that expressing motion as simple, abstract, geometric visuals can promote its perception.

The “HJ-Playground” application, detailed in Chapter~\ref{chapter:hj-playground}, visualizes pre-recorded motion data (3D time-series of body parts) as moving point clouds in a 3D space. Users can draw auxiliary lines (e.g., segments, circles) between points to create “Hyojo figures.” As they engage with these figures, users name perceived Hyojo using onomatopoeia and reflect on emerging questions through embodied meta-cognition. These experiences can be saved and revisited, enriching the perception of Hyojo in a single movement.

Chapter~\ref{chapter:jissenresult} reports on practical applications with learners. 
Participant A, a street dancer, explored the Hyojo evoked by a fundamental movement called “oldman,” which belongs to a genre different from A’s area of expertise.  
A discovered that a certain virtual tetrahedron within A’s own body could be perceived as two distinct Hyojo.  
By connecting a sequence of points in a single stroke from left to right, A perceived a Hyojo reminiscent of drifting in the wind.  
Furthermore, by observing the motion of a line drawn between the shoulders, A perceived a Hyojo resembling “a pencil lightly held at the fingertips and shaken,” and noticed an unexpectedly unstable swaying in that motion.
Participant B, a triple jumper, used the application to explore the Hyojo evoked by a peculiar movement involving a single wooden clog (“ippon-geta”) and interpersonal interaction.  
By connecting the shoulder, hand, and elbow to form a triangle, B perceived a Hyojo of “a triangle striking the opponent,” evoking a vision of powerful ground contact—an ideal to be reached in triple jump.  
By creating figure incorporating circular forms, B perceived a Hyojo of “the space between self and other bubbling,” which led to a personal understanding of the concept of relaxation—a notion previously undervalued in their practice as a triple jumper.  
By connecting the clog and foot with a line and extending it outward, B perceived a Hyojo of “striking with a sword using the ankle,” which inspired a novel approach to the first step in the run-up of the triple jump.
Through playful interaction with the app, both learners generated rich Hyojo and re-questioned their movement.

Chapter~\ref{chapter:analysis} analyzes the types of questions evoked by Hyojo figures. Two main patterns emerged: (1) summoning Hyojo figures into the body as virtual forms to evoke sensations, and (2) metaphorically interpreting abstract figures through scenes from unrelated life domains. The second pattern was more prominent in retired athletes than active learners. Although the figures were originally abstract forms derived from bodily movement, a remarkably diverse range of domains was used as metaphorical sources.


In the conclusion, I revisited the above developments and, based on the findings, reconstructed a conceptual model of learning embodied wisdom.  
The mutually constraining relationship between “questioning with the body” (cognitive act = noesis) and “question” (cognitive content = noema) arises at the interface between one’s Umwelt—shaped through active living and learning—and the broader field of everyday life, which I refer to as “the wild.”  
When Hyojo is richly perceived, or when “meaning” is deeply cultivated, it is because the encountered thing in the wild appears, through the confluence of diverse domains within one’s Umwelt, in a multivalent and layered manner.  
I conclude that learners can cultivate an embodied readiness to perceive in this way—that is, to face and respond to the world as it appears through such manner.


\vspace{2cm}


\textbf{Keywords: \\Learning Embodied Wisdom, Motor Learning, Questioning with the Body, Living, Meaning-Making Process, Umwelt, Situation, Hyojo, the Wild, Design, Research from The First-Person's Viewpoint}

\vspace{2cm}

Graduate School of Media and Governance, Keio University  

Doctoral Program

Takato Horiuchi
