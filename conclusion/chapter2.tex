\chapter{物語ということと関連方法論について}

K.マイネルは、毛沢東の格言を引いている。「梨の味を知ろうとするなら、口に入れ噛み砕いてみなければならない」という格言を弾きながら、
スポーツ運動を本質的に完全に把握しようとする者はそれを自ら行なってみなければなるまい。これは運動研究者に課せられた大切な要求んおである。
と述べている。

% エピソード記述では、以下三種類の語りが含まれることを論じる
% \begin{description}
% \item[背景]体験が生じた舞台にまず読み手を招じ入れる
% \item[エピソード]書き手の心揺さぶられた様を語る
% \item[考察(メタ観察)]書き手が心揺さぶられた理由を添える
% \end{description}