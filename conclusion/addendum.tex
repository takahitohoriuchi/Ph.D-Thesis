\part*{付録}
% 目次にも表示させる
\addcontentsline{toc}{part}{付録}
% \setcounter{chapter}{0} % 章番号をリセット
\chapter{補論}
% \chapter{物語についての付録}
% \section{二元論・心脳問題・現象学をめぐる本研究の位置づけ}
% 本研究は、意識のありよう、すなわち主観を扱った。
% 途中、神経科学の知見を拠りどころにすることもあった。
% ホムンクルス問題や、

% たしかに脳は、意識の座であることは疑いようがない。
% 脳がなければ、意識現象は成り立たない。

% 池谷のいうように、
% \begin{itemize}
%   \item なぜ、(脳神経の)電気・化学的現象が、「感じ」を生むのか?
%   \item なぜ、すべては(脳神経の)電気・化学的現象であるにもかかわらず、「みる感じ」「きく感じ」「においの感じ」「感情」は、それぞれ違った「感じ」として感じられるのか?
% \end{itemize}

% あまりに深淵な問題である。
% 現代科学では、意識の統合情報理論(IIT)や--など、
% 期待の高まるアプローチはある。
% しかしいまだ、意識現象は謎に包まれている。

% ゆえに、本研究で深く立ち入れないのだが、少なくとも、

% 本章の議論に対して、次のようなツッコミが浮かぶかもしれない。

% 「この研究では、物と心が未分な『一元的なみえ』が原基的であるということの根拠に、脳神経科学、生態心理学など、経験科学の知見を援用していますよね?
% でも、それらの経験科学は『物』と『心』という二元論あるいはその一方のみ、という考えにもとづきます。
% つまり、「一元的である」ということを主張するために、二元的に舞い戻ってしまいませんか?あるいは混乱します。」

% 本節では補足的に、この疑問へ回答しておく。

% まず再確認だが、本研究は「一人称視点からのみえ」を扱うものであり、
% 一人称視点からのみえが「表情という一元的な構図」で描ける、ということが著者の主張の根幹である。
% ここに「なぜ、一人称視点からのみえが一元的なのか?」というツッコミはありうる。
% それにこたえるときに、諸経験科学の知見を援用している。

% 本研究では、二種類の二元論のあいだに線引きしている。
% 否定しているのは、「実物対象@外部世界ーその表象@内部世界」という二元論である。
% 受け入れているのは、「物質現象ー意識現象」という二元論である。
% 一人称的なみえを「意識現象」とみなしたとき、それを成り立たせるしくみとしての「物質現象」がある、ということだ。
% そのしくみが、諸経験科学で記述されてきた、脳身体環境の因果的相互作用である。
% それも、80s情報処理モデルだとダメだが、身体性認知科学のモデルを、根拠にしているのである。

% たしかに、究極的につきつめるなら、この2種類の二元論には違いはなくなる。
% じじつ、大森荘蔵の一連の仕事は、いっさいがっさいの物-心という二元構造を排し、
% すべての「立ち現れ」という一元へと帰そうとしている。

% それら経験科学社が、そのような研究をできたのも、まずは立ち現れ・表情・質なるものを感得したからであろう。
% あたかも、ポランニーが言ったように、カエルの分類学的研究をするとき、
% まずは、分類体系や基準をもたぬままに、「それ」を「カエルだ」とわからなければ、そもそもカエルの研究はできない。
% これは明晰的な知というより、それをささえる暗黙知にほかならない、ということである。

% それを明晰的な理性でもって反省し・分解することで、



% \section{身体の「表情」あれこれ}
% 「身体」にも表情があるのであり、
% 実践家は、鑑賞対象となる動く身体の「表情」を感得すればよい。
% これが、\ref{sec:学びにおける、動く身体を鑑賞するという体験}節で投げかけた問いへの著者のアンサーである。

% 日常の身振り手振りそぶりは、身体の表情の典型例である。
% 「ーー」と言いながらもモジモジしていたり、
% (発話内容と身体表情が「矛盾」しているが、私たちは身体表情のほうを「真」であると受けとるわけである!)。
% 腕組みしてガードしたり、
% マジシャンは華麗な仕草で、観ている私たちの注意を違うほうへいざない、私たちの認識枠を操る。
% 観ている私たちは、目の前のマジシャンのふるまいに、「タネとなる部分の仕草」を見破ることができない。
% タネの仕草とは異なる表情をみてしまう。

% 実践家の身体運動も、この線でとらえなおしたい。
% ボクシングの試合を観戦しているとき、
% その選手の殴り姿にはすでに、相手を殴る相手を「ブン殴る」という強い闘志がこもっている。
% 選手が相手選手へ抱く「誠意」や、
% 競馬では「前進気勢」ということばもある。

% 観ている私たちは、つい力む。じっとしてはいられなくなる。
% これを即応価とみるならば、上述した「運動共感」が強く生じるケースだろう。
% 格闘技全般において、相手に対して「自らの気配を殺す」ことをする。

% 運動主体(観測対象の)強い「意志」でもない、別種の即応価もある。
% もっとプリミティヴな、「うごめき」としてなにかを感じるのかもしれない。

% たとえば、羽生結弦のスケーティングに、
% ラグビーニュージーランド代表オールブラックスのハカに、
% その全体に、
% これも表情である。


% ほかにも、たとえば、バキシリーズから、宮本武蔵の動きが。


% 表情論のまとめからいえば、
% \begin{itemize}
%   \item 「見慣れた事物的姿」を保留すること
%   \item 鑑賞主体との関係性に帰着すること
%   \item 動きのなかにこそ現れるようにすること
% \end{itemize}

% たしかにそう考えてみれば、

% \section{実践者のことば}
% 奥平(2020)は,棒高跳選手である自身を対象とした一人称研究を行い,自分なりの身体運用の原理を追究した.
% 彼の学びに,曖昧模糊とした問いの重要性を如実に示す雄弁な例が見られる.
% 一例を示すと,彼は「会陰をキュッと持ち上げ,『おえっ』とえずくようにしてその持ち上げた意識を百会につなぎ止めておく」という体感的な表現を言葉に残している.これらの<感触>は,日々の練習での体感に意識を向けて言語に残したり,身体スキルに関する様々な文献を読み考えたりする中で,彼の意識に強く印象付けられた言語表現である.

%  国内トップレベルの100m走選手であった土江(2004)が走りのフォーム改善を図った事例報告にも,体感的な問いを垣間見ることができる。
% 土江は,自身の走りについて「接地時間を短くするため,はじくようにバネのような動き(p.15)」という意識から,「接地中に重心を浮かび上がらないように,むしろやや下向きにすべり落とすようにスライドさせる(p.16)」という意識に変えることで(これらは(ibid.)の一部に過ぎない),フォームを改善した.
% (ibid.)に掲載される改善前後の走りの連続写真を見ると,確実にフォームが変容していることをTは理解できる.
% そして,このフォーム改善後に迎えたシーズンで,土江は自己ベスト記録を更新し,オリンピック代表に選出されたのだ(ibid.).

