\part*{付録}
% 目次にも表示させる
\addcontentsline{toc}{part}{付録}
% \setcounter{chapter}{0} % 章番号をリセット
\chapter{補論}
% \chapter{物語についての付録}
% \section{二元論・心脳問題・現象学をめぐる本研究の位置づけ}
% 本研究は、意識のありよう、すなわち主観を扱った。
% 途中、神経科学の知見を拠りどころにすることもあった。
% ホムンクルス問題や、

% たしかに脳は、意識の座であることは疑いようがない。
% 脳がなければ、意識現象は成り立たない。

% 池谷のいうように、
% \begin{itemize}
%   \item なぜ、(脳神経の)電気・化学的現象が、「感じ」を生むのか?
%   \item なぜ、すべては(脳神経の)電気・化学的現象であるにもかかわらず、「みる感じ」「きく感じ」「においの感じ」「感情」は、それぞれ違った「感じ」として感じられるのか?
% \end{itemize}

% あまりに深淵な問題である。
% 現代科学では、意識の統合情報理論(IIT)や--など、
% 期待の高まるアプローチはある。
% しかしいまだ、意識現象は謎に包まれている。

% ゆえに、本研究で深く立ち入れないのだが、少なくとも、

% 本章の議論に対して、次のようなツッコミが浮かぶかもしれない。

% 「この研究では、物と心が未分な『一元的なみえ』が原基的であるということの根拠に、脳神経科学、生態心理学など、経験科学の知見を援用していますよね?
% でも、それらの経験科学は『物』と『心』という二元論あるいはその一方のみ、という考えにもとづきます。
% つまり、「一元的である」ということを主張するために、二元的に舞い戻ってしまいませんか?あるいは混乱します。」

% 本節では補足的に、この疑問へ回答しておく。

% まず再確認だが、本研究は「一人称視点からのみえ」を扱うものであり、
% 一人称視点からのみえが「表情という一元的な構図」で描ける、ということが著者の主張の根幹である。
% ここに「なぜ、一人称視点からのみえが一元的なのか?」というツッコミはありうる。
% それにこたえるときに、諸経験科学の知見を援用している。

% 本研究では、二種類の二元論のあいだに線引きしている。
% 否定しているのは、「実物対象@外部世界ーその表象@内部世界」という二元論である。
% 受け入れているのは、「物質現象ー意識現象」という二元論である。
% 一人称的なみえを「意識現象」とみなしたとき、それを成り立たせるしくみとしての「物質現象」がある、ということだ。
% そのしくみが、諸経験科学で記述されてきた、脳身体環境の因果的相互作用である。
% それも、80s情報処理モデルだとダメだが、身体性認知科学のモデルを、根拠にしているのである。

% たしかに、究極的につきつめるなら、この2種類の二元論には違いはなくなる。
% じじつ、大森荘蔵の一連の仕事は、いっさいがっさいの物-心という二元構造を排し、
% すべての「立ち現れ」という一元へと帰そうとしている。

% それら経験科学社が、そのような研究をできたのも、まずは立ち現れ・表情・質なるものを感得したからであろう。
% あたかも、ポランニーが言ったように、カエルの分類学的研究をするとき、
% まずは、分類体系や基準をもたぬままに、「それ」を「カエルだ」とわからなければ、そもそもカエルの研究はできない。
% これは明晰的な知というより、それをささえる暗黙知にほかならない、ということである。

% それを明晰的な理性でもって反省し・分解することで、

\section{環世界概念の別解釈について}
\label{sec:horon_kansekai}

ユクスキュルは環世界の考え方によって、「生物は固有な世界を生きている主体である」ということを「機械操作係である」という言い方でも書いている。
2つの言い方はいずれも、「生物が客観的世界に組み込まれた機械である(生理学ではそのように生物を記述する)」ことを批判的に飛び越えようとしている。
しかし、本研究からすればこの2つは互いに異なる着地点であることを補足しておく。
本研究は前者の言い方に賛成である。

後者の「機械操作係」という表現は、本研究にとっては好ましくない。
すこし後の知能科学の歴史からみれば、機械操作係の考え方は、生物と機械とを統一した制御・通信モデルでとらえる\textbf{サイバネティクス}\cite{wiener:1948}」と類同している。
サイバネティクスは、その後認知科学で「人間の心」のモデル化する「情報処理モデル」として輸入された。
情報処理モデルは、心と身体とを分離して扱っており、本研究はまさにそこを問題視しているのであった。
また。
サイバネティクスと情報処理の「フィードバック」の考え方は、「環(Krais)」のかたちが意味する「部分の相互関係で全体が維持されると同時に、全体のなかでこそ部分は存続できる」という関係性とは異なる。
したがって、身体知の思想を推し進める本研究からすると、機械操作係という表現は好ましくない。
ヴァイツゼッカーもこの後者の点に批判的である。
\footnote{ユクスキュルの息子で医師であったトゥーレ・ユクスキュルもまた、父ヤーコプの環世界の考え方を人間へと敷衍した「状況環」という考え方を展開している。
ちなみに、状況環はヴァイツゼッカーに近い考え方ではあるが、トゥーレはヴァイツゼカーを引用しなかった。
これには、感情的な事情があるらしいとのことである。
}。
\section{「表情」なき世界:離人症}

読者のなかには、「そこに建っているビルには表情を感得できない。あれはただの無機質な物体にすぎないじゃないか」という疑問をもった者もいるかもしれない。
こうした指摘に対する反駁となりうる記述も、文献\cite{hiromatsu:1989}には書いてある。
\begin{quotation}  
  なるほど、現相のうちには、これというほどの感情価やこれというほどの即応価を帯びていないものもある。
  だが、その場合でも、表情価が端的に\ruby{零}{ゼロ}なのではない。よしんば零としか言いようのない“欠如態”の相にあるとしても、
  欠如態は(いわゆる“無色透明”が一種の色であるのと類比的に)それ自身、れっきとした価値態であることを忘れてはなるまい。\\
  (『表情』, p.79)
\end{quotation}

このようなかたちで廣松は、感得される表情現相は
「人物や動物の顔面表情や身体的挙措表情には限られない。
原基的な相においては(中略)、一切の現相が\ruby{悉}{ことごと}く表情性を帯びて
\footnote{鋭い読者は「Xが表情性を帯びる」という表現方式は表情にふさわしくないのでは?
と思ったかもしれない。
それは正しい。
しかしその部分こそ、「表情に対して語彙が貧困」という廣松が指摘する問題でもあろう。
実は廣松は、「表現の便宜上、以下では事物が表情性を帯びた相で現前するかのように記す方式をも辞せないようにしよう(『表情』, p.10)」と
断りをいれたうえで「Xが表情性を帯びる」という書き方をしている。
}
感得される」と説明する。

では反対に、もし私たちが、ほんとうに、「表情」をまったく感得できないのだとするならば、どうなるのか?
私たちの体験の前にくりひろがる環世界は、どのようなものになるはずなのだろうか?

離人症という精神疾患がある。
離人症患者は、次のような体験をする。
\cite{nakamura:1979}によれば、離人症と診断された24歳のある女性は次のように語ったという。
\begin{quote}
  音楽を聞いても、いろいろの音が耳の中にはいりこんでくるだけだし、
  絵を見ていても、いろいろの色や形が眼の中にはいり込んでくるだけ。
  何の内容もないし、何の意味もない。(\cite{nakamura:1979}, p.47)
\end{quote}
また、42歳のある女性\footnote{診断は未確定とのこと}の症例では、患者からの手紙のなかで次のように綴った。
\begin{quote}
  暑い寒いという温度の高低はわかりますが、暑い寒いといった感じはどうもピンと来ません。
・・・本当にただ視聴覚に訴え、肉体的に感じることだけで、精神的な感じの方は相変わらず何も感じることができません。  
(\cite{nakamura:1979}, p.47)
\end{quote}

中村によれば、私たちの「共通感覚」が喪失しているのである。
共通感覚とは、体性感覚を中心として諸知覚が統合された感覚であり、
私たちが生きるうえでの基本的な感受性・常識の基盤となっていると中村はいう。
共通感覚を失えば、その当人の環世界からは「表情」が失われてしまう、と著者は考える。
それが上記の事例である。
なんとも殺風景的である。
「そこのビルに表情がない」と考えるひとは、
\section{立ち現れ一元論 by 大森荘蔵}
\label{sec:horon_tathiaraware}
大森荘蔵による一連の哲学「立ち現れ一元論」の象徴的な文言をいくつか載せておく。


% \todo{岩肌のやつがあったよなああ。}

たとえば私たちは、色や形を「世界」に属する性質として、感情を「私」に属する性質として描きがちである。
大森はそれすら否定してみせる。象徴的な言明をいくつか引いておこう。
\begin{quote}
  心という袋をひっくり返しにして、風景の立ち現れに吐き出す。
\end{quote}

\begin{quote}
  一本のネクタイの色はさまざまに見える。朝日の中で、木陰の中で、夕闇の中で、蛍光灯の下で、その色合いは微妙に変わる。 また、黄疸の人、色盲の人、呉服屋さんにはまた別様に見えよう。 これらの十人十色が全て「心に映じた」色であるというのであれば、ネクタイの客観的な色は一体何色であればいいのか。それは、カメレオンの本当の色は何かというのと同じように意味をなさない問いであろう。(\cite{ohmori:1976}, p.107)
\end{quote}
\begin{quote}
  一本の樹木もネクタイの色と同様、陽炎の向こうで、霧の中で、安物の窓ガラスの向こうで、二日酔いの人の目に、近視の人、老眼の人の目に、形を変えて見える。 この時、その樹の客観的な形とは正常な状況で正常な人に見える形だという人は、単に一つの「標準形」を指定しただけである。 それはカメレオンの「標準色」として緑を、ネクタイの「標準色」として、売り場の店員に見える色を(売手市場の場合だが)、指定するのと変わりはない。(\cite{ohmori:1976}, p.108)
\end{quote}
\begin{quote}
  (※気分や「心地」が我々の「心の内」にあるとしか言えないという考えに対して、)
しかしはたして、例えば恐ろしさは、すっぽり「心の内」に抱かれているのだろうか。歯医者と、あのピカピカ光る拷問器具をこわがるとき、恐ろしいのは、これらの道具と拷問者である。恐ろしさは、それらの人と事物に、いわば「附着」しているのである。 それを引き剥がして、一方に、怖くもなんともない歯医者と道具、そして今一方に、純粋結晶のように取り出された、純粋の恐怖(恐怖のエッセンス、恐怖のエキス)、 そして、この純粋恐怖だけが、私の「心の内」にある。しかし、もしそうなら、私は一体何が恐ろしいのだろう?(\cite{ohmori:1976}, p.116)
\end{quote}

このように大森は、「立ち現れ」の一元論を展開している。
「外なる物、内なる心」という「二段構え」の構図のなかに、「表情」はない。
「表情」は立ち現れてくるものである。

ほかにも大森は、以下のようにも書いている。
\begin{quotation}
    他人の「胃が痛い」という発言をその人の「胃痛」を構成する振舞の一部として受け取る。
    この発言以外に「胃痛」を構成する振舞は多々ある。
    身をよじる振舞、ものを食べられないという振舞、冷汗という振舞、
    ある種の表情という振舞、動作の不活性という振舞等を、あげればきりがない。
    これら無数の振舞とならんで「胃が痛い」という発言の振舞が「彼は胃痛」という情景を構成しているのである。
    「胃が痛い」という発言はこの「彼は胃痛」という情景の「報告」ではなく、その情景の一部なのである。(\cite{ohmori:1971}, p.26)
  
\end{quotation}

% \section{身体の「表情」あれこれ}
% 「身体」にも表情があるのであり、
% 実践家は、鑑賞対象となる動く身体の「表情」を感得すればよい。
% これが、\ref{sec:学びにおける、動く身体を鑑賞するという体験}節で投げかけた問いへの著者のアンサーである。

% 日常の身振り手振りそぶりは、身体の表情の典型例である。
% 「ーー」と言いながらもモジモジしていたり、
% (発話内容と身体表情が「矛盾」しているが、私たちは身体表情のほうを「真」であると受けとるわけである!)。
% 腕組みしてガードしたり、
% マジシャンは華麗な仕草で、観ている私たちの注意を違うほうへいざない、私たちの認識枠を操る。
% 観ている私たちは、目の前のマジシャンのふるまいに、「タネとなる部分の仕草」を見破ることができない。
% タネの仕草とは異なる表情をみてしまう。

% 実践家の身体運動も、この線でとらえなおしたい。
% ボクシングの試合を観戦しているとき、
% その選手の殴り姿にはすでに、相手を殴る相手を「ブン殴る」という強い闘志がこもっている。
% 選手が相手選手へ抱く「誠意」や、
% 競馬では「前進気勢」ということばもある。

% 観ている私たちは、つい力む。じっとしてはいられなくなる。
% これを即応価とみるならば、上述した「運動共感」が強く生じるケースだろう。
% 格闘技全般において、相手に対して「自らの気配を殺す」ことをする。

% 運動主体(観測対象の)強い「意志」でもない、別種の即応価もある。
% もっとプリミティヴな、「うごめき」としてなにかを感じるのかもしれない。

% たとえば、羽生結弦のスケーティングに、
% ラグビーニュージーランド代表オールブラックスのハカに、
% その全体に、
% これも表情である。


% ほかにも、たとえば、バキシリーズから、宮本武蔵の動きが。


% 表情論のまとめからいえば、
% \begin{itemize}
%   \item 「見慣れた事物的姿」を保留すること
%   \item 鑑賞主体との関係性に帰着すること
%   \item 動きのなかにこそ現れるようにすること
% \end{itemize}

% たしかにそう考えてみれば、

% \section{実践者のことば}
% 奥平(2020)は,棒高跳選手である自身を対象とした一人称研究を行い,自分なりの身体運用の原理を追究した.
% 彼の学びに,曖昧模糊とした問いの重要性を如実に示す雄弁な例が見られる.
% 一例を示すと,彼は「会陰をキュッと持ち上げ,『おえっ』とえずくようにしてその持ち上げた意識を百会につなぎ止めておく」という体感的な表現を言葉に残している.これらの<感触>は,日々の練習での体感に意識を向けて言語に残したり,身体スキルに関する様々な文献を読み考えたりする中で,彼の意識に強く印象付けられた言語表現である.

%  国内トップレベルの100m走選手であった土江(2004)が走りのフォーム改善を図った事例報告にも,体感的な問いを垣間見ることができる。
% 土江は,自身の走りについて「接地時間を短くするため,はじくようにバネのような動き(p.15)」という意識から,「接地中に重心を浮かび上がらないように,むしろやや下向きにすべり落とすようにスライドさせる(p.16)」という意識に変えることで(これらは(ibid.)の一部に過ぎない),フォームを改善した.
% (ibid.)に掲載される改善前後の走りの連続写真を見ると,確実にフォームが変容していることをTは理解できる.
% そして,このフォーム改善後に迎えたシーズンで,土江は自己ベスト記録を更新し,オリンピック代表に選出されたのだ(ibid.).

