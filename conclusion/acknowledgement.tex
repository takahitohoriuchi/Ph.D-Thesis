\section*{謝辞}
本研究は、さまざまな方の支えのうえに成り立っています。

諏訪先生。
実に24学期ものあいだ、研究者としての心構えや作法からなにからなにまで、さまざまなことを導いてくださいました。
本研究が形になったのは、諏訪先生のきめ細やかなアドバイスがあったからです。
諏訪先生が、生活者であり、アスリートであり、研究者であり、教育者であることは、私にとってあまりにも有難いことでした。
博士になるということは、一人前の認証バッジを得たというわけなので、もうこれからは、遠慮なく突き進みたいと思います。
これからもどうぞよろしくお願い申し上げます。
% なんといっても、ちょうどよい「事例」を瞬時に引き出す力。わからくなっていることに、わかりやすいことばをもってきて、解きほぐし、問題を実にシンプルにとらえてみせる力。メタに考える力。物怖じせずむかってゆく転校生マインド。

副査を快く引き受けてくださった石川初先生、中西先生、西村ユミ先生にも感謝を申し上げます。
先生方の鋭くエキセントリックなご指摘は、いつも数歩先をみており、私の研究をより深く、より広く、よりおもしろくしてくれました。
% 「ケツとしかいいようのないなにかなんだよなー」というのは、私たちの出会いで初めて交わされたことばでした。私の卒論発表を聴いてくださりコメントくださったこと、いまでも鮮明に蘇ります。HJ-Playgroundをみて「犬の散歩の達人になれる」という話、ほかの誰がそんなコメントをくださるでしょうか。修士のときも博士のときも、石川先生の発する言葉は、いつも数歩先を突いていて、アドバイスをもらったその瞬間はいつも、それがあとになってじわじわと効いてきて「あのとき石川先生が言ってたのって、こういうことだったのか」と気付かされる、なんてことも少なくありません。
引き続き審査よろしくお願い申し上げます。

HJ-Playground実践参加者のみなさま。
どうもありがとう。みなさんこそが、私がHJ-Playgroundの制作者本人でありながら気づいていない可能性を切り拓いてくれました。
みなさんがおもしろい使い方をしてくれたから、研究になりました。

諏訪研のみなさま。
あまりに諏訪研長くいたので、先輩方や同期はもちろん、後輩の皆さんにもお世話になってしまいました。
みなさんの卒論の発表をきいたりアドバイスをするとき、それはそのまま、自分の研究のことを考えることにもなっていました。
何気ない日々の交流もふくめて、本当にありがとうございました。

児玉先生。
本研究の議論や、モーキャプ環境の多大なるお手伝いをしてくださったからこそ、本研究は完成に至りました。
博論執筆中もさまざまにご配慮いただき、感謝申し上げます。
% 思えば、私の卒論の引用に登場する。互いの研究の距離感が

家族へ。
様々な面から支えてくれてありがとう。この博論は献本するので、それなりに目立つところにでも飾ってください。
私がこまめに実家に帰ったり、家族イベントに欠かさず参加しているのは、実は寂しい僕を、変わらず温かく迎えてくれるからです。

その他、私の長い博士過程生活のあいだ、研究にかかわらず、私と交流をしてくれたすべての方々に感謝を申し上げます。
ひとつひとつの体験が、生きる私の身体に降り積もり、本研究の礎をなしてくれていると、私は実感しています。
