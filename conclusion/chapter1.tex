\part{結論}
% \addcontentsline{toc}{part}{結論}

\chapter{考察}

\section{本論文の議論の振り返り}
本研究では、運動学習を、生きるなかで主体的に意味をつくってゆく「\textbf{身体知の学び}」としてとらえ、\textbf{一人称研究}による実践をとおしてその\textbf{意味生成プロセスを記述}した。
イントロダクションからの流れを振り返る。

知の科学の歴史は、心と身体を切り離し、心に知を求める認知主義(情報処理モデル)が主流であった。
しかし情報処理モデルでは、日常の現実生活において混沌とした状況から意味を創りだすひとの知をとらえることができない。
身体性認知の考え方は、認知主義に対抗し、知には身体も必要であり、脳・身体・環境の全体が相互作用するシステムによっていかに知が成り立つかを探究する。
脳神経科学をも巻き込むかたちで、脳-身体-環境全体からなる「認知作用」の機序や、それによってどう「認知内容」が生まれてくるか、というむきが強く、
認知内容そのものは、必ずしも主題的には扱われていない。

本研究はそのどちらとも異なる「身体知の学び」という考え方を推し進めている。
木村敏の論じる主体と世界との関わりの一般原理--フッサールやヴァイツゼカーやユクスキュルの思想を統合したもの--を足場にして、
\textbf{認知とは「認知作用(ノエシス)と認知内容(ノエマ)との相互限定」であり、
認知作用(ノエシス)とは、知覚する-行為する-思考するが三位一体の作用である}、という見方を呈示した。
また、ユクスキュルの環世界論を参照して、ひとは(学習者は)「個人に固有な意味世界」を生きうることを述べ、
運動学習における認知も\textbf{個人固有}でありうるという考えにつなげた。
これを踏まえ、「生きるなかで学ぶ」ということに焦点を置く本研究では、
身体的な認知作用と認知内容との相互限定は、互いに収束させあうだけでなく、むしろ、もつれあい・ずらしあう関係でもありうることに着目した。
それを巻き起こすことを「\textbf{身体で問う}」と呼んだ。
身体で問うこと自体が、主体性の萌芽であり、身体知の学びプロセスの最小単位的プロセスであり、
\textbf{身体知の学びとは、身体で問うことによって、主体的に環世界を創り変えてゆく営み}だとした。

従来の運動学習研究群は、
スポーツ科学は物体的身体の客観的特徴を解明することで、運動学習の一助となる知見を蓄積してきた。
学習プロセスをモデル化しようとする試みでも量的アプローチが盛んである。
運動学習を、
「情報処理システムが機械としての身体の制御パラメータを修正する」としてみる情報処理的アプローチ、
「知覚-行為システムが身体に成り立つ協調的構造を相転移する」としてみるダイナミカルシステムアプローチがある。
これらの多くは客観的な観測・記述であるのにくわえ、「expert-novice間」や「何かの実践の前-後」間を比較して、学習の時間を止めて「差」を示すものが多く、
意味づくりプロセスを陽には扱わない。
質的アプローチは意味の領域を扱うものではある。
現象学的なアプローチであるスポーツ運動学は、機械運動として身体運動を記述することの限界を指摘し、運動の「形態(ゲシュタルト)」を扱った。
形態は、植物の形態発生のごとく成長のなかにあるかたちであり、自己観察や他者観察によって内的に感得されるべきものである。
本研究の「表情」とも通ずるものであるが、金子らのアプローチは運動学習を体系的に記述することをめざすという点では、本研究とは異なる。
状況論アプローチでは、伝統的な教える-教えられるといった「脱分脈化された学校教室での知識教授の構図」を乗り越え、
学びを、状況・他者との関係性のなかで生きながら学び合うことだとしてとらえてきた。
意味づくりを「環世界の変容」すなわち「本人からみた自己と世界(環境や他者)との関係性」でとらえる本研究とは、その点では力点が異なる。
そのうえで状況論アプローチは、「意味は生きているなかの状況から創りだされる」といった思想や、ことばの位置づけなど、本研究の思想と近しいところも多い。

本研究全体に通底する態度である「\textbf{構成的}」ということについても論じた。
認知は構成的であり、構成的な認知を探究するためには、構成的な手法が必要であること、
そして、研究という営みもそもそも構成的な営みであることを指摘した。
本論文は、構成的という態度を大事にして、通常の科学論文では省かれてしまうような研究者の試行錯誤そのものについても、
(平たく言えばストーリーテリングのようなかたちで)説明してきた。

身体知の意味づくりのプロセスを記述する方法として、本研究では諏訪の提唱する\textbf{認知的手法:からだメタ認知}を採用した。
からだメタ認知は身体で問う具体的手法であり、
みずからの思考・知覚・行動、すなわち「自分からみた自身と世界との関係性」について、
それがあいまいな違和感的なものであっても、はっきりした問題意識的なものであっても、積極的にことばにしてみて書きつづり問うてみる。
ことばにしてみることで、連想や推論など、自身とことばとの予期せぬインタラクションを起こし、自らの認知を変容させることになる。
データ記録方法でありながら同時に、身体で問うことをうながし、構成のループを駆動する実践手法でもある。
本研究での「ことば」の位置づけは、ものごとを正確に記述したり他者に伝えたりというよりも、本源的に自身の身体との共創を起こす媒体である。

第一部研究では、
本研究では、私自身の身体知の学び、すなわち、私がひとりの\textbf{アスリートとして生活と競技を分けずに「走り」を学ぶ(アスリートとして生きる)様}を、\textbf{物語}として描き出した。
私は、自らのままならない身体と付き合い、体感に傾聴しながら、問題意識を醸成し、自分にとって納得できる動きの意味を試行錯誤的に探った。
怪我や生活上の出来事をきっかけとして、百均製LEDをもちいたトラッキングをDIY的に実施することで、自身の身体と動きを手触るようにして問うてみたり、
日常生活における、立つ・歩く動きをスキルとみなして根本的な再構築を図ったりした。
そして、日常生活で自身を取り巻く、競技に一見関係ないモノをツールへと転用しながら、
それらを通して、よりよい身体運用スキル、そして根本的な身体のあり方を問うことにすら試みた。
このようにして、私は数々の問題意識を醸成してきた。
これらの努力の結果として私の走フォームにいかなる変化が生じたかを考察した。
私が自覚的にアスリートとして生きようとする態度を「\textbf{学びの野生化}」と命名し、その意義を論じた。
物語が他者にもたらす意義を議論した。

第一部研究を終えた私は、自身の身体を触発するさまざまな「トイ」づくりをしてみたり、
ある種「悟った」ようになった思惟を、「身体知輻輳性」と命名しながら論じてみたりしていた。
そうした問題意識のなかで私は、第二部研究の着想となる哲学概念\textbf{「表情」}に出会った。

第二部研究では、\textbf{動いている身体の「表情」の感得をうながす運動学習支援webアプリ『HJ-Playground』を制作}し、アプリをもちいた身体知の学びの実践をおこなった。
私たち人間は「表情」の満ち満ちた環世界を生きている。
「表情」とは、視覚が主題になるような現象でありながら、
行為の契機や情動や感情の契機をも孕みもつかたちで、生々しい現象である。
みずからの環世界を主体的に創り変えながら生き・学ぶ運動学習者にとって、
\textbf{動いている身体が醸し出す「表情」は、意味づくりプロセスの源}になるという仮説を立てた。
それをもとに、アプリの制作と実践をおこなった。
身体知の学びとしての運動学習支援研究を概観したのち、
動いている身体の「表情」に近しいものに迫っているプロジェクト(研究や作品)をみやり、
アプリ制作のヒントを探り、
\textbf{「運動をもとにして、素朴で抽象的で図形的な見た目を表現する」}ことが、「表情」の感得をうながすひとつの方法であると考えた。

制作したwebアプリ『HJ-Playground』は、あらかじめ計測したユーザ自身/他者の運動データ(各部位の三次元時系列位置情報)を、画面内の三次元空間に動く点群として描き、
ユーザに、それら点どうしのあいだに線分や円などの\textbf{補助線を描きくわえ「表情図形」を作図する}ことを促す。
ユーザには、作図した表情図形を鑑賞しながら、感得している「表情」をオノマトペで命名し、そのさなかで生まれる問いをからだメタ認知で内省記述することを促す。
これらは保存したり、再鑑賞することができる。
これらによって、ひとつの身体運動にさまざまな「表情」を感得することをうながす。

アプリをもちいた実践をおこなった。
対象者のストリートダンサーAは、「oldman」というAの専門とは別ジャンル基本動作を対象に、それが醸し出す「表情」をアプリで探った。
身体内部のあるひとつの仮想的四面体が2通りの「表情」として感じられることを発見したり、
点同士左から右へ一筆書きでつなげることで、風に吹かれて移動するような「表情」を得たり、
両肩をむすんだ線の動きに「鉛筆を指先で軽くつまんで振るような情景」の「表情」をみてとって、予想外に不安定な揺れ方をしているのに気付いたりした。

別の対象者の三段跳選手Bは、一本ゲタ対人運動という不思議な動きを対象に、それが醸し出す「表情」をアプリで探った。
肩と手と肘の3点をむすび、「三角形が相手を突く」という攻撃性ある「表情」をみてとり、自分が到達すべき三段跳の力強い接地のビジョンをみたり、
円をつかった図形を作図することで「自身と他者とのあいだの空間が泡立つ」ような「表情」をみてとることで、それまで三段跳選手である自分が重要視していなかった「脱力」という深い概念について、その意味するところを自分なりに納得したり、
ゲタと足を線でむすびそれを延長してみることで「足首で剣を打ち込む」ような「表情」をみて、三段跳びの助走の一歩目の新しい踏み出し方を発想したりした。
このように対象者らは、自身の運動を新規なかたちで身体で問い、

そののち、実践者らが表情図形の作図をとおして生み出した問いを分析した。
表情図形が、実践者らの問い立てをうながすパタンには、少なくとも2種類あり、
ひとつは、表情図形をみずからの身体に「仮想的図形」として召喚してそこに身体感覚を呼び起こすパタン、
ふたつめは、抽象的な表情図形にたいし、元の身体運動とは異なる日常生活のドメインの情景(できごとやシーン)に見立てる、メタファ的パタンであった。
ふたつめのパタンは、現役の運動学習者であるAとBよりも、すでに現役を引退しているCとBのほうが顕著にみられた。
また、ふたつめのパタンでは、どういうドメインの情景がメタファのソースとしてもちいられているのか、そのバラエティを調べた。
元々は身体運動をもとにした抽象的図形にもかかわらず、実に多様なドメインが、ソースとしてもちいられていた。


\section{身体知の学びの概念モデルの再構成}
以上のように、本研究では、運動学習を身体知の学びとしてとらえ、一人称研究によって実践をとおしてその意味生成プロセスを記述してきた。

\chapter{身体知の学びの概念モデルを再構成する}
\begin{figure}[H]
  \centering
  \includegraphics[width=\textwidth]{./images/zentai.pdf}
  \caption{身体知の学び:輻輳性}          %和文 caption
  % \ecaption{Number of Hyojo entries with/without metaphor (A \& B vs. C \& D)} %英文 caption
  \label{fig:zentai}
\end{figure}

第一部でも第二部でも、本一人称研究がつまびらかにした身体知の学びのすがたには、
運動学習者は、表面的な学習ドメイン内部に縛られず、学習者が生きている全体のなかの他ドメインのものごとをも巻きこみながら問いを立てる、という共通点がみられた。
そのうえで両部にみられた学びのプロセスのちがいには、
第一部は、日常生活の運動やモノを、積極的に運動学習ドメインの文脈へ取り込むありよう、
第二部は、自身の身体運動をもとに作図した抽象的な図形を、日常生活のできごとに見立てること、さらにふたたび運動学習ドメインの文脈に取り込もうとするありよう、
があった。
本論文の最後のまとめとして、この2点を組み込めるかたちで、身体知の学びの概念モデルを再構成する(\autoref{fig:zentai})。
この終わりかたは、終わりなき構成のループ\cite{suwa_fujii:2015}を意識したものである。

\subsection{日常生活全体の「野」(薄グレー平面)}
\autoref{fig:zentai}は三次元の図である。
最上部に位置する横-奥行き方向に拡がる薄グレー平面について説明する。
これは、日常生活全体の場そのものである。
第一部研究の考察(\ref{chapter:yaseika}章)で言う「\textbf{\ruby{野}{の}}」に相当するものである。
薄グレー平面上に、色で塗りつぶされた歪形が配置されている。
それぞれの歪形は、なんらかの「ものごと」を指す。
歪形の色が重要で、その歪形の「ドメイン」を表している。
\autoref{fig:zentai}では、黄色い歪形は、その学習者にとっての実践ドメインのものごとである。
第一部の私の事例なら「陸上競技ドメイン」、第二部の対象者Aの事例ならば「ダンスドメイン」とも呼びうるものである。
「ドメイン\ruby{?}{はてな}」については後述する。
薄グレー平面上に歪形は4つしか描いていないが、実際には無数にありうる。
日常生活の場には、さまざまなドメインのものごとがある、ということを表している。
ドメインとは相対的なものであり、「運動学習ドメイン」や「家事ドメイン」というレベルもあれば、
「陸上競技ドメイン」や「ダンスドメイン」というレベル、
「走りというドメイン」といったひとつひとつのスキルのレベルまで、さまざまありうるし、
さまざまなレベルのドメインは混在しうる(それが「野」である)。


薄グレー平面領域のうち、赤青緑の三色円環(これまで\autoref{fig:noesisnoema}として登場してきた「身体で問う」)と、
その「内部」と「外部」についてくわしくは後述するが、
「外部」に上述の歪形群は位置している、ということはポイントだと述べておく。

\subsection{学習者が生き・学び・創り変えてきた「環世界」(薄グレー円柱)}
\autoref{fig:zentai}の下から上まで伸びる薄グレー円柱について説明する。
これは、身体知の学びのプロセスである。
つまり、これまで登場してきた\autoref{fig:shintaichinomanabi}と同じである。
だがそれを下から上にむかって積み重なってゆくように描くことで、これまでの\autoref{fig:shintaichinomanabi}版よりも、「歴史的」なありようを強調している。
薄グレー円柱は、学習者が主体的に生き・学び、創り変えてきた環世界そのものなのである。

薄グレー円柱の内部に注目してほしい。
歪形群が縦方向に重なるように配置されている。
この各歪形の色は、前項で述べた薄グレー平面上の各歪形の色と対応している。

\subsection{「野」と「環世界」の界面=身体で問う(薄グレー平面と薄グレー円柱の関係性)}
\label{subsec:grayheimenuandgrayenchu}
赤青緑の三色円環は、\autoref{fig:noesisnoema}と同じく「身体で問う」ということを表している。
三色円環とその内部は、薄グレー円柱と薄グレー平面との接面にひとしい。
これはとても重要な事態を表現している。
\textbf{
  「身体で問うこと(認知作用=ノエシス)と問い(認知内容=ノエマ)との相互限定関係」
  は、
  「主体的に生き・学んできた環世界」
  と
  「日常生活全般の野」
  との「界面」として生じる
  }
ということである。
だとすると、\textbf{身体知の学びにおける意味の源としての「表情」とは、}単に環世界(薄グレー円柱)の事物が帯びるもの、というだけでなく、
\textbf{「環世界」と「野」の界面において生じている}(薄グレー円柱と薄グレー平面の接面)ことになる。
そして\textbf{「意味」とは、}創り変えられていく環世界そのもの(薄グレー円柱そのもの)なのだが、
それすなわち、\textbf{「環世界」と「野」とが出会い続けた歴史に宿っている}のである。

「野」と「環世界」の関係性について、さらに、「ドメイン」の観点も盛り込むと、第一部・第二部の結果をより統合できるようにある。
まずは、\autoref{fig:zentai}のとおりに、図を三人称視点から眺めよう。
薄グレー平面(日常生活全体の野)には、さまざまなドメインの歪形群が広く散らばっている。
いっぽう薄グレー円柱(学習者がひとつの身体でもって生き・学び・創ってきた環世界)においては、薄グレー円柱断面と歪形の大きさはそう違わず、
歪形群は、縦方向に積み重なるように、所狭しと存在している。
さまざまなドメインごとがギュッと寄り集まっている。
\autoref{fig:zentai}を真上から眺めることを考えればわかりやすいが、
積み重なった歴史(環世界)を見通すようにすると、
「輻輳」している、という構図になるのである。
\autoref{fig:zentai}は「\textbf{輻輳性}」(第二部\ref{subsec:fukusousei}項で述べた)を表しているのである。



\subsection{事例で考える}
これをふまえて、いまこのとき、身体知の学びの実践者が、ある状況に出くわしたとしよう。
このとき生まれる「問い」は、上述したように、野と環世界との出会いである。
だからこの問いの「色」は、あらかじめ決まるものではない。
たとえば、\autoref{fig:zentai}の「ものごとb」はブルーグレイだが、
学習者がこのものごとbについて問うたからといって、
その問いがブルーグレイになるとは限らない。
野と環世界とが接面を保っているかぎりにおいて、問いは生まれるのだから、
問いの色は、環世界に輻輳したものごと群のいろいろな色からも照らされてこそ決まるものだからである。
すなわち、野で出くわしているものごと群のドメイン群(歪形の色群)と、
環世界をなしているものごとのドメイン群との、
かけあわせによって問いのドメインは決まる。

本研究の第一・二部の結果を事例に考えよう。
まずは第一部前半のケースである。
私が陸上競技場で、陸上競技の練習を文字通りしていた段階である(第一部物語考察\ref{chapter:yaseika}章でいう、「栽培的」\cite{levi-strauss:1962}な学び)。
陸上競技場という場には黄色い歪形ばかりがあり、
私の環世界にも黄色い歪形ばかりが降り積もっていたのだろう。
このときに生まれる問いは、
野と環世界との接面で生じる歪形は、黄色くなるのだろうと考えられる。

第一部後半のケースを考えよう。
野生的\cite{levi-strauss:1962}に学ぶようになった段階である。
日常生活の野において、さまざまな色の歪形群と出くわす。
私の環世界には黄色い歪形ばかりではなく、いろいろな色の歪形が降り積もっていたかもしれない。
だが重要なのは、このときに生まれる問いは、
野と環世界との接面で生じる歪形が黄色くすることができるということだ。
つまり、野にある状態の色と、問いの内容になるときに色を塗り替えることができるのである。

第二部のケースを考えよう。
第二部の結果で見られたのは、直接に運動学習の実践に紐づけるのではなく、
運動学習ドメインではない、日常生活の別のドメインのできごとに喩え(抽象的な表情図形にそういうドメインの情景を召喚し)ていた。
さらには明示的にそれを介することで運動学習の実践につなげているような学びかたもあった。
ではこの図のとおりこの図を三人称視点から眺めてみよう。
HJ-Playgroundで作成する表情図形は、抽象的な見た目をしている。
いわば、もとの身体運動というドメインを離れて、抜け殻のように「ドメインが未定」とでも言いうる状態になっている。
\autoref{fig:zentai}のものごとxは黒く塗りつぶしてあるのが「ドメイン未定」を表している。
野にあるほうの対象のものごとxはドメイン未定(だが抽象的な図形)、としてみる。
環世界のほうはどうなっているのか?
環世界のほうでは、いろいろな色の歪形と、実践ドメインの歪形がありうる。
このときに生まれる問いの色(つまり「表情」)は、いろいろな色の歪形群の「輻輳」したそれに照らされて決まってくるものである。
つまり、見立ての構造になっている。

ものごとが「表情」豊かに感得されるとき、
あるいは、
「意味」が含蓄あるものとして醸成されているとき、
それは、
野で出くわした対象が、
環世界の多ドメインのものごとの輻輳として(輻輳に照らされて)、
多義的・多重的に、立ち現れてくるということなのだろう。
そしてやはり、主体的な身体知の学びなのだから、
そのように立ち現れてくるというのも、そのように主体的に身構える、ということなのであろう。
ひとは、「世界に身を挺した主体」なのである\cite{merleau-ponty:1945}。
以上が、研究をとおして獲得した身体知の学びについての視座である。


% \section{MEMO}}

% インゴルドも言っている。
% 客観性・普遍性・再現性・論理性。
% 客観性とはなんなのか。
% 「アトラス」(生物や物理現象の図鑑)を例に、客観性の歴史を紐解く。
% R.ダストンは、
% 佐伯はおもしろい研究とは・・・
% ロレインダストン


% プラグマティズム哲学の流れ。
% ヴァレラでもそうだった。
% Shustermanでもそうだった。
% 樋口でもそうだった。
% HCI領域では、近年、「自伝的デザイン」の機運が高まっている。
% K.Hookがそうである。
% 本研究では、かすってはいたし、におわせてはいたのだが、そのプロセスを丹念に扱うことはしなかった。
% 彼らの

% 「守られた予定調和な場」に収束してゆくと、そこに「意味」はつくられない。
% 意味とは、生活と実践の界面にある。
% 「学び」には
% 日常生活で出くわす「状況」は混沌としており、予測もできない。
% そうした状況に居合わせたとき、なんとか、やりくりする。
% そうやって、なんとかやっていかざるを得ない。
% 守られた学びから、そういうままならない、生きている場に学びの場を転じた。
% スキルの学びに、そのように意味をもたせたのである。
% 思い通りになる場から、思い通りにならない場へと、身を転じた。
% AIにはできない学び方である。

% \todo{質的研究の成果呈示:「一式」をつくりみせる}

% \todo{
%   ひとは、身体的なノエシスとそれにともなうノエマとが、相互限定的な関係性を維持することで、当人の身体に連関した環世界を生きている。
% ひとは世界に身を挺した主体\cite{merleau-ponty:1967}として、世界にむかって身の構えをとっている。
% 生きていることの全体から「意味」を浮かびあげる。
% 新奇の状況に居合わせたとき、身体でもって問いかえす。
% すると状況から「意味」が、図として浮かび上がってくる。
% なんとかやりくりする。
% 一時的にあえて滑らかな協応を崩すことさえある。
% それが意味づくりであり、意味を作りながら生きている。
% }

% (身体性認知でもたびたび引用されるヴァレラは、仏教や現象学を援用しながら、研究者自身が身体としてある反省を実践することで認知を研究すべきことを唱導していた)。

% 自己矛盾のポイント
% \begin{itemize}
%   \item 知には、身体知という知と非身体知があり、研究対象は身体知だが、研究という営みは非身体知だと考えている。
%   \item 知はすべからく身体知であり、研究対象が身体知だし、研究という営みも非身体知なのだから、そういう研究せよ
% \end{itemize}

% \cite{neustaedter:2012}
% 「みえ」を呈示(これこれこんな着眼点たちがこう関係しています、という一覧の図示)
% ツールキットの呈示。
% パタンランゲージの提出
% KJ法のA型図解化の提出。
% 概念モデル。

% どれもが、「一式」をみせている。
% ひとつの着眼点やパタンだけを取り出しても、それは成果としての意義はそう大きくない。
% そういう実利主義的な考え方ではなく、
% 「全体のセット」にこそ意味がある。


% \todo{Somaesthetics:「机上の哲学」から「現場での実践」へ
% 美学の流れでは、伝統的には西洋哲学で形而上での論述にあふれていた。

% カントしかり、バウムガルデンしかりである。

% そこから、シュスタマンは、プラグマティズムの流れを合流させることで、
% 学者自身が身体(soma)をつかって実践することの重要性を説き、
% 「Somaesthetics」を提唱した。

% 実践するということは、その自らの身体を研ぎ澄まし、より望ましいありかたへと「鍛錬」してゆくことにほかならない。

% この鍛錬をとおして、さまざまな問いを生み出し、そこから洞察を得て、「哲学」へと醸成してゆく。

% こうしたボトムアップなアプローチこそが健全なのである。

% 本研究はまったく同意する。}

\section{補足}
\subsection{環世界概念の別解釈について}
\label{sec:horon_kansekai}

ユクスキュルは環世界の考え方によって、「生物は固有な世界を生きている主体である」ということを「機械操作係である」という言い方でも書いている。
2つの言い方はいずれも、「生物が客観的世界に組み込まれた機械である(生理学ではそのように生物を記述する)」ことを批判的に飛び越えようとしている。
しかし、本研究からすればこの2つは互いに異なる着地点であることを補足しておく。
本研究は前者の言い方に賛成である。

後者の「機械操作係」という表現は、本研究にとっては好ましくない。
すこし後の知能科学の歴史からみれば、機械操作係の考え方は、生物と機械とを統一した制御・通信モデルでとらえる\textbf{サイバネティクス}\cite{wiener:1948}」と類同している。
サイバネティクスは、その後認知科学で「人間の心」のモデル化する「情報処理モデル」として輸入された。
情報処理モデルは、心と身体とを分離して扱っており、本研究はまさにそこを問題視しているのであった。
また。
サイバネティクスと情報処理の「フィードバック」の考え方は、「環(Krais)」のかたちが意味する「部分の相互関係で全体が維持されると同時に、全体のなかでこそ部分は存続できる」という関係性とは異なる。
したがって、身体知の思想を推し進める本研究からすると、機械操作係という表現は好ましくない。
ヴァイツゼッカーもこの後者の点に批判的である。
\footnote{ユクスキュルの息子で医師であったトゥーレ・ユクスキュルもまた、父ヤーコプの環世界の考え方を人間へと敷衍した「状況環」という考え方を展開している。
ちなみに、状況環はヴァイツゼッカーに近い考え方ではあるが、トゥーレはヴァイツゼカーを引用しなかった。
これには、感情的な事情があるらしいとのことである。
}。
\subection{「表情」なき世界:離人症}

読者のなかには、「そこに建っているビルには表情を感得できない。あれはただの無機質な物体にすぎないじゃないか」という疑問をもった者もいるかもしれない。
こうした指摘に対する反駁となりうる記述も、文献\cite{hiromatsu:1989}には書いてある。
\begin{quotation}  
  なるほど、現相のうちには、これというほどの感情価やこれというほどの即応価を帯びていないものもある。
  だが、その場合でも、表情価が端的に\ruby{零}{ゼロ}なのではない。よしんば零としか言いようのない“欠如態”の相にあるとしても、
  欠如態は(いわゆる“無色透明”が一種の色であるのと類比的に)それ自身、れっきとした価値態であることを忘れてはなるまい。\\
  (『表情』, p.79)
\end{quotation}

このようなかたちで廣松は、感得される表情現相は
「人物や動物の顔面表情や身体的挙措表情には限られない。
原基的な相においては(中略)、一切の現相が\ruby{悉}{ことごと}く表情性を帯びて
\footnote{鋭い読者は「Xが表情性を帯びる」という表現方式は表情にふさわしくないのでは?
と思ったかもしれない。
それは正しい。
しかしその部分こそ、「表情に対して語彙が貧困」という廣松が指摘する問題でもあろう。
実は廣松は、「表現の便宜上、以下では事物が表情性を帯びた相で現前するかのように記す方式をも辞せないようにしよう(『表情』, p.10)」と
断りをいれたうえで「Xが表情性を帯びる」という書き方をしている。
}
感得される」と説明する。

では反対に、もし私たちが、ほんとうに、「表情」をまったく感得できないのだとするならば、どうなるのか?
私たちの体験の前にくりひろがる環世界は、どのようなものになるはずなのだろうか?

離人症という精神疾患がある。
離人症患者は、次のような体験をする。
\cite{nakamura:1979}によれば、離人症と診断された24歳のある女性は次のように語ったという。
\begin{quote}
  音楽を聞いても、いろいろの音が耳の中にはいりこんでくるだけだし、
  絵を見ていても、いろいろの色や形が眼の中にはいり込んでくるだけ。
  何の内容もないし、何の意味もない。(\cite{nakamura:1979}, p.47)
\end{quote}
また、42歳のある女性\footnote{診断は未確定とのこと}の症例では、患者からの手紙のなかで次のように綴った。
\begin{quote}
  暑い寒いという温度の高低はわかりますが、暑い寒いといった感じはどうもピンと来ません。
・・・本当にただ視聴覚に訴え、肉体的に感じることだけで、精神的な感じの方は相変わらず何も感じることができません。  
(\cite{nakamura:1979}, p.47)
\end{quote}

中村によれば、私たちの「共通感覚」が喪失しているのである。
共通感覚とは、体性感覚を中心として諸知覚が統合された感覚であり、
私たちが生きるうえでの基本的な感受性・常識の基盤となっていると中村はいう。
共通感覚を失えば、その当人の環世界からは「表情」が失われてしまう、と著者は考える。
それが上記の事例である。
なんとも殺風景的である。
「そこのビルに表情がない」と考えるひとは、
\subsection{立ち現れ一元論 by 大森荘蔵}
\label{sec:horon_tathiaraware}
大森荘蔵による一連の哲学「立ち現れ一元論」の象徴的な文言をいくつか載せておく。


% \todo{岩肌のやつがあったよなああ。}

たとえば私たちは、色や形を「世界」に属する性質として、感情を「私」に属する性質として描きがちである。
大森はそれすら否定してみせる。象徴的な言明をいくつか引いておこう。
\begin{quote}
  心という袋をひっくり返しにして、風景の立ち現れに吐き出す。
\end{quote}

\begin{quote}
  一本のネクタイの色はさまざまに見える。朝日の中で、木陰の中で、夕闇の中で、蛍光灯の下で、その色合いは微妙に変わる。 また、黄疸の人、色盲の人、呉服屋さんにはまた別様に見えよう。 これらの十人十色が全て「心に映じた」色であるというのであれば、ネクタイの客観的な色は一体何色であればいいのか。それは、カメレオンの本当の色は何かというのと同じように意味をなさない問いであろう。(\cite{ohmori:1976}, p.107)
\end{quote}
\begin{quote}
  一本の樹木もネクタイの色と同様、陽炎の向こうで、霧の中で、安物の窓ガラスの向こうで、二日酔いの人の目に、近視の人、老眼の人の目に、形を変えて見える。 この時、その樹の客観的な形とは正常な状況で正常な人に見える形だという人は、単に一つの「標準形」を指定しただけである。 それはカメレオンの「標準色」として緑を、ネクタイの「標準色」として、売り場の店員に見える色を(売手市場の場合だが)、指定するのと変わりはない。(\cite{ohmori:1976}, p.108)
\end{quote}
\begin{quote}
  (※気分や「心地」が我々の「心の内」にあるとしか言えないという考えに対して、)
しかしはたして、例えば恐ろしさは、すっぽり「心の内」に抱かれているのだろうか。歯医者と、あのピカピカ光る拷問器具をこわがるとき、恐ろしいのは、これらの道具と拷問者である。恐ろしさは、それらの人と事物に、いわば「附着」しているのである。 それを引き剥がして、一方に、怖くもなんともない歯医者と道具、そして今一方に、純粋結晶のように取り出された、純粋の恐怖(恐怖のエッセンス、恐怖のエキス)、 そして、この純粋恐怖だけが、私の「心の内」にある。しかし、もしそうなら、私は一体何が恐ろしいのだろう?(\cite{ohmori:1976}, p.116)
\end{quote}

このように大森は、「立ち現れ」の一元論を展開している。
「外なる物、内なる心」という「二段構え」の構図のなかに、「表情」はない。
「表情」は立ち現れてくるものである。

ほかにも大森は、以下のようにも書いている。
\begin{quotation}
    他人の「胃が痛い」という発言をその人の「胃痛」を構成する振舞の一部として受け取る。
    この発言以外に「胃痛」を構成する振舞は多々ある。
    身をよじる振舞、ものを食べられないという振舞、冷汗という振舞、
    ある種の表情という振舞、動作の不活性という振舞等を、あげればきりがない。
    これら無数の振舞とならんで「胃が痛い」という発言の振舞が「彼は胃痛」という情景を構成しているのである。
    「胃が痛い」という発言はこの「彼は胃痛」という情景の「報告」ではなく、その情景の一部なのである。(\cite{ohmori:1971}, p.26)
  
\end{quotation}