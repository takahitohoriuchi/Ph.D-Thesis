\chapter{物語の意義}
\label{chapter:monogatarinoigi}
本稿では、Tがアスリートとして、生活と競技を分けずにスキルを学ぶ様を物語として描いた。
研究論文を物語の形で執筆することは、どういう意義を持つのだろうか?
研究分野に何をもたらすのであろうか?

物語は、一人称視点に立つからだメタ認知の記述を拠り所にして、構成されたものである。
従来の科学観に則るならば、「一人称研究による物語は、普遍的に検証された知見ではないから、知の蓄積としての価値はない」という言説も成り立つかもしれない。
しかしながら、本稿に示した学びの物語は、普遍的に証明された「情報」の提示ではなく、一人のアスリートがスキルを学ぶプロセスで醸成した「意味」の例示である。
一人の人間に生じた事例であり、個人固有性やそのアスリートTが置かれた状況に色濃く依存した物語であるが、Tがどのような意味を醸成しながら「生きた」のか、そのリアリティを描き出したものである。
そこには、アスリートがスキルを学ぶとはどういうものごとなのか、どんな問題意識を育まれ、生活と競技がどのように一体となるのか、という知の姿が描かれている。
それを論文として伝えることは、今後のスキル研究に新たな視座を与えるものである。

科学論文は、普遍性・客観性・論理性という三大原則に則って、普遍的な知見の獲得を目論むものである。
しかし、一方で、それらの原則は「生きる」姿を削ぎ落としてしまうことに、つまり、科学的方法論は人が「生きる」ことから乖離してしまっていることに、哲学者・中村雄二郎\cite{nakamura:1979}は懸念を示している。
人が生きる上で醸成した「意味」を記述した物語には、人が生きる姿が色濃く描かれている。

作家・小川は、臨床心理学者・河合が言うところの「物語(物語るという行為)」に、「事実を否定する絵空事ではなく、『いのち』や『たましい』を手触りあるものとして刻みつける」(\cite{ogawa_el_al:2011}、p.129)意義があると解説する。
すなわち、「科学論文になくて物語にあり」得るものは、手触りである。物語には手触りがあるからこそ、読者に「感触」\cite{uragami:2015,suwa:2016}を届けることができるのではないだろうか?
そして、読者は、その物語を自分ごととして咀嚼し、自分自身の生きる身体で以って「問い」、自らの生に活かすことができるはずだ。

FolkPsychologyという分野を提言したブルーナー\cite{bruner:1990}も、物語は「本当らしさ」や「実生活らしさ」を伝える媒体であると説く。
(物語に内在する)時系列性を以ってプロットを語り、通常から逸脱したもの(「必ずしも普遍的ではない知の姿」のことであると著者は理解している)を読者に理解可能な形で提示するものである。
ブルーナーは物語を、「人のコミュニケーションにおいて、もっとも身近にあり、もっとも力強い談話形式の一つ」(p.108)だと明言している。
つまり、個別具体的で、状況依存的で、必ずしも普遍的ではない(一人の人間に生じた)ものごとであっても、十分他者に伝わる媒体であると著者は解釈している。
研究論文として物語を提示する意義は、そこに描かれた知の姿が内包する「意味」を(本論文の場合は、アスリートが生活の中から、様々な問題意識の変遷を経て、スキルを学ぶプロセスにおける知の立ち上がりの姿)、読者である各々の研究者が「感触」を伴って納得する場を提供し、今後のスキル研究に新たな視座を提示することにあるのではないか。


% エピソード記述では、以下三種類の語りが含まれることを論じる
% \begin{description}
% \item[背景]体験が生じた舞台にまず読み手を招じ入れる
% \item[エピソード]書き手の心揺さぶられた様を語る
% \item[考察(メタ観察)]書き手が心揺さぶられた理由を添える
% \end{description}








