\part{「アスリートとして生きる」学びの物語}
\chapter{物語をはじめる前に}

\section{物語の概要}
\label{sec:deca}
これまで書いてきたように私は、
研究者かつ陸上十種競技の実践者として、
からだメタ認知を駆使しながら「走り」の学びを実践してきた。
私は十種競技選手として、「走り」に十種競技の基礎があると信じている(この考えは、物語執筆中の現在も変わっていない)。
陸上競技の(十種競技の)文化に生きてきた私からすると、十種競技選手として走りに基礎を求める態度は決して珍しくはない。
十種競技得点における走種目得点の重要性を示す報告(e.g.\cite{yasuda:2013})もある。
しかも私は「走り」を不得意としていたのだ。
第一部のメインパート\ref{chapter:monogatari}章において、陸上十種競技選手である私が、生きていることと競技が渾然一体となりながら「走り」をより望ましいように創りかえてゆく意味生成プロセスを、物語として描く。


なぜ「物語」なのか?
アスリートである私が試行錯誤しながら、問題意識が変遷する様を描こうとするならば、私本人の一人称視点から見える世界と自身の関係性を時系列的に、そして、個別具体性を色濃く残したまま語る他ないからである。
それは必然的に「物語」になる。
物語であることの意義は、\ref{chapter:monogatarinoigi}章で説明する。

物語本編の\ref{chapter:monogatari}章では、私が記録し続けたからだメタ認知の記述を拠り所としながら、私が語り手となって、スキルの学びの物語を描く。
物語において、実践を私が生み出した問題意識は、文中でハイライトして表記する。
文中で鉤括弧「」で括った文言は、私にとって重要な変数や意識である。
先にも述べたが、新しい変数に着眼することが、一般に学びの始まり(必要条件)である。

2015年4月23日から2019年10月9日までの間に、623 件(総文字数 496,900 字)の記述を書き溜めてきた。
これを拠りどころとしながら物語として構成した。
623件の記述から、「生きていることと渾然一体であるアスリートの学び」の物語の骨子になりうるものを、私が厳選した(2015年12月5日から2017年9月23日の範囲であり、全623件のうちほんの一部であることは述べておく)。

身体知の学びのプロセスが、生きているなかで営まれ、生きていることと渾然一体であることを主張してきたが、実は、物語開始時点では「主人公の私」は渾然一体がよいことを未だ自覚できていない。
主人公である私は、アスリートおよび自身の学びを探究する研究者として、学びを進める中で様々な出来事をきっかけとして、2016年11月頃(\ref{sec:kegagenin}節)から次第に、「生活と競技の渾然一体」を自覚的に為すようになる。
「物語の語り手である私」は、「主人公である私」の学びが生活の上にあることを強調するために、可能な限り、学びに起因・関連したと思われる生活上の出来事にも随所で触れながら、物語ることにする。




