\chapter{学びの物語本編}
\label{chapter:monogatari}

\section{接地面から体幹を介して、肩甲骨へと繋がる意識}
\label{sec:denpa}
学部4年シーズンが終了した2015年11〜12月、私は「接地」に留意していた。
「フラット接地」(足裏全面が同時に接地)によって、「臀筋\footnote{
お尻の筋肉のこと。
大腿骨と骨盤をつなぎ、股関節の動きを制御する。
人体のうち最大の筋肉であり、陸上競技において重要な筋肉である。
}
で地面を押し、高出力を発揮しよう」と試みていた。
「身体が固く力みやすい」という癖をもつ私は、足首が曲がりにくく、つま先から接地しがちである。
自らの癖を強く自覚してからは、むしろ「カカトを振り下ろす」ことを意識することによって、フラット接地に近づけるという考えに至っていた(問題意識1:カカトを振り下ろすことによって足裏前面で接地し、臀筋を使って地面を押す)。

そんな中、私は学士卒業論文を執筆しなければならなかった。
執筆作業に追われ、練習時間はあまり確保できなくなった。
本格的な運動ができない状況で私が優先的に取り組んだのは、自室でできる「体幹トレーニング」である。
体幹トレーニングとは、腹筋群・背筋群を鍛える運動をさす。
その頃、体幹トレーニングには、自分なりに価値を見出していたのだ。
それが垣間見える記述を\autoref{fig:20151205}に示す。
「ボディバランスのトレーニング」が、体幹トレーニングのことである。
下線部で語られているのは、身体を意識的にあれこれ動かす以前に、まず「体幹部のコンディション(=ボディバランス)」を整えておく(問題意識2)ことの重要性である。
こうして私は毎日30分の体幹トレーニングに取り組むことを習慣づけた。

\begin{figure}[h]
\centering
\begin{mynote}
ボディバランスを重視したトレーニングは毎日のように欠かさないべきだということ。
(中略)久しぶりの走りだというのに、普通に過去最高級の走りだった。
芝での流しだが、重心の高い位置がキープしたまま、下に落ちずに進んでいく。
(中略)間違いなくここ数日間のボディバランスのトレーニングの成果だと思っている。
身体の、いろいろな部位の、いろいろなインナーマッスルに力が入る状態になっていると、力を使わずに脚が上がったりする。

\textbf{[2015年12月5日、からだメタ認知記述から抜粋]}
\end{mynote}
\caption{体幹トレーニングに自分なりの価値を見出した、私の記述}
\label{fig:20151205}
\end{figure}

やがて卒論執筆が終了し、2016年2月からは冬季練習に移った。
すると、前述のフラット接地による地面反力が、肩甲骨あたりにまで伝わってくる体感を掴んだ。
卒論執筆期に注力的に鍛えた体幹部が、臀部から肩甲骨付近へ橋渡ししたのだろうか。
私は接地面から肩甲骨までを「ひとつながり」に感じるという実感を得た。
走りにおいて肩甲骨の重要性が叫ばれる意味を、このとき初めて自分なりに納得できたのだ。
\autoref{fig:dempa}は、当時の私がその体感を探る様子である。
接地直前に踵にアクセントを置きつつ、肩甲骨から腕を振り込み(局面1)、接地瞬間に臀筋と体幹部を締めることで(局面2)、直後に地面反力が肩甲骨あたりまで伝播するのを感じる(局面3)のだ。

\begin{figure}[h]
\centering
\includegraphics[width=10cm]{./images/dempa.pdf}
\caption{地面反力が右踵から肩甲骨に伝播する体感」を探る私の様子(2016年2月)}
\label{fig:dempa}
\end{figure}

その発見以降は、肩甲骨を大きく動かすことを鍵とした。
2016年11月上旬まで、
% すなわち修士一年シーズンを通して、
すなわち2016年シーズンを通して、
私は肩甲骨の意識を拠点としながらスキルを模索した。
すると、私がこれまで気にかけてこなかった「腕振り」が自ずと意識の前面に浮上した(\autoref{fig:20160831})。
記述後半の「後ろ寄り」とは、腕の動きが、両肩を結んだラインより後方に広い可動域をもった状態をさす。
それまで無自覚的に「腕振り」として受け容れていた動きを、肩甲骨を大きく動かす体感をもとにして、「腕掘り(下線部)」という独特な表現へ再編成したということである(問題意識3:接地面から肩甲骨までのつながりの結果としての「腕掘り」)。

本節で登場した私の問題意識を列挙すると、以下のようになる。

\begin{itemize}
\item 問題意識1:カカトを振り下ろすことによって、足裏前面で接地し、臀筋を使って地面を押す
\item 問題意識2:体幹部のコンディションを整えておく
\item 問題意識3:接地面から肩甲骨までのつながりの結果としての「腕掘り」
\end{itemize}

\begin{figure}[h!]
\centering
\begin{mynote}
S(※私の練習パートナー)と話しながら、腕振り、肩甲骨の使い方について考える。
\underline{「腕掘る」という表現について、(中略)掘る位置は、自分のかなり手前なのだと。}
最\underline{初の加速のときは、もしかしたら、大きく大きく、遠くから掘ってこないといけな}
い\underline{のかもしれない。}
(手前を掘ることによって、自然に腕振りのレンジが、後ろ寄りになるということだ。
自分の身体より後ろの可動域が拡がる。
後ろで腕がしっかり動く。
すなわち、肩甲骨がよく動いているということも満たされる。

\textbf{[2016年8月31日、からだメタ認知記述から抜粋]}

「※」は本稿執筆時に著者が加えた注釈を指す。
以降の記述でも、適宜注釈を付してある。
\end{mynote}
\caption{体振りを「腕掘り」と解釈する、私の記述}
\label{fig:20160831}
\end{figure}

\section{「百均LEDトラッキング」の実践}
\label{sec:led}
% 修士課程に進んで
2016年シーズンに入って
まもなく(2016年4月)、私は、研究プロジェクト型授業のメンバーであるX氏(Tが在籍する大学の教員)に出会うことになる。
Xはインタラクション研究者及びメディアアーティストである、
それからというもの、私は、私自身の学びと研究について、Xと時折議論するようになった。

私は、自身のスキルを言葉以外の方法で外化する手法の可能性をXに持ちかけた。
私がおぼろげながらに想定していた外化手法は、スポーツ科学で常套的な「動きの可視化」である。
この観点で、Xとの議論から、「LEDライトを使えば、動きの軌跡がとれる」という案が湧き出て、これに着手した。
名付けて「百均LEDトラッキング」である。
私はさっそく百円ショップでLEDライトを購入した(\autoref{fig:ledlight})。
\begin{figure}[h]
\centering
\includegraphics[width=5cm]{./images/LED.pdf}
\caption{私が百円ショップで購入したLEDライト}
\label{fig:ledlight}
\end{figure}


LEDライトを身体に(以下の事例では右手と右膝)装着した状態で、暗所で走る。
運動の様子を固定点から長時間露光撮影すると、装着部位の動きが光の軌跡として表れる(\autoref{fig:ledpath}は、ある日の撮影結果である)。
自宅最寄りの百円ショップで発見したドーム型LED(直径3.5cm)を使用し、撮影はスマートフォンカメラアプリケーション「MagicShutter(当時数百円)」をもちいた擬似露光撮影効果で賄った。
実践は、ナイター照明の点く普段の陸上競技場ではなく、夜陰に包まれる土グラウンドで行なった。

\begin{figure}[h]
\centering
\includegraphics[width=10cm]{./images/ledpath.pdf}
\caption{百均LEDトラッキングの光軌跡(画像右手前に向かって、6歩走っている。
上部の軌跡は右手、下部の軌跡は右膝。
)}
\label{fig:ledpath}
\end{figure}

「動きの可視化」といえば、本実践よりも精密なテクノロジーを用いた「計測」(例えば、モーションキャプチャ手法など)が一般的だが、なぜそれを選ばなかったのか。
それらの計測には、使いこなすための知識が必要であるし、実際の計測に際しては多大な労力がいる。
私は、アスリートとして学ぶ実践を最優先にしていたため、練習時間を削ってまで、
テクノロジーを使いこなすための一連のものごとに労力を割くわけにはいかなかったのだ。

スポーツ科学では計測に時間をかけがちであるが、計測にかかる労力が、実は、学び手本人の問題意識の醸成プロセスの継続性を阻害する致命的要因になってしまうことを、諏訪・矢島・筧・仰木\cite{suwa_et_al:2012}は指摘している。
問題意識が生じたら、それをこのようなツール制作にすぐ反映させ、実践して可視化する。
早いサイクルで、つくり、いじることを通して、新しい変数を見出したり、問題意識を生み出すことは、学びにおいて本質的である。
「百均LEDトラッキング」実践のような「手軽さ」は鍵なのだ。
こういう意図からも、私は本実践に取り組んだわけである。

右手と右膝にライトを装着した。
右手を選んだのは、私にとって、上述した問題意識の「腕掘り」が軌跡としてはどうなっているのかを観察して、深く考えることが切に必要であると考えたからである。
右膝は、股関節の動きを表す重要部位として選んだ。

テクノロジー(モーションキャプチャ)によって
可視化する場合、一般に、身体は「可視化の対象」として、グラフ化されて描かれる。
これに対し、本実践で得た\autoref{fig:ledpath}は、その意味での「グラフ」ではない。
自分が主体として、自身の身体を空間に塗りつけるように描いた軌跡なのである。
それ故であろうが、私自身が図6の軌跡を見たとき、単に「グラフ」を眺める以上の、自分が自分を塗りつけたような感覚が迫ってきた。
すなわち、前節の「腕掘り」という体感が、観察される軌跡と符合したのだ。

すると興味深いことに、私は、「右手が掘った対象」(それは空気である)をも、この軌跡の中に見出すことになった。
そして、その空気は、「粘性のある空気」であるという解釈も生まれた(問題意識4:「粘性のある空気」を腕で掘る)。
以下に、該当する記述を掲載する(\autoref{fig:20160915})。

もし、テクノロジーによって可視化していたら、軌跡は「右手」以上のものを表さないので、それを「グラフ」として見るだけで、空気やその粘性にまで意識を及ばせることはできなかったであろう。
百均LEDトラッキングの実践は、「モノ(この場合はLED)を通して、自分の身体を手触るように問う」ことの有効性を、身を以て知った経験であった。

別の日にも、右手の百均LEDトラッキングをX氏(Xは陸上未経験者)とともに実施した(\autoref{fig:ledpath2})。
Xの軌跡と比較してみると、「粘性のある空気を腕で掘る」感覚(問題意識4)はまた少し違うしかたで増長される。
丸で囲った頂点近傍にに着目してほしい。
これは腕振りの腕を振り上げから振り下ろしに切り替わるフェーズである。
私の軌跡は滑らかだが、X氏のは尖っている。
これは、語り手の私には以下のようにみえてくる。
ふつう「腕振り」とは、「振り上げ」と「振り下ろし」2フェーズの繰り返す振り子のようにとしてとらえられ、そういう動きになりがちなのだろう。
すると、振り上げ切った直後に振り下ろしという別フェーズへ切り替わり、それが頂点の「とんがり」として表れるのではないか、と私は考える。
いっぽう「腕掘り」の意識は、腕の振り上げから振り下ろしのフェーズを「途切れ」させることなく、滑らかにつなげる意識でもある。
粘性のある空気を「掘る」感覚(問題意識4)のはこの振り下ろしへの切り替わりと振り下ろしフェーズに生じる感覚である。

\begin{figure}[h]
\centering
\includegraphics[width=\textwidth]{./images/ledpath2.pdf}
\caption{右手の百均LEDトラッキング:私(左)とX氏(右)の比較}
\label{fig:ledpath2}
\end{figure}


\begin{figure}[h]
\centering
\begin{mynote}
この光の軌跡のカーブ(※\autoref{fig:20160915}内でポインティングした部分)が、なんとなく大きくゆるやかになっているのがいい走りなのだ。
(中略腕を振り下ろす、振り下ろすというか、腕を掻くように掘るようにするときの「質感」が重要なのだと!!S(※私の練習パートナー)
も「『ぐ〜っっぐ〜っっ』っていう感じなんですよね」と前に言っていた。
そのことと同じことだと思うのだが、何も抵抗がなく一瞬で軽く振り下ろすように腕は動いてはいけない。
(中略)\underline{空気よりも密度が高い、もう少し粘性もあるような、そういうものをま}
さ
\underline{に「掘る」感じで腕は動くべきなのだ。}
(中略)一歩一歩確実に加速していくためには、この軽すぎない質感が大事なのである。

\textbf{[2016年9月15日、からだメタ認知記述から抜粋]}
\end{mynote}
\caption{「腕で掘る」対象は「粘性のある空気」と解釈する、私の記述}
\label{fig:20160915}
\end{figure}

本節で登場した私の問題意識は、以下である。
\begin{itemize}
\item 問題意識4:「粘性のある空気」を腕で掘る
\end{itemize}

\section{「軸」の意味を納得する}
\label{sec:jikunoimi}
まもなく私は、「粘性のある空気を腕で掘る体感」を「軸」という表現にも紐付けた。
実は、私はそれまで「軸」への着眼を敢えて避けてきた。
なぜなら、軸という定型句を安易に使ってわかった気になると、思考停止に陥ると考えていたからである。
「粘性のある空気を腕で掘る体感」という問題意識を持つに至って、ようやく、「軸」の概念が私の腑に落ちた。
どう腑に落ちたのだろうか?当時のからだメタ認知記述をみてみよう(\autoref{fig:20160917})。
「体幹まわりの筋肉(\ref{sec:denpa}節)が、全身連動を生むための拘束条件として機能すれば軸が形成される(問題意識5)」という解釈に至ったのである(下線部参照)。
詳しく説明する。
腕掘りが起こる「接地面から肩甲骨までの『伝播する体感』(問題意識1)」や、「掘られる空気の『粘性』(問題意識4)」は、動きの「重み」を表していると、私は解釈する。
それが生じるとき、腕(四肢)を「拘束」できている状態にある(拘束されていない四肢は、各々独立して「軽々しく」動いてしまう)。
そして、拘束を生むのは「体幹部の筋群(問題意識2)」であり、それこそが軸なのだ、という解釈である。

\begin{figure}[h]
\centering
\begin{mynote}
やはり「軸」が大事なのだと。
(中略)\underline{体軸は保たれたまま、その上でうねうね}し
て
\underline{いることが重要。}
(中略)自分から軸を形成するもの(骨か?)を動かしていってはいけない。
あくまで連動。
\underline{連動するためには、自由度があまりに高すぎる状態で}は
\underline{だめで、それなりの束縛条件をつくっておく必要があるのだ。そのひとつが軸}。
(中略)
\underline{最近ホットな腹横筋腹斜筋は、軸を意識したときにちょっと使われる感覚があっ}
\underline{た}。
\textbf{[2016年9月17日、からだメタ認知記述から抜粋]}
\end{mynote}
\caption{「軸」の概念が腑に落ちた、私の記述}
\label{fig:20160917}
\end{figure}

\begin{figure}[h]
\centering
\includegraphics[width=10cm]{./images/boltrun.pdf}
\caption{ウサイン・ボルトの走りに私が付す解釈}
\label{fig:boltrun}
\end{figure}

100mと200m走の世界記録保持者であるウサイン・ボルトの走り\footnote{https://youtube/89J4pgVVsQcより引用。ボルトが現世界記録9.58を記録した世界陸上2009年ベルリン大会時の走りである}には、
その性質が顕著である(\autoref{fig:boltrun})。
白丸部を四肢の連結部(程よい拘束条件)として機能させた上で、四肢を脱力させると、四肢が自動的に連動し、うねうねした走り
\footnote{ボルトが「脊柱側弯症(脊柱が身体の左右方向に湾曲している)」という持病を抱えることも、この走りには関係していると思われる。
}
になるのだと、私は解釈している。

本節で登場した私の問題意識は、以下である。
\begin{itemize}
\item 問題意識5:体幹まわりの筋肉が、「全身連動を生むための拘束条件」として機能すれば軸が形成される
\end{itemize}

\section{怪我の原因を「歩き」に見出す}
\label{sec:kegagenin}
こうした気付き(\ref{sec:denpa}〜\ref{sec:jikunoimi}節)の連鎖にあった
% 修士1年シーズン(2016年4月〜10月下旬)は、
2016年シーズン(2016年4月〜10月下旬)は、
十種競技全体、そして構成する多くの種目で自己ベストを更新する飛躍のシーズンとなった。
しかし、実は、学部4年時に受傷した右膝の怪我(膝蓋靭帯炎)が完治せず、ごまかしながら競技に出続けたシーズンでもあった。

シーズンを終え、きたる冬季鍛練期に備え、怪我の根本的な治療を決意した。
丹念にリハビリする中で、怪我を引きずり続けた根本原因をようやく痛感した。
それは、腰を落としながら、ゆっくりと前に大きく1歩踏み出すリハビリをした時のことである。
怪我の右足を踏み出すときに、痛みを恐れ不自然な動きになったのだ。その瞬間踏み出す時に「下腿(ひざ下)が振り出される」癖を感知した。
この癖は左右ともにあった(痛みは右だけだが)。

すぐに、走りでもその癖が悪さをしていると気づいた。
\autoref{fig:kiryu}に、その分析的な解釈を示す。

\begin{figure}[h]
\centering
\includegraphics[width=10cm]{./images/kiryu.pdf}
\caption{膝蓋靱帯を痛める私の走り(上段)。桐生選手の走り(下段と比較する形で示す。}
\label{fig:kiryu}
\end{figure}

「腕で掘ろう」(問題意識3)と、右腕を肩甲骨(体軸に近い部位)から大きく動かそうとするあまり、右下腿が大きく振り出され(局面1)、身体重心(白丸)より過度な前方接地となってしまうのだ(局面2)。
「真下接地」が理想なのに比べ、前方接地はブレーキをかけることになる。
更に、膝に、曲がる向きに力が加わり(「a」の矢印)、膝蓋靱帯に激しくダメージを与える。
そしてその直後(局面3)、接地脚の膝関節が屈曲し(「潰れた接地」)、接地脚である右脚側の膝関節を伸展させるように離地(局面4)しながら左脚のスイングが行われる。
つまり、非効率的かつ、膝蓋靱帯にダメージを与えるフォームだったから怪我をしたのだということに気づいた(問題意識6:怪我の原因は「前方接地」にある)。

下段の桐生選手(100m走で日本歴代3位\footnote{
本物語と同時期の2017年9月9日、桐生選手は日本人史上初となる9秒台、9.98秒(+1.8)を記録し、日本陸上界の新しい扉をこじあけた。
福井県でおこなわれた全日本インカレの決勝レースでのことである。
\autoref{fig:kiryu}の動画のURLは以下である。 https://youtube/Sg1fyPh294w
})の走りと比べれば、違いは明らかである。
桐生選手も下腿が大きく振り出される特徴を有しているが(局面1)、接地は重心真下で(局面2:スイングする脚の膝が、私と比べてすでに接地脚(右脚)の膝よりも前方に出ていることに着目)、右膝が潰れることなくスムーズに前に進めている(局面3、4)。
畳んだ左脚の素早いスイング(局面2、3)がこの接地に関係しているのだろう(なお、スイング脚に着目したのは本物語執筆時である)。
私のこの動き(膝屈伸)は、皮肉なことに、力みすら生んでいた。

本節で登場した私の問題意識は、以下である。
\begin{itemize}
\item 問題意識6:怪我の原因は「前方接地」にある
\end{itemize}

\section{「立つ・歩く」を見つめ直す}
\label{sec:tatuaruku}
この経験(リハビリ動作で前方接地という欠点を自覚)を通して、ある仮説が私に芽生えた。
前方接地の悪癖は、日常生活における何気ない「歩き方」に根付いているのではないか?
だからこそ走りにも表出してしまうのではないか?という仮説である。

そうであるならば、歩きを改造せねばなるまい
(問題意識7:走りのフォームの悪い動きは、歩きのフォームに原因がある)。
私は、より良い歩きを探究せんと模索を始め、数多くの新しい変数を見出し始めた。
それが如実に表れた記述を以下に掲載する(\autoref{fig:20161125})。
この記述には、数多くの変数への言及がみられるが、記述中最後の変数「インパクトの瞬間に一番ヒットして、そのあとは『軽く引っ掻く?』ような感じでスカッと」以外は、どれもほぼ同様のことを表す体感や解釈である。
色々な角度からサーチライトを当てるようにして、アクセントの微妙な差異を確かめているのだ。

\begin{figure}[h]
\centering
\begin{mynote}
歩きが練習なのだと胸を張って言える。
靴によって感覚が全然異なるが。「スイっとグイッと接地中最後の最
後まで力が加わっている」必要がある。
それが「膝抜き」であり「ハムウォーク」であり、効率よく力を加
えている歩き方なのである。
そして、別名「体重移動だけで歩く」という動きなのである。
新たなチェックポイントとして、離地した直後に、足裏が後ろから見えてはいけない。
なるべく「足裏を見せずに歩く」のだ。中学生への指導として「足の裏を見せないように歩くんだよ」というもの、それの意味も今は「ナンバ歩き」、この歩き方の結果としての話だという理解になっている。
「接地中の地面への力の加わり方は、あまりアクセントが無く一定な感じ」。
普通の今までのダメな歩き方だと、「インパクトの瞬間に一番ヒットして、そのあとは『軽く引っ掻く?』ような感じでスカッと」というような、力のかかり方の時系列変化なのだ。

\textbf{[2016年11月25日、からだメタ認知記述から抜粋]}
\end{mynote}
\caption{「歩き方」に多くの変数を見出している、私の記述}
\label{fig:20161125}
\end{figure}

私はさらに、日常生活において「立つ」ことを、走りを改善するための基礎スキルとして位置付けた。
この思想は、原初的な身体のあり様を説く野口\cite{noguchi:2003}に多大な影響を受けている。
野口は卵が立つ様子から、筋力に頼らずに骨だけで立つことこそ良い立ち方であると洞察した。
卵はほんの一点だけで支えられ、その状態から少しでも外れると、すぐに倒れる。
しかし、すぐ倒れるということはわずかの力で動き出せることでもある、という逆説めいた説を、野口は以下のように説明する。
\begin{quote}
運動能力が高いということは、その動きに必要
な状態の差異を、自分のからだの中に、自由に
創り出すことができることである。
(\cite{noguchi:2003}、p.22)

\end{quote}

私は、以前から野口の思想を知りながらも、実は軽視してきた。
怪我と向き合って初めて、野口の弁が身に染みるように理解できるようになった。
私はこう悟った。
自分は、本質的には日常生活で「立つ」こと(自然に立つ)すらできていないのだ、と(問題意識8:自分は「立つ(骨で立つ)」ことすらできていない)。
それまでの私は(身体障害や身体機能不全を抱えていないということもあり)、「立つ」という運動に「立つことの達成具合」といった見方をしたことなど一度もなかったし、
「日本語を喋れる」とか「自転車に乗れる」とかと同様に、「立つ」ことを、当たり前にできていることとして不問に付していたが、
その「立てている」とはしょせん、「日常生活に不自由なく立つということができている」という意味にすぎなかったわけである。
私にとって「立つ」という運動の意味が大きく変革された。
このように「立つ」を「自然に立つ・骨で立つ」という意味で捉え直したとき、力んで走ってしまっている自分は、まともに「立つ」ことすらできていないこと思い知ったのである。
走っている時に「真下接地」を達成できていないのが何よりの証拠であろう。
よく考えてみると、真下接地は、「自然に立つ」ことの必要条件そのものである。

それからというもの、私は、「無駄な筋力を使わず、重力を最大限利用した高効率な動き」を生む身体(問題意識9)のあり方を目指すようになる。
「立つ」ことはその基礎である。
これに則した形で、Tが重視してきた体幹トレーニング(\ref{sec:denpa}節)の意味も更新される。
以下記述を掲載しよう(\autoref{fig:20161203})。

\begin{figure}[h]
\centering
\begin{mynote}
体幹にいい具合に力とか刺激が入っている状態というのは、何も力を入れなくても、「骨で立つ」状態が自然に維持できるような状態である。
そうなって初めて「椎骨(※連結して脊椎をなす骨)を積み上げる」感覚が芽生える。
(中略)つまり、「バランスの良い体幹トレーニングは、骨で立つため」に行っているのである。
力をいれずとも正しい姿勢になるようなトレーニングなのである。

\textbf{[2016年12月3日、からだメタ認知記述から抜粋]}
\end{mynote}
\caption{「骨で立つ」ことから、体幹トレーニングの意味を見直した、私の記述}
\label{fig:20161203}
\end{figure}

こうして私は、立つことを基礎として、変数を続々と開拓しながら歩きを改造し、それらを走りや他の動きと関連付けることを続けた。
代表的な変数を以下に挙げよう。

\begin{itemize}
\item「上から吊られている」(\autoref{fig:walk}の全局面)
\item「接地直後に一瞬『ふっ』と膝を抜く、崩す」(\autoref{fig:walk}の局面1)
\item「接地位置中心に転がっていく(全身が)」(\autoref{fig:walk}の局面2)
\item「足裏が地面から『剥がれる』」(\autoref{fig:walk}の局面3)
\end{itemize}

\autoref{fig:walk}は、私がこれらの体感と向き合いながら「歩き」を試行錯誤するようすである。
\begin{figure}[h]
\centering
\includegraphics[width=10cm]{./images/walk.pdf}
\caption{私の歩きに生じる体感}
\label{fig:walk}
\end{figure}

「上から吊られている」ような重心の高い姿勢によって、接地が潰れるのを防ぐ。
その上で、「接地直後に『ふっ』と膝を抜き・崩す」ことにより、膝関節屈伸による「跳ねる動作」が抑えられ、自然な倒れこみとして次の1歩が繰り出される。
その結果、離地では「足裏が剥がれる」感覚が生じるのだ。これらのことの総体として、全身に「転がる」体感が生じる。
「転がる」ことは、走りでは素早い股関節回転となる。
なお、図内には体感の効果線や注釈を描きくわえているが、この図の歩きにおいて、確実にこれらの体感が生じているとは言い難い。
まだこの時点では、「歩き」を「できている」とは言えない状況であり、あくまで試行錯誤のさなかにあり、上記の体感が生じたり消えたりしているところである。
日常動作を、根本からとらえなおして、創りなおしてゆくのは難しいものである。
上述したように、なんら疑いなく当たり前に「できている」と思って年月を生きてきてしまっていたため、慣れてしまっており、
歩きを試行錯誤的に試すなかで生じる体感は非常に繊細で微妙なもので、ある。
このように、歩きかたを無駄な筋力をつかわないしかたに創りなおすのはとても微妙な現象であり、

残念ながら、
% 修士2年シーズンの
2017年シーズンの
幕開け直前(2017年3月)に、私は大怪我を負ってしまう(右足舟状骨の疲労骨折)。
コンディションに気遣っていたつもりでも、蓄積した疲労に十分にケアしきれなかったのだろう。
スキルを学ぶこととは、こうも儘ならないのかと肩を落とした。
% 修士2年シーズンにあたる、
2017年シーズンにあたる、
2017年3月〜2017年9月(本物語終点)は、怪我からの復帰が叶わなかった。
ジョギングすら出来ず、足への衝撃がない穏やかな運動のみ許された(2017年8月に1週間だけ一時復帰するが、痛みが再発した)。
しかし、今や、立つ・歩くことも練習と化している。
したがって、学びの道は決して閉ざされていないと、私は自らをそう信じることに決めた。

本節で登場した私の問題意識を列挙すると、以下のようになる。

\begin{itemize}
\item 問題意識7:走りのフォームの悪い動きは、歩きのフォームに原因がある
\item 問題意識8:自分は「立つ(骨で立つ)」ことすらできていない
\item 問題意識9:「無駄な筋力を使わず、重力を最大限利用した高効率な動き」を生む身体
\end{itemize}

\section{日常生活にあるモノをツールに転じて、身体を問う}
\label{sec:monowotool}
怪我によって、運動が厳しく制限されると、競技場での練習時間が短くなり、生活の時間に新たな余白が生まれた。
% こうした修士2年開始のタイミング(2017年4月)で、X(\ref{sec:led}節に登場)が私の修士研究副査となった。

2017年シーズンでも、X(\ref{sec:led}節に登場)から、私の取り組みに対し定期的にアドバイスをもらう機会をもらえるようになった。
これらの事態が重なったことは、さらなる学びの開拓を目論む私にとって、「モノを通して自分の身体を手触るように問う(\ref{sec:led}節)」態度を加速させた。

まず、私は、「コンピュテーショナルツールの試作」に取り組むようになる。
自らの身体に新しい気づきを得るためのツールである。
より本研究に即したかたちで言い換えれば、自らの身体とことばの関係性を進化させるためのツールである。
電子工作やプログラミングのスキルが皆無に等しかった私にとって、制作はかなり大変だった。
だが、その大変さはむしろ、ふだんの練習や百均トラッキング(\ref{sec:led}節)のような手軽い取り組みとは異なった新鮮なしかたで自らの身体のありかたに迫ろうとするゆえでもあり、私は楽しく取り組むことができた。

各ツールからアスリートとして価値高い問題意識を得られたわけではないのだが、
アスリートとしてツールを作ってみる経験そのものが、少なくとも、私のアスリートとして「モノ」とじっくり向き合う鍛錬になったろうと私は考える。
ツール試作の取り組みについて以下に簡単に述べておく。
2017年4月〜5月、はじめに、上下の加速度を音の高さにリアルタイムに変換する「加速度可聴化ベルト」を制作した。
2017年7〜8月(\ref{sec:jikunoimi}節と\ref{sec:tatuaruku}節の間の時期)には、
PCのwebカメラからの映像を入力とし、常に各ピクセルにおけるフレーム差分を計算して、それをもとにして効果を付与したドット映像を出力し続けるインタラクティブな鏡型の「FusionMirror」を試作した(\autoref{fig:fusionmirror})。
\begin{figure}[h]
\centering
\includegraphics[width=10cm]{./images/fusionmirror.pdf}
\caption{FusionMirrorの画面キャプチャ(私が片脚で立ってバランスをとるようす)}
\label{fig:fusionmirror}
\end{figure}
%\footenote{
%制作にはProcessing言語をもちいた。ProcessingはJavaベースのプログラミング言語であり、画面上に動的なスケッチを生成するのに優れている。
%}。
FusionMirrorの制作にはProcessing言語をもちいている。

Processing言語をもちいたスケッチ生成や画像処理の可能性をみた私は、簡易的な2次元モーションキャプチャリングシステム「DIYモーションキャプチャキット(以下DIYモーキャプキット)」にも取り組んだ(\autoref{fig:diymocapkit}〜\autoref{fig:diymocapgraph2})。
DIYモーキャプキットは、映像をProcessingで画像処理して動きを可視化するソフト面だけでなく、発泡スチロール球に色を塗った手作りマーカや撮影背景用の黒布も自作している(\autoref{fig:diymocapkit})。
なるべくささっと手軽に撮影環境を準備できるようなキットを作ろうと工夫した(\autoref{fig:diymocapenv})。
実際にDIYモーキャプキットをもちいて、私と陸上未経験者とを比較するかたちで「歩き」の撮影や独自の可視化にも試みた(\autoref{fig:diymocapgraph1}、\autoref{fig:diymocapgraph2})。
歩きという運動を扱っているのはもちろん当時の私が問題意識7・8・9(\ref{sec:tatuaruku}節)を抱いていたからである。
例えば\autoref{fig:diymocapgraph1}は、左肩・左肘・左膝・左足首の4点にマーカを装着して撮影したものを、
その4点を上から順に線で結び、その軌跡を映像に重ね描きするようにしたものである。
当時の私にとって、黄色プロット(膝)の軌跡が密なエリアは「接地中」なのだが、著者(右側)と陸上未経験者(左側)では、
その疎密変化の具合が異なり、著者のほうが疎密変化が緩慢であることがみてとれる。
私にとってこのことは、\ref{sec:tatuaruku}節で登場した「接地直後に『ふっ』と膝を抜く、崩す」という変数とむすびつくものであった。

\begin{figure}[h]
\centering
\begin{minipage}[b]{0.45\linewidth}
\centering
\includegraphics[width=\linewidth]{./images/diymocapkit.pdf}
\caption{DIYモーキャプキットの手作りマーカと黒布}
\label{fig:diymocapkit}
\end{minipage}
\hspace{0.04\linewidth}%画像間の余白
\begin{minipage}[b]{0.45\linewidth}
\centering
\includegraphics[width=\linewidth]{./images/diymocapenv.pdf}
\caption{DIYモーキャプキットの撮影環境}
\label{fig:diymocapenv}
\end{minipage}
\end{figure}

\begin{figure}[h]
\centering
\includegraphics[width=\textwidth]{./images/diymocapwalk.pdf}
\caption{DIYモーキャプで撮影した歩き(上段:私、下段:陸上未経験者、右肩・右肘・右膝・右外踝の4点に自作マーカを取り付けている。}
\label{fig:diymocapwalk}
\end{figure}

\begin{figure}[H]
\centering
\begin{minipage}[b]{0.45\linewidth}
\centering
\includegraphics[width=\linewidth]{./images/mocapwalkplot.pdf}
\caption{DIYモーキャプキットで試した可視化1}
\label{fig:diymocapgraph1}
\end{minipage}
\hspace{0.04\linewidth}%画像間の余白
\begin{minipage}[b]{0.45\linewidth}
\centering
\includegraphics[width=\linewidth]{./images/mocapwalkphase.pdf}
\caption{DIYモーキャプキットで試した可視化2(相空間のグラフを映像に重ねる)}
\label{fig:diymocapgraph2}
\end{minipage}
\end{figure}


このように、ツールの制作には、
% 修士2年シーズン
2017年シーズン
を通して断続的に取り組んだ。
上述のように、各ツールからアスリートとして価値高い問題意識を得られた訳ではない。
しかし、アスリートとしてツールを作ってみる経験そのものが、私の「モノへのまなざし」を鋭敏にした可能性がある。

そして、興味深いことに、この態度が日常生活にも表れ出たのである。
私は、日常生活で、一見競技に関係ないモノと自身の身体の関係を積極的に取り結び、ツールとして転用することによって身体を問い、スキルを学ぼうとした(問題意識10:日常生活で身を取り巻くモノは、身体を問うツールになる)。
生活における些細な気づきの瞬間を逃さずに、アスリートとして生きようとしたのである。
その片鱗を以下に物語ろう。
\subsection{「石花」にみる、「骨で立つ身体」}

2017年5月、とあるコワーキングスペースに、友人と赴く機会があった。
イベント等に積極的な場所だからか、様々な展示や広告が陳列されている。
その場で、「石花(石を絶妙な形に積み上げる遊び)」の活動団体ポスターに目線が吸い込まれた。
積み上がった石に「骨で立つ身体(問題意識8)」を見出してしまったのだ。
私は、怪我で走れない状態だが、スキルの学びを止める気は一切なかった。
その押し込められた「学び欲」が、漏れ出た瞬間であったかもしれない。

早速後日、自宅から程近い相模川へ赴き、石花を実践してみた。
石の選定から始める必要がある。
選定段階からすでに、朧げながら、自分の身体(骨)と石の関係を探っていたのかもしれない。
積み上げるのにも苦戦しながら、ようやく作品が1つできた(\autoref{fig:ishihana})。
この作品は、安定したDの上に、下から順にC、B、Aと積み上げるのでは成立しない。
BとCを手で支持しながら、同時にAを「真ん中を貫く」ように置くことで初めて、安定が生まれる。
私は作りながらそれを実感した。

\begin{figure}[h]
\centering
\includegraphics[width=10cm]{./images/ishihana.pdf}
\caption{私の石花作品}
\label{fig:ishihana}
\end{figure}

石同士を結ぶ「筋肉」は一切ない。
4つの石の間に素朴に成立する絶妙なバランスを見て、筋の緊張能力に頼らず「骨だけで立つ」ような自らの体感を体内に湧き起こらせたのだ。
石花作品からは、「軸(問題意識5)の姿」すら見えてくる。
軸の姿とは、「そのモノの見た目が棒状」であることよりも、「どこから眺めても『スッ』と1本通っている感じ」なのだと、私は見抜いたのである(\autoref{fig:ishihana}の右側写真)。

\subsection{リュックを「腹負う」}

バランスの意識は、重心の意識と密に関係するものである。
春雨の降る新宿で、私は建物を出た。
傘をささねばならず、それを開いて歩き出した時、愛用のリュックが濡れるのを回避するために、とっさに前に抱えてみた。
この瞬間、新鮮な体感が舞い降りたのである。
その時の私の意識が、\autoref{fig:20170513}のメタ認知記述に表れている。

\begin{figure}[h]
\centering
\begin{mynote}
雨の新宿を歩いていて、リュックを前に背負う。
いや、「『腹』負う」。
ここで気づく。
リュックを前に下げると、「正しい歩き」の感覚がおりてきやすくなる。
\underline{リュック}-
\underline{身体系という1つのモノの重心は、体ひとつよりも全然前に移動するからだろう}。
\underline{ふつうに接地したときの感覚が、リュック-身体系の真ん中を貫いてくれる}感覚がある。
末端だけで歩くような感覚にはなかなかならない。

\textbf{[2017年5月13日、からだメタ認知記述から抜粋]}
\end{mynote}
\caption{リュックを「『腹』負う」体感について解釈する、私の記述}
\label{fig:20170513}
\end{figure}

身体重心よりも前方に位置する「リュックと身体を合わせた重心」を、あたかも身体重心のように感知することで、接地位置に乗り込むという、良い歩きの感覚(問題意識9)に迫ることができる。
これを「リュックを『腹』負う」と命名した。
動作だけでなく、日常生活のモノと動作をセットで括った、
この命名行為は価値高い。
このエピソードは、「生きること自体を、スキルの学びの前面に出す態度」の顕れであると私は解釈している。

\subsection{洗濯機を一人で運ぶ、体幹トレーニング術}

夏休みのある日のこと、私は兄から引っ越しの手伝いを頼まれた。
大きく重く持ち手もない洗濯機を、玄関から所定の位置へと運び入れる必要があった。
その経路(廊下)は狭く、複数人で運搬するのは至難の業である。

このように「問題」が発生した時こそ、一般に、人は自身の置かれた状況に思索を巡らすものである。
私は次のようなことを考えた。
「一人で運ぶしかない。
運び得るのはこの場には自分しかいない。
洗濯機には持ち手もない。
鍛えてきた体幹の筋群(\ref{sec:denpa}節)をうまく使って抱えれば、良質なトレーニングになるし、結果的に一人で運び切れるのではないか?」。
そう決断し、それを成し遂げた。
この試行にはいかなる問いが生じていたのだろうか(\autoref{fig:20170825})。

\begin{figure}[h]
\centering
\begin{mynote}
R(※兄)には運べないのになぜ自分に運べるのか。
これは間違いなく、体幹部の使い方であろう。
もちろん少なくとも体幹の筋肉が発達していない限りは、「体幹で支える」ことはできない。
腕などではない。
「体幹で持っている」のだ。「ものの重心」と、「自分の重心」と、「そのあわせた重心」と。
これらをすべて身体で感じて、体幹を用いてうまく操作することが肝心なのだ。

\textbf{[2017年8月25日、からだメタ認知記述から抜粋]}
\end{mynote}
\caption{洗濯機を体幹で運ぶことを解釈する、私の記述}
\label{fig:20170825}
\end{figure}

身体とモノが一体としたときの重心を見極めんとする態度(下線部)は、前節で語ったリュックの事例と相通じており、興味深い。
私にとって、これらの変数関係は「引き出せる術」になっていると考えられる。
「ウエイトトレーニング(高重量ウエイトを挙上する筋トレ)」では、ややもすると、その見た目の雄々しさや豪快さに酔いしれたり、挙上できる最大重量値に囚われる。
私も過去に、そのマインドに嵌まり、自身の「力み癖」を助長し、身体の連動性を減じてしまったことがあった。
そんな私は洗濯機を、「連動性を生む体幹(問題意識5)」を注力的にトレーニングするための、いわば「数値なきウエイト」と見立てたのだ。

本節で登場した問題意識は以下である。
\begin{itemize}
\item 問題意識10:日常生活で身を取り巻くモノは、身体を問うツールになる
\end{itemize}

\section{醸成された問題意識群と、それらの相互関係性}
以上で物語を終える。
物語のまとめとして、私の物語で登場した全10個の問題意識の関係を、\autoref{fig:zentaizou}に示す。
私が自身の走りのスキルを学んだ「howのプロセス」は、こうした問題意識(とその関係性)に支えられているのだと私は解釈している。

\begin{figure}[h]
\centering
\includegraphics[angle=90, height=\textheight]{./images/mondaiishikizentaizou.pdf}
\caption{私の物語に登場した問題意識の全体像}
\label{fig:zentaizou}
\end{figure}

問題意識1〜6において、走りにおける身体を自分なりに分節・焦点化し、それらや地面の関係を考えた。
1〜6は、立つこと歩くことと分離しているわけではない。
例えば6は、走る以前の、「歩くリハビリ動作」に発見した問題意識であった(\ref{sec:tatuaruku}節)。

「粘性」や「拘束(軸)」といった仮想的・比喩的な体感(4や5)さえも問題意識として浮上した。
そののちに、「歩く」こと(7)、「立つ」こと(8、9)や重力が走りと重要な関係を有することに考察が及び、遂に10では、「生活におけるモノ」を、「身体」
を問うためのツールとして転用した。
次第に走るスキルについての学びを私が「生活」と交えながら、拡張する様子が見てとれる。


