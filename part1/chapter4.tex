\chapter{物語の考察:学びの野生化}
\label{chapter:yaseika}

私が、生活と競技を切り離さず、生活の中にも様々なヒントを見出しながら、スキルを学ぶ態度は、野生の思考\cite{levi-strauss:1962}と解釈できる現象なのではなかろうか。
レヴィ=ストロースら人類学者は、世界各地の未開文化をフィールドワークによって詳らかにしてきた。
未開民族は、彼らの生きる環境をなす自然種を精緻な関心によって弁別し、それらの関係を隠喩的な思考媒体として活用することで社会文化を作り上げていた。
例えば、南ボルネオのイバン族は、各種の鳥の性質の差異を解釈する。
アジアキヌバネドリ属の一種の警戒の鳴き声は、喉を切られた動物の喘ぎに似ていることから豊猟の前兆であり、同属の別種は、その「笑声」で交易旅行がうまくゆく前兆とされたり、その派手な赤い頸毛が戦勝や遠征に伴う威光の印だという。

このように、身の回りにある事物を組み合わせて、そこに生じる関係から象徴的に秩序や法則をつくり出す態度を、レヴィ=ストロースは「野生の思考」と呼ぶ。
いっぽう、法則や秩序から事物をトップダウンに(一意的に)位置付ける、近代科学的態度を「栽培的思考\cite{levi-strauss:1962}」と呼ぶ。
レヴィ=ストロースは、野生の思考という基礎の上に栽培的思考があるべきだと説く。
また、ありあわせの資材の断片から、なんとかやりくりする器用仕事を「ブリコラージュ」と呼び、現代の我々に表れる野生の思考だとレヴィ=ストロースは言う。

私の一連の学びを俯瞰すると、まさに、レヴィ=ストロースが言うところの「野生」に符合する。
当初の私は、競技場やトレーニングルームだけで学んでいた。
特に陸上競技場は、競技用に洗練されたタータン(合成ゴムの地面)に、均等かつ精密に引かれたライン等によって、秩序づけられた空間である。
その空間に則する形で距離やタイムを計ったり陸上競技用の道具を用いたりしながらスキルを学ぶのは、レヴィ=ストロースのいう「栽培的」な学びである。

私はそこから脱皮し、自らの学びの場を日常生活へと開放したのだと解釈できる。
例えば、立つ・歩くことをスキルとして問うたり(\ref{sec:tatuaruku}節)、競技に一見関係ない石ころや洗濯機やリュックを、私自身の身体を問うツールへと再解釈した(\ref{sec:monowotool}節)のである。
\ref{sec:monowotool}節の各試行は、ブリコラージュだと捉えることができる。

このように、私は自らの「生」において、持ち合わせの日常生活動作や、身を取り巻くモノと、私自身の身体との関係を取り結び、その関係を通して身体を問い、競技スキルを形作らんとしてきた。
これを、スキルの学びに「野生」の思考が浮上してきた形として、本稿では\textbf{野生化}と命名する。
野生化と命名する積極的な意義は、その学び方を、奇抜というより、むしろ、狭い場からより広い場への拡張したのだと理解するためである。
野生/栽培のアナロジーを強調して表現するならば、守られた環境で栄養をもらう(栽培的な学び)だけではなく、主体的に野に身を晒しながら栄養を獲得する(野生的)ようになったということである。

本稿では、あらゆるアスリートが自らの学びを野生化するべきであると主張するつもりはない。
しかし、私の過去数年間においては、走りが苦手であったが故に、そのスキルを根本から見直そうと、体感の微妙な差異を感じ取り、そこから様々な問題意識を立て、日常生活にもヒントを見出そうとしてきた軌跡が、次第に私の学びを「野生化」に導くことになった。
少なくとも私にとっては、「野生化」は、走りのスキルを磨く上で大きな学びをもたらしてきたと断言できる。

学びの野生化という、学びの実践が拡張していったことこそ、認知が構成的な姿そのものと言える。
そして、狭義の一人称研究で構成的手法をとったからこそ、野生化が観測できた(仮説を立てて検証する、という1サイクルを研究とみなすスタイルでは、こうした野生化の現象は観測できなかったであろう)

