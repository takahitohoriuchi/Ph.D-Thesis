\chapter{走りはどう変わったのか?連続写真からの考察}
\label{chapter:hashirinohenka}

前章で物語った実践とそこで得た問題意識群は、私のパフォーマンスに変化を及ぼした。
本章では私の「走り」に着目し、4つの時期の連続写真(\autoref{fig:runtransformation})を用いてそれを考察する。
全て30fpsで撮影された連続写真を、右脚接地瞬間(局面3)で4つの走りを揃えて表示した。
私の身体に書き込んだ線は、「上体の傾き」「右大腿骨」「右下腿」の代表線である。
カメラは固定カメラではなく、カメラの場所や角度は4つで統一していない。
その上でなお読み取り得る変化を、本章では考察する。


\begin{figure}[h]
  \centering
  \includegraphics[width=10cm]{./images/runtransformation.pdf}
  \caption{走りに表れた変化}
  \label{fig:runtransformation}
\end{figure}

\begin{description}
\item[A:2016年7月24日(\ref{sec:led}節に相当)]
\end{description}
「腕で掘る」意識がまだ濃かった時期である。局面3で、真下接地できていない。
\begin{description}
\item[B:2016年9月18日(\ref{sec:jikunoimi}節に相当)]
\end{description}
「軸」の意味を、再解釈した時期である。
Aに比べて、局面1〜2で背筋が伸びているのに加え、局面3で、より接地位置真上に近い位置に腰があり、軸の意識が表れているように見える。
いっぽうで、右接地位置を手前にしようと、右腰がより引けてしまって左側の骨盤が前傾してしまい、左脚の「膝のたたみ」が甘くなり、左脚スイングが右膝関節の屈曲を助長してしまっている(局面4)。
Bの局面4は、接地脚膝関節の大きな屈曲に伴って、踵も浮いており、Aの局面4は踵が付いている。
局面5でAよりも深く右膝が潰れ、局面6でAよりも上に伸び上がってしまっている。

\begin{description}
\item[C:2017年1月17日(\ref{sec:kegagenin}節・\ref{sec:tatuaruku}節に相当)]
\end{description}
立つ・歩くを作り直していた冬季である。全局面を通して、AやBよりも前傾姿勢気味である。局面3〜5で、AとBは右腰が引けてしまっている(真上から見て骨盤が右回転)のに対し、Cは骨盤が進行方向に正対している。
すなわち、接地において、AとBよりも重心が乗り込めている。
局面5では、右膝は屈曲しているが「潰れて」いるのではない。
乗り込めており、意識的に「膝を抜いて」(\ref{sec:tatuaruku}節)いるのだ。
局面6では、上に伸び上がらず、前に進めている。
また、全局面にかけて、AとBに比べて、つま先があがっており。踵から接地に入り、接地位置の上を身体が「転がる」ようにみてとれる。
\begin{description}
\item[D:2017年7月29日(\ref{sec:monowotool}節に相当)]
\end{description}
大怪我から1週間だけの一時復帰時である。
Cと比較して、全局面にかけて上体はより直立に近い。
Cよりも、右膝の屈曲がない、すなわち、局面4・6での右下腿が前傾しすぎていない。
より最後まで踵が浮かずに、地面に力を加えられている可能性があり、これは「足裏が剥がれる」(\ref{sec:tatuaruku}節)感覚と符合する。
また、Cよりも膝をたたんで左脚スイングできている。
局面5〜6でそれが顕著である。
局面4〜5でCよりも腰が高く、局面6で、上半身と右大腿のなす角の大きさ(局面6の図内に線で表した)が、Cより小さい。
すなわち、離地で右股関節が開きすぎておらず、次なる右大腿のスイングが早まる。
石花や洗濯機の実践を経て、「軸」という点で、動きがより研ぎ澄まされたかもしれない。
ただし、Dの地面のみ、「タータン(合成ゴム)」ではなく「硬い人工芝」である。
後者の方が、滑りやすくて私にとっては走りにくい。

以上のように、物語の期間において、私はさまざまに身体での問いを発し問題意識を醸成してきたのにともなうかたちで、
私の走りは変化してきたのである。

