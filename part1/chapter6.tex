\chapter{物語前史}
\label{chapter:monogatarizenshi}
私は陸上を愛する陸上選手でありながら、みずからの「走り」に大きなコンプレックスを抱え、「苦手」である走りと長らく向き合い続き、一人称研究を遂行した。
この「矛盾」が私を突き動かし、この一人称研究を創りあげてゆく原動力になったと言ってもよい。
この一人称研究するもの、「走り」の身体知をつくる物語を始めるまえに、「物語前史」として、私が矛盾に生きる一人称研究者としてやる経緯を語っておく。

\section{幼少期から小学生時代}

私は幼少時、年上の大きいお兄ちゃんたちに囲まれて
\footnote{
  当時は公務員住宅に暮らしており、近所には子どもたちがたくさんいた。
二つ年上の兄の存在も、。
}
、ドッジボール・ドロケイ・ロクムシなどなど、たくさん運動をして過ごしてきた。
運動が大好きだった。
小学校のクラスではつねに足が一番速かった。
小学校高学年に上がると、クラブ活動
\footnote{
  毎週水曜日の5・6時間目はクラブ活動の時間だった。
}がはじまり、私は「陸上クラブ」に所属していた。
すでに陸上競技と出会ってはいたのだ。

周りの友だちが続々と地域の少年野球チームにも入るようになったころ、私は、中学受験のために学習塾に通いはじめた。
私が過ごした残りの小学校生活はというもの、スポーツの習い事をする経験もないまま、机にむかってお勉強するばかりであった。
お勉強中心生活がたたって、私は徐々に太っていってしまった。

\section{中学時代:私は「足が遅い」というレッテルを貼ってしまう}
中学受験を経て、第一希望の中学ではないところに入学した\footnote{
むしろチャレンジ受験した中学に受かってしまったのだ。
当時の私はというと、チャレンジ校に受かった喜びは皆無、第一志望に落ちたことショックが実に大きかった。
第一志望校での青春を小学生なりに思い描いていたからだ。
そんな複雑ななか説得されていったのだが、という予想だにしない結果となった。
}。
志望校ではない中学でなにをやるべきなのか、やりたいことがわからない。
そしてに思春期に突入してもいる。
その中学での部活動というよりも、中二病で音楽活動が好きだった。
受験生活の名残で身体各部にまとわりついた贅肉も、コンプレックスだった。

私は惰性で陸上部へ入部した。
理由があるとすれば、部活動見学したときに、陸上部の先輩たちがなんともゆるゆると自由な雰囲気でやっていて、それが魅力的に思えてしまったからだ。
それだけである。
あまり部活が盛んな中学ではなかったし、私は部活動というものに興味が湧かなかった。
自主性を重んじる校風が関係しているのか、じっさい陸上部の活動はかなりの部分が生徒に任されていた。
中学陸上部での私はというと、後輩らと裸足で鬼ごっこをしたりして、ほぼ遊んで過ごしていた。
それでもクラスでは一番足が速かったから、それで満足してしまっていた。
陸上競技というものは好きだった\footnote{
  私が中学2年の夏に世界陸上大阪大会、3年生の夏には北京オリンピックが開催された。
  どちらもテレビで楽しく観戦したし、北京オリンピックでは、日本が男子4×100mリレーで夢の銅メダルを獲得したのをみて、
  「陸上で活躍するってこんなにかっこいいんだ」と憧れを抱いた記憶はある。
なおこの日本の銅メダルはのちに銀メダルに繰り上がった。
}が、練習のしかたもよくわからなかった。
それに、思春期真っ只中の学生が、みずから主体的に「自身の身体の動き」と向き合うには、あまりに条件がそろっていなかったと今の私は思う。

そんななか劇的な体験をする。
中学一年生の春、初めて「大会」に出場した。
100mに出場した。
会場は東京都江東区の「夢の島競技場」である。
家から会場までは、それまでの人生でもっとも長い電車ひとり旅であった。
途中迷子になりかけながらも、なんとか会場にたどり着いたのだった。

レースは学年別に分かれており、私が走ったのは男子中学一年生の部である。
自分の組が走る番を待っているなか、周りの選手たちのみるからに速そうな風貌に、衝撃を受けた。
彼らの多くは、学校名を冠したユニフォーム(袖なしランニングシャツに短いランニングパンツ)を着こなしていた。
ユニフォームから覗くのは、こんがりと焼けた肌、立体的に浮き出た筋肉と血管、各部位に施されたテーピングや磁気テープ。
ペチペチと自分の筋肉を叩いたりストレッチをしたりして、彼らはレースのそのときにむけてなにやら準備している。
どれもが私にはないものだった。
私はというと、肌は白いし、足首は浮腫んでいて、学校指定の体操着を「着させられ」ている。
スパイクだけは「いっちょまえ」\footnote{
  ミズノ社の真っ赤なクロノインクスである。
}
というちぐはぐさも、なんとも不格好だった。
まわりの選手たちと自分とのあまりに違いに、レース前から私は気圧されていた
\footnote{
あの感じは、当時の私が経験していた中学受験の思い出にも重なるところがあったのではないかと思われる。
それは、受験会場にて、「N」の1文字がドンと刻まれた青いバッグを背負った集団に出くわしたときの感じである。
}。
まるで「夢の島」という遠いどこかの別世界に迷い込んだかのような錯覚に陥った。

レース結果は、同組内で5着くらいだった(タイムは13.
74秒)。
まわりとのタイム差は歴然である。
同じ一年生で12秒台が続出しているのだ。
風貌が物語っていたとおり、彼らは自分より圧倒的に速かったのだ。
私は悟ってしまった。
「俺は一般人世界で、彼らは陸上人世界、まったく別の世界に生きているんだな。
」
といったふうにである。
小学生時代にスポーツの習い事をしなかったし、中学時代もコーチに習っているわけでもない私は、そんなふうに線引きをしてしまった。
さらには、走るというのは、俗に言う「運動神経の良さ」だ、といった間違った考え方も、まだ子どもだった自分には根強く残っていたように思う。
つまり、自分は「足が遅い」\footnote{
  という慣用表現があるが、それも、走るということを、所与の才能と捉えるのと関係していそうだと著者は思う。
}
のだから、きっといくら頑張ったところで、超えられぬラインがある、と。
がっくり落ち込むとか、逆転に燃えるとかを超えて、すんなりと私は受け容れてしまっていた。

その後、中学2年、3年と、陸上部に在籍しながらも、大した練習はせず、それでも成長期の身体の発達もあって、100mのタイムは伸びていったのだった。
だが、こうしたコンプレックスや線引きが払拭されることはなかった。

\section{高校時代:投てき選手としての成長、陸上競技そのものへのめり込む}

高校に入学しても、陸上部に入部した。
なぜか?
短距離については諦めてしまっていても、「投てき」という新しい可能性を見出したからである。
投てきへ着目するきっかけには、2つ年上の兄の存在があった。
兄は円盤投とやり投で活躍していたのだが、
当時は私のほうが兄よりも腕相撲が強かった。
当時の私からすれば、次のように短絡的に考えていたのだ。
投てきも腕相撲も「腕力」がものをいうのであり、兄よりも腕相撲が強いならば、投てき種目でも兄以上に活躍できるはずだ、と
\footnote{むろん、実際には投てき種目と腕相撲はまったくの別物だし、投てきは腕力だけでやるものでもない。
}。
私は円盤投をメイン種目に、砲丸投をサブ種目に選んだ。
円盤投は、直径約20cm・1.
75kgの円盤を投げる距離を競う投てき種目であり、直径2.
5mの「サークル」のなかから1回転半の「ターン」で円盤に勢いをつけて前方へ放り投げる(\autoref{fig:discus})。

私はすっかり円盤投の技術磨きにハマってしまった。
顧問の先生は、投てきの指導者ではなかった(自身が元々陸上中・長距離を専門としており)が、私が熱心に取り組んでいるのを汲み取り、さまざまな「武者修行」するきっかけをさまざまに与えてくれた
\footnote{他校の練習に参加したり、合宿に参加したり、である。
}
。

\begin{figure}[h]
  \centering
  \includegraphics[width=\textwidth]{./images/discus.pdf}
  \caption{著者の高校時代の円盤投のパフォーマンス}
  \label{fig:discus}
\end{figure}

武者修行をとおして私は、「自分で考える」ということの重要性をまざまざと実感した。
あるコーチAは円盤投の「ファーストターン」と呼ばれる局面(\autoref{fig:discus}の局面2-3である)について、「重心を外側に大きくまわれ」とアドバイスするかもしれない。
別のコーチBは同局面に対して「左膝を深く曲げろ」とアドバイスするかもしれない。
これらのアドバイスを鵜呑みにしても、私の身体ではどうもうまくいかない。
私は試行錯誤のなかで、「左足で踏む」がしっくりくることに気づいた。
上記3つはほぼ同じ全身運動状態を言い表しているものである。
だが言及部位は明らかにちがうのだ。
あるひとつの全身運動をことばで表現してみるとき、幾通りもの言い表しかたがありうる\footnote{これを諏訪\cite{suwa:2016}は「入力変数」と呼ぶ}。
当時の私の身体にとっては、「左足で踏む」という表現が「しっくり」きたのである。
ことばの表現のしかた(意識の注ぎ方)のちがいで、まるで違う感覚が起こったり、実際に生じる身体の動きも違うものになる。
技術のことばとは、そもそも、身体にそっくりそのまま「インストール」するような類のものではない。
だから、アドバイスのことばの「意味」を、自分の身体でもって咀嚼したり取捨選択しなくてはならない(これは前述したとおりである。
)。
その取り組みを私は武者修行しながらやっており、また、この取り組みこそ肝要なのだという学習観をも私は培っていった。
ひとりの指導者に縛られなかったからこそ、こういう思想を育むことができたのだと私は思う。
さらに言えば、「なぜうまくいかないのだろう?」という試行錯誤そのものが、私にとってなんとも至福の時間であった。

私は「こだわりの塊」とも言える円盤投技術を磨いてくうち、自身が「技巧派の投てき選手である」というアイデンティティすら自負するようになっていた。
私は投てき選手としては決して身体は大きくないし膂力も強くなかった。
大きな体躯で強い膂力をもつライバルたちに勝つためには、円盤投の技術を磨くことは必要不可欠であった。
ここでいう「技巧」とは、「パワー頼りではなく技巧だ」といったニュアンスである。
より一般的に言えば、陸上界あるいはスポーツ界では、「基礎体力と技術」という二元論的な構図でパフォーマンスや選手のレベルがよく語られる。
$パフォーマンス=基礎体力+技術$といった足し算的とらえかたであり、
私が自負していた「技巧派」とは、この構図のなかで「技術」側面を突出させることでパフォーマンスを高めることを指す。
なお、投てき選手として勝負するためのフィジカルな身体づくり
\footnote{
  投てき選手にとっては、ある程度「高体重」であることが望ましい。
  重いものを遠くまで投げ飛ばすには、投てき物に大きな力積を与えるために「体重」が重要なのだ。
  体重を増やすために投てき選手は、筋肉だけでなく脂肪をもたくわえる。
}
の努力をしていなかったわけではない。
じじつ私は、投てき選手として高校時代に体重は8kgほど増やした。

だが私はチームの4×100mリレーを走っていたので、体重を増やしすぎることは避けていた。
100mのタイムは、高校時代で11.
83まで伸ばすことはできた。
このタイムは投てき選手としては「走れる」ほうだが、
短距離の選手としては、まったく勝負することはできない。

高校時代をとおして私は、陸上競技の世界にどんどんのめり込んでいった。
気づけば、短距離ではないにしろ、中学時代に「あちら側」だと線引いて追いやってしまっていた「陸上人の世界」に立っていたのである。
円盤投の記録はうなぎのぼりに成長した。
記録の成長は、自分に選手として自信をもたらした。
記録の成長にともない、陸上競技への愛は深まり、円盤投の技術へのこだわりは強まってゆく。
円盤投の技術を磨かんと試行錯誤すればするほど、陸上競技への愛は深まってゆく。
うまくいかなくても、試行錯誤そのものが楽しくなる。
記録もまた伸びる。
このような相乗効果の渦のさなかに私はいたと言えるなあ。

すでに私の興味は、自分がやっている投てき種目だけにとどまらくなっていた。
それが陸上競技の世界にのめり込むということである。
陸上競技の「文化」に身を浸すということである。
短距離ふくめさまざまな種目の記録や相場についてもくわしくなっていた。
技巧派の自負が関係してか、他種目の「技術論」を語るのも好きになっていた。
いつしか私には「語っているだけでは足りない。
もう、全部の種目を、自分でやりたい。
」と思い描くようになった。
十種競技が、まさにそれなのだ。
「大学生になったら十種競技をやろう」。
私はそう決意したのであった。

\section{大学時代:十種競技に取り組み殻を破って泥臭い学びに至る}

大学に入学した私は、いまかいまかと待ち望んだ大学陸上部の門を叩く。
当時100年近い伝統誇る大学陸上部は、私のいた高校陸上部とは比べものにならないほどレベルが高い。
日本代表経験がある者も何人もいる。
そういう環境である。
そんななかで私は十種競技をはじめた。

\begin{itemize}
\item 十種には走りが重要
\item 私は投てき選手であり、「走り」は中学時代からの引きずりがある。
別に、走りの解決はしていなかったのだ。
\item 大1・2年は、技巧派のプライドが悪いほうへはたらいていたのか、「外形をなぞる」ようなやりかたになっていた。
\end{itemize}
  - とくに、まわりの中でやるのが怖かった。
\begin{itemize}
\item 大3の冬に空をやぶった。
\end{itemize}
  - プライドをぶち壊した
  - 十種競技の特性からいって、「技巧派」プライドをぶち壊すことが重要だと。
  - 

十種競技の総合得点を上げるには「走り」がまともにできることが重要である(\ref{sec:deca}節参照)。
私にとってはそれが通年の課題であった。
大学1・2年時は、ほんとうの意味では、走りの改良はできなかった。
それに、高校時代の投てき選手としての名残で、脂肪が増えていた。
脂肪の多く重い身体では、速くは走ったり跳ぶこともできない。
これまで述べてきたとおり、私は走りが苦手であった。
私は当時の私なりに、あの手この手で、走りのフォームを改善しようと取り組んでいたのだが、
じじつ、投てき選手であった高校時代の100mのベスト記録を、大1・2年時は、更新することができなかったのである。
私は「私は十種競技選手である」というアイデンティティがもてずにいたことである。
もちろん、十種競技全体の記録が低記録にとどまっていたことも大きい。
強くなるためには、十種競技選手としてワイルドにアイデンティティをもつためには、
ワイルドさは、まさしく、細かいことを気にしないというか、
「表面的になぞる」ようなやりかたになってしまっていた。
「泥臭く、練習しなければならない。
」
うだうだし続けていたのだ。
なにより辛かったのは、

慶應競走部というレベルの高い環境であったことも大事だ。
重大な問題は、「みんなの前で遅い走りをトラックでさらすの怖い」ということもある。

\section{大学学部まで}
% 一人称研究をやっていた。
% 卒業研究
実は\ref{sec:embodiedwisdomlearning}で登場した\autoref{fig:kostubankorobashi}は、大4シーズン6月のことである。