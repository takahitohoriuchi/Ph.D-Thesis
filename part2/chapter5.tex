\chapter{実践結果}
\label{chapter:jissenresult}
本アプリをもちいて、いかにして「表情」を感得し、いかなる問いを生んだのか?
次節において、対象者AとBがおこなったその意味づくりの様相の一端を、AとBが生成した表情エントリ(と補助的インタビュー)を拠りどころにしながら、
著者の視点から記述する。
前述したように、これは著者がAとBに二人称的にかかわる\cite{saeki:2017}ことによって可能になる。

以降では、対象者が作成した表情エントリは《A4》や《B13》のように《対象者+番号》の形でID表記し、《》の中身には適宜当該エントリのオノマトペを併記する。
対象者がインタビューで語った言葉の直接引用は『』でくくって表記する。

\section{対象者Aの学び}
Aは、流れてくる曲にあわせて即興的に踊るヒップホップダンサーである。
Aがめざしているのは『豊かな技の引き出しをもち、それらを様々に組み合わせて自分らしく楽しく踊れる』ようになることである。
本研究以前からAには、時おり練習しているがいまいち『しっくり』こず、まだ技の引き出しに収められていない技があった。
上半身を下半身より先行させて左右に移動する基本技「oldman」である(\autoref{fig:oldman1})。
oldmanはポップダンスの基本技であるが、『各部位を独立的に動かすアイソレーションあるいは連動や身体軸の制御』など、ヒップホップのエッセンスを多く含んでいるとAは考える。
そこで、oldmanをAの撮影対象動作(プレイ対象動作)として選定した。
% oldman1
\begin{figure}[tb]
  \begin{continuousphoto}                      
  \begin{center}
    \begin{overpic}[width=0.18\columnwidth]{./images/hjplayground/oldman1c-1_page1.pdf}
      \put(2,90){1:始動}
    \end{overpic}
    \hspace{-0.5em}
    \begin{overpic}[width=0.18\columnwidth]{./images/hjplayground/oldman1c-2_page1.pdf}
      \put(2,90){2:上体先行}
    \end{overpic}
    \hspace{-0.5em}
    \begin{overpic}[width=0.18\columnwidth]{./images/hjplayground/oldman1c-3_page1.pdf}
      \put(2,90){3:右足接地}
    \end{overpic}
    \hspace{-0.5em}
    \begin{overpic}[width=0.18\columnwidth]{./images/hjplayground/oldman1c-4_page1.pdf}
      \put(2,90){4:左足接地}
    \end{overpic}
    \hspace{-0.5em}
    \begin{overpic}[width=0.18\columnwidth]{./images/hjplayground/oldman1c-5_page1.pdf}
      \put(2,90){5:上体先行}
    \end{overpic}
  \end{center}
  \end{continuousphoto}
  \caption{Aがoldmanを踊る様子(oldman1)}            
  \label{fig:oldman1}
\end{figure}

Aは2025年3月25日〜4月19日の25日間の期間で、全19個の表情エントリ(《A1》〜《A19》)を作成した(\autoref{table:hyojoentries_a})。
実践開始前に撮影したデータを「oldman1」と記載する(\autoref{fig:oldman1}は実際のoldman1の映像である)。
\begin{longtable}{llllr}
\caption{Aの作成した表情エントリ一覧} \\
\hline
表情エントリID & 作成日付 & 運動データ & 表情オノマトペ & 文字数\\
\hline
\endfirsthead
\hline
表情エントリID & 作成日付 & 運動データ & 表情オノマトペ & 文字数\\
\hline
\endhead
《A1》 & 03/25/25 & oldman1 & フリフリ & 393\\
《A2》 & 03/25/25 & oldman1 & ぐういペ & 512\\
《A3》 & 04/01/25 & oldman1 & ブふぁああーーー & 420\\
《A4》 & 04/01/25 & oldman1 & ガチョベチョガキョ & 494\\
《A5》 & 04/01/25 & oldman1 & バラらららららら & 212\\
《A6》 & 04/02/25 & oldman1 & なぬななぬな & 348 \\
《A7》 & 04/08/25 & oldman1 & ンーコツコツンー & 228 \\
《A8》 & 04/08/25 & oldman1 & グリグリ & 184 \\
《A9》 & 04/08/25 & oldman1 & フラフラぶんフラフラ& 328 \\
《A10》 & 04/13/25 & oldman1 & フサササフサ & 273\\
《A11》 & 04/13/25 & oldman1 & ブふぁああーーーペ & 23\\
《A12》 & 04/17/25 & oldman2 & うー、、わっダラララ & 122\\
《A13》 & 04/17/25 & oldman2 & ねーねねねね... & 200\\
《A14》 & 04/18/25 & oldman3 & mm & 18\\
《A15》 & 04/18/25 & oldman3 & 擬音が思いつかない、、 & 63\\
《A16》 & 04/18/25 & oldman3 & パラララららら & 11\\
《A17》 & 04/18/25 & oldman3 & 惑星 & 9\\
《A18》 & 04/19/25 & oldman1 & バラバラ & 60\\
《A19》 & 04/19/25 & oldman3 & へふいへ & 303\\
\hline
\label{table:hyojoentries_a}
\end{longtable}


\subsection{《フリフリ(A1)》}
\label{subsec:a1}
Aのプレイのしかたには、本実践以前からもっていた「身体内の仮想的な線」を表情図形へ具現してみるというものがあった。
たとえば《フリフリ(A1)》や《ンーコツコツンー(A7)》である。
《フリフリ(A1)》は「oldman1」で、両肩を結ぶ補助線を引き、補助線スタイル「10倍延長」で延長し、それを斜め前から眺める、という図形に感得した表情である(\autoref{fig:hyojonotea1})。
当初の作図のねらいは「動作全局面をとおした両肩線の水平具合」を評価するためであった。
だが実際に鑑賞してみると、「鉛筆を優しくつまんで振った」ような表情が立ち現れた。
その表情は、動作の最後の瞬間(\autoref{fig:oldman1}局面5)をきわだて、
「両肩線が上下に微妙に揺れている」という新事実への着眼をAにもたらしたのだった。
% 《フリフリ(A1)》
\begin{figure}[H]
  \begin{hyojoentry}    
  \textbf{《フリフリ(A1)》} (oldman1)

  \vspace{0.5em} \hrule height 0.5pt \vspace{0.5em}

  \begin{center}
    \begin{overpic}[width=0.3\columnwidth]{./images/hjplayground/A1-1.pdf}
      \put(2,90){1}
    \end{overpic}
    \hspace{0.05em}
    \begin{overpic}[width=0.3\columnwidth]{./images/hjplayground/A1-2.pdf}
      \put(2,90){2}
    \end{overpic}
    \hspace{0.05em}
    \begin{overpic}[width=0.3\columnwidth]{./images/hjplayground/A1-3.pdf}
      \put(2,90){3}
    \end{overpic}
  \end{center}

  \vspace{0.5em}

  \small
  \textit{
    % 鎖骨の二点を結び、線延長することで鎖骨をぶっさすような形を作ってみた。
    (前略)
    最後の数コマの間、棒がフリフリと揺れる様子を見つけた。
    この結び方をすることで鎖骨の動きを見てみようと思って結んでみた。
    鎖骨はなるべく水平を維持できると見た目が綺麗になると思っており、この線を見ることでその平行具合を見ようとしたが、傾きよりも 最後の一瞬に存在する棒が上下に揺れる姿が発見的だった。
    左右の点が交互に上下運動していて、それによってその点を貫く棒が上下に揺れる。
    振れ幅は小さく、周波数は高い。小刻みに細い棒が揺れる姿が、鉛筆を指で持って振るやつに似ている。
    あれは結構ブンブンする感じだけど、これはもっと優しげ。フリフリといった感じ。悪く言えばブレと言える。
    良く言えば、、あんまり良く言えないな、、。
    (後略)
    % 傾きのようなもっと大きなブレを気にしていたが、こんな小さなブレも起きているのか、、これを治すのは骨が折れそう
  }

  % \vspace{0.5em} \hrule height 0.5pt \vspace{0.5em}

  \end{hyojoentry}

  \caption{表情エントリ《A1》}            
  \label{fig:hyojonotea1}
\end{figure}

\subsection{《ンーコツコツンー(A7)》}
《ンーコツコツンー(A7)》(\autoref{fig:hyojonotea7})は、「線分」スタイルの補助線を「右肩-左つま先」「左肩-右つま先」の各2点間に引き、全点を非表示にすることで作図した表情図形から得た表情である。
斜め方向に伸び切る線のようす(局面1)を「限界まで倒れそうになる軸」とし、それを支えるために慌てて直後の足の踏み出しが起こっているように「見えるような見えないような」、
そういう絶妙な違和感をAは生々しく書き綴っている(\autoref{fig:hyojonotea7})。
人体形に似て非なる図形に移入しようとすると、どこかもつれた身体感覚が芽生える。
そうやって違和感(問い)を触発するという、表情感得の事例だと著者は考える。
% 《ンーコツコツンー(A7)》
\begin{figure}[H]
  \begin{hyojoentry}

  \textbf{《ンーコツコツンー(A7)》} (oldman1)

  \vspace{0.5em} \hrule height 0.5pt \vspace{0.5em}

  \begin{center}
    \begin{overpic}[width=0.18\columnwidth]{./images/hjplayground/A7b-1.pdf}
      \put(2,90){1(始動)}
    \end{overpic}
    \hspace{-0.5em}
    \begin{overpic}[width=0.18\columnwidth]{./images/hjplayground/A7b-2.pdf}
      \put(2,90){2(ンー)}
    \end{overpic}
    \hspace{-0.5em}
    \begin{overpic}[width=0.18\columnwidth]{./images/hjplayground/A7b-3.pdf}
      \put(2,90){3(コツ)}
    \end{overpic}
    \hspace{-0.5em}
    \begin{overpic}[width=0.18\columnwidth]{./images/hjplayground/A7b-4.pdf}
      \put(2,90){4(コツ)}
    \end{overpic}
    \hspace{-0.5em}
    \begin{overpic}[width=0.18\columnwidth]{./images/hjplayground/A7b-5.pdf}
      \put(2,90){5(ンー)}
    \end{overpic}
  \end{center}

  \vspace{0.5em}

  \small
  \textit{
    限界まで倒れる軸を支えるために慌てて足が出てるように見えるような気がする。
    実際にはそんなバランス崩れるような動きではないが、この棒二本だとそう見える。
    と思ったが、そうでもない感じする。
    なにかというと、斜めが伸びきっている時、線が伸び縮みしているからだ。と思う。
    (後略)
    % 立体だからか。
    % 後ろにそれば、長さが変わるんだな。奥行きのあるダンス?
    % 真正面から見たら平面的?奥行きのない身体などないな。
    % 地面表示してみたら、この棒だけで、もうオールドマン"
  }

  % \vspace{0.5em} \hrule height 0.5pt \vspace{0.5em}

  \end{hyojoentry}

  \caption{表情エントリ《A7》}            
  \label{fig:hyojonotea7}
\end{figure}

\subsection{《ブふぁああーーー(A3)》}
《ブふぁああーーー(A3)》(\autoref{fig:hyojonotea3anda19}上段)は、
胸骨1点と骨盤4点とに補助線を引いて「四角錐」を作り、それが動くようすから得た表情である。
Aはこの表情図形を、四角錐の胸骨頂点を中心に底面を振り動かす感覚としてとらえた。
同時にAは「固さ」を感じとった。
普段、体内には「ゴム」や「風船」といった弾性体モチーフでアプローチしてきたAにとって、
体内で四角錐を動かす感覚は、いっそう新しく感じられたのである。
(\autoref{fig:hyojonotea3anda19}の内省記述参照)。
「底面の浮き上がる感じ(\autoref{fig:hyojonotea3anda19}上段、局面1〜2)」など、Aはおもしろさを感じていた。
Aは通勤中さえも、歩く動きのなかでこの表情図形の咀嚼を続けていた。
(しかしそう簡単に体得できるものではないという素直な現実が語られている)。
%《ブふぁああーーー(A3)》と《へふいへ(A19)》
\begin{figure}[H]
  \begin{hyojoentry}

  上段:\textbf{《ブふぁああーーー(A3)》} (oldman1)\\
  下段:\textbf{《へふいへ(A19)》} (oldman3)

  \vspace{0.5em} \hrule height 0.5pt \vspace{0.5em}

  
  \setlength{\tabcolsep}{1pt} % ← 横のスキマを詰める!
  \renewcommand{\arraystretch}{0.95} % ← 縦のスキマ(ラベルと画像間)
  % 上段
  \begin{center}
    \begin{overpic}[width=0.18\columnwidth]{./images/hjplayground/A3b-1.pdf}
      \put(2,90){1}
    \end{overpic}
    \hspace{-0.5em}
    \begin{overpic}[width=0.18\columnwidth]{./images/hjplayground/A3b-2.pdf}
      \put(2,90){2}
    \end{overpic}
    \hspace{-0.5em}
    \begin{overpic}[width=0.18\columnwidth]{./images/hjplayground/A3b-3.pdf}
      \put(2,90){3}
    \end{overpic}
    \hspace{-0.5em}
    \begin{overpic}[width=0.18\columnwidth]{./images/hjplayground/A3b-4.pdf}
      \put(2,90){4}
    \end{overpic}
    \hspace{-0.5em}
    \begin{overpic}[width=0.18\columnwidth]{./images/hjplayground/A3b-5.pdf}
      \put(2,90){5}
    \end{overpic}
  \end{center}

  \vspace{0em}

  % 下段
  \begin{center}
    \begin{overpic}[width=0.18\columnwidth]{./images/hjplayground/A19b-1.pdf}
      \put(2,90){1(へ)}
    \end{overpic}
    \hspace{-0.5em}
    \begin{overpic}[width=0.18\columnwidth]{./images/hjplayground/A19b-2.pdf}
      \put(2,90){2(ふい)}
    \end{overpic}
    \hspace{-0.5em}
    \begin{overpic}[width=0.18\columnwidth]{./images/hjplayground/A19b-3.pdf}
      \put(2,90){3(へ)}
    \end{overpic}
    \hspace{-0.5em}
    \begin{overpic}[width=0.18\columnwidth]{./images/hjplayground/A19b-4.pdf}
      \put(2,90){4}
    \end{overpic}
    \hspace{-0.5em}
    \begin{overpic}[width=0.18\columnwidth]{./images/hjplayground/A19b-5.pdf}
      \put(2,90){5}
    \end{overpic}
  \end{center}  
  
  
  \small  
  \textbf{《ブふぁああーーー(A3)》の内省記述}\\
  \vspace{-1.0em} \hrule height 0.5pt \vspace{0.5em}  
    骨盤を底面、胸骨の真ん中を頂点とする四角錐。
    底面がこちら側に浮き上がるように持ち上がる。
    浮き上がる最初の勢いは結構あって、ブワってくるかと思いきや。
    そのあとはふぁあああーーって感じでふわって持ち上がる。
    これおもろそう。
 
    20250401追記。
    今朝、駅で考えながら歩いてみた。
    これ、結構面白い。あんまりまだ意識しこなせないけど。
    手と足が同時に出そうになったりした。
    身体の中に知らない関係図を採用したためじゃない?
    体内がゴムとか風船とかそういう比喩意識で動くことはちょいちょいあったけど、幾何学立体ってのは初めての感覚だな。
    柔らかく動かないせいで、扱いが難しい感じ。
    なぜかこのあと、猫の捻れ問題に思いを馳せた。
    上手いダンサーは割と身体単体でエネルギー収支合わせてそうな気がする。
    仕事量0のダンス?疲れなさそーー。ダンス初めて、ハウスとかhiphopとか、だんだん疲れなくなっていくのを感じる時期があったことを思い出す。  

  % \vspace{0.5em} \hrule height 0.5pt \vspace{0.5em}
  \vspace{0.5em}
  \small  
  \textbf{《へふいへ(A19)》の内省記述}\\
  \vspace{-1.0em} \hrule height 0.5pt \vspace{0.5em}  
    前回は、四角錐の頂点を持って四角錐を振るという想像をしていたが、これ、底面が頂点を振るというふうに見える。
    頂点が思ったよりくるくるブンブン動いていて、底面は底面として移動している感じ。
    底面の角度が頂点を動かしている。
    へ で底面が浮く。 ふい で真ん中を次の位置に向かって移動。へ で次の位置に着地。これは底面の動きに対する擬音。
    この、連動におけるどっちが主導?というはなし、結構大事な要素な気がする。
    同じステップでも、頂点を主とするか、底面を主とするかで2通りの捉え方ができる。
    他の動きにおいても、主導する部位を入れ替えればそれだけで動きのパートリーを倍にできる。
    可能性がある。  

  \end{hyojoentry}

  \caption{表情エントリ《ブふぁああーーー(A3)》と《へふいへ(A19)》}            
  \label{fig:hyojonotea3anda19}
\end{figure}

\subsection{《フサササフサ(A10)》}
% 《フサササフサ(A10)》
《フサササフサ(A10)》は、Aが備えつけ補助線パタン「左から右に一筆がき」をもちいて作図した表情図形から感得した表情である。
くしくもAはStrandbeest\cite{jansen:online}を想起しており、風に吹かれて動くような身体感覚を得た(\autoref{fig:hyojonotea10})。
Aは内省記述内で、以前作成した《ンーコツコツンー(A7)》(\autoref{fig:hyojonotea7})の表情図形についても言及している。
類似の表情図形をも引き込みつつ、両者に通底する表情として感じようとすることは、意味に厚みを与えることでもあると著者は解釈する。
\begin{figure}[H]
  \begin{hyojoentry}

  \textbf{《フサササフサ(A10)》} (oldman1)

  \vspace{0.5em} \hrule height 0.5pt \vspace{0.5em}

  \begin{center}
    \begin{overpic}[width=0.18\columnwidth]{./images/hjplayground/A10-1.pdf}
      \put(2,90){1}
    \end{overpic}
    \hspace{-0.5em}
    \begin{overpic}[width=0.18\columnwidth]{./images/hjplayground/A10-2.pdf}
      \put(2,90){2(フサ)}
    \end{overpic}
    \hspace{-0.5em}
    \begin{overpic}[width=0.18\columnwidth]{./images/hjplayground/A10-3.pdf}
      \put(2,90){3(サ)}
    \end{overpic}
    \hspace{-0.5em}
    \begin{overpic}[width=0.18\columnwidth]{./images/hjplayground/A10-4.pdf}
      \put(2,90){4(サ)}
    \end{overpic}
    \hspace{-0.5em}
    \begin{overpic}[width=0.18\columnwidth]{./images/hjplayground/A10-5.pdf}
      \put(2,90){5(フサ)}
    \end{overpic}
  \end{center}

  \vspace{0.5em}

  \small
  \textit{
    "https://www.youtube.com/watch?v=Pj-NqWDH2qE
    これを感じる。風で動く木組みのモンスターみたいなやつ。 
    風が吹くと連動して全体が動く様
    無駄のない動きってやつはこのように、一つの力が全体に連動していって、勝手に身体が動いていくって感じなんじゃないか。
    軸棒クロスだけのノートを前に作ったけど、あれも可動域の限界がきてトントンって足が出る感じだった。そういうことか。 
    擬音の話で言うと、これはもはやあの木組のモンスターの動きの擬音である。風のフと、軽くて静かな動きササの合わせ技。
  }

  \end{hyojoentry}

  \caption{表情エントリ《フサササフサ(A10)》}            
  \label{fig:hyojonotea10}
\end{figure}

\subsection{《へふいへ(A19)》と《ブふぁああーーー(A3)》}
Aは、《フサササフサ(A10)》を作成した同日、2回目の撮影を行った。
2通りの意識のしかたでoldmanを踊った。
その2通りとは、Aがこれまで作成した10個の表情エントリのなかで、A自身がとくに気になっている表情であった。
それをもとに意識をつくって踊ってみるとどうなるのかをAは試してみたくなったのである。
Aは、ひとつは、《フササフサ(A10)》にもとづき「フササフサ」というオノマトペを意識して踊り(運動データ「oldman2」)
もうひとつは、《ブふぁああーーー(A3)》の表情図形を動かすことをイメージして踊る(運動データ「oldman3」)
ことで、撮影をおこなった。

oldman2に対しては、《うー、わっダラララ(A12)》や《ねーねねねね…(A13)》などの表情を感得したが、あまり新しい問いを生み出せなかったとAはインタビューで語った。
oldman3に対しAは、《へふいへ(A19)》(\autoref{fig:hyojonotea3anda19})を作図した。
これは、oldman1で作図した《ブふぁああーーー(A3)》と同じく胸骨と骨盤からなる四角錐だが、そこに軌跡(120fpsで10フレーム一単位とした15単位=1.25秒ぶん)をつけた図形である。

Aはここで大きな気づきを得る。
両図形とも同じ部位からなる四角錐であるいっぽうで、その表情には大きな違いがあることをAは見出した。
《ブふぁああーーー(A3)》では局面1-2にかけて頂点が右に移動し(ぶ)、
局面2-4にかけて底面が頂点に振られる(ふぁー)。
いっぽう《へふいへ(A19)》では、局面1で底面が浮き(へ)、
局面2-3にかけて頂点が底面に振られ(ふい)(3で底面は自身が振り動かした頂点に追従し)、
局面3-4で底面が着地する(へ)。
つまり《A19》は、《A3》とは底面-頂点の連動主従関係が逆転した別の表情である。
そうしてAは「他の動きにおいても、主導する部位を入れ替えればそれだけで動きのレパートリーを倍にできる」という、踊りの引き出しについての一段階メタな仮説へと昇華したのである(\autoref{fig:hyojonotea3anda19}の内省記述)。

\subsection{アプリの実践がAのダンスにもたらしたこと}
\label{subsec:caseAsummay}

\textcolor{red}{
  以上、Aはoldmanでのアプリ使用をとおして、自身のダンスに以下のような問いを展開した。
\begin{itemize}
  \item 両肩のラインに意外な揺れ動きが生じているという事実に着眼した(《フリフリ(A1)》)
  \item oldman自体のしっくりくるよう踊る意識のしかたを発見した(《フサササフサ(A10)》)
  \item 左右移動するときの新しい感覚を獲得した(《ンーコツコツンー(A7)》)、
  \item 部位間の主従を転換しようと意識するだけで動きのレパートリーを倍増しうるという一段メタな仮説を形成した(《へふいへ(A19)》と《ブふぁああーーー(A3)》)
\end{itemize}
}


本アプリで感得した「表情」が、Aが日常で踊るダンスに確かな影響があったことを、Aはインタビューで語った。
2025年4月16日に踊ったダンスのなかで、Aは《ブふぁああーーー(A3)》を思い浮かべた動きを繰り出した(\autoref{fig:a3i})。
なお、この時点では《A11》までを作成していた。
BPM=約88の曲のフォーカウントからフォーエンカウントの動きである。
局面1で「骨盤底面」が「ブ」と浮き、局面2-4で胸骨頂点を中心に骨盤底面を「ふぁああーーー」と振っている。
この動きはoldmanとは右足と左足を着く順番が逆転しているが、
実際の即興ダンスのなかでは、基本技は、状況(音楽や環境)に即応するよう変形して繰り出される。
そのように踊るためには、状況のもつ表情を感得するのはもちろんのこと、
自らの踊りそのものの表情をも感得することも重要だろうと著者は考える。
% 《ブふぁああーーー(A3)》の実践編
\begin{figure}[H]
  \begin{continuousphoto}

  \begin{center}
    \begin{overpic}[width=0.240\columnwidth]{./images/hjplayground/A3i-1.pdf}
      \put(2,90){1}
    \end{overpic}
    \hspace{-0.5em}
    \begin{overpic}[width=0.240\columnwidth]{./images/hjplayground/A3i-2.pdf}
      \put(2,90){2}
    \end{overpic}
    \hspace{-0.5em}
    \begin{overpic}[width=0.240\columnwidth]{./images/hjplayground/A3i-3.pdf}
      \put(2,90){3}
    \end{overpic}
    \hspace{-0.5em}
    \begin{overpic}[width=0.240\columnwidth]{./images/hjplayground/A3i-4.pdf}
      \put(2,90){4}
    \end{overpic}    
  \end{center}

  \end{continuousphoto}

  \caption{《ブふぁああーーー(A3)》を発動しているAのダンス}
  \label{fig:a3i}
\end{figure}

実践終了直後の2025年4月19日に踊ったダンスで、Aは《へふいへ(A19)》から影響を受けた動きを試してみた(\autoref{fig:a19i})。
BPM=約82の曲のツーカウント(2)からツーエンカウント(2.5)の動きである。
右足に体重をかけて踏み締め、その地面反力を利用して左後ろに身体を放る(局面4の直後)。
骨盤を右足にしっかりと乗せたまま傾け、それを伝播させるように、胸骨をわずかに前を通るようにしながら(局面2-3)、左後ろにむかって(局面4)小さく振り回している。
図からもただちには見てとれないほど微妙で繊細である。
Aはインタビューにて、これを動画で観ると、踊っている最中に自分が感じていたほど《へふいへ》の動きにはなっていないと反省をした。
動いているさなかで本人に立ち現れる感覚(あるいは醸し出そうとする表情)と、それを外からみて感得しうる表情は同じとは限らない。
しかし、動いている身体に感得した表情から問いを展開し、動きかたの引き出しを増やした(=意味をつくった)ことは、
本アプリがAにもたらしたひとつの意義であると考える。
% 《へふいへ(A19)》の実践編
\begin{figure}[H]
  \begin{continuousphoto}
  \begin{center}
    \begin{overpic}[width=0.240\columnwidth]{./images/hjplayground/A19i-1.pdf}
      \put(2,90){\color{white}1}
    \end{overpic}
    \hspace{-0.5em}
    \begin{overpic}[width=0.240\columnwidth]{./images/hjplayground/A19i-2.pdf}
      \put(2,90){\color{white}2}
    \end{overpic}
    \hspace{-0.5em}
    \begin{overpic}[width=0.240\columnwidth]{./images/hjplayground/A19i-3.pdf}
      \put(2,90){\color{white}3}
    \end{overpic}
    \hspace{-0.5em}
    \begin{overpic}[width=0.240\columnwidth]{./images/hjplayground/A19i-4.pdf}
      \put(2,90){\color{white}4}
    \end{overpic}    
  \end{center}

  \end{continuousphoto}

  \caption{《へふいへ(A19)》を発動しているAのダンス}          
  \label{fig:a19i}
\end{figure}

その他Aは、日常の身体運用においても、表情図形を自らの身体に召喚していた。
いつも20分徒歩で通勤しているAは、歩いているときに《ガチョベチョカキョ(A4)》を試して、金属的な足音をまざまざと感じたという。
また表情《ブふぁああーーーペ(A11)》は、《ブふぁああーーー(A3)》と《ぐういぺ(A2)》の2つの表情図形を足し引きして作った表情図形である。
実際に踊りに踊るときの感覚に近づけるためのAのこのやりくりは、本アプリの仕様が促したものとも考えられる。


\clearpage
\section{対象者Bの学び}
\label{sec:caseB}

Bは陸上三段跳選手である。
三段跳は、助走からホップ→ステップ→ジャンプの3歩で跳ぶ距離を競う種目である(\autoref{fig:triplejump})。
\begin{figure}[H]
  \centering
  \includegraphics[width=\textwidth]{./images/hjplayground/triplejump.pdf}
  \caption{三段跳(Bのパフォーマンス動画から著者作成)}          %和文 cap tion
  \label{fig:triplejump}
\end{figure}
三段跳びでは、\autoref{fig:triplejump}のような複雑で激しい動きのなかで、巧みに姿勢を調整することが重要である。
本実践では、Bの姿勢調整技術を研ぎ澄ますために、2種類の「対人一本ゲタ運動」すなわち、BとCが一本ゲタを履いて立ち、対面状態で「互いに手を握り合って姿勢を保持しあう運動(以下、協力)」と「手押し相撲(以下、相撲)」を対象動作として考案した。

一本ゲタではまっすぐ立つことさえ難しい。
Bは本実践以前から、一本ゲタを履いてさまざまな動きを試し、姿勢調整技術を磨かんとしていた。
本実践ではここにさらに「対人運動」という予測不可能性を盛り込むことにしたわけである。
予測不可能にすることで「表情」を漏れ出させよう、とするねらいもあった。

本実践では、「協力 or 相撲」の条件と、Bが履く一本ゲタの種類「歯の位置が中央寄り or 後ろ寄り」という条件(CはBと異なるほうの一本ゲタを履く)とを合わせた
計4パタンの運動データ「協力B後ろゲタ」「協力B中央ゲタ」「相撲B後ろゲタ」「相撲B中央ゲタ」を撮影した。
4条件設けたねらいは、比較実験をするためというよりも、多様な状況をつくり多様な表情から姿勢調整を学習する可能性を拡げるためである。

\begin{figure}[H]
  \centering
  \includegraphics[width=\textwidth]{./images/hjplayground/interpersonal_geta.pdf}
  \caption{対人一本ゲタ運動4条件と、Bが作成した表情エントリ群}          %和文 cap tion
  \label{fig:ippongeta}
\end{figure}

Bは2025年3月25日〜4月9日の15日間で、全23個の表情エントリ(《B1》〜《B23》)を作成した(\autoref{table:hyojoentries_b})。

% Bの表情エントリ一覧
\begin{longtable}{lllp{5cm}r}
\caption{Bの作成した表情エントリ一覧} \\ 
\toprule
表情エントリID & 作成日付 & 運動データ & 表情オノマトペ & 文字数 \\
\midrule
\endfirsthead
\toprule
表情エントリID & 作成日付 & 運動データ & 表情オノマトペ & 文字数 \\
\midrule
\endhead
《B1》 & 03/25/25 & 相撲B後ろゲタ & ほー・ふぐっ & 22 \\
《B2》 & 03/25/25 & 相撲B後ろゲタ & びよびよーー・ふぐっ & 353 \\
《B3》 & 03/25/25 & 相撲B後ろゲタ & ツン(+攻撃) & 332 \\
《B4》 & 03/25/25 & 相撲B後ろゲタ & お局orませた女の子(気が強そう) & 121 \\
《B5》 & 03/25/25 & 相撲B中央ゲタ & すぽーーん & 159 \\
《B6》 & 03/26/25 & 協力B後ろゲタ & ふん・ふん & 56 \\
《B7》 & 03/26/25 & 協力B後ろゲタ & びょーん・がちゃぐん & 86 \\
《B8》 & 04/08/25 & 相撲B後ろゲタ & んぽん(ビリヤード) & 167 \\
《B9》 & 04/08/25 & 相撲B後ろゲタ & ブンブン(んぽんのコピー) & 182 \\
《B10》 & 04/08/25 & 相撲B後ろゲタ & & 35 \\
《B11》 & 04/08/25 & 協力B中央ゲタ & プシュプシュぷわっ(気泡) & 215 \\
《B12》 & 04/08/25 & 協力B中央ゲタ & 横隔膜or綱渡 & 223 \\
《B13》 & 04/08/25 & 協力B中央ゲタ & ぬぬぬんんふわふわふわ & 197 \\
《B14》 & 04/08/25 & 協力B後ろゲタ & ぬぐむぐ(イモムシ はいはい) & 187 \\
《B15》 & 04/08/25 & 相撲B中央ゲタ & ギーコギコ & 270 \\
《B16》 & 04/08/25 & 相撲B中央ゲタ & うぉーーっとっとごめん & 268 \\
《B17》 & 04/08/25 & 相撲B中央ゲタ & ずんずんずんズコーー & 221 \\
《B18》 & 04/08/25 & 協力B後ろゲタ & わん、つ & 163 \\
《B19》 & 04/08/25 & 協力B中央ゲタ & ぱきっつつ & 100 \\
《B20》 & 04/09/25 & 協力B後ろゲタ & あいやっ & 185 \\
《B21》 & 04/09/25 & 協力B後ろゲタ & ぼぶわっ! & 428 \\
《B22》 & 04/09/25 & 協力B後ろゲタ & ディフェンス & 8 \\
《B23》 & 04/09/25 & 協力B後ろゲタ & うーみょううみょうんぶん & 450 \\
\bottomrule
\label{table:hyojoentries_b}
\end{longtable}

Bがこれらの表情エントリを対人一本ゲタ運動のどのシーンに対して作ったのか、時間的対応関係を\autoref{fig:ippongeta}に示す。
Bの実践のダイジェストを、インタビューにてBが語った4つの表情《B3》《B11》《B20》《B21》を取り上げながら説明する。
4つの表情は\autoref{fig:ippongeta}にて赤で示したものであり、図内写真に番号づけた各局面は、本節の\autoref{fig:hyojonoteb3}・\ref{fig:hyojonoteb11}・\ref{fig:hyojonoteb20andb21}の同番号局面と同じである。

\subsection{《ツン(+攻撃)(B3)》}
《ツン(+攻撃)(B3)》(\autoref{fig:hyojonoteb3})は
運動データ「相撲B後ろゲタ」における
「BがCを手押すのをCが受け流す」
0.7秒間のシーンに対し、
備え付け補助線パタン「全体の輪郭」を適用し、
補助線スタイル「線分」で図形を作図し、そこに感得した表情である。
Bは図形の「三角形の鋭角(Bの肩-手首-肘)のふるまい」に、「照準を定めて攻撃する意志をもって突く」ようすを見てとった。
興味深いことに、これは三段跳の各踏切時に「地面に足を衝突させるさま」としても立ち現れた。
それまでBが抱いていた接地のモチーフは「ボールが弾む感覚」であった
(奇しくも、Aが《ブふぁああーーー(A3)》(\autoref{fig:a3i})で語ったのと同じく弾性体のモチーフであることも興味深い)。
そんなBにとってこの「三角形で攻撃的に打突」という新しい接地感覚となった。
同時にそれは「現在の自分はできていない、ひとつの望ましい踏切のありかた」だとして、一種の「指針」として解釈した
\footnote{三段跳のステップ踏切時は、接地直前から着く脚の主動筋群を収縮させ、自ら地面に着きにいくような「積極的着地」が良いともされている。}
。
このようにBは、抽象的な三角形の動きを媒介に、《ツン(+攻撃)》という、「手で押す動作」とも「三段跳の接地動作」ともなりうるような表情を感得したのである。
% 《ツン(+攻撃)(B3)》
\begin{figure}[H]
  \begin{hyojoentry}
  \textbf{《ツン(+攻撃)(B3)》} (相撲B後ろゲタ)

  \vspace{0.5em} \hrule height 0.5pt \vspace{0.5em}

  \begin{center}
    \begin{overpic}[width=0.3\columnwidth]{./images/hjplayground/B3-1.pdf}
      \put(2,90){1}
    \end{overpic}
    \hspace{0.05em}
    \begin{overpic}[width=0.3\columnwidth]{./images/hjplayground/B3-2.pdf}
      \put(2,90){2}
    \end{overpic}
    \hspace{0.05em}
    \begin{overpic}[width=0.3\columnwidth]{./images/hjplayground/B3-3.pdf}
      \put(2,90){3}
    \end{overpic}
  \end{center}

  \vspace{0.5em}

  \small
  \textit{
    ツン とはしたけど表現したいものはもっと攻撃的。
    上体の部分でできている三角形が相手に攻撃していく、刺さる部分を表現したいんだけどいい言葉が思いつかない。
    三角形は相手を攻撃する意志がみえる。
    「ツ」の音がしっくりくる。ただツンって優しく指しているから微妙。
    最初見えなかった三角形がつくられる。
    照準をしぼる、的をねらう。
    三段でいうなら踏切板の近くだろうか。
    そういえば三段跳び、三角形、両方3だな。
    これまで体の中には球体しか作ってこなかったけど(弾ませたくて)案外角のある三角形もやくだつのかなぁ。
    三角形の方が球体よりも能動的に動かさないとはずまない。
    最近の三段の課題はあおられる、後傾することで、もっと突っ込みたい。
    三角の意思つくってみようかな。
  }  

  \end{hyojoentry}

  \caption{表情エントリ《B3》}          
  \label{fig:hyojonoteb3}
\end{figure}

\subsection{《プシュプシュぷわっ(B11)》}
《プシュプシュぷわっ(B11)》(\autoref{fig:hyojonoteb11})は
運動データ「協力B中央ゲタ」における
「BがCが前にバランスをくずして前に体重をかけてきたのをのけぞりつつ耐えたのち、押し返して直立状態に復帰する」
3.4秒間のシーンに対し、
Bが備えつけ補助線パタンの「三角形で埋める」を適用し、
補助線スタイル「内接円+線分」で図形を作図し、そこに感得した表情である。
BC二人の身体の間を満たす円群が大小関係を様々に変化させながら蠢く。
両者のあいだが「泡立つ」かのようである。
Bはここに「全体のエネルギーをある程度保ちつつ、内部のエネルギー疎密を柔軟に移動させる」ことを感じた。
これはBの三段跳と太くつながった。

説明しよう。
Bには「全身が力んでしまいやすい」癖があった。
それを防ぐためにBは三段跳において、体幹部に注力し『常に腹圧をかける』ようにしていた。
しかし空中にて四肢のほうをうまくつかえずにいた
(これをBは『手足に神経が通わない感じ』だと表現する)。
だからといって腕ばかり注力しても、今度は要である体幹部とのつながりが途切れてしまう。
あるいはただ「全身を脱力させる」意識が良いというわけでもない。
三段跳ではステップ・ジャンプ踏切時の衝撃負荷に耐えねばならず
\footnote{ステップの瞬間は体重の7〜12倍もの負荷がかかるという報告\cite{ramey_et_al:1985}もある。}
、完全な脱力はそれになじまないからである。

こうした脱力と緊張との関係性に悩んでいたところ、
Bは《B11》をつうじて「全体を保ちつつ、中身の疎密だけを移動をさせる」といった新しいとらえかたを獲得したのであった。
インタビューでBは、このことについて語るなかで『よく、上手な選手は「腕で舵を取る」と言われる』と言及した。
船で舵を取る動作には、重さや手応えを伴う。
《B11》をふまえて言い換えるならば、上手な選手は「腕から体幹部のほうへと蜜を流し込む」かのように腕を使う、ということだろう。
《B11》は、Bが三段跳の文脈で耳にしていた「意味深な言葉」の意味を自分なりに咀嚼するそのきっかけを与えたのではないかと著者は考える。

% 《プシュプシュぷわっ(気泡)(B11)》
\begin{figure}[H]
  \begin{hyojoentry}

  \textbf{《プシュプシュぷわっ(気泡)(B11)》} (協力B中央ゲタ)

  \vspace{0.5em} \hrule height 0.5pt \vspace{0.5em}

  \begin{center}
    \begin{overpic}[width=0.18\columnwidth]{./images/hjplayground/B11-1.pdf}
      \put(2,90){1}
    \end{overpic}
    \hspace{-0.5em}
    \begin{overpic}[width=0.18\columnwidth]{./images/hjplayground/B11-2.pdf}
      \put(2,90){2}
    \end{overpic}
    \hspace{-0.5em}
    \begin{overpic}[width=0.18\columnwidth]{./images/hjplayground/B11-3.pdf}
      \put(2,90){3}
    \end{overpic}
    \hspace{-0.5em}
    \begin{overpic}[width=0.18\columnwidth]{./images/hjplayground/B11-4.pdf}
      \put(2,90){4}
    \end{overpic}
    \hspace{-0.5em}
    \begin{overpic}[width=0.18\columnwidth]{./images/hjplayground/B11-5.pdf}
      \put(2,90){5}
    \end{overpic}
  \end{center}

  \vspace{0.5em}

  \small
  \textit{
    泡、気泡がこの膜の中にある。
    この気泡はなるべく多く存在していたい。
    でもどちらかに引っ張られれば、無駄に抵抗せずに一旦気泡を潰す。
    そうすると弾けた気泡の勢いでそちら側に体勢を立て直すことができる。 
    変に形を保とうと力んだりせず、柔軟に対応していけばいい、一旦なくなる部分があっていい、そういう肩肘張らない大事さ、抜重、力みと脱力の関係見たいのが見える気がする。
    中心部の気泡は潰れていないのを見ると、守るべき場所もあるようだ。"
  }  

  \end{hyojoentry}

  \caption{表情《B11》}          
  \label{fig:hyojonoteb11}
\end{figure}

\subsection{《あいやっ(B20)》と《ぼぶわっ!(B21)》}
《あいやっ(B20)》(\autoref{fig:hyojonoteb20andb21}上段)は
運動データ「協力B後ろゲタ」における
「Cが前方にバランスをくずして左足を踏み出したのをBがわずかに後傾しつつ耐えた」
1.6秒間のシーンに対して、
「Bの右ゲタの歯先と前端」と「Cの左ゲタの歯先と前端」をそれぞれ結び、
補助線スタイル「10倍延長」で図形を作図し、そこに感得した表情である。
Bはここに、剣道の打ち込みの情景をみた(「あいやっ」は打ち込み時の発声に近い)。

Bはすかさず《あいやっ(B20)》から点群を表示状態に変えた表情図形をつくり、
《ぼぶわっ!(B21)》(\autoref{fig:hyojonoteb20andb21}下段)というオノマトペを名付けた。
表示した点群が、竹刀を握る「手首」として現れた。
打つさいの迷いや、淡々と待ち構える相手すらも見えてくることが内省記述には表れている。
そのさまにBは「テコの原理」を思った。
Bは、元の動作では「足首」に相当する部分なのだという事実知覚相\cite{hiromatsu:1989}へと持ち帰り、
遂に「足首で剣を打ち込むような接地」を発想するに至ったのである。
作図した表情図形に(偶然にも)「線どうしの交差」という幾何学的関係性が生じたからこそ、
剣を切り結ぶ情景が立ち現れたのだろうと推察できる。
% あいやとぼぶわ
\begin{figure}[H]
  \begin{hyojoentry}
  上段:\textbf{《あいやっ(B20)》} (協力B後ろゲタ)\\
  下段:\textbf{《ぼぶわっ!(B21)》}(協力B後ろゲタ)

  \vspace{0.5em} \hrule height 0.5pt \vspace{0.5em}

  
  \setlength{\tabcolsep}{2pt} % ← 横のスキマを詰める!
  \renewcommand{\arraystretch}{0.95} % ← 縦のスキマ(ラベルと画像間)
  % 上段
  \begin{center}
    \begin{tabular}{cccc}
      初動 & 剣を振り上げ & 打ち込む \\
      \begin{overpic}[width=0.3\linewidth]{./images/hjplayground/B20-1.pdf}
        \put(2,90){}
      \end{overpic} &
      \begin{overpic}[width=0.3\linewidth]{./images/hjplayground/B20-2.pdf}
        \put(2,90){}
      \end{overpic} &
      \begin{overpic}[width=0.3\linewidth]{./images/hjplayground/B20-3.pdf}
        \put(2,90){}
      \end{overpic} &      
    \end{tabular}
  \end{center}
  
  \vspace{-0.5em}
  
  % 下段
  \begin{center}
  \begin{tabular}{cccc}
    \begin{overpic}[width=0.3\linewidth]{./images/hjplayground/B21-1.pdf}
      \put(2,90){}
    \end{overpic} &
    \begin{overpic}[width=0.3\linewidth]{./images/hjplayground/B21-2.pdf}
      \put(2,90){}
    \end{overpic} &
    \begin{overpic}[width=0.3\linewidth]{./images/hjplayground/B21-3.pdf}
      \put(2,90){}
    \end{overpic} &    
  \end{tabular}
  \end{center}
  

  
  \small  
  \textbf{《あいやっ(B20)》の内省記述}\\
  \vspace{-1.0em} \hrule height 0.5pt \vspace{0.5em}  
    剣道的な。剣で打ち込んでいる。
    左の打ち込む剣はぶれている。剣に迷いがあるようだ。
    打ち込んだ後引いているところを見ると成功しなかったらしい。
    一方で右の剣は穏やかだ。
    淡々と構えて受けている。
    右の剣の方が長さもある。
    そういえば、剣を打ち込んで止まるとき剣どうしは割と中心付近で交わるけど中心よりは手前で止まる。
    これが両者の間合いだろうか
    近すぎたくはないのか?    

  \vspace{0.5em}

  \small  

    \textbf{《ぼぶわっ!(B21)》の内省記述}\\
    \vspace{-1.0em} \hrule height 0.5pt \vspace{0.5em}  
    あいやっのコピー。点をつけてみた。
    点をつけると棒を持つ手が見える(5つの点で構成)
    そして左の人の手首(棒をもたない3つの点)が震えていることに気づく。 
    はじめは、こんなに震えているようでは渾身の一撃は加えられないぞと批判的に感じた。
    だが、打ち込むスピードに注目してみると、結構はやく動いているのでは?と。
    手首側の2つの点が激しく運動(上下に?回転もかかっている?)することによってその力は剣に効率よく伝わる。
    ちょうどテコの原理のように。
    そう言えば下駄は、というか人間の足はテコの原理を使っていることを思い出す。
    踵が力点で、支点である歯、つま先が作用点といった具合のはず。
    踵にグンと力をほんの一瞬かけることで素早い推進力を得る。
    これまで身体操作をする際に「足」としてテコを意識しても素早く動かせる感覚が生まれにくかったが、手首であれば神経が鮮明なので思い浮かべやすい。
    足の延長線上に剣があると思ってそれを打ち込むのやってみよう。  
  

  \end{hyojoentry}

  \caption{《あいやっ(B20)》と《ぼぶわっ!(B21)》}          
  \label{fig:hyojonoteb20andb21}
\end{figure}

《あいやっ(B20)》と《ぼぶわっ!(B21)》では、もうひとつ興味深いことが起きていた。
\autoref{fig:hyojonoteb20andb21}で左側から打ち込んでいる人物はCであるが、Bは作図中にこれがB自身だと思い込んでいた。
Bはアプリ内カメラ位置の緯度・経度をまわして操作するうち、抽象的な映像内でBとCどちらがどちらかわからなくなり(気にならなくなり)、知らずうちにCへ移入してしまったのだろう。
そうならば本アプリには、自他のきびしい区別が取り払ってより自由な移入をうながす、という効果もあると考えうる。

\subsection{アプリの実践がBの三段跳にもたらしたこと}
\label{subsec:caseBsummary}
\textcolor{red}{
以上、Bは一本ゲタ対人運動でのアプリ使用をとおして、自身の三段跳に以下のような問いを展開した。
\begin{itemize}
  \item (今はまだできないが)めざすべき力強い接地はこうあるべきというビジョンを思い描いた(《ツン(+攻撃)(B3)》)、  
  \item 「脱力」と「腕で舵をとる」という、よく耳にしていた「意味深な言葉」の意味を、ひとつなぎに自分なりに納得した(《プシュプシュぷわっ(B11)》)、
  \item 助走一歩目の新しい踏み込み方(竹刀を打ち込むような踏み込み)を発想した(《《あいやっ(B20)》と《ぼぶわっ!(B21)》)。
\end{itemize}
}

Bは《あいやっ(B20)》と《ぼぶわっ!(B21)》をもとにして、三段跳の「助走」における新しい意識のしかたを創りだした。
「踵を踏み込むテコの原理で、すばやく打ち込むような次の一歩を生む」というものである。
Bはその意識でもって助走を探ってみた。
しかし結果は『無駄な動きを作るだけになってしま』い、『しっくりこなかった』とBは補助的インタビューにて語った。
その理由として、足首は手首にくらべ『神経が通っていない』ことにある、とBなりに解釈を紡いだ。
これについてBは、『(本アプリで得た剣道性を)「ものにする」には、はっきりと意識しながらやるというより像として浮かべながらなんとなくで取り入れられるといいんだろうな』
とインタビューにて反省した。

その他Bは、実践期間において、三段跳の練習中に妄想が起きたという。
「踏切板の両端と自分の腰を結んだパチンコ武器(スリングショット)のような長い鋭角二等辺三角形があり、引っ張ったゴムから指を放しバチンと球が放たれるように、助走からホップ踏切をする」というものだ。
この発想元は《B3》にあるという。
本アプリで補助線を引くという行為をした経験そのものも、Bのアスリートとしての実践に影響したことが伺える。

% \begin{figure}[H]
%   \centering
%   \includegraphics[width=\textwidth]{./images/hjplayground/triplejump.pdf}
%   \caption{三段跳(Bのパフォーマンス動画から著者作成)(再掲)}          %和文 cap tion
%   \label{fig:triplejump}
% \end{figure}