\chapter{分析}
\label{chapter:analysis}
\section{データ全貌}

前章では、AとBによるアプリをもちいた運動学習について、
すなわちAとBがどのような表情図形を作図し、それを媒介にいかなる表情を感得し、問いを深めたのかを語った。
本章では、アプリをもちいた実践の全体像を、A〜Dの表情エントリのデータから分析する。

まずはA〜D4名が実践で残した表情エントリの全一覧を以下に示す(\autoref{table:hyojoentries_a}〜\autoref{table:hyojoentries_d})。
なお表内の「文字数」とは、からだメタ認知による内省記述の文字数であり、
AとBのものについては、前章で載せた表の再掲である。

% Aの表情エントリ一覧
\begin{longtable}{llllr}
\caption{Aの作成した表情エントリ一覧} \\
\hline
表情エントリID & 作成日付 & 運動データ & 表情オノマトペ & 内省記述文字数\\
\hline
\endfirsthead
\hline
表情エントリID & 作成日付 & 運動データ & 表情オノマトペ & 内省記述文字数\\
\hline
\endhead
《A1》 & 03/25/25 & oldman1 & フリフリ & 393\\
《A2》 & 03/25/25 & oldman1 & ぐういペ & 512\\
《A3》 & 04/01/25 & oldman1 & ブふぁああーーー & 420\\
《A4》 & 04/01/25 & oldman1 & ガチョベチョガキョ & 494\\
《A5》 & 04/01/25 & oldman1 & バラらららららら & 212\\
《A6》 & 04/02/25 & oldman1 & なぬななぬな & 348 \\
《A7》 & 04/08/25 & oldman1 & ンーコツコツンー & 228 \\
《A8》 & 04/08/25 & oldman1 & グリグリ & 184 \\
《A9》 & 04/08/25 & oldman1 & フラフラぶんフラフラ& 328 \\
《A10》 & 04/13/25 & oldman1 & フサササフサ & 273\\
《A11》 & 04/13/25 & oldman1 & ブふぁああーーーペ & 23\\
《A12》 & 04/17/25 & oldman2 & うー、、わっダラララ & 122\\
《A13》 & 04/17/25 & oldman2 & ねーねねねね... & 200\\
《A14》 & 04/18/25 & oldman3 & mm & 18\\
《A15》 & 04/18/25 & oldman3 & 擬音が思いつかない、、 & 63\\
《A16》 & 04/18/25 & oldman3 & パラララららら & 11\\
《A17》 & 04/18/25 & oldman3 & 惑星 & 9\\
《A18》 & 04/19/25 & oldman1 & バラバラ & 60\\
《A19》 & 04/19/25 & oldman3 & へふいへ & 303\\
\hline
\label{table:hyojoentries_a}
\end{longtable}

% Bの表情エントリ一覧
\begin{longtable}{lllp{5cm}r}
\caption{Bの作成した表情エントリ一覧} \\ 
\toprule
表情エントリID & 作成日付 & 運動データ & 表情オノマトペ & 文字数 \\
\midrule
\endfirsthead
\toprule
表情エントリID & 作成日付 & 運動データ & 表情オノマトペ & 文字数 \\
\midrule
\endhead
《B1》 & 03/25/25 & 相撲B後ろゲタ & ほー・ふぐっ & 22 \\
《B2》 & 03/25/25 & 相撲B後ろゲタ & びよびよーー・ふぐっ & 353 \\
《B3》 & 03/25/25 & 相撲B後ろゲタ & ツン(+攻撃) & 332 \\
《B4》 & 03/25/25 & 相撲B後ろゲタ & お局orませた女の子(気が強そう) & 121 \\
《B5》 & 03/25/25 & 相撲B中央ゲタ & すぽーーん & 159 \\
《B6》 & 03/26/25 & 協力B後ろゲタ & ふん・ふん & 56 \\
《B7》 & 03/26/25 & 協力B後ろゲタ & びょーん・がちゃぐん & 86 \\
《B8》 & 04/08/25 & 相撲B後ろゲタ & んぽん(ビリヤード) & 167 \\
《B9》 & 04/08/25 & 相撲B後ろゲタ & ブンブン(んぽんのコピー) & 182 \\
《B10》 & 04/08/25 & 相撲B後ろゲタ & & 35 \\
《B11》 & 04/08/25 & 協力B中央ゲタ & プシュプシュぷわっ(気泡) & 215 \\
《B12》 & 04/08/25 & 協力B中央ゲタ & 横隔膜or綱渡 & 223 \\
《B13》 & 04/08/25 & 協力B中央ゲタ & ぬぬぬんんふわふわふわ & 197 \\
《B14》 & 04/08/25 & 協力B後ろゲタ & ぬぐむぐ(イモムシ はいはい) & 187 \\
《B15》 & 04/08/25 & 相撲B中央ゲタ & ギーコギコ & 270 \\
《B16》 & 04/08/25 & 相撲B中央ゲタ & うぉーーっとっとごめん & 268 \\
《B17》 & 04/08/25 & 相撲B中央ゲタ & ずんずんずんズコーー & 221 \\
《B18》 & 04/08/25 & 協力B後ろゲタ & わん、つ & 163 \\
《B19》 & 04/08/25 & 協力B中央ゲタ & ぱきっつつ & 100 \\
《B20》 & 04/09/25 & 協力B後ろゲタ & あいやっ & 185 \\
《B21》 & 04/09/25 & 協力B後ろゲタ & ぼぶわっ! & 428 \\
《B22》 & 04/09/25 & 協力B後ろゲタ & ディフェンス & 8 \\
《B23》 & 04/09/25 & 協力B後ろゲタ & うーみょううみょうんぶん & 450 \\
\bottomrule
\label{table:hyojoentries_b}
\end{longtable}

% Cの表情エントリ一覧
\begin{longtable}{llllr}
\caption{Cの作成した表情エントリ一覧} \\ 
\toprule
表情エントリID & 作成日付 & 運動データ & 表情オノマトペ & 文字数 \\
\midrule
\endfirsthead
\toprule
表情エントリID & 作成日付 & 運動データ & 表情オノマトペ & 文字数 \\
\midrule
\endhead
《C1》 & 03/25/25 & 相撲B後ろゲタ & ぐいーーーん & 313 \\
《C2》 & 03/25/25 & 相撲B中央ゲタ & ふえーん & 280 \\
《C3》 & 03/25/25 & 協力B後ろゲタ & すろりーす & 190 \\
《C4》 & 03/29/25 & 協力B中央ゲタ & ぼわあ 〜ん & 304 \\
《C5》 & 03/29/25 & 相撲B後ろゲタ & キリキリ & 296 \\
《C6》 & 04/02/25 & 相撲B後ろゲタ & グギーーーぎっぐ & 67 \\
《C7》 & 04/02/25 & 相撲B後ろゲタ & うーーーーーんしょ & 12 \\
《C8》 & 04/03/25 & 協力B中央ゲタ & すしゅんすシュンスゥイーーーん & 76 \\
《C9》 & 04/08/25 & 相撲B中央ゲタ & 何も生まれなかった & 24 \\
《C10》 & 04/08/25 & 相撲B中央ゲタ & 何も生まれなかったのコピー & 24 \\
《C11》 & 04/08/25 & 協力B後ろゲタ & よーいよぃよう & 181 \\
《C12》 & 04/08/25 & 協力B後ろゲタ & んがんがあ〜〜 & 101 \\
《C13》 & 04/09/25 & 協力B中央ゲタ & オフフォーっ & 495 \\
《C14》 & 04/09/25 & 協力B後ろゲタ & 一旦 & 0 \\
《C15》 & 04/09/25 & 協力B後ろゲタ & うぅーーーいおぉーーい & 279 \\
《C16》 & 04/09/25 & 協力B後ろゲタ & こそこそ & 208 \\
《C17》 & 04/09/25 & 協力B後ろゲタ & ういおい & 215 \\
《C18》 & 04/09/25 & 協力B後ろゲタ & ゆるゆる & 115 \\
《C19》 & 04/09/25 & 協力B後ろゲタ & ふんふんほいふんふんふんほい & 36 \\
《C20》 & 04/16/25 & 協力B中央ゲタ & ウェーーーい & 228 \\
《C21》 & 04/16/25 & 相撲B中央ゲタ & う〜〜〜〜〜 & 199 \\
\bottomrule
\label{table:hyojoentries_c}
\end{longtable}

% Dの表情エントリ一覧
\begin{longtable}{lllp{5cm}r}
\caption{Dの作成した表情エントリ一覧} \\
\toprule
表情エントリID & 作成日付 & 運動データ & 表情オノマトペ & 文字数 \\
\midrule
\endfirsthead
\toprule
表情エントリID & 作成日付 & 運動データ & 表情オノマトペ & 文字数 \\
\midrule
\endhead
《D1》 & 03/21/25 & oldman1 & するんりゅ & 51 \\
《D2》 & 03/21/25 & oldman1 & つゔぁーんグ & 85 \\
《D3》 & 03/21/25 & oldman1 & ひょいった & 168 \\
《D4》 & 03/21/25 & oldman1 & ひょいった のコピー & 168 \\
《D5》 & 03/21/25 & oldman1 & にゃるヴーんぱ & 98 \\
《D6》 & 03/21/25 & 相撲B後ろゲタ & はっふっは & 81 \\
《D7》 & 03/21/25 & 相撲B後ろゲタ & ギーコっとギコくわぁあ & 130 \\
《D8》 & 04/02/25 & oldman1 & フルマラソンのゴール直後、倒れそうになるも踏ん張る膝 & 250 \\
《D9》 & 04/02/25 & oldman1 & ふぶるんふぶるん & 225 \\
《D10》 & 04/03/25 & 相撲B後ろゲタ & ぐぐっグ & 166 \\
《D11》 & 04/03/25 & 相撲B中央ゲタ & ふぅおっと & 162 \\
《D12》 & 04/05/25 & oldman1 & くわうんくわ & 151 \\
《D13》 & 04/06/25 & 協力B後ろゲタ & おっと危ない! & 154 \\
《D14》 & 04/06/25 & 協力B後ろゲタ & クンらりらんと & 204 \\
《D15》 & 04/07/25 & 協力B中央ゲタ & ドゥルン & 154 \\
《D16》 & 04/09/25 & 相撲B後ろゲタ & ふぅわああ & 125 \\
《D17》 & 04/11/25 & 相撲B中央ゲタ & アタタタタタ! & 133 \\
《D18》 & 04/13/25 & 協力B後ろゲタ & 逃げるな!こっちだ! & 99 \\
《D19》 & 04/13/25 & 協力B中央ゲタ & 引っ張ってドン! & 98 \\
《D20》 & 04/17/25 & oldman2 & んふあ!あんフ & 91 \\
\bottomrule
\label{table:hyojoentries_d}
\end{longtable}



A〜D4名の対象者が実践で残した表情エントリの量を以下に示す(\autoref{table:hyojoentrysnumber}・\autoref{table:naiseitoukei_subject})。
Aはoldman1・2・3の3つ、、BとCは対人運動4つ、Dは「oldman3以外」の6つの運動データでプレイをしている(\autoref{table:hyojoentrysnumber})。
なお、繰り返すが、D自身の動作は撮影していない。


\begin{table}[htbp]
  \centering
  \begin{minipage}[t]{0.52\textwidth}
    \raggedright
    \vspace{0pt}
    \caption{運動データごとの表情エントリの数}
    \label{table:hyojoentrysnumber} 
      \begin{tabular}{lrrrrr}
        \hline
        \bf 運動データ & \bf A & \bf B & \bf C & \bf D & \bf 合計 \\
        \hline
        oldman1 & 12 & 0 & 0 & 8 & 20 \\
        oldman2 & 2 & 0 & 0 & 1 & 3 \\
        oldman3 & 5 & 0 & 0 & 0 & 5 \\
        協力B後ろゲタ & 0 & 8 & 9 & 3 & 20 \\
        協力B中央ゲタ & 0 & 4 & 4 & 2 & 10 \\
        相撲B後ろゲタ & 0 & 7 & 4 & 4 & 15 \\
        相撲B中央ゲタ & 0 & 4 & 4 & 2 & 10 \\
        \hline
        合計 & 19 & 23 & 21 & 20 &  \\
        \hline
      \end{tabular}
  \end{minipage}  
  \hfill
  \begin{minipage}[t]{0.45\textwidth}
    \raggedright
    \vspace{0pt}
    \caption{対象者別の内省記述の統計}
    \label{table:naiseitoukei_subject}    
  % \ecaption{Statistics of self-reflection description by subject}  
    \begin{tabular}{lrrr}
      \hline
      \bf 対象者 & \bf エントリ数 & \bf 平均文字数 & \bf SD \\
      \hline
      A & 19 & 221.2 & 166.4 \\
      B & 23 & 192.5 & 119.7 \\
      C & 21 & 173.5 & 129.1 \\
      D & 20 & 139.7 & 50.8 \\
      \hline
      合計 & 83 &  &  \\
      \hline
    \end{tabular}
  \end{minipage}
  
\end{table}




% \begin{table}[htbp]
%   \centering
%   \caption{対象運動データ別の内省記述の統計}
%   \ecaption{Statistics of self-reflection description by target movement data}
%   \label{table:naiseitoukei}
%   \begin{tabular}{lrrr}
%     \hline
%     \bf 運動データ & \bf エントリ数 & \bf 内省記述平均文字数 & \bf SD \\
%     \hline
%     oldman1 & 20 & 236 & 145 \\
%     oldman2 &  3 & 163 &  56 \\
%     oldman3 &  5 &  81 & 126 \\
%     協力B後ろゲタ & 20 & 151 & 119 \\
%     協力B中央ゲタ & 10 & 200 & 123 \\
%     相撲B後ろゲタ & 15 & 173 & 115 \\
%     相撲B中央ゲタ & 10 & 183 &  94 \\
%     \hline
%     合計 & 83 & 182 & 124 \\
%     \hline
%   \end{tabular}
% \end{table}


\section{表情図形の2種類の媒介様式}
表情図形は、ユーザに身体運動の表情の感得と問い立てを促す媒体である。
実践で得られた表情エントリの内省記述を眺めてみると、表情図形は少なくとも2種類のしかたで媒介をしていることに気づく。

ひとつめは、表情図形を「図形」としてとらえつつ、元の動いている身体や学習ドメインの身体運動と関連づけながら、そこに身体感覚を現前させるしかたである。
前章で語った《A3》や《A19》(\autoref{fig:hyojofigurea3anda19})では、四角錐が元の動いている身体(の体内)に埋め込まれる存在となって身体感覚を駆動しており、
《A7》(\autoref{fig:hyojofigurea7})では長い線分が、元映像の自分の脚であるかのようでそうでない曖昧になり、「慌てる」感覚を得たりする。
% 《B3》(\autoref{fig:hyojonoteb3})の「三角形は相手を攻撃する」のように、図形の意志や行為といった擬人化をすることもある。
% このとき表情図形は、映像を「元の身体運動」としてみせている。

% TODO:四角錐の図
\begin{figure}[H]    
  \setlength{\tabcolsep}{2pt} % ← 横のスキマを詰める!
  \renewcommand{\arraystretch}{0.95} % ← 縦のスキマ(ラベルと画像間)
  % 上段
  \begin{center}
    \begin{tabular}{ccccc}
      % 初動 & 剣を振り上げ & 打ち込む \\
      \begin{overpic}[width=0.18\linewidth]{./images/hjplayground/A3b-1.pdf}
        \put(2,90){1}
      \end{overpic} &
      \hspace{-0.5em}
      \begin{overpic}[width=0.18\linewidth]{./images/hjplayground/A3b-2.pdf}
        \put(2,90){2}
      \end{overpic} &
      \hspace{-0.5em}
      \begin{overpic}[width=0.18\linewidth]{./images/hjplayground/A3b-3.pdf}
        \put(2,90){3}
      \end{overpic} &
      \hspace{-0.5em}      
      \begin{overpic}[width=0.18\linewidth]{./images/hjplayground/A3b-4.pdf}
        \put(2,90){4}
      \end{overpic} &    
      \hspace{-0.5em}
      \begin{overpic}[width=0.18\linewidth]{./images/hjplayground/A3b-5.pdf}
        \put(2,90){5}
      \end{overpic}
    \end{tabular}
  \end{center}
  
  \vspace{-0.5em}
  
  % 下段
  \begin{center}
    \begin{tabular}{ccccc}
      % 初動 & 剣を振り上げ & 打ち込む \\
      \begin{overpic}[width=0.18\linewidth]{./images/hjplayground/A19b-1.pdf}
        \put(2,90){1}
      \end{overpic} &
      \hspace{-0.5em}
      \begin{overpic}[width=0.18\linewidth]{./images/hjplayground/A19b-2.pdf}
        \put(2,90){2}
      \end{overpic} &
      \hspace{-0.5em}
      \begin{overpic}[width=0.18\linewidth]{./images/hjplayground/A19b-3.pdf}
        \put(2,90){3}
      \end{overpic} &      
      \hspace{-0.5em}
      \begin{overpic}[width=0.18\linewidth]{./images/hjplayground/A19b-4.pdf}
        \put(2,90){4}
      \end{overpic} &
      \hspace{-0.5em}    
      \begin{overpic}[width=0.18\linewidth]{./images/hjplayground/A19b-5.pdf}
        \put(2,90){5}
      \end{overpic}
    \end{tabular}
  \end{center}
      
  \caption{《ブふぁああーーー(A3)》と《へふいへ(A19)》の表情図形}          
  \label{fig:hyojofigurea3anda19}
\end{figure}


% TODO:ンーコツコツンーの図
\begin{figure}[H]
  \begin{continuousphoto}
  \begin{center}
    \begin{overpic}[width=0.18\columnwidth]{./images/hjplayground/A7b-1.pdf}
      \put(2,90){1}
    \end{overpic}
    \hspace{-0.5em}
    \begin{overpic}[width=0.18\columnwidth]{./images/hjplayground/A7b-2.pdf}
      \put(2,90){2}
    \end{overpic}
    \hspace{-0.5em}
    \begin{overpic}[width=0.18\columnwidth]{./images/hjplayground/A7b-3.pdf}
      \put(2,90){3}
    \end{overpic}
    \hspace{-0.5em}
    \begin{overpic}[width=0.18\columnwidth]{./images/hjplayground/A7b-4.pdf}
      \put(2,90){4}
    \end{overpic}
    \hspace{-0.5em}
    \begin{overpic}[width=0.18\columnwidth]{./images/hjplayground/A7b-5.pdf}
      \put(2,90){5}
    \end{overpic}                
  \end{center}
  \end{continuousphoto}
  \caption{《ンーコツコツンー(A7)》の表情図形}
  \label{fig:hyojofigurea7}
\end{figure}


ふたつめは、表情図形を、実践の身体運動とは異なるドメイン情景(心象風景)のできごと(の一部)になぞらえるしかたである。
前章の例で言えば、《A1》(\autoref{fig:hyojofigurea1})で「鉛筆を指先で軽くつまんで振る」情景や、
《B20》《B21》(\autoref{fig:hyojofigureb20})では「足首」に「剣道での打ち込み」の情景を立ち上げたのである。
前章の事例で言えば《A1》《A10》《B11》《B20》《B21》もそうである。
このようになぞらえることで、対象者らは自身の学習ドメインでの発見を生み出していた。
  \ref{subsec:caseAsummay}項と\ref{subsec:caseBsummary}項にて箇条書きしたとおりだが、
  たとえば\autoref{fig:hyojofigurea1}の表情図形にAは、「鉛筆を指先で軽くつまんで振るあの情景」を立ち上げながら、それは自身のoldmanの両肩ラインの不安定な揺れという気づきにもなった。
  \autoref{fig:hyojofigureb20andb21}の表情図形にBは、「剣道の打ち込みの情景」を立ち上げながら、それは自身の三段跳の助走(とくに一歩目)で「足首で剣を打ち込むように踏み込む」という新しい意識の仕方を発想していた。


\begin{figure}[H]
  \begin{continuousphoto}
  \begin{center}
    \begin{overpic}[width=0.240\columnwidth]{./images/hjplayground/A1-1.pdf}
      \put(2,90){1}
    \end{overpic}
    \hspace{-0.5em}
    \begin{overpic}[width=0.240\columnwidth]{./images/hjplayground/A1-2.pdf}
      \put(2,90){2}
    \end{overpic}
    \hspace{-0.5em}
    \begin{overpic}[width=0.240\columnwidth]{./images/hjplayground/A1-3.pdf}
      \put(2,90){3}
    \end{overpic}
    % \hspace{-0.5em}
    % \begin{overpic}[width=0.240\columnwidth]{./images/hjplayground/A3i-4.pdf}
    %   \put(2,90){4}
    % \end{overpic}    
  \end{center}
  \end{continuousphoto}
  \caption{《フリフリ(A10)》の表情図形}
  \label{fig:hyojofigurea1}
\end{figure}


\begin{figure}[H]    
  \setlength{\tabcolsep}{2pt} % ← 横のスキマを詰める!
  \renewcommand{\arraystretch}{0.95} % ← 縦のスキマ(ラベルと画像間)
  % 上段
  \begin{center}
    \begin{tabular}{cccc}
      % 初動 & 剣を振り上げ & 打ち込む \\
      \begin{overpic}[width=0.3\linewidth]{./images/hjplayground/B20-1.pdf}
        \put(2,90){}
      \end{overpic} &
      \begin{overpic}[width=0.3\linewidth]{./images/hjplayground/B20-2.pdf}
        \put(2,90){}
      \end{overpic} &
      \begin{overpic}[width=0.3\linewidth]{./images/hjplayground/B20-3.pdf}
        \put(2,90){}
      \end{overpic} &      
    \end{tabular}
  \end{center}
  
  \vspace{-0.5em}
  
  % 下段
  \begin{center}
  \begin{tabular}{cccc}
    \begin{overpic}[width=0.3\linewidth]{./images/hjplayground/B21-1.pdf}
      \put(2,90){}
    \end{overpic} &
    \begin{overpic}[width=0.3\linewidth]{./images/hjplayground/B21-2.pdf}
      \put(2,90){}
    \end{overpic} &
    \begin{overpic}[width=0.3\linewidth]{./images/hjplayground/B21-3.pdf}
      \put(2,90){}
    \end{overpic} &    
  \end{tabular}
  \end{center}
      
  \caption{《あいやっ(B20)》(上段)と《ぼぶわっ!(B21)》(下段)の表情図形}          
  \label{fig:hyojofigureb20andb21}
\end{figure}



このふたつめは、概念メタファ\cite{lakoff:1993}と似た構造をしていることに気づく。
たとえば「議論は戦争だ」と言えば、
喩え元となる「ソースドメイン(戦争)」に成り立つ諸関係性(敵陣への攻撃、防御不可能性、戦略、勝利、領地獲得など)を、
喩え先となる「ターゲットドメイン(議論)」に潜んでいた関係性として浮かび上がらせる。
概念メタファは、抽象的な概念(議論や愛など)を、より具体的で、身体的に把握できるものに喩える、ということが
文献\cite{lakoff:1993}が『Metaphors We Live By』と題されているように、
概念メタファはただのレトリック(修辞的な技法)なのではない。
身体をそなえて生きる私たちがもつ本源的な認知能力・様式である、というのがLakoffらの主張である。
メタファを引き出すということは、動いている身体の「表情」を感得をうながすという点で、本アプリ(表情図形)が果たす重要なはたらきであるように思われる。
『HJ-Playground』の「表情図形の抽象性」と「ユーザみずから作図する」という仕様の掛け合わせが、意味解釈を紡ごうとする学習者にとって、とりあえず(実践ドメインに限らず)なにかに見立てよう、とうながすことになった可能性がある。


\section{エントリからモチーフ語を抽出}
そこで、表情図形がどのようなソースドメインの拡がりを生んでいるのかを分析する。
そのためにまず本節では、各表情エントリに対して内省記述から、ソースドメインの主題となる名詞的概念:「モチーフ語」を抽出する。
抽出ルールを以下に示す。
\begin{itemize}
  \item Xみたいに/のように/に似ている/に見えるなどの比喩表現にともなってもしくは暗喩表現で登場する語はモチーフ語として抽出する。
  \item あるソースドメインに属する複数の名詞的概念をもちいてひとつの情景をつまびらかに描写している場合も、それらひとつひとつをモチーフ語として抽出する。
  \item 情景が語られながらもモチーフ語が記述のなかに不足している場合は、モチーフ語を補完する。
  % \item 「身体/身体部位」については、どういうケースであれ情景を描くのに必ず身体はかかわっているはずので、直接「身体/身体部位に喩えている」ケースを除き、モチーフ語として抽出しない。
\end{itemize}
事例で説明する。
《A1》の場合(\ref{subsec:a1}項の\autoref{fig:hyojonotea1}参照)「鉛筆を指で持って振るやつに似ている」という記述が情景描写の箇所である。
ここから「鉛筆」と「おもちゃ」をモチーフ語として抽出する。
この情景を描写するには、鉛筆だけではなく、「指でつまんで振る」という遊びの行為も盛り込む必要があるので、それに符号する名詞概念「おもちゃ」を補完する。
《B12》の場合「横隔膜」と「綱渡」という複数のソースドメインから名詞概念が登場するので、それぞれモチーフ語として抽出する。
このようにして、すべての表情エントリに対してモチーフ語を抽出する。

抽出の結果、対象者A〜Dが作成した全83エントリのうち57エントリがモチーフを有し、57エントリから計87語のモチーフ語が得られた。
対象者それぞれの「モチーフありエントリ数/全エントリ数」の内訳は、
Aは9/19、Bは15/23、Cは16/21、Dは17/20、となった。
運動学習の実践者であるAとBよりも、そうではないCとDのほうが、メタファをもちいるエントリ(モチーフ語をもつエントリ)の割合が高い。
% そこでAとB、CとDをそれぞれ合算し、A\&BとC\&Dではメタファあり/なしエントリの分布に有意な差があるかどうか、$\chi^2$検定により調べた。

% $\chi^2$検定の結果、有意差が認められた($\chi^2(1) = 4.23,\ p = .04$)(\autoref{fig:metaphor_bargraph})。
% この差に大きく寄与しているのは、C\&Dのメタファなしエントリ(1.83)とメタファありエントリ(1.78)である。
非運動学習者のほうが「実践に生かそう」という強い目的をもたぬゆえに、「比較的自由に」妄想をはたらかせることが表れた結果ではないかと著者は考察する。

\begin{figure}[H]
  \centering
  \includegraphics[width=0.8\textwidth]{./images/hjplayground/metaphor_bargraph.pdf}
  \caption{メタファあり/なしエントリの数(A\&B vs C\&D)}          %和文 caption
  % \ecaption{Number of Hyojo entries with/without metaphor (A \& B vs. C \& D)} %英文 caption
  \label{fig:metaphor_bargraph}
\end{figure}
 
\section{モチーフ語を比喩分類ツリーに位置付ける分析}
抽出したモチーフ語がどのような意味空間の拡がりをもっているのかを分析するために、比喩辞典\cite{nakamura:2023}をもちいた。
この比喩辞典では、約1600語に及ぶ比喩のソースとなる語が、【大分類(自然/人間/文化の3種)】・\{中分類(全17種)\}・(小分類(全93種))の3階層からなる分類ツリーに位置付けられている
\footnote{著者中村は、通常の比喩辞典がターゲットごとにソースの語を整理するのに対し、この辞書はソースの語ごとにターゲット用例を整理するのが特徴だと語る。
この辞書では、各ソースの語について、実際の文学作品での用例が複数掲載されている。}。

各エントリから抽出したモチーフ語を比喩辞典から探し、その語が属する【大】-\{中\}-(小)分類を、その表情エントリに割り当てた。
例えば、《A4》の「歯軋り」には【人間】-\{生理\}-(生理現象)、《B2》の「納豆」には 【文化】-\{産物・製品\}-(食)を割り当てた。
すべてのモチーフ語に必ずひとつの【大】-\{中\}-(小)分類を割り当てた。
モチーフ語が比喩辞典に掲載されていない場合は、分類ツリーのなかでそのモチーフ語がありうる場所に割り当てた。

モチーフ語が分類ツリーにおいてどう分布しているのか、サンバースト図で示す。
全体の図(\autoref{fig:sunburst_A})とA〜Dの図(\autoref{fig:sunburst_A2}、\{中分類\}まで)に分けて示している。
サンバースト図の見方を説明する。
中心から順に大・中・小の3階層を示し、それぞれ階層内に全分類を並べており、
各分類に当てはまるモチーフの度数を色合いで示した。
度数が0であればグレー、1以上であれば、度数が高いほどエリアの青色が濃くなる。

\begin{figure}[tb]
  \centering
  \includegraphics[width=\textwidth]{./images/hjplayground/sunburst_all.pdf}
  \caption{比喩分類ツリーにおけるモチーフ語の分布(全体)}          %和文 caption  
  \label{fig:sunburst_A}
\end{figure}

\begin{figure}[tb]
  \centering
  \includegraphics[width=\textwidth]{./images/hjplayground/sunburst_abcd.pdf}
  \caption{比喩分類ツリーにおけるモチーフ語の分布(対象者ごと)}          %和文 caption
  % \ecaption{Distribution of motif words in the metaphor classification tree (by subject)} %英文 caption
  \label{fig:sunburst_A2}
\end{figure}
全体としては、モチーフ語87語で分類ツリーの大分類3種全て・中分類13種(17種中)・小分類33種(93種中)をカバーした。
身体運動をもとにした抽象図形から、これだけ豊かな意味の拡がりがあるのは興味深い。
【大分類】でみると【文化】が突出しており、【人間】と【自然】はほぼ同数である。
【文化】では\{産物・製品\}が多く、そのなかで(日用品)と(武器)が多い。
(武器)の内訳はすべて対人一本ゲタ運動の表情エントリ由来である。
\{抽象\}や\{社会\}は0であるが、\{抽象\}は今回の抽象映像のデータでは原則モチーフにはならない。
\{社会\}は傘下の(小分類)からするに、身体運動からは遠すぎるのかもしれない。
【人間】では\{生理\}が多く、そのすべてが小分類\{生理現象\}に集約されている。
【自然】では\{動物\}が最多で、\{自然物\}と\{天文\}・\{気象\}が続き、\{小物・土地\}が少ないが、
今回のターゲットドメインが身体運動だということからすると、「動きのあるもの」が多くなったという可能性も考えられる。

対象者ごとのサンバーストグラフ(\autoref{fig:sunburst_A2})について述べる。
全体の分布と同じく【文化】が多いが、
なかでもAの分布は【人間】1語、【自然】6語、【文化】7語と特徴的である。
前節の結果もあわせると、Aはメタファなしのエントリも19個中9個と多く(つまり図形あるいは身体運動のドメインで語る)、
モチーフ語も自然のものが多い。このAとBCDの違いは、Aだけが、プレイした動作がすべて「ひとりでダンスをする動作」であったことが関係している可能性もある。
Bは【文化】【人間】【自然】の順に多い。
CとDは【大分類】3種に比較的満遍なくモチーフ語が分布しているが、中分類ではかなり分布が異なる。
Cの【自然】8語はすべて\{動物\}に位置するのに対し、Dの【自然】7語は満遍なく散らばる。
また、Cの【文化】は、ABDと比べて、\{産物・製品\}が少ないというのは特徴的である。
% A〜Dの間でモチーフ語の分類ツリーにおける分布に差があるかどうか、期待度数がある程度高い【大】のレベルで$\chi^2$検定をおこなったが、有意差はなかった
% ($\chi^2(6) = 8.08,\ p = .23$)。
