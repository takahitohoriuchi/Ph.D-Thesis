\chapter{分析}
\label{chapter:analysis}
\section{データ全貌}

前章では、AとBによるアプリをもちいた運動学習について、
すなわちAとBがどのような表情図形を作図し、それを媒介にいかなる表情を感得し、問いを深めたのかを語った。
本章では、アプリをもちいた実践の全体像を、A〜Dの表情エントリのデータから分析する。

まずはA〜D4名が実践で残した表情エントリの全一覧を以下に示す(\autoref{table:hyojoentries_a}〜\autoref{table:hyojoentries_d})。
なお表内の「文字数」とは、からだメタ認知によるからだメタ認知の文字数であり、
AとBのものについては、前章で載せた表の再掲である。

% Aの表情エントリ一覧
\begin{longtable}{llllr}
\caption{Aの作成した表情エントリ一覧} \\
\hline
表情エントリID & 作成日付 & 運動データ & 表情オノマトペ & からだメタ認知文字数\\
\hline
\endfirsthead
\hline
表情エントリID & 作成日付 & 運動データ & 表情オノマトペ & からだメタ認知文字数\\
\hline
\endhead
《A1》 & 03/25/25 & oldman1 & フリフリ & 393\\
《A2》 & 03/25/25 & oldman1 & ぐういペ & 512\\
《A3》 & 04/01/25 & oldman1 & ブふぁああーーー & 420\\
《A4》 & 04/01/25 & oldman1 & ガチョベチョガキョ & 494\\
《A5》 & 04/01/25 & oldman1 & バラらららららら & 212\\
《A6》 & 04/02/25 & oldman1 & なぬななぬな & 348 \\
《A7》 & 04/08/25 & oldman1 & ンーコツコツンー & 228 \\
《A8》 & 04/08/25 & oldman1 & グリグリ & 184 \\
《A9》 & 04/08/25 & oldman1 & フラフラぶんフラフラ& 328 \\
《A10》 & 04/13/25 & oldman1 & フサササフサ & 273\\
《A11》 & 04/13/25 & oldman1 & ブふぁああーーーペ & 23\\
《A12》 & 04/17/25 & oldman2 & うー、、わっダラララ & 122\\
《A13》 & 04/17/25 & oldman2 & ねーねねねね... & 200\\
《A14》 & 04/18/25 & oldman3 & mm & 18\\
《A15》 & 04/18/25 & oldman3 & 擬音が思いつかない、、 & 63\\
《A16》 & 04/18/25 & oldman3 & パラララららら & 11\\
《A17》 & 04/18/25 & oldman3 & 惑星 & 9\\
《A18》 & 04/19/25 & oldman1 & バラバラ & 60\\
《A19》 & 04/19/25 & oldman3 & へふいへ & 303\\
\hline
\label{table:hyojoentries_a}
\end{longtable}

% Bの表情エントリ一覧
\begin{longtable}{lllp{5cm}r}
\caption{Bの作成した表情エントリ一覧} \\ 
\toprule
表情エントリID & 作成日付 & 運動データ & 表情オノマトペ & 文字数 \\
\midrule
\endfirsthead
\toprule
表情エントリID & 作成日付 & 運動データ & 表情オノマトペ & 文字数 \\
\midrule
\endhead
《B1》 & 03/25/25 & 相撲B後ろゲタ & ほー・ふぐっ & 22 \\
《B2》 & 03/25/25 & 相撲B後ろゲタ & びよびよーー・ふぐっ & 353 \\
《B3》 & 03/25/25 & 相撲B後ろゲタ & ツン(+攻撃) & 332 \\
《B4》 & 03/25/25 & 相撲B後ろゲタ & お局orませた女の子(気が強そう) & 121 \\
《B5》 & 03/25/25 & 相撲B中央ゲタ & すぽーーん & 159 \\
《B6》 & 03/26/25 & 協力B後ろゲタ & ふん・ふん & 56 \\
《B7》 & 03/26/25 & 協力B後ろゲタ & びょーん・がちゃぐん & 86 \\
《B8》 & 04/08/25 & 相撲B後ろゲタ & んぽん(ビリヤード) & 167 \\
《B9》 & 04/08/25 & 相撲B後ろゲタ & ブンブン(んぽんのコピー) & 182 \\
《B10》 & 04/08/25 & 相撲B後ろゲタ & & 35 \\
《B11》 & 04/08/25 & 協力B中央ゲタ & プシュプシュぷわっ(気泡) & 215 \\
《B12》 & 04/08/25 & 協力B中央ゲタ & 横隔膜or綱渡 & 223 \\
《B13》 & 04/08/25 & 協力B中央ゲタ & ぬぬぬんんふわふわふわ & 197 \\
《B14》 & 04/08/25 & 協力B後ろゲタ & ぬぐむぐ(イモムシ はいはい) & 187 \\
《B15》 & 04/08/25 & 相撲B中央ゲタ & ギーコギコ & 270 \\
《B16》 & 04/08/25 & 相撲B中央ゲタ & うぉーーっとっとごめん & 268 \\
《B17》 & 04/08/25 & 相撲B中央ゲタ & ずんずんずんズコーー & 221 \\
《B18》 & 04/08/25 & 協力B後ろゲタ & わん、つ & 163 \\
《B19》 & 04/08/25 & 協力B中央ゲタ & ぱきっつつ & 100 \\
《B20》 & 04/09/25 & 協力B後ろゲタ & あいやっ & 185 \\
《B21》 & 04/09/25 & 協力B後ろゲタ & ぼぶわっ! & 428 \\
《B22》 & 04/09/25 & 協力B後ろゲタ & ディフェンス & 8 \\
《B23》 & 04/09/25 & 協力B後ろゲタ & うーみょううみょうんぶん & 450 \\
\bottomrule
\label{table:hyojoentries_b}
\end{longtable}

% Cの表情エントリ一覧
\begin{longtable}{llllr}
\caption{Cの作成した表情エントリ一覧} \\ 
\toprule
表情エントリID & 作成日付 & 運動データ & 表情オノマトペ & 文字数 \\
\midrule
\endfirsthead
\toprule
表情エントリID & 作成日付 & 運動データ & 表情オノマトペ & 文字数 \\
\midrule
\endhead
《C1》 & 03/25/25 & 相撲B後ろゲタ & ぐいーーーん & 313 \\
《C2》 & 03/25/25 & 相撲B中央ゲタ & ふえーん & 280 \\
《C3》 & 03/25/25 & 協力B後ろゲタ & すろりーす & 190 \\
《C4》 & 03/29/25 & 協力B中央ゲタ & ぼわあ 〜ん & 304 \\
《C5》 & 03/29/25 & 相撲B後ろゲタ & キリキリ & 296 \\
《C6》 & 04/02/25 & 相撲B後ろゲタ & グギーーーぎっぐ & 67 \\
《C7》 & 04/02/25 & 相撲B後ろゲタ & うーーーーーんしょ & 12 \\
《C8》 & 04/03/25 & 協力B中央ゲタ & すしゅんすシュンスゥイーーーん & 76 \\
《C9》 & 04/08/25 & 相撲B中央ゲタ & 何も生まれなかった & 24 \\
《C10》 & 04/08/25 & 相撲B中央ゲタ & 何も生まれなかったのコピー & 24 \\
《C11》 & 04/08/25 & 協力B後ろゲタ & よーいよぃよう & 181 \\
《C12》 & 04/08/25 & 協力B後ろゲタ & んがんがあ〜〜 & 101 \\
《C13》 & 04/09/25 & 協力B中央ゲタ & オフフォーっ & 495 \\
《C14》 & 04/09/25 & 協力B後ろゲタ & 一旦 & 0 \\
《C15》 & 04/09/25 & 協力B後ろゲタ & うぅーーーいおぉーーい & 279 \\
《C16》 & 04/09/25 & 協力B後ろゲタ & こそこそ & 208 \\
《C17》 & 04/09/25 & 協力B後ろゲタ & ういおい & 215 \\
《C18》 & 04/09/25 & 協力B後ろゲタ & ゆるゆる & 115 \\
《C19》 & 04/09/25 & 協力B後ろゲタ & ふんふんほいふんふんふんほい & 36 \\
《C20》 & 04/16/25 & 協力B中央ゲタ & ウェーーーい & 228 \\
《C21》 & 04/16/25 & 相撲B中央ゲタ & う〜〜〜〜〜 & 199 \\
\bottomrule
\label{table:hyojoentries_c}
\end{longtable}

% Dの表情エントリ一覧
\begin{longtable}{lllp{5cm}r}
\caption{Dの作成した表情エントリ一覧} \\
\toprule
表情エントリID & 作成日付 & 運動データ & 表情オノマトペ & 文字数 \\
\midrule
\endfirsthead
\toprule
表情エントリID & 作成日付 & 運動データ & 表情オノマトペ & 文字数 \\
\midrule
\endhead
《D1》 & 03/21/25 & oldman1 & するんりゅ & 51 \\
《D2》 & 03/21/25 & oldman1 & つゔぁーんグ & 85 \\
《D3》 & 03/21/25 & oldman1 & ひょいった & 168 \\
《D4》 & 03/21/25 & oldman1 & ひょいった のコピー & 168 \\
《D5》 & 03/21/25 & oldman1 & にゃるヴーんぱ & 98 \\
《D6》 & 03/21/25 & 相撲B後ろゲタ & はっふっは & 81 \\
《D7》 & 03/21/25 & 相撲B後ろゲタ & ギーコっとギコくわぁあ & 130 \\
《D8》 & 04/02/25 & oldman1 & フルマラソンのゴール直後、倒れそうになるも踏ん張る膝 & 250 \\
《D9》 & 04/02/25 & oldman1 & ふぶるんふぶるん & 225 \\
《D10》 & 04/03/25 & 相撲B後ろゲタ & ぐぐっグ & 166 \\
《D11》 & 04/03/25 & 相撲B中央ゲタ & ふぅおっと & 162 \\
《D12》 & 04/05/25 & oldman1 & くわうんくわ & 151 \\
《D13》 & 04/06/25 & 協力B後ろゲタ & おっと危ない! & 154 \\
《D14》 & 04/06/25 & 協力B後ろゲタ & クンらりらんと & 204 \\
《D15》 & 04/07/25 & 協力B中央ゲタ & ドゥルン & 154 \\
《D16》 & 04/09/25 & 相撲B後ろゲタ & ふぅわああ & 125 \\
《D17》 & 04/11/25 & 相撲B中央ゲタ & アタタタタタ! & 133 \\
《D18》 & 04/13/25 & 協力B後ろゲタ & 逃げるな!こっちだ! & 99 \\
《D19》 & 04/13/25 & 協力B中央ゲタ & 引っ張ってドン! & 98 \\
《D20》 & 04/17/25 & oldman2 & んふあ!あんフ & 91 \\
\bottomrule
\label{table:hyojoentries_d}
\end{longtable}



A〜D4名の対象者が実践で残した表情エントリの量を以下に示す(\autoref{table:hyojoentrysnumber}・\autoref{table:naiseitoukei_subject})。
Aはoldman1・2・3の3つ、、BとCは対人運動4つ、Dは「oldman3以外」の6つの運動データでプレイをしている(\autoref{table:hyojoentrysnumber})。
なお、繰り返すが、D自身の動作は撮影していない。


\begin{table}[htbp]
  \centering
  \begin{minipage}[t]{0.52\textwidth}
    \raggedright
    \vspace{0pt}
    \caption{運動データごとの表情エントリの数}
    \label{table:hyojoentrysnumber} 
      \begin{tabular}{lrrrrr}
        \hline
        \bf 運動データ & \bf A & \bf B & \bf C & \bf D & \bf 合計 \\
        \hline
        oldman1 & 12 & 0 & 0 & 8 & 20 \\
        oldman2 & 2 & 0 & 0 & 1 & 3 \\
        oldman3 & 5 & 0 & 0 & 0 & 5 \\
        協力B後ろゲタ & 0 & 8 & 9 & 3 & 20 \\
        協力B中央ゲタ & 0 & 4 & 4 & 2 & 10 \\
        相撲B後ろゲタ & 0 & 7 & 4 & 4 & 15 \\
        相撲B中央ゲタ & 0 & 4 & 4 & 2 & 10 \\
        \hline
        合計 & 19 & 23 & 21 & 20 &  \\
        \hline
      \end{tabular}
  \end{minipage}  
  \hfill
  \begin{minipage}[t]{0.45\textwidth}
    \raggedright
    \vspace{0pt}
    \caption{対象者別のからだメタ認知の統計}
    \label{table:naiseitoukei_subject}    
  % \ecaption{Statistics of self-reflection description by subject}  
    \begin{tabular}{lrrr}
      \hline
      \bf 対象者 & \bf エントリ数 & \bf 平均文字数 & \bf SD \\
      \hline
      A & 19 & 221.2 & 166.4 \\
      B & 23 & 192.5 & 119.7 \\
      C & 21 & 173.5 & 129.1 \\
      D & 20 & 139.7 & 50.8 \\
      \hline
      合計 & 83 &  &  \\
      \hline
    \end{tabular}
  \end{minipage}
  
\end{table}




% \begin{table}[htbp]
%   \centering
%   \caption{対象運動データ別のからだメタ認知の統計}
%   \ecaption{Statistics of self-reflection description by target movement data}
%   \label{table:naiseitoukei}
%   \begin{tabular}{lrrr}
%     \hline
%     \bf 運動データ & \bf エントリ数 & \bf からだメタ認知平均文字数 & \bf SD \\
%     \hline
%     oldman1 & 20 & 236 & 145 \\
%     oldman2 &  3 & 163 &  56 \\
%     oldman3 &  5 &  81 & 126 \\
%     協力B後ろゲタ & 20 & 151 & 119 \\
%     協力B中央ゲタ & 10 & 200 & 123 \\
%     相撲B後ろゲタ & 15 & 173 & 115 \\
%     相撲B中央ゲタ & 10 & 183 &  94 \\
%     \hline
%     合計 & 83 & 182 & 124 \\
%     \hline
%   \end{tabular}
% \end{table}



\section{表情エントリの派生関係}
自他の表情エントリ相互の影響・派生関係はあったか?
  厳密な判別は不可能だが、エントリ内の記述やインタビューから判明した範囲で述べる\footnote{
    HJPでは、エントリ相互の参照情報は保管されない。  
  }。
  Aの19エントリ中2つ以上は派生である。
  《ブふぁああーーーペ(A11)》は、《ブふぁああーーー(A3)》と《ぐういぺ(A2)》の2つの図形をAなりに足し引きして作った表情図形だという(図\ref{fig:hasei}左上)。
  このようにAは、実際に踊るときの感覚に近づけるためにHJPの仕様でやりくりしていた。
  また、《へふいへ(A19)》は《ブふぁああーーー(A3)》に軌跡をつけた図形である(図\ref{fig:hyojonotea3anda19})。
  その他Aは、自身の一作図方略(\ref{subsubsec:hefuihe}項)からして、
  それまで作った図形のバリエーションを意識するという点で、
  % \ridX{X4}
  エントリどうしの影響関係はあるだろうことを、インタビューで語った。
  Aは他者(D)のエントリからの影響は受けなかったという。

  Bは23エントリ中2つ以上は派生である。
  《ブンブン(んぽんのコピー)(B9)》が《んぽん(ビリヤード)(B8)》から(図\ref{fig:hasei}右上)、
  《ぼぶわっ!(B21)》が《あいやっ(B20)》から派生した(図\ref{fig:hyojonoteb20andb21})。
  《んがんがあ〜〜(C12)》は、Bの《ぬぐむぐ(イモムシ はいはい)(B14)》をコピーして編集した。
  その他、BはCのエントリ全体からの影響を語った。
  Bは当初、全身の「軸」を重要視するあまり、縦方向を意識しつつ複雑な図形をつくる態度でいた。
  いっぽうCは点を非表示にしたり、線が少なかったりなど、比較的シンプルな表情図形をつくっていた(例:《よーいよぃよう(C11)》)。
  それをみてBは、『そういうのもアリなんだな』と、作図のしかたのヒントを得たという。
  《ぬぐむぐ(イモムシ はいはい)(B14)》は、その発想から作ったものである。
  それを受けさらにCは《B14》をコピーしてその図形のまま《んがんがあ〜〜(C12)》を作成したのだった(図\ref{fig:hasei})。
  「学び合い」の一端と言える。

  \begin{figure}[H]
    \centering
    \includegraphics[width=\columnwidth]{./images/hjplayground/hasei.pdf}
    \caption{表情エントリどうしの顕著な派生事例}          %和文 caption    
    \label{fig:hasei}
  \end{figure}

\section{表情図形を情景に見立てることへの着目}
実践で得られた表情エントリを眺めてみると、抽象的な表情図形を、元の身体運動とは異なる情景として見立てているケースが散見される。
前章の例で言えば、《A1》(図\ref{fig:hyojonotea1})で「鉛筆を指先でつまんで振る」情景や、《B20》《B21》(図\ref{fig:hyojonoteb20andb21})では足首に「剣道での打ち込み」の情景を立ち上げたのである。
《A10》《B11》《B20》《B21》もそうである。

ここでメタファ理論\cite{lakoff_1993}\cite{lakoff_johnson}を参照しよう。
メタファでは「議論は戦争だ」というふうに喩える。
「議論」という抽象的な概念に対して、より具体的・身体経験的な「戦争」というドメインをあてがうことで、喩え元のドメイン(=戦争)に成り立つ諸関係性(敵陣への攻撃、防御不可能性、戦略、勝利、領地獲得など)が喩え先のドメイン(=議論)に転写され、「議論」のなんたるかを身体的に把握するのである。
このようにメタファは(単なる修辞技法ではなく)身体をそなえた私たちの原基的な認知様式なのだとLakoffらは論じる。

メタファ理論は、抽象的な表情図形に対して具体的なドメインを動員して表情感得しようとすることを説明しうるものである。
B.Munariのデザイン教本『空想旅行』\cite{munari:1992}でも、あるランダムな点群の布置に、線を様々なスタイルで描きくわえて「ふたりひとくみ」や「音符」や「摩天楼」や「発芽」など、様々に点群の関係づけて見立てることの重要性を示している。

そこで、見立てという観点から本実践のデータを眺め、表情図形はどの程度/どのようなドメインからの見立てを促したのか、その全体像を調査する。

\section{モチーフ語を抽出する}
からだメタ認知記述から、以下のルールで、見立ての元である名詞的概念:「モチーフ語」を抽出した。
  \begin{itemize}
  \item Xみたいに/のように/に似ている/に見えるなどの比喩表現にともなってもしくは暗喩表現で登場する語はモチーフ語として抽出する。
  \item あるソースドメインに属する複数の名詞的概念をもちいひとつの情景を詳細に描写している場合も、ひとつ一つをモチーフ語として抽出する。
  % \item 情景が語られながらもモチーフ語が記述のなかに不足している場合は、モチーフ語を補完する。
  \end{itemize}
  事例で説明する。
  《A1》の場合(図\ref{fig:hyojonotea1})「鉛筆を指で持って振るやつに似ている」という記述が情景描写の箇所である。
  ここから「鉛筆」を抽出する。
  《A1》では鉛筆を奇抜な使い方をする情景ではあるが、使い方にかかわらず「鉛筆」を抽出する
  \footnote{    
    《B1》の例を抽出ルールを補足しておく。
    仮に、「〇〇に似ている」の〇〇が    
    「棒」ならば棒がモチーフ、
    「野球」ならば野球(という娯楽・文化)がモチーフとなる。
  }。
  % 《B11》の場合(図\ref{fig:Hyojonoteb11})、「泡」と「細胞」
  《B12》の場合、図形を「横隔膜」「綱渡をする人」「バネ」という三重に喩えているのだが、これら3つともモチーフ語として抽出する。
  このようにして、すべての表情エントリに対しモチーフ語を抽出した。

  \section{モチーフ語の抽出結果}
  モチーフ語抽出の結果、A〜Dが作成した全83エントリ中57エントリがモチーフ語を有し、57エントリから計87語のモチーフ語を抽出した。
  対象者それぞれの「モチーフありエントリ数/全エントリ数」の内訳は、Aは9/19、Bは15/23、Cは16/21、Dは17/20、となった。
  これを運動学習者(AとB)と非運動学習(CとD)にまとめると、
  運動学習者群では42エントリ中24エントリ(57.1\%)で、非運動学習者群では41エントリ中33エントリ(80.5\%)でモチーフ語が使用された。
  非運動学習者群でより高いモチーフ語使用傾向が観察されたが、被験者数が限られているため(N=4)、統計的検定は行わず記述統計として報告した。
  非運動学習者のほうが「学びに活かそう」という強い目的をもたぬゆえに、「比較的自由に」妄想をはたらかせやすいのかもしれない。

  \section{モチーフ語をドメインツリーに位置付ける}
  表情図形のモチーフが属するドメインを調査するために、比喩辞典\cite{nakamura:2023}をもちいる方法を考案した。
この辞典では、約1600語に及ぶ比喩の喩え元となる語が、【大分類(自然/人間/文化の3種)】・\{中分類(全17種)\}・(小分類(全93種))の3階層からなるドメイン分類ツリーに体系的に位置付けられている。
各エントリから抽出した各モチーフ語をこの比喩辞典から探し、その語が属する【大】-\{中\}-(小)ドメインに位置付けた。
例えば、《A4》の「歯軋り」は【人間】-\{生理\}-(生理現象)、《B2》の「納豆」は 【文化】-\{産物・製品\}-(食)に位置付けた。
すべてのモチーフ語を必ずひとつの【大】-\{中\}-(小)ドメインに位置付けた。
モチーフ語が比喩辞典に存在しない場合は、そのモチーフ語がありうるツリーの枝に位置付けた。

\section{ドメインツリーにおけるモチーフ語分布}
% sunburst_all 2
全モチーフ語をツリーに位置付けた結果を、網羅性を把握しやすいようサンバースト図(図\ref{fig:sunburst_A})で示す。
  図は、中心から順に大・中の2階層を示し、各階層内に全ドメインを並べ、
  各ドメインは度数が0ならグレー、1以上なら青で示した(度数が高いほど濃い青)。
  本分析では詳細な度数よりも、そのドメインのモチーフがもちいられたか否か(青かグレーか)を重視する。
  表情感得は本質的に一回性のある現象だという前提に立ち、表情図形のもつ多様性を積極的に拾うためである。
  ドメインの占める角度に意味はない。
  該当ドメインのモチーフ語を有する《エントリID》をドメイン内に付置した(Aは赤、Bは緑、Cは青、Dは黄)。
  ドメイン内に付置されたエントリの数とモチーフ語数が一致しないケースは、
  そのエントリが当該ドメインのモチーフ語を複数有することを意味する。

  \begin{figure}[tb]
    \centering
    \includegraphics[width=\columnwidth]{./images/hjplayground/sunburst_all2.pdf}
    \caption{比喩ドメインツリーにおけるモチーフ語の分布}          %和文 caption
    \ecaption{Distribution of motif words in the metaphor domain tree} %英文 caption
    \label{fig:sunburst_A}
  \end{figure}

  % \begin{figure}[tb]
  %   \centering
  %   \includegraphics[width=\columnwidth]{./images/sunburst_abcd.pdf}
  %   \caption{比喩ドメインツリーにおけるモチーフ語の分布(対象者毎)}          %和文 caption
  %   \ecaption{Distribution of motif words in the domain tree (by participant)} %英文 caption
  %   \label{fig:sunburst_A2}
  % \end{figure}

  全体としては(図\ref{fig:sunburst_A})、モチーフ語87語でツリーの大分類3種全て・中分類13種(17種中)・小分類33種(93種中)をカバーした。
  【大分類】でみると、【文化】が突出しており【人間】と【自然】はほぼ同数である。
  \{中分類\}では不足ドメインが\{物象\}\{感性\}\{社会\}\{抽象\}の4種あったが、
  \{抽象\}は本実践ではむしろ「喩え先」に相当するからかモチーフ語としては登場しなかった。
  少なくとも\{社会\}ついては小分類(政治・経済・法律)と(教育)が属する)身体運動から遠すぎるということも関係していよう。

  エントリの付置にも着目しよう。  
  A〜D全員、【大分類】では3種すべてを網羅しているが、その度数や、
  \{中分類\}でのカバーするドメインには、それぞれ違いが表れる。  
  Aのエントリは【自然】と【文化】に偏り(【人間】は《A4》のみ)、
  Bは【文化】に偏る。
  C・Dでは大分類3種が満遍なく分布する。
  A〜Dの個人特性とHJPとの関係性、プレイした対象動作の違いが、こうした違いを生んでいるのだろう。

  \section{考察}
表情図形は、身体部位どうしの関係性や身体感覚への志向を促しただけでなく、
「見立て」をも促した。
2種の身体運動から4人が創った計83エントリ中57個で見立てが起きた。
見立て元のドメインは、メタファ喩え元辞典の中分類17種中13種をカバーした。
興味深い結果である。

身体知の学びの基本思想とメタファ理論とに照らせば、
この結果は、表情図形作図の果たすひとつの役割を示唆する。
元の身体運動というドメインに縛られずに、
本人の日常を生きる全体経験から一部を動員し、その異ドメインに成り立つ関係性を図形に転写しながら、身体運動を問うことを促す、という役割である。
この結果は、身体知の学びが本人の「生きている」ことと地続きであるという本研究の基本思想とも符号する。
「だからこそ抽象的な表情図形は、通常の映像よりもかえって『ありあり』と立ち現れてくるのではないか?」という考えも著者には浮かんでくる。

AもBも、実践後の学びでも「図形的な意識」が癖づいていた。
ここにもHJPの価値が垣間見える。
ツールは使用中に便利なだけではなく、
使うなかでひとの身体知は進化する。
進化したならば、同ツールに頼る必要はない。
「卒業」である。
ぬか床ロボット「Nukabot」をデザインしたドミニクはインタビュー\cite{nakanishi_2024}にて「卒業するツール」の考えを展開する。
HJPも卒業まで伴走してくれるツールなのかもしれない。

AとBは、それぞれの問題意識や方略に基づいて、自分なりにHJPを駆使し(図形を作図)、問いを紡いでいた。
では、HJPはどういう学習者にとって有用なのか?
著者は、万人に有用であるとまでは主張しない。
諏訪による「見立て(構成的知覚)」能力の向上の研究\cite{suwa_2004}を参照しよう。
対象者ら(大学生)は9ヶ月間、写真や空間や場に対してそこにある要素や位置関係についてからだメタ認知する研鑽を続けた。
鍛錬の前後で、曖昧図形解釈課題(与えられた曖昧図形を何通りに見立てられるかという解釈数をテスト)を実施すると、解釈数は1.6倍に増加し、この増加率は、研鑽をしなかった統制群と比較して有意に高かったという。
この結果が支持するのは、からだメタ認知の鍛錬が見立ての能力向上に寄与するということである。

本実践で観測された多彩な「見立て」による表情感得は、表情図形(の作図)が可能にしたことなのか、それともからだメタ認知だけでも可能だったのか?
本研究の結果からは、明確な線引きはできない。
しかしながら、AとBがその両方を駆使し、見立てにおいては「抽象的図形を」見立て、表情感得して問いを紡いだことは揺るぎない事実である。
少なくともからだメタ認知をする学習者にとってはHJPが有用であることを支持する結果が得られたと、著者は考える。

本研究は、表情図形の詳細な作図プロセスを扱わなかったが、
作図プロセスを細かく調べれば、さらなる表情感得の様相や図形と言葉の共創様態に迫れる可能性がある。
また、HJPでのユーザどうしでの「学び合い」についても、本実践では十分に促せなかったが、実践的検討の余地は残されている。