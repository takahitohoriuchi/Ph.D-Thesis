\part{動いている身体の「表情」の感得をうながすアプリのデザイン実践}

\chapter{背景と問題意識}
第二部は、運動学習の身体知創造プロセスを支援するツールをデザインする研究である。
動いている身体が醸し出す「表情」を感得することを運動学習者に促すツールである。
まずは本章にて、私が「表情」に出会ういきさつ、「表情」とはなんなのか、「表情」が身体知創造プロセスとどう関係するものなのかを述べることで、
第二部研究の問題意識を展開する。


\section{問題意識の醸成}
\label{sec:tetsugakujousei}
本節では、
\textcolor{red}{第一部の実践を終えた私が(修士論文としてまとめた私が)、}
第二部の研究に至るようになったのかを示す。

\subsection{身体感覚を触発するおもちゃを試作する}
\label{subsec:toy}
私は博士過程に入ってからも、自分の身体感覚をつつき、ことばを触発するようなメディアの試作を続けていた。
% たとえば、MRIをつかって、生成()Ultra-Regeneration of MRI\cite{horiuchi_suwa:2018}は、MRIで撮影した人体の横断面\footnote{人体を仮にきゅうりに見立てたときの「輪切り」の面である}
% の断面図動画を素材にもちいる映像型のおもちゃである\footnote{
%   制作には、openFrameworksをもちいた。openFrameworksはProcessingに似たプログラミング言語(C++言語のソフトウェアフレームワーク)であり
%   グラフィックス、オーディオ、PC外部の各種センサやアクチュエータなど、様々な技術どうしをつなげたものを制作することができる。
%   Processingよりも高速な処理が可能であり、主にメディアアートの分野でもちいられている。  
% }。
% PC画面内で、MRI動画をふたたび横断面方向に重ねることで「人体を再生」したうえで、画面上

たとえば、骨盤の「傾き」と「回転軸」\footnote{
  「軸」という言葉そのものは、武術やスポーツでしきりに意味深げに語られる概念でもある。
}をリアルタイムに映像可視化するインスタレーション型ツールを、他の陸上競技実践者とともにデザインすることに試みたりもした\cite{horiuchi_suwa:2020b}。
ユーザは加速度センサ搭載のベルト(\autoref{fig:jikubelt})を腰に巻いて運動すると、
(それがPCにUDP通信でリアルタイムに送信され、)
目の前のスクリーンに、骨盤の3Dモデルによって骨盤の傾きと回転軸が可視化される(\autoref{fig:jikuscreen})。
\autoref{fig:jikuplay}はユーザがプレイするようすである。
立った状態で右脚をゆっくりももあげする動作で試したとき、2種類の意識のしかたでは、
回転軸のふるまいがまったく異なることを私たちは発見した(\autoref{fig:jikuplay})。

\begin{figure}[H]
\centering
\begin{minipage}[b]{0.45\linewidth}
\centering
\includegraphics[width=\linewidth]{./images/JIKU/belt.pdf}
\caption{加速度センサ搭載ベルト}
\label{fig:jikubelt}
\end{minipage}
\hspace{0.01\linewidth}%画像間の余白
\begin{minipage}[b]{0.53\linewidth}
\centering
\includegraphics[width=\linewidth]{./images/JIKU/jikuscreen.pdf}
\caption{映像の内容(黄色が回転軸)}
\label{fig:jikuscreen}
\end{minipage}
\end{figure}

\begin{figure}[h]
  \centering
  \includegraphics[width=\textwidth]{./images/JIKU/junkun.pdf}
  \caption{ユーザがプレイするようす(上段と下段は2種類の意識で動きを比較した)}
  \label{fig:jikuplay}
\end{figure}

ほかにも私は、自らの「足」の模型を作り、それとともに暮らしてみることもした。
模型は、足を3Dスキャンしたのち、それを3Dプリンタで出力した硬質樹脂製のもの(\autoref{fig:foot_tate}・\autoref{fig:foot_yoko})
と、樹脂製モデルで型取りしてそこに軟質樹脂を流し込んで固めたもの(\autoref{fig:foot_yawa})を作った。
扁平足である自身の足裏だが、
通常の生活では、自分の足裏が天を向いているのを眺める、ということもあまりない。
足裏が扁平足なりにちゃんと「地形」的になっており、それを握ってみたり、指の腹で実際になぞってみたりして、
特異なしかたで、自らの足裏の感覚をつつくような経験であった。

\begin{figure}[H]

\centering
  \begin{minipage}[b]{0.3\linewidth}
    \centering
    \includegraphics[width=\linewidth]{./images/footmake/foot_tate.pdf}
    \caption{硬質樹脂性の私の足模型を握ってみる}
    \label{fig:foot_tate}
  \end{minipage}
  \hspace{0.01\linewidth}%画像間の余白
  \begin{minipage}[b]{0.3\linewidth}
    \centering
    \includegraphics[width=\linewidth]{./images/footmake/foot_yoko.pdf}
    \caption{硬質樹脂性の私の足模型を上向きに机の上に置く}
    \label{fig:foot_yoko}
  \end{minipage}
  \hspace{0.01\linewidth}%画像間の余白
  \begin{minipage}[b]{0.3\linewidth}
    \centering
    \includegraphics[width=\linewidth]{./images/footmake/foot_yawa.pdf}
    \caption{軟質樹脂性の私の足模型を指でなぞる}
    \label{fig:foot_yawa}
  \end{minipage}
\end{figure}

なお、
% 修士終了後の
\textcolor{red}{2017年シーズン終了後の}
私は、十種競技からは引退し\footnote{
  より精密にいえば、博士課程1年目にふたたび大学陸上部に所属して現役復帰したのだが、1シーズンで競技者としては引退することになった。
}、競技中心の生活は送らなくなり、たっぷりと時間をつかって練習したり、ベスト記録をめざして競技会に出場する機会はほぼなくなったことは述べておく。
だが相変わらず、空いた時間をつかって走ったり身体運動をしてより望ましい動きの試行錯誤は続けていた。
そんななかで私はこうした「おもちゃ」によって、身体運動の「なにか」に迫ろうとしていたのである。

当時の私はこうしたメディアを「トイ(おもちゃ)」と仮称していた\footnote{
  「トイ」はで本研究のキーワードである「身体で問う」の「問い」とダブルミーニングである(つまり、「\ruby{トイ}{問い}」)。
}\cite{horiuchi_suwa:2018b}。
呼び方にはこだわりがあった。
「ツール」と呼ぶとどこか合目的的な響きをもってしまうと、当時の私は感じていた。
あたかも、身体運動のみるべき変数があらかじめ決まっていてその変数の状態を厳密に評価するためのメディアである、というようにである。
私がもくろんでいたのはそういうメディアのありかたではなく、
とりあえずメディアと「戯れてみる」ことによって、打算を超えたしかたで身体を触発して、新しい感覚やことばを生み出そうとするものだった。
それが「トイ」である。

\subsection{哲学的な思惟を抱く:身体知輻輳性の提唱}
\label{subsec:fukusousei}
身体運動をめぐる私の思惟は、以前よりも哲学的(現象学的)な傾向をもつようになっていた。
こうしたは傾向は、
% 修士2年時に
\textcolor{red}{2017年シーズンに}
積極的・自覚的に「アスリートとして生きる」ようになっていったころ(第一部の\ref{sec:tatuaruku}あたり)からであった。
そんな私は、学習者が積極的に「アスリートとして生きる」のが良いのは、身体知に\textbf{輻輳性(confluential structure)}があるからだ、と提唱した\cite{horiuchi_suwa:2019b}。

輻輳性とはなにか、説明しよう。
\textcolor{red}{
そもそも「\ruby{輻湊}{ふくそう}」とはどういう意味のことばか。
輻輳の語源は、車輪において\ruby{輻}{や}(=スポーク)が、中心軸の(\ruby{轂}{こしき})に集中する様子からきており、ものごとが四方からひとところに寄り集まっているようすをいう。
現代でも、交通や通信が一部へ集中し混雑することを輻輳と呼ぶ。
}

「身体運動の名前」と「その身体運動の意味」という対応関係を考えたときに、私たちはつい技名と意味との一対一対応で考えてしまう。
野球のバッティングとサッカーのシュートは「異なる」とみなしがちだろうし、また、AさんのバッティングとBさんのバッティングは運動名が同じだから「同じ」運動だとしがちだろう。
2つのバッティングが「同じ」運動だとしたうえで、両運動のあいだにどういう差異があるのかを見極めようとするのが、
運動学習者であれ、スポーツ科学の研究者であれ、人々の自然的な傾向であろうと私は考える。
輻輳性はその考え方を否定する。

たしかに、「名前」のほうは体系的に整理されていたり、技どうしの区分けもはっきりしている。
しかし「意味」のほうもそうなのか?そんなに意味どうしがきっちりと棲み分けられているものだろうか?
そんなことはないだろう。
それが証拠に、スキルの「転移」という現象がある。
バッティングを例にとれば、バッティングで学んだことがゴルフスイングにも影響する、というのがスキルの転移である。
上記の一対一対応関係が否定されるのである。


各種運動名は分かれていても、そのそれぞれに対応する「意味」はそんなに分かれていないのだとすると、それは、
運動の名前レベルでは運動名(あるいは名指されたひとつ一つの運動体験)に整理整頓され象られているそれらが、
意味レベルではひとつに「輻輳」している、という構図になる。
運動学習者はたったひとつの身体でもって人生を生き、生活も競技もひっくるめたさまざまな身体運動をその身でこなしてきている。
いろいろなシーンで体験してきた多種多様なスキルは、言うなれば「ひとつの身体(という場)」に輻輳して、身体知をなしているのである。
もともとスキルが輻輳してあるのならば、「転移」という表現も実は奇妙である。
転移という表現をもちだすことになるのは、「名前ありき」でとらえる見方をしてしまっているせいである。
「輻輳ありき」ならば、その「輻輳したひとつの塊」に、運動名によって分節化しただけである。
分節化によって際立つものごとはあるにせよ、ひとつの塊をそれぞれ別角度から眺めたようなスキルAとBに影響関係がみてとれるのは当たり前である。
これが輻輳性の考え方である。

輻輳性の考え方に拠れば、以下のことも成り立ちうる。
Aさんにとって、Aさんの「走り」とBさんの「走り」とでは、同じ動作名がついていても、現実的にはまったく異なる意味をもつ。
あるいは、Aさんにとって、Aさんの「歩き」とAさんの「走り」は、動作名は異なっても、ほぼ同じ意味をもったりもする。

私が実践者として日常生活と競技を積極的に融合させていったのは(第一部\ref{sec:tatuaruku}〜\ref{sec:monowotool}節)、
「意味」の輻輳的なありかたを認めたうえで、
運動名(ラベル)にはとらわれず(だが参考にはしながら)に、
身体の声に耳を傾けながらスキルを根本的なとらえなおしをはかったということになる。
私が「リュックを腹負う」(第一部\ref{sec:monowotool}節参照)とわざわざ技として命名していたのも、そういう態度の表れとも解釈できる。
そうやって、生きる全体から「意味」を「図」として浮かび上がらせようとしたのだろうと私は解釈する。

私にとって輻輳性の考え方は、運動名とその意味という対応関係の構図でとらえるときに、その構図の内側から構図そのものを超克しうるような概念装置となった。
見てわかるとおり、輻輳性は現象学の考え方と類似している。
現象学をよく知らなかった当時の私は、
自らの運動学習の実践からなかば自力で現象学の基本的な考え方を醸成したのである。
トイによって身体を触発しようとしていた(\ref{subsec:toy}項)のも、
方向性としては、こういう現象学的なありかたを欲していたのだと、私は確信している。

\subsection{「表情」と出会う}
そうして、哲学書をも読み漁るようになっていた私は、
ある書物との衝撃的な出会いを果たす。
哲学者・廣松渉の著作「表情」\cite{hiromatsu:1989}である。
「表情」は(次節で詳述するが)意味の源たる現象である。
私が実践者かつ研究者としてつかもうとしていた現象の正体こそ「表情」なのではないか?
そう思えてならなかった。
というのも、自分や他者の身体運動を観ながらなにかを学びとろうとするときに、
私はそこ(観測対象の身体運動)に、まずもって「生々しいなにか」を感得しかけるのである。
だが「ことば」の力だけを借りてそれと関わってみても、なかなかしっくりくる共創が巻き起こせていない、という実感があった。
運動学習者は、自他の動いている身体が醸し出す「表情」を豊かに感得できるのが良いのだと、
私は仮説を生成した。


こうして、「表情」という概念を手に入れた私は、
運動学習者が、動いている身体の「表情」を感得することを促すツールを制作することにした。
このように、私が実践者かつ研究者として一人称研究を遂行し、学んできたからこそ、
「表情」に着眼することになったわけである。
本節ではそのいきさつを簡単に示した。

\section{「表情」}
\label{sec:hyojo}
\subsection{表情論導入}
「表情」とはなんなのか。
E.カッシーラー\cite{cassirer:1929}(1874-1945)が記号論の文脈で論じ、
M.メルロ=ポンティ\cite{merleau-ponty:1945}(1908-1961)はそれを身体的な現象学に引き継ぐ形で、
Ausdruck(ドイツ語)・Expression(英語)を論じた。
Expressionは日本語だと「表情/表現」という二重の意味をもつ。
廣松渉(1933-1994)はExpressionに「表情」という日本語訳のほうをあて、それらを引き継ぎながら、独自の表情論を展開した。

廣松は以下のように書いている。
\begin{quotation}
  風景に眼を向けて見よう。われわれの日常如実の体験相においては、いま例えば、
  「いま裏山の松の樹はガッシリとしているが大枝はノタウッテいる。
  崖にかけて淡竹がスクスクと伸びており、刃先はピンと張っている。
  ・・・小川はサラサラと流れ、魚はスイスイと泳いでいる。
  雪がヒラヒラと舞い始め、やがてシズシズと降りしきる。
  松はコンモリと雪帽子を被り、いよいよドッシリと落ち付いて見える。
  一陣の風がサッと捲き起こり、雪がパッと散る。が、松はカタクナに立っている。
  竹はタワワに軋み、雀がピョンピョンと枝渡りすると、ドタドタと雪が零れる。
  夕陽がノンビリと傾き、月影がソッと忍び寄って来る・・・。」

  環界的情景は、表情性に満ち充ちている。\\
  (\cite{hiromatsu:1989}, p.9)
\end{quotation}
表情におよそ相当するのは、カタカナで表記された、オノマトペを主とした述語的部分である。
廣松はこのように、「表情」を言葉の問題にむすびつけながら考究している。
末尾にある「環界的情景」とは、(私たちひとの生きる)環世界\cite{uexküll:1934}のことである。
ユクスキュルの環世界\cite{uexkull:1934}は、それぞれの生物が客観的には同じ環境に存在していても独自な意味世界を生きていることを主張するものである(\ref{subsec:umwelt}節)。
廣松の「表情」は、とくに人間の環世界に着目し、人間の環世界は無味乾燥で殺風景なものではなく、
意味に満ち充ちた生々しいものである、ということを強調するものだと解釈できよう。
生々しさとは、オノマトペに代表されるように「身体感覚的」に把握されるものである。
廣松は続ける。

\begin{quotation}
  直接的な体験意識に即するとき、事物(というものが在って、それ)が表情性を帯びている、 
  という表現方式は実態には合わない。
  右の文章では、松がグネグネしているとか、淡竹がスクスク伸びているとか、
  事物的分節体が表情性を呈するかのような表現方式になっているが、
  原基的にはむしろ、グネグネしているあれ、スクスク伸びているこれの覚知が先であって、
  その覚知与件が松・竹として事物的に認知・命名されるというのが実情であろう。 \\
  (\cite{hiromatsu:1989}, pp.9-10)
\end{quotation}
つまり、私たちに生きられた世界(環世界)は、あらかじめすでに、「表情」に満ちている。
私たちの常識からすると、(1)「竹」という物がまずあって、(2)私がそれを認識する、という二段構えで理解される。
いわゆる物心二元論的な構図である。
二元論的構図で描くかぎり、世界は殺風景になってしまう。
「グネグネ」は、この構図には描けない体験である。
グネグネは、殺風景な世界にはないし、「各自が心に生成した、二次的で私秘的なもの」かといえば、そうではない。
廣松は「表情」という概念をもちこむことで、この構図そのものを覆すのである。
私たちは(1)「グネグネ」といった表情こそをまず先に感得しており、
そのあと事後的に、(2)その表情を「竹」といった事物的なものとして捉えている(捉えなおす)だけだ、
と廣松はいうのである。
表情感得こそが、私たちの「生まの体験」だと廣松は説いている。

\begin{quote}
  「純然たる知覚現相」などというものは如実には存在せず、如実の現相はその都度すでに“情意的な\ruby{契機}{モメント}を孕んで”おり、本源的に表情的である。
  より正確に言えば、如実の\ruby{環境世界}{ウムヴェルト}的現相は本源的に情動的価値性を“懐胎”せる表情性現相である。
  従って、表情性現相は汎通的である。(p.17)
\end{quote}


菅野\cite{sugeno:1998}も、表情を環世界に絡めて説明する。
「記号の精神からの音楽の誕生」と題する論考のなかで菅野は、
「生物は環世界の事物に対して問いを発し、これに事物が応答するという、問いと答えの応酬のさなかから、事物の意味が立ち現れる(\cite{sugeno:1998}, p.134)」
と説明したうえで、環世界の事物がもつ「意味のトーン」が「表情」なのだと論じている。
こうしたことからも著者は、「表情」感得が、意味づくりの源になるのではないか?と考える。
菅野が考察の対象にする「トーン」という概念は、ユクスキュルの環世界論に由来する。
ユクスキュルは、トーンという概念\cite{uexkull:1934}を、主体の気分によって対象のもつ意味(どういう行為でかかわるか)が変わる、というニュアンスでもちいている。
ヤドカリにとってのイソギンチャクの知覚像は、気分に応じて変わる(異なるトーンを帯びる)。
自分の家にイソギンチャクをつけているヤドカリにとっては「保護」のトーン、
家を失ったときには「居住」のトーン、
腹が減ったときには「餌食」のトーンを、イソギンチャクは帯びるのである。 
「表情」は、観察主体の気分に応じて変わりうるものであると考えられる。


「表情」は顔のそれ(表情)を万物のそれへと拡張した概念だ、ととらえてみるとわかりやすい。
私たちは、相手の顔面に表情をみずにはいられない。
「表情抜きの物的顔面の動き」だけをみてとることは逆にむずかしい。
顔面は物的パーツの集合体であるとわかっていてもなお、
その動きに表情をみてしまう。
よもや、相手の目鼻口といったパーツのひとつひとつを別々に認識したのちそれらを「合算」して「ということは、このひとは喜んでいるのだな」と計算しているわけでもあるまい。
私たちはじかに、相手の動く顔の全体としての表情を感得しまえているだろう。
それが、身ぶり手振りを交わす滑らかなコミュニケーションの成立にも寄与しているのだろうし、
その逆も然りであろう。
「表情」はそういう刺激の単純な総和ではなく、
「ゲシュタルト的全一態(\cite{hiromatsu:1989}, p.75)」で感得されるものである。
廣松がExpressionの訳語に「表情」という日本語を採用したのも、「表現」よりも「表情」と呼ぶほうが、上記してきた表情論の問題を顕著にできるからである。

\subsection{表情の中身}
表情の中身(構成)をもう少しくわしくみておこう。
廣松\cite{hiromatsu:1989}は表情感得がゲシュタルト的であることを前提しつつも、だがそのうえであえて下位的に分解的に説明できるともいう。
そのように「表情」を説明した文言も、\cite{hiromatsu:1989}には散見される。
以下に示す。
\begin{quote}
  知覚相・情緒価・即応価の“融合態”(p.78。太字は著者による)
\end{quote}
\begin{quote}
  表情性感得は、知覚的認知と感情的興発と反応的態勢との融合的感受である(p.77)
\end{quote}
\begin{quote}
  表情感得とは、情緒価と即応価とを\ruby{内自化}{イン・ジッヒ}\footnote{}せる知覚的現認(p.79)
\end{quote}

\textbf{知覚相}とは、「事実知覚相」とも表記ゆれしている。事物的・物的なあらわれである。
「相(phase)」とは、物質でいう「固体相・液体相・気体相」のように、表層的にあらわれ出ている状態であると考えられる。

\textbf{情緒価}は、「感情価」「情動価」とも表記ゆれしうるという。末尾の「価(value・potential)」とは、知覚「相」とくらべて、「こもっている」「帯びている」ようなありようを示しているのだと著者は解釈している。
色彩心理学が示すように、赤に「情熱」を・青に「冷静」を見てとってしまうというのが簡単な例になるだろう。
ただ廣松によれば、感情価はこうした単純な「喜怒哀楽」だけに限らない。

\textbf{即応価}は、廣松は「行動価」または「信号価」とも呼んでいる。
認知科学領域でも、即応価に相当するさまざまな概念は述べられている。
まずは、廣松もとりあげているように、認知主体が環境から直接に知覚しうる「行為の可能性」としての\textbf{アフォーダンス(affordance)}\cite{gibson:1979}の考え方は最たるものである。
また、郵便受けが「入れてください」と言っているように知覚されるという、ゲシュタルト心理学でいわれる\textbf{要求特性}\cite{kofka:1935}も、即応価の契機のわかりやすい例であろう。
「運動共感」という考え方\cite{kaneko:2005}も、即応価に相当するだろう。
ミラーニューロン\cite{Rizzolatti:2004}やカノニカルニューロンと呼ばれるニューロン群の存在は、即応価の存在の根拠ともなりうる神経的基盤であろう。

知覚相・情緒価・即応価をさらに、下位的に分析してみるならば、
\begin{quote}
  知覚相・情緒価・即応価という三契機の各々が、質態値・度量値・趨勢値を内自化せしめた相で現前する(p.79)
\end{quote}
のだという。
質態値とは質である。
度量値とは量である。「大小・強弱・濃淡といった度合的(p.76)」であり「スカラー的に配位され」るものだという。
趨勢値とは、質と量がつねに変化のさなかにあるということを示し、「生滅的であれ変様的であれ移動的であれ、ともあれ予期的に覚知されるディスポジショナルな変化様態値(p.77)」であり、
「ヴェクトル的に描出される(p.77)」ものだという。

ここで再び、廣松がとりあげる「表情」の例をみてみよう。
「ゾクッ」という表情である。
「ゾクッ」の体験は、(1)敵の姿の対象認知が起こり、(2)恐怖の感情が興発し、(3)逃走の行動が誘発される
という情報処理的順番では描けない。
「ゾクッ」は(2)だけに相当するものではないのだ。
そうではなく、「ゾクッ」がまず生じて、それがそののち、敵の姿となり、恐怖感情となり、逃走行動になる、という説明のほうが、
ありのままの体験を表せているのではないだろうか。
いわば、「ゾクッ」という表情が、これら知覚・感情(思考)・行動という3つのはたらきに分凝\cite{hiromatsu:1989}するのである
(ゾクッの時点ではまだはっきりとした的の姿をとらえ終えていない。)。
ありのままの体験は、対象の知覚は、感情価と行動価をはらんだ知覚相でもって生じる。

表情は、ただ私たちに\textbf{立ち現れ}てくるのである。
「立ち現れる」という動詞は、現象学者・大森荘蔵\cite{ohmori:1976}(1921-1997)がしきりにもちいる動詞であり、本研究ではれっきとした用語としてもちいている。
大森は「立ち現れ一元論」を唱導し、物vs心(世界vs私)という構図をまるごと脱した一元論的な見方である\footnote{余談だが、\ruby{惣}{すべて}ということばがあることがここで思い出される}。
「立ち現れる」とは、そういう一元論的な構図がよく表れた動詞なのである。
補論(\ref{sec:horon_tathiaraware}節)にて、大森による象徴的なフレーズを掲載する。
「表情」がどういう現象かの理解の助けにもなるはずである。

ここまでの議論を踏まえ、「表情」という概念を図示するならば、\autoref{fig:hyojo}のように描ける。

\begin{figure}[h]
  \centering
  \includegraphics[width=0.5\textwidth]{./images/hyojo.pdf}
  \caption{「表情」の概念の図示}
  \label{fig:hyojo}
\end{figure}

\autoref{fig:hyojo}は、
知覚相・行動価・情緒価の3つを、それぞれ「まがたま形」にすることで、「3つでひとつの○になる」ように描いた\footnote{
  「三つ巴」の図をヒントにしている(が、3者が争っているというよりは、本質的に融合し一体的であることを表現している)。
}。
これは、全体性を表現するのと同時に、「つねに動きつづけてこそ在る」ことを表現している。

このようにして「表情」の構成をみてみると、表情は、表面的には視覚という知覚体験としてありながら、
その内実は、すでにして「行為・行動」の契機、「思考」の契機が溶け込んでいる。
要するに「表情」は身体知の「源泉」である、と考えることができる。
\autoref{fig:hyojoandshintaichinomanabi}に、「身体知の学び」の概念モデル(\autoref{fig:shintaichinomanabi})と「表情」(\autoref{fig:hyojo})
とをあわせた図として、両者の関係性を示す。
「表情」は、身体で問うことにともなって立ち現れる問い(ノエマ)の、
原基的なありようである。
問うというノエシスは、知覚・行為・思考の三位一体の作用である。
問いというノエマは、(原基的には、)知覚相・行動価・情緒価の未分的融合としての「表情」である。
両者は相互限定関係にある。
また、上記の大森の思想も踏まえ、この図は、まだ世界vs私といった構図の生まれる「以前」を表すものでもあるから、
この図には、「身体をもったひと」「対象」「ひとの頭から出る吹き出し」といった要素は登場し得ないのである。
図では、赤は思考、青は知覚、緑は行為、というふうに色付けを対応させている。


\begin{figure}[h]
  \centering
  \includegraphics[width=\textwidth]{./images/hyojoandshintaichinomanabi.pdf}
  \caption{「表情」と「身体で問う」ことの関係性}
  \label{fig:hyojoandshintaichinomanabi}
\end{figure}

\section{動いている身体の「表情」を感得することのむずかしさ}
\label{sec:学びにおける、動く身体を鑑賞するという体験}
序論の\ref{sec:ugoiteirushintainohyojo_intro}節でも頭出ししておいた問題を再度述べよう。
運動学習者はしばしば、自身の運動を撮影した動画や、他者の運動をみて観察して、そこからさまざまに内省する。
このとき、その動いている身体が醸し出す「表情」はつい、認識や内省から取りこぼされてしまう。
「表情」は、観察者の身構え(要するにノエシス)に応じて立ち現れてくる。
「学び取ってやろう」と身構えるほど、その志向は、物的な身体運動の一部の特徴に絞られてしまいがちである。
そこには豊かな「表情」があるのに、観察者が取りこぼしているとするならば、それは意味づくりの可能性を狭めていることにもなってしまう。
そういう意味では、実は、眼の前の全身運動を「そのままの物的事実としてフラットにみる」、ということも、とてもむずかしいことである。
観察における私たちの志向は、物的なありかたになりがちなのにくわえ、そのなかでも「ごく一部」にむきがちなのである。

このことは、中村\cite{nakamura:1979}の指摘を参照すれば、
視覚という知覚の「性格」のせいであると考えられる。
視覚は「明晰的な意識」と結びついてはたらきやすい。
自己(みる主体)と世界とを引き離し、そうすることで世界を「みられる対象」として定位する、という形ではたらくのである。
それはさらに、視覚がほかの諸知覚から「独走」してしまうことでもある、中村は指摘する。

\textcolor{red}{
からだメタ認知\cite{suwa:2016}によって「ことば」をうまくつかうことは、「表情」を感得するためのひとつのアプローチであると本研究は考える。
}
だが、それだけでは必ずしも十分ではない、という問題意識も、私は抱いている。
じっさい、廣松は「表情」をことばの問題とむすびつけながら論じているが、
「表情」が生々しく豊かに立ち現れてくるのにくらべ、私たちはそれらを表現する確かな語彙や分類体系を持ちあわせていないことを指摘している。
少なくとも日本語においては、表情の種別を各一語であらわすような語彙は十分ではない。
知覚相に相当する部分で言えば、色や音に限定されれば、語彙はよくよく配備されているが、香や形ともなると、とたんに語彙は貧困になる。
感情価や即応価についても、語彙は貧困である
\footnote{感情については、漢語では語彙は豊かだと廣松は指摘する。}。
だから廣松はそこで「オノマトペ」に着目したのであろう。

ことばも合わせ技でつかいながら、表情感得を促す工夫がしたい。
その工夫によって、からだメタ認知による内省を促したい。
豊かな「表情」感得をうながし、より良い意味づくりを支援するしたい。
そういう問題意識のもと、私は、動いている身体の「表情」の感得をうながすツールを制作した。