
\chapter{ツールの制作にむけて}
\label{chapter:fortoolmaking}
では、どのようなツールをデザインするのがよいのか?
本章で議論する。

\section{身体知の学びとしての運動学習を支援する研究}
身体知の学びとしての運動学習の支援研究には、どのようなものがあるかみておく。

スポーツ科学的視座からみれば、学習者に「良い運動が有する客観的・量的な特徴を知らせる」といったかたちの支援がありうる。
DSAの視座からみれば、この潮流では近年、(コーチが)学習者の運動に相転移を促す「環境」をデザインする、という支援のありかた\cite{ueda:2023}も起こっている。
本研究はこれらの支援を否定するものでは決してないが、「表情」の問題へアプローチするには、少なくとも運動の「質的な」側面に直接アプローチすることが必要であろうと考えられる。

運動の質的側面を重視するアプローチをみてみよう。
モダン・ダンスのパイオニアであるR.Laban\cite{laban:1971}は、運動をうみだす内的なはたらき(Labanは"Effort"と呼んでいる)
\footnote{Labanは、運動の外的な形や様式を重視するクラシックバレエの考え方に反発した。}
に着目し、あらゆる運動のEffortを、Space(直線的/曲線的)・Time(速い/遅い)・Weight(強い/弱い)・Flow(コントロールされた/自由な)の4側面の組み合わせによって描画する記法"Effort Graph"を編み出した。
Effort Graphは、自らのパフォーマンスを分節したり、新しい動き(振り付け)を創造することにおいて有用であろう。

からだメタ認知を促すことは身体知の学びの支援となる。
前述したように、からだメタ認知のひとつの肝は、体性感覚的で曖昧模糊としたものごとについても積極的にことばにすることであった。
「ことばに正解はない」というのは大前提だとしても、そうだとしても、感覚的なものごとをメタ認知することはそう簡単なことではない(「表情」もこの類の体験である)。
\textbf{創作オノマトペ}\cite{otsuka:2015}は、そうした曖昧模糊とした感性的体験に対して、独自のオノマトペ(例:ぬぴゃふー)で表現してみるからだメタ認知促進メソッドである。
オノマトペにすると、音節という分節構造が自然と生まれるが(例えば、ぬ/ぴゃ/ふー)、
各音節ごとに、どういうニュアンスがこもった/をこめたのか、をからだメタ認知によってことばにするのである。
運動学習ではないが、日本酒の味わいという感性的体験に対して、創作オノマトペによってからだメタ認知感性開拓を促した事例がある\cite{otsuka:2015}。
創作オノマトペは「原初音韻論遊び\cite{noguchi:2003}」という、色々な言葉をとりあげてそれを構成するひとつひとつの音韻に対し音韻そのものがもつ感触やニュアンスを言葉で表現してみる営みをヒントにした手法である。
廣松が「表情」をオノマトペとむすびつけた\cite{hiromatsu:1989}こともふまえつつ、
本研究でもオノマトペは「表情」から問いを立てるための一媒体であると考える。

運動の量的特徴をユーザに提示してユーザにからだメタ認知を促すことを目論んだツールもある。
いわば量と質の両側面から攻めるアプローチである。
Nishiyamaら\cite{nishiyama:2010}が開発した『Motion Prism』は、
時系列データである身体運動の各フレームを「姿勢情報」によりクラスタリングし、各クラスタに色を割り当てる。
各フレームに対して「セル」をつくり、各クラスタ傘下のセルにクラスタの色を塗る。
ふたたび元の時系列でセルを積み上げることで「カラーバー」として姿勢変化を可視化し、ユーザにフィードバックする。
ユーザには能動的に時間的分節を見出したりその意味を自分なりに解釈することを促す。
「身体の物的事実をもとにした意味ありげな運動の分節」をフィードバックすることで、ユーザに「能動的に問うことを促す」というやりかたは、本研究のヒントになる。
\section{動いている身体の「表情」に近しいものを扱うプロジェクト}
\label{sec:design}
\todo{ここは修正なの赤くする}
動いている身体の「表情」の感得を促そうとするときには、
また、スポーツ科学的なアプローチでよく示される「グラフ」や、Labanの「記譜」、あるいは、西山の
といった、表象性質、情報の表象、指標性だけでは、どこかが足りない。
これは、記号論でいうところの「指標性(index)」なものでは、記号論・記号学的議論には深くは立ち入らないことにする。
どこかかゆいところに手が届かない感覚が、私にはある。
それは、廣松がことばのなかでも「オノマトペ」に着目したということにも関連するだろう(もっとも、オノマトペをもってしても、表情にたいしては無力)。
そこで本研究は、オノマトペは重要なものとして残しながら、
もっと豊かにしたい。

動いている身体の「表情」がどういうものなのか、関連するであろう研究や作品群を眺めながら考え、
その感得を促すための方法を議論する。

ヒューマンコンピュータインタラクション(HCI)領域では、質的な側面からダンス学習を支援する研究もある。
Labanのアプローチを足場にしつつもより実践的に磨いたダンス理論参照した、
Fdili Alaouiらによる『THE DOUBLE SKIN / DOUBLE MIND INTERACTIVE INSTALLATION』\cite{fdilialaoui_et_al:2015}は、
巨大スクリーンに映し出された抽象的なバネモデルの映像の前で動くことを体験者に促すインスタレーション作品である。
体験者の動きの速度・加速度・躍度がリアルタイムに検知・計算され、それをバネモデルに対応づけている。
ダンサーを対象にしたユーザ実験では、動きの探索を促し、内省することを助けたという。

Konnoらによる『RAM Dance Toolkit』\cite{konno_et_al:2016}では、
ダンスのためのインスタレーション環境(ダンサーの動きに応じてインタラクティブに変化するさまざまな映像)を、クリエイティブコーディングによってシンプルな手順で作り出すことができるツールキットである。
ユーザ(ダンサー)は、各部に慣性センサが搭載された専用スーツを着用して、スクリーンの前で踊る。
ユーザは踊りながら、「少し先の未来を予測したや関節角度や回転を表示」したり「身体外部に描画する立方体を表示」したりなどの動きに応じるインスタレーション環境(映像効果)を、GUIをつうじて設定・選択したりすることができる。
動きのデータ自体を解析・利用するための機能も備えている。
\footnote{
  世界的なコンテンポラリダンサー安藤洋子氏と、プログラマーやメディアアーティストらが協同して制作したプロジェクトである。
}。
ただし、『THE DOUBLE SKIN / DOUBLE MIND INTERACTIVE INSTALLATION』\cite{fdilialaoui_et_al:2015}も『RAM Dance Toolkit』\cite{konno_et_al:2016}も、
リアルタイムなインタラクションに重きが置かれ、それと関連してか、長期的な運動学習の支援に対して必ずしも直接的な工夫や検討がされているわけではない。
身体知の学びの支援としては、それも重要な点である\footnote{
私は、2017年1月にX氏(第一部\ref{chapter:monogatari}章の物語でも何度か登場。2017年1月は物語の\ref{sec:tatuaruku}節の時期にあたる)から紹介されたイベント:
「Perception Engineeringキックオフ―つなげる身体」@山口情報芸術センター、2017年1月21日)に参加したのだが、
上記『RAM Dance Toolkit』\cite{konno_et_al:2016}は、そのイベントにて紹介されていたプロジェクトである。
思えばそれに触れた当初から、研究者・実践者として生きている私には、動いている身体の「表情」の感得を促す工夫についての「伏線」が貼られていたように思う。
}。

% バイオロジカルモーション
ひとは、点群が動く単純な映像をみるだけで、それが「人体」であり、なにをしているところか、さらに人物の性別・感情・意図さえも感じとることができる\cite{johansson:1973}。
バイオロジカルモーション(以下、BM)として知られるこの認知現象は、動いている身体の表情の一種だと言えそうではある。
しかしオノマトペと結びつけてとらえる「表情」、それも運動学習者が動いている身体に感じとるべき「表情」は、必ずしもBMに限定されるものでもないと著者は考える。
というのも、動いている身体が醸し出す「表情」は、必ずしも「人体の形」にみているとは限らないのではないか、と著者は感あげる。

% 『ISSEY MIYAKE A-POC INSIDE』
そこで短編映像『ISSEY MIYAKE A-POC INSIDE』(2006)\cite{euphrates:online}にも着目しよう。
この作品では、モデルウォークする人物(や駆ける動物の姿)から作られた動く白点群が黒背景に描かれ、
これら点群を頂点とするシンプルな図形、次々と表示される
\footnote{
本作品は、New York ADC Gold Prize in 2007、第11回文化庁メディア芸術祭における優秀賞受賞作品である。
}。
棒人間とは異なる形だが、「ファッションモデルたちの、生き生きとした動きなどを如実に表現する\cite{euphrates:online}」と述べられている。

% 階段を降りる裸体
M.デュシャン(1887-1968)による絵画『階段を降りる裸体 No.2』(1912)は、表題のモチーフで純粋な動きを抽象画として描いたものである\footnote{
  E.マイブリッジ(1830-1904)による連続写真『Woman Walking Downstairs』(1887)に影響を受けた作品である。}。
それにインスパイアされたQuayolaとM.Aktenによる映像作品『Forms』(2011)\cite{akten:online}は、トップアスリートのパフォーマンス映像をもとに計算・生成した3Dの抽象的なCG映像である
\footnote{
本作品は、2013年度Ars Electronicaにおいて、Golden Nica賞(アニメーション部門最高賞)を受賞作品である。
}。
美的な観点から、身体と環境との見えない関係性(力やバランスや優雅さや葛藤)を「抽象的形態として彫刻するかような外挿的な可視化」技法を探究したものだと作者Aktenは説明する。

% 『Strandbeest』
『Strandbeest』(1990〜)\cite{jansen:online}は、複数の木製のリンク機構が軸方向に並列的に連なったキネティックアートであり、風を動力にして作動し砂浜を移動する。
むきだしの木組みが蠢くさまを観て著者は、(あえてことばで描写するならば)負傷した多脚生物が何者かから逃げているかのような表情を感得せずにはいられない。
上記Aktenの作品を踏まえて言えば、リンク機構のふるまいそのものが、逃げ歩く表情をありありと浮かび上がらせる「外挿」なのかもしれない。
キネティックアートは「動きそのもの\cite{miyoshi:2022}」がアートの主題になっているからか、
概して、作品は動きを生む内部構造や部材が隠されたりせず、むしろ「むきだし」になっている。
作品の動きと形とが必然的なむすびつきをもっており、形の動きと動きの形がひとつになっている。
動いている身体の「表情」とは、そういうむきだしな姿において\ruby{顕}{あらわ}になりやすいのではないかと著者は考えた。

\section{どういうツールを作るのがよいか}
本章の議論や「表情」の考え方(\ref{sec:hyojo}項)や「身体知の学びの性質」(\ref{subsec:shintaichinomanabi}項)を総合して、動いている身体の「表情」の感得をうながす身体知学習支援ツールのヒントになる項目は以下である。
\begin{itemize}
  \item 実際の動きをもとにした「かたち」を描くこと
  \item かたちは、抽象的で、素朴で、動きがむきだしになったような見た目であること。
  \item 「人に似て非なる形」や「人ならざる形」のような、人体形をあえて保留した形であること
  \item からだメタ認知(ことば)を促し、かつ、長期的な学習を支援できるようになっていること
  \item 身体知の学びの主体性をうながすこと
\end{itemize}
私はこれらのヒントをふまえてツールを制作した。
次章で説明する。











