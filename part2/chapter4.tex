\chapter{実践方法}
\label{chapter:jissenmethod}


\section{対象者}
4名(A〜D)を対象にアプリをもちいた実践をおこなった(\autoref{table:subjects})。
A〜Dいずれもからだメタ認知をもちいて学ぶ経験を1年以上有している。
主たる実践対象者(以下、対象者と呼ぶ)は現役の運動学習者であるAとBである。
CとDは副対象者とした。
副対象者として参加してもらう目的は、主対象者であるAやBの学習を触発することにある。
前章で述べたとおり、HJ-Playgroundでは他者の表情コレクションも閲覧できる。
AとBがCやDの表情エントリに影響を受けて自身の表情エントリを作成するなど、学びの可能性をひらいておくため、というのがCとDの参加の第一目的である。
また、対象者の属性によってHJ-Playgroundの効果を比較しやすくするためでもある。

後述するが、Aは自身が踊る「oldman」というダンス技をプレイ対象動作とし、
BとCは「一本ゲタ対人運動」をペアでおこない、これをプレイ対象動作とし、
Dは自身の運動を撮影せずに、上記A〜Cの動作をもちいてプレイした。

\begin{table}[htbp]
  \centering
  \caption{実践対象者の概要}
  \begin{tabular}{c|clll}
    \hline
    \bf ID & \bf 年齢(歳) & \bf 実践ドメイン(現役/引退) & \bf プレイ対象動作 & \bf 学習対象者区分 \\
    \hline
    A  & 24 & ストリートダンサー(現役) & oldman                     & 主たる対象者 \\
    B  & 21 & 陸上・三段跳(現役)         & 一本ゲタ対人運動(with C) & 主たる対象者 \\
    C  & 24 & 陸上・棒高跳&走高跳(引退) & 一本ゲタ対人運動(with B) & 副対象者     \\
    D  & 23 & サッカー(引退) & oldman\&一本ゲタ対人運動  & 副対象者     \\
    \hline
  \end{tabular}
\label{table:subjects}
\end{table}


\section{実践手順}
本実践の実施にあたっては、SFC研究倫理審査委員会の承認を得ている(2023年10月17日承認)。
対象者に実践について説明し、同意書へのサインを得たのち、以下の手順で実践した。
\begin{enumerate}
  \item プレイ対象動作を考案・選定する(著者と学習者とでおこなう)
  \item 対象者のプレイ対象動作をモーキャプで撮影する
  \begin{itemize}
    \item 撮影データをアプリに読み込む
  \end{itemize}  
  \item 対象者がHJ-Playgroundでプレイしながら日々過ごす(実践期間)
  \item 学びを解釈するインタビューの実施
\end{enumerate}

それぞれ説明する。
実践の説明ののち、対象者自身のどういう動作をアプリでプレイするかを著者と共同で考案・選定した。

対象者の動作をモーションキャプチャOptiTrack社製 V120: Trio\footnote{
解像度30万画素、レンズ視野角47°
}をもちいて120Hzで撮影した。
撮影データをアプリDBに登録したのち、
対象者らにwebアプリのURLとアプリ使用説明書を共有し、
この時点をアプリでの「実践期間開始」とした。

実践期間中には、Zoomをもちいたオンラインセッションも2度おこない、不明点の聞き取りや対象者をエンカレッジしたりした。

実践終了後、AとBに2回ずつ、学びを解釈するための半構造化インタビューを実施した。
オンライン会議アプリZoomで著者と1対1で、実際に対象者が作成した表情エントリを鑑賞しながら進めた。
1回目インタビュー(Aは約84分、Bは約70分、ともに2025年4月19日)の予め用意した質問項目は、「実践期間に運動学習者として抱いていた問題意識」、「どのような表情図形を作図し、いかなる問いを生んだか?」、「アプリでの表情感得体験が、身体知の学びになにをもたらしたか?」の3点である。
2回目インタビュー(Aは2025年8月26日に、Bは2025年8月29日、ABともに50分)の予め用意した質問項目は、「表情エントリ相互の派生関係」「その後の学びの活動はどうか」「HJPの経験とその後の学びの関連」の3点である。

インタビューの目的を補足する。あくまで実践の主要データはAとBが作成した表情エントリである。
だが、とくにからだメタ認知記述は、AとBが「自分のため」に問いながら書きつけた言葉であって、必ずしも他者にわかる説明の言葉になっているわけではないし、上記質問項目のような、自身の学びを俯瞰した記述も必ずしもなされない。
本研究では、それを著者が解釈し、個別具体的な学びの様相を読者になるべく了解可能なかたちで解説するという手立てをとる(それが次章である)。
そういう解釈の補助としてインタビューを実施した。
2回目インタビューは、身体知の学びは時間をかけて醸されるという考えから、時間をおいてもう一度実施した。

AとB(インタビュイー)は著者(インタビュワー) と、同研究室で学びながら、からだメタ認知を駆使して身体知の学びに取り組んだ仲間であり、かつ、そのことを相互理解していた。
半構造化インタビューに臨むためのラポール形成の観点からも、このことは重要であろう。
半構造化インタビューでは、対象者は自身の表情エントリと自身の学びとを架橋しうるもっともらしい解釈を語りだし、自身も身体知の学び手である著者は、対象者からそれを引き出すように深掘る質問を投げかけた。
そのように両者のあいだで「折り合いをつける」ようにして、解釈を紡いだ。


\section{対象動作の身体運動データの取得について}

対象者Aは、以下のようにマーカを貼り付けた(\autoref{fig:markers_a_f}〜\autoref{fig:markers_a_b})。

% サトゥー
\begin{figure}[h]
\centering
  \begin{minipage}[b]{0.3\linewidth}
  \centering
  \includegraphics[width=\linewidth]{./images/jikken/sato_f.pdf}
  \caption{対象者Aのマーカ貼り付け位置(正面から)}
  \label{fig:markers_a_f}
  \end{minipage}
\hspace{0.02\linewidth}%画像間の余白
  \begin{minipage}[b]{0.3\linewidth}
  \centering
  \includegraphics[width=\linewidth]{./images/jikken/sato_l.pdf}
  \caption{対象者Aのマーカ貼り付け位置(左から)}
  \label{fig:markers_a_l}
  \end{minipage}
\hspace{0.02\linewidth}%画像間の余白
  \begin{minipage}[b]{0.3\linewidth}
  \centering
  \includegraphics[width=\linewidth]{./images/jikken/sato_b.pdf}
  \caption{対象者Aのマーカ貼り付け位置(後ろから)}
  \label{fig:markers_a_b}
  \end{minipage}
\end{figure}

対象者B・Cは、以下のようにマーカを貼り付けた(\autoref{fig:markers_bc_f}〜\autoref{fig:geta_ushiro})。
% キサラとサラ一緒に
\begin{figure}[H]
\centering
  \begin{minipage}[b]{0.3\linewidth}
  \centering
  \includegraphics[width=\linewidth]{./images/jikken/sarakisara_s.pdf}
  \caption{対象者B・Cのマーカ貼り付け位置(真横から)}
  \label{fig:markers_bc_f}
  \end{minipage}
\hspace{0.02\linewidth}%画像間の余白
  \begin{minipage}[b]{0.3\linewidth}
  \centering
  \includegraphics[width=\linewidth]{./images/jikken/sarakisara_f.pdf}
  \caption{対象者B・Cマーカ貼り付け位置(Bの後ろから)}
  \label{fig:markers_bc_b}
  \end{minipage}
\hspace{0.02\linewidth}%画像間の余白
  \begin{minipage}[b]{0.3\linewidth}
  \centering
  \includegraphics[width=\linewidth]{./images/jikken/sarakisara_k.pdf}
  \caption{対象者B・Cのマーカ貼り付け位置(Cの後ろから)}
  \label{fig:markers_bc_c}
  \end{minipage}
\end{figure}

% キサラとサラ別々
\begin{figure}[H]
\centering
  \begin{minipage}[b]{0.21\linewidth}
  \centering
  \includegraphics[width=\linewidth]{./images/jikken/sara_f.pdf}
  \caption{対象者Bのマーカ貼り付け位置(正面から)}
  \label{fig:markers_b_f}
  \end{minipage}
\hspace{0.01\linewidth}%画像間の余白
  \begin{minipage}[b]{0.21\linewidth}
  \centering
  \includegraphics[width=\linewidth]{./images/jikken/sara_r.pdf}
  \caption{対象者Bのマーカ貼り付け位置(真横から)}
  \label{fig:markers_b_l}
  \end{minipage}
\hspace{0.01\linewidth}%画像間の余白
  \begin{minipage}[b]{0.21\linewidth}
  \centering
  \includegraphics[width=\linewidth]{./images/jikken/kisara_l.pdf}
  \caption{対象者Cのマーカ貼り付け位置(真横から)}
  \label{fig:markers_c_r}
  \end{minipage}
\hspace{0.01\linewidth}%画像間の余白
  \begin{minipage}[b]{0.21\linewidth}
  \centering
  \includegraphics[width=\linewidth]{./images/jikken/kisara_f.pdf}
  \caption{対象者BCのマーカ貼り付け位置(正面から)}
  \label{fig:markers_c_f}
  \end{minipage}
\end{figure}

% ゲタ
\begin{figure}[H]
\centering
  \begin{minipage}[b]{0.21\linewidth}
  \centering
  \includegraphics[width=\linewidth]{./images/jikken/geta_chuo.pdf}
  \caption{中央ゲタのマーカ貼り付け位置}
  \label{fig:geta_chuo}
  \end{minipage}
\hspace{0.01\linewidth}%画像間の余白
  \begin{minipage}[b]{0.21\linewidth}
  \centering
  \includegraphics[width=\linewidth]{./images/jikken/geta_ushiro.pdf}
  \caption{後ろゲタのマーカ貼り付け位置}
  \label{fig:geta_ushiro}
  \end{minipage}
\end{figure}
