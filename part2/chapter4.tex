\chapter{実践方法}
\label{chapter:jissenmethod}


\section{対象者}
4名を対象にアプリをもちいた実践をおこなった(\autoref{table:subjects})。
A〜Dいずれもからだメタ認知をもちいて学ぶ経験を1年以上有している。
主たる対象学習者は現役の運動学習者であるAとBである。
CとDは副対象者とした。
副対象者として参加してもらう目的は、主たる学習者であるAやBの学習を触発したり、本アプリが「表情」の感得をうながすしかたが(それぞれの属性をもつ)対象者によってどのように異なりうるのかの考察しやすくするためである。
BとCは一本ゲタ対人運動をペアでおこない、同じ運動データをもちいてアプリをプレイした。
Dは、自身の運動を撮影したわけではなく、A〜Cの運動のデータをもちいてプレイした。

\begin{table}[htbp]
  \centering
  \caption{実践対象者の概要}
  \begin{tabular}{c|clll}
    \hline
    \bf ID & \bf 年齢(歳) & \bf 実践ドメイン(現役/引退) & \bf プレイ対象動作 & \bf 学習対象者区分 \\
    \hline
    A  & 24 & ストリートダンサー(現役) & oldman                     & 主たる対象者 \\
    B  & 21 & 陸上・三段跳(現役)         & 一本ゲタ対人運動(with C) & 主たる対象者 \\
    C  & 24 & 陸上・棒高跳&走高跳(引退) & 一本ゲタ対人運動(with B) & 副対象者     \\
    D  & 23 & サッカー(引退) & oldman\&一本ゲタ対人運動  & 副対象者     \\
    \hline
  \end{tabular}
\label{table:subjects}
\end{table}


\section{実践手順}
本実践の実施にあたっては、SFC研究倫理審査委員会の承認を得ている(2023年10月17日承認)。
対象学習者に実践について説明し、同意書へのサインを得たのち、以下の手順で実践した。
\begin{enumerate}
  \item プレイ対象動作を考案・選定する(著者と学習者とでおこなう)
  \item 対象学習者のプレイ対象動作をモーキャプで撮影する
  \begin{itemize}
    \item 撮影データをアプリに読み込む
  \end{itemize}  
  \item 対象学習者がアプリでプレイしながら日々過ごす(実践期間)
  \item 補助的インタビュー
\end{enumerate}
実践の説明ののち、対象者自身のどういう動作をアプリでプレイするかを著者と共同で考案・選定した。
対象者の動作をモーションキャプチャOptiTrack社製 V120: Trio\footnote{
解像度30万画素、レンズ視野角47°
}をもちいて120Hzで撮影した。

撮影データをアプリDBに登録したのち、
対象者らにwebアプリのURLとアプリ使用説明書を共有し、
この時点をアプリでの「実践期間開始」とした。
実践期間中には、Zoomをもちいたオンラインセッションも2度おこない、不明点の聞き取りや対象者をエンカレッジしたりした。

実践期間終了後、AとB\&Cに約90分の補助的なインタビューをおこなった。
補助的インタビューの目的は、対象者らが作成した表情コレクションだけからでは必ずしも読みとれない実践の全容をつかむためとし、
あくまで補助的な位置付けのインタビューである。

\section{対象動作の身体運動データの取得について}

対象者Aは、以下のようにマーカを貼り付けた(\autoref{fig:markers_a_f}〜\autoref{fig:markers_a_b})。

% サトゥー
\begin{figure}[h]
\centering
  \begin{minipage}[b]{0.3\linewidth}
  \centering
  \includegraphics[width=\linewidth]{./images/jikken/sato_f.pdf}
  \caption{対象者Aのマーカ貼り付け位置(正面から)}
  \label{fig:markers_a_f}
  \end{minipage}
\hspace{0.02\linewidth}%画像間の余白
  \begin{minipage}[b]{0.3\linewidth}
  \centering
  \includegraphics[width=\linewidth]{./images/jikken/sato_l.pdf}
  \caption{対象者Aのマーカ貼り付け位置(左から)}
  \label{fig:markers_a_l}
  \end{minipage}
\hspace{0.02\linewidth}%画像間の余白
  \begin{minipage}[b]{0.3\linewidth}
  \centering
  \includegraphics[width=\linewidth]{./images/jikken/sato_b.pdf}
  \caption{対象者Aのマーカ貼り付け位置(後ろから)}
  \label{fig:markers_a_b}
  \end{minipage}
\end{figure}

対象者B・Cは、以下のようにマーカを貼り付けた(\autoref{fig:markers_bc_f}〜\autoref{fig:geta_ushiro})。
% キサラとサラ一緒に
\begin{figure}[H]
\centering
  \begin{minipage}[b]{0.3\linewidth}
  \centering
  \includegraphics[width=\linewidth]{./images/jikken/sarakisara_s.pdf}
  \caption{対象者B・Cのマーカ貼り付け位置(真横から)}
  \label{fig:markers_bc_f}
  \end{minipage}
\hspace{0.02\linewidth}%画像間の余白
  \begin{minipage}[b]{0.3\linewidth}
  \centering
  \includegraphics[width=\linewidth]{./images/jikken/sarakisara_f.pdf}
  \caption{対象者B・Cマーカ貼り付け位置(Bの後ろから)}
  \label{fig:markers_bc_b}
  \end{minipage}
\hspace{0.02\linewidth}%画像間の余白
  \begin{minipage}[b]{0.3\linewidth}
  \centering
  \includegraphics[width=\linewidth]{./images/jikken/sarakisara_k.pdf}
  \caption{対象者B・Cのマーカ貼り付け位置(Cの後ろから)}
  \label{fig:markers_bc_c}
  \end{minipage}
\end{figure}

% キサラとサラ別々
\begin{figure}[H]
\centering
  \begin{minipage}[b]{0.21\linewidth}
  \centering
  \includegraphics[width=\linewidth]{./images/jikken/sara_f.pdf}
  \caption{対象者Bのマーカ貼り付け位置(正面から)}
  \label{fig:markers_b_f}
  \end{minipage}
\hspace{0.01\linewidth}%画像間の余白
  \begin{minipage}[b]{0.21\linewidth}
  \centering
  \includegraphics[width=\linewidth]{./images/jikken/sara_r.pdf}
  \caption{対象者Bのマーカ貼り付け位置(真横から)}
  \label{fig:markers_b_l}
  \end{minipage}
\hspace{0.01\linewidth}%画像間の余白
  \begin{minipage}[b]{0.21\linewidth}
  \centering
  \includegraphics[width=\linewidth]{./images/jikken/kisara_l.pdf}
  \caption{対象者Cのマーカ貼り付け位置(真横から)}
  \label{fig:markers_c_r}
  \end{minipage}
\hspace{0.01\linewidth}%画像間の余白
  \begin{minipage}[b]{0.21\linewidth}
  \centering
  \includegraphics[width=\linewidth]{./images/jikken/kisara_f.pdf}
  \caption{対象者BCのマーカ貼り付け位置(正面から)}
  \label{fig:markers_c_f}
  \end{minipage}
\end{figure}

% ゲタ
\begin{figure}[H]
\centering
  \begin{minipage}[b]{0.21\linewidth}
  \centering
  \includegraphics[width=\linewidth]{./images/jikken/geta_chuo.pdf}
  \caption{中央ゲタのマーカ貼り付け位置}
  \label{fig:geta_chuo}
  \end{minipage}
\hspace{0.01\linewidth}%画像間の余白
  \begin{minipage}[b]{0.21\linewidth}
  \centering
  \includegraphics[width=\linewidth]{./images/jikken/geta_ushiro.pdf}
  \caption{後ろゲタのマーカ貼り付け位置}
  \label{fig:geta_ushiro}
  \end{minipage}
\end{figure}
