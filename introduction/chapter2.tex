\chapter{知の学問における本研究の位置づけ}
\label{chapter:embodiedwisdom}
本章では、認知科学や人工知能を中心とした、広く知の学問を概観し、
本研究の思想(身体知の学び)の位置づけを明確にする。

\section{情報処理モデルの思想}
\label{sec:aiboom}
認知科学や人工知能が誕生した1950年代以降、知の学問のメインストリームをなしてきたのは、身体と心とを分離し、そのうえで、心のほうに知を求める思想・方法論である
\footnote{17世紀のデカルトによる「我思うゆえに我あり」とした心と身体を分離する考え方\cite{descartes:1637}は、色濃く後世に影響を残している。
}。
いわゆる\textbf{認知主義}である。
認知主義は、\textbf{行動主義}---動物実験のように刺激と反応のセットすなわち客観的に観測可能な物的身体のふるまいだけに知を求める---へ反発する立場である。
認知主義では、知は「コンピュータ」に見立ててモデル化された。
\textbf{情報処理モデル}である。
情報処理モデルは、システム内部(あたまのなか)に「知識」が格納されており、知覚を外界からシステムへの「入力」、行動をシステムから外界への「出力」とみなす。
システムは、入力をもとに、それをもとに、すでに貯蔵済の内部知識のなかから最適と判断されたものを選び計算し、行動として出力する。
出力の結果はシステムにフィードバックされ、内部知識を更新する、といったものだ。
80〜90年代には、医学診断などの専門ドメインにおける膨大な知識をif-then形式で格納・活用するエキスパートシステムが注目され、第二次AIブームが巻き起こった。
こうした情報処理モデル(シンボリックAI)は、well-definedな(かたちで用意された)世界のなかでは威力を発揮した。

しかし、である。
いざ現実世界に放り出されてみれば、事態は異なった。
シンボリックAIは現実世界で出くわす問題に適切に対処することができなかったのだ。
シンボリックAIは、ill-definedな現実世界において、当座の問題にはなにが関係していて/関係していないのか、という思考の枠を準備しておくことができない。
\textbf{状況}に対して適切に対処できないのである(枠があっても、上記の選別に無限の時間がかかってしまう)。
いわゆるフレーム問題\cite{McCHay:69,dennett:1984}である。

いっぽう人間はといえば、状況に応じて、フレームを狭めたり拡げたりしながら、なんとかやっていく\cite{decerteau:1980}ことができている\footnote{もちろん「人間はフレーム問題を完全に解決している」わけではない}。
\ref{sec:embodiedwisdomlearning}節で述べた「状況から知を創りだす」ということに相当する。
それを可能にするのは「身体」であるというのが、有力な見方であり\footnote{
  諏訪\cite{suwa:2018}は、お笑い芸人の大喜利やデザイナーのスケッチを例に、クリエイティブな発想の源は、身体ごと世界に没入して世界に触り、身体や感情の発露にある(さらに、そこに「ことば」を駆使するのがよい)と説いている。
}、本研究もそう考える。
ひとは、身体をそなえて生きており、ひとの知は身体あってこそ成り立つ。
シンボリックAIは「身体」をそなえていないのである。

この問題は、2025年現在第四次AIブームの中心を担う生成AIについても、本研究からみればさして変わらない。
生成AI
\footnote{ChatGPTをはじめ生成AIは非常に便利であり、仕事や学術や芸術といった様々な領域において、私たち人間との共同する時代が到来している。
}は、人間の脳のニューラルネットワークを模したモデル
\footnote{ニューラルネットとそれをビッグデータをつかって深層学習させることは、2010年代に起きた第三次AIブームからの中核技術である。
}であり、
ChatGPTに代表されるように、テキスト・画像・音声などマルチモーダル学習をした大規模言語モデルが実現されている。
そこでの内部知識は明示的表象としてあるわけではなく、ニューラルネット全体へ\textbf{分散的に表象}されている(サブシンボリックAI)という点で、
フレーム問題に対して一歩前進しているとは言えるのかもしれない。
しかしそれでもなお、サブシンボリックAIもまた「身体」をそなえているわけではない。
それに、サブシンボリックAIは「人間の用意したビッグデータ」から「学習させられる」のであって、
主体的に「学ぶ」\cite{saeki:1995}ことはしない。
サブシンボリックAIは「情報」を処理しているだけであって、ものごとの「意味」を知らないのである。
極論を言ってしまえば、サブシンボリックAIは身体をそなえて生きていないから、ひとの知とは異なるのである。

\section{身体性認知の思想}
\label{sec:shintaisei}

知の成立には「身体」が欠かせないと説く研究思想もある。
生態心理学の祖・J.J.ギブソン(1904-1979)\cite{gibson:1979}はそのひとりと言えよう。
認知主体は内的表象の構成というプロセスを介さずとも、「環境から直接に情報を知覚」できるとギブソンはいう。
生態心理学では原則的に行為と知覚は一体であると考え(この関係性は情報処理モデルのフィードバック機構とは異なる)、
そのうえで知覚-行為システムとして認知を記述しようとする。

工学的アプローチ(ロボティクス)から「身体」の重要性を示したのは「\textbf{知能の身体性}」という考え方である。
ロボットは、プログラムとして明示的に表象された命令どおりにふるまうわけではない。
ロボットの物理的な身体と環境との相互作用もあってこそはじめて、(プログラムもふくめた)その全体から何らかのふるまいが\textbf{創発(emergence)}するのである
\footnote{PheiferによるDidabot\cite{pheifer:2001}を例にとろう。
Didabotは小さな車型ロボットであり、前部の左右それぞれに近接センサを搭載している。
「直進せよ。左右どちらかの近接センサに物体を検知したときは、曲がれ(回避)」という命令がプログラムされている。
だが、実際にサイコロ状の物体をたくさん散りばめたフィールドでDidabotを一定時間動かしてみたところ、驚くべきことが起きた。
Didabotはフィールドの物体群を「お掃除」してしまっていたのである。
そんなプログラムはどこにも書いていないのに、なぜなのか?
こういうことだ。
物体aが真正面にあるとき、左右どちらの近接センサにも反応しない。
Didabotは物体aをそのまま押し進める(プログラムの視点からすれば、何事も起きていない)。
その状態で左右どちらかに物体bを検知すると、Didabotは曲がる。
このとき、物体aがその場(物体bのそば)に置き去りにされる。
これを繰り返すと、はじめ散らばっていた物体群がだんだんと集積してゆく、というカラクリである。
「おそうじ」行動は、プログラムの中身のみならず、物体重量や形、Didabotの馬力や形、センサの検知可能範囲やとりつけ位置・・・などの物理的相互作用によって、創発したのである。
}。
哲学者・Clark\cite{clark:1997}は、こうした知見を踏まえて、知は、脳内から身体、世界へと「漏れ出」ているのだと説いた。
Clarkは、脳-身体-環境の相互作用によって脳が計算の負荷を減らせる可能性を指摘している。
Clarkの立場は、表象が必要ないとまで主張するギブソンよりも急進性は低いと言える。

ヴァレラ(1946-2001)らによる\textbf{エナクティヴィズム}の考え方\cite{varela_et_al:1991}もある。
エナクティヴィズムでは、まずなにより「行為」を考え、環境も知覚も「行為」ありきで成り立つ。
みずから行為することによる、感覚運動カップリングがはたらき、
感覚運動カップリングの歴史が、世界を、知覚を、産出する、と考える。
ミツバチと花とが、ミツバチが密を吸いやすように、花が蜜を吸われやすいように、というかたちで共進化する。
また、ヴァレラは、知の問題を探究するときには科学的なアプローチだけでは足りないことを指摘していた。
知を「研究するわたし」から切り離して対象化したうえで、知に「ついて」扱おうとするだけでは、知の本質には迫れない。
そこでヴァレラは、仏教や現象学やプラグマティズム哲学をふまえて、三昧/開かれた反省によって知が現れると説いた。
それは、自己(わたし)や反省プロセスそのものをも巻き込むような、身体的な反省\footnote{ヴァレラは「身体としてある反省」と呼んだ。}である。
ギブソンの考えと、表象主義を乗り越えようとする点では共通しているが、ギブソンは環境に実在すると考えるいっぽう、行為が産出すると考えることがちがう。
また、Clarkらの考え方はヴァレラの考え方よりもいくぶん機能主義的、と言えよう。

近年は、本節で挙げてきた研究思想群(知能の身体性、生態心理学、エナクティヴィズムなど)を、脳神経科学をも巻き込むかたちで「身体性認知」として統合的にとらえようとする動向もある。
知を「脳内」だけに閉じ込めず(脳を単なる中枢とせず)、\textbf{脳-身体-環境のダイナミカルなシステム}(力学系)のなかで知をとらえる。
そのうえで立場はいろいろ分かれている。
ギブソンに似た、認知作用にそもそも表象を認めず脳-身体-環境システムだけで考えるような急進的な立場、
Clarkに似た、認知作用は脳-身体-環境システムが「脳がする計算負荷を減らす」役割を果たすのであって脳の表象・計算そのものはあるとするような立場、
ヴァレラに似た(あるいはLakoff\&Johnson\cite{lakoff_johnson}のように)、認知内容の身体づけられたありようを扱う立場、
などである(本研究の焦点は本文の3つめに挙げた立場に近い)。
Gallagher\cite{gallagher:2023}はこうした流れを、4E(embodied, embedded, extended, enactive)という用語をもちいて展望している。

本研究は身体性認知へ一定の賛意は示しつつ、その焦点は「身体づけられた意味のありよう」にある。
本研究は、認知作用が脳-身体-環境のダイナミカルなシステムがあって成り立つことは認めつつも、
認知作用を客観的かつミクロに記述することはしない。
それにともなう認知内容の側にこそ、本研究の焦点はある。
身体と「意味」とは不可分である。
身体がどのように意味をつくりだし、意味はどのように身体づけられているのか、を真っ向から扱うのが本研究である。
もしここにも、心脳問題で指摘される「説明のギャップ」\cite{levine:1983}を持ち込んで良いのならば、
「脳-身体環境の物的相互作用」を記述しようが「心的体験そのもの」には迫れない、と考えることもできる。
両者のあいだには埋まることのない「ギャップ」が残らざるを得ないのではないか、と著者は考える。

そればかりではない。
本研究は思考(やことば)といった「認知内容」が、(身体性)認知作用が原因となって結果的に生じるものだ、とは考えない。
それだけでなく、思考が認知をかたちづくる、という視点に本研究は立つ(詳細は\ref{subsec:questioningwithbody}、\ref{subsec:constructionloop}、\ref{subsec:embodiedmetacognition}項などを参照)。
重要なことだが、思考(やことば)をそのように位置づけるとき、「意味」とは、単なる内観とか記号的な表象とは異なるものである。
身体性認知の研究群のなかでさえ、科学的にあつかうのがむずかしいからなのか、思考やことばが認知をつくる(知覚や行為をつくる)ありようを迫るアプローチは盛んとは言えないのが現状だろう。
しかし意味の問題を真っ向から扱うためには、この性質こそ重要なことなのである。

以上を鑑みると、意味を主題的に扱う本研究に意義を認められるだろう。
\section{身体知の学びの思想}
本節では、前節で主張したような身体と意味の関係性を理論立てながら、本研究の中核概念である身体知の考え方を打ち立てる。

\subsection{環世界:固有な意味世界を生きる}
\label{subsec:umwelt}
「意味」の次元に迫りうる概念を早くから提唱していたのは、生物学者・ユクスキュル(1864-1944)である。
ユクスキュルの提唱した\textbf{環世界(Umwelt)}\cite{uexkull:1934}論は、本研究が拠って立つ思想家のひとりであるメルロ=ポンティ(1908-1961)をはじめ、現象学・哲学領域にも大きな影響をあたえた。
環世界とはなにか。
ダニの例が手っとりばやい。
ダニは、灌木につかまって待ち構えている。
哺乳類の酪酸のにおいを嗅ぎつけると、
木から手を離し、落下する。
落下場所の衝撃を感じ、
哺乳類の体表の毛を這いまわり、
毛のないところをみつけると、そこに喰いこみ、吸血する。
それだけなのだ。
ダニの生きる世界には、酪酸、温度、接触という3つの刺激だけがあり、
その他の刺激は「存在しない」に等しい。

ダニの例が示すのは、生物が世界をどのように「経験」するかは、その生物の身体(どのような感覚器官をもち、どのような運動をなすか)に深く依存しているということである。
言い換えれば、生物が世界に出会うしかたは、その生物固有の知覚と作用によって象られている。
このように、それぞれの生物種は、固有な知覚と作用の器官を有し、知覚と作用(行動・運動)によって環境の対象とかかわっている(知覚と作用というツメをもつピンセットで対象をはさみこむかのごとく、である)。
これを\textbf{機能環}と呼ぶ。
それぞれの生物は、その生物固有な知覚と作用だけに連関した世界、機能環の集まり/連なりによってなす世界を生きている。
この意味世界を、ユクスキュルは「環世界」と呼んだ。
ダニのばあい環世界は単純で3つの機能環のみからなるが、犬や人間ともなれば環世界はもっと複雑である。
ダニと犬と人間はそれぞれ、客観的には同じ環境(Umgebung)に存在していても、異なる環世界(Umwelt)を生きている。
環世界概念は、私たち一人ひとりが固有な環世界を生きていることを含意している。
本研究の身体知の\textbf{個人固有性}\cite{suwa:2016}の考え方は、一人ひとりが自らの身体に照応した固有の環世界を生きていることとして、環世界とあわせて理解できる。

環世界の考え方は、身体知の学びの実践者が、当人固有の意味世界を生きるということ(\ref{sec:konzenittai}節参照)を下支えする。
先に述べておくと、私たち人間が生きる環世界は「表情」に満ち充ちており\cite{hiromatsu:1989}、そのことが第二部研究の根幹テーマになる。
なお、ユクスキュルの環世界論には、本研究と主張とはなじまない別の解釈にもつながりうる面もあるので、それについては、補論\ref{sec:horon_kansekai}で述べる。

\subsection{ノエシスとノエマの相互限定}
\label{subsec:noesisnoema}
木村敏(1931-2021)はユクスキュルやWeizsäcker(1886-1957)の仕事を踏まえながら、それを現象学の考え方と紐づけることで、人間が生きているということについての原理を論じている。
木村の論から、「身体で問う」ということを説明しよう。

Weizsäcker\cite{weizsaecker:1950}の提唱した\textbf{ゲシュタルトクライス}を確認することからはじめよう。
有機体は、知覚と運動をたえまなく「立て直し」続ける\footnote{Weizsäckerはこれを転機(クリーゼ)と呼んでいる}ことで、つど世界(対象)と出会い続け、みずからの生を維持している。
知覚と運動とでなすひとつの円環に有機体と客体が挟まれており、
同時に、この全体の円環のなかにこそ知覚-運動、有機体-客体の関係は維持される。
この関係性を、Weizsäckerはゲシュタルトクライス\footnote{クライスは円環という意味のドイツ語である。}と呼んだ。
象徴的なフレーズを\cite{weizsaecker:1950}から以下に直接引用しておこう\footnote{
ほかにも\cite{weizsaecker:1950}には以下のような象徴的なフレーズが書かれている。
「私が自分で動くときに私は一つの知覚を感じるという事態として、また私が或るものを知覚するときに私にとって一つの運動が現前するという事態として成立し(p.58)」
「知覚が自らを生じせしめる要因として自己運動を含んでいるというのではない。むしろ、知覚はそれ自体、自己運動なのである(p.59)」
}
\begin{quote}
ゲシュタルトクライスの要点は、一切の生物的行為において知覚と運動が互いに一方を代理しうる2つの状態であること、この両者は常に相互に隠蔽されていること、このからみ合い、代理、隠蔽には主体と客体の両者も関与していることにある(\cite{weizsaecker:1950}, p.17)。
\end{quote}

有機体の知覚-運動の関係性は「フィードバック機構」のそれとはちがう。
情報処理のフィードバックは知覚と運動を互いに独立した入力・出力とするが、クライスには入力も出力もない。
有機体の知覚と運動は、いわば「回転扉の両面」のように相互に隠蔽しあい交互に現出しあいながら回っているような関係をなすとWeizsäckerは言う。
それがゲシュタルトクライスである\footnote{
ゲシュタルトクライスは機能環(\ref{subsec:umwelt}項参照)よりもいっそう知覚と運動が「ひとつのはたらき」であることを強調していると言える。
機能環が強調するのは、知覚と行為の連関が各生物によって「固有」であることであろう。
}。

木村\cite{kimura:2005}はゲシュタルトクライスの考え方を、現象学の祖・フッサール(1859-1938)の考え方と統合する。
フッサールは「思考」という現象は、\textbf{ノエシス}(「思考する」という作用)と\textbf{ノエマ}(思考された内容・対象)という2面から成り立つことを述べた。
これらは\ref{sec:shintaisei}節で登場させていた「認知作用」「認知内容」に対応する語である。
木村は、人間のノエシス(思考する)とは、単に「あたまのなかだけで起こる精神的作用」ではなく
まさしくこのゲシュタルトクライス(知覚する-運動する)なのだと論じた
\footnote{たしかに私たちが日常で思考するとき、ロダンの『考える人』のようにじっとして思いめぐらすだけでなく、動き回りながら思考する\cite{saeki:1990}であろう。}
。
木村は音楽演奏を例にとる。
音楽演奏では、まさしく演奏者は、メロディを聴きとりながら(知覚)メロディを奏でている(運動)。
メロディとは過去・現在・未来をふくみもったなにか(ゲシュタルト)にほかならず、知覚運動カップリング(=ノエシス)のさなかでこそ、演奏者はメロディを感得できるのである(=ノエマ)。
木村はこれについて「演奏するというノエシス的行為が音楽のノエマ的表象を意識に送り込むのではあるけれども、ノエマ的な音形態を知覚しないで演奏行為をおこなうことは不可能(\cite{kimura:2005}, p.52)」だと説明している。
すなわちノエシスとノエマは、ノエシスがノエマを生み出すのと同時に「ノエマ的面がノエシスを限定する」という「相互限定」的な関係にある。

\subsection{身体で問う}
\label{subsec:questioningwithbody}
こうして、思考する、知覚する、行為する、という3種の作用は、「3つ独立した作用がインタラクションしている」
というよりも「本来的に三位一体だが、見かけ上3つの作用になっている」という関係としてとらえることが可能になる
(\ref{sec:embodiedwisdomlearning}節で頭出ししておいたことである)。
これが「認知」である。

本研究では、「生きている」ことにおける「認知」の説明をベースにしながら、「生きている」よりも主体性の高まった「学ぶ」における場合に転用して考えよう。
「認知」は「\textbf{身体で問う}」(という認知)と呼ぶべきものになる。
一点注意しておきたい。
上記した木村の音楽演奏の事例だと、3種の作用が「うまく滑らかに連関している」にみえるかもしれない。
ところが、一般に学習者が「ああでもない、こうでもない」というふうに身体で問うとき、必ずしもそうはならない。
むしろ、3種の作用は「もつれあう」ようにしてはたらき、「ぎこちない」関係でありうる。
そのもつれあいにおいて、新しく認知内容を創りなおし/気づきなおしてゆくのである。
「反省」は頭のなかだけで起こる活動ではなく身体的行為そのものだ、とするヴァレラ\cite{varela_et_al:1991}の考え方にも通底する。
身体で問うことの概念モデルを\autoref{fig:noesisnoema}に示す。

\begin{figure}[h]
  \centering
  \includegraphics[width=\textwidth]{./images/noesisnoema.pdf}
  \caption{「身体で問う」ことの概念モデル}
  \label{fig:noesisnoema}
\end{figure}
\begin{figure}[h]
  \centering
  \includegraphics[width=0.4\textwidth]{./images/cognitivecoupling.pdf}
  \caption{認知カップリングの概念モデル\cite{suwa:2016}}
  \label{fig:cognitivecoupling}
\end{figure}

図の描き方にこめた意味を解説する。
認知は、認知するというノエシス(認知作用)とノエマ(認知内容)からなり\footnote{
  「問エシス」と「問エマ」という用語を定義してもよいのかもしれない。
}
、両者が相互に限定しあっているという関係性を、円(枠線)と円の内部領域との関係によって表現している。
認知するというノエシスは、知覚する-行為する-思考する、という三位一体のひとつの作用であることを、(ゲシュタルトクライスの図に倣って)3分円が3つでひとつの円をなす関係で表現している。
完全な円として閉じきらず「互い違い」になっているのは、もつれあいの関係性を表すためである。
言うなれば、3種の作用のもちつ・もたれつ・もつれあう関係性を図示している。
そして、図に薄いグレーで描いたクエスチョンマーク(?)は、いまその瞬間には本人が窺い知ることのできないなにがしかである。
このあともつれによって生じたすきまから、新しいなにかとして問いへと流れこみうる存在である。
\footnote{
「?」は、郡司\cite{gunji:2019}がいうところの「外部」に相当しうるものだろう。
この図をみて、枠線の外になにかがありうること、それが「外部」たりうることを指摘してくれたのは
山本篤氏や児玉謙太郎氏である。
}。
思考内容や知覚内容と呼びうるものは、\autoref{fig:noesisnoema}内ではノエマに相当する(このことは、第二部の研究で扱う「表情」\cite{hiromatsu:1989}とも本質的にかかわる)。

\autoref{fig:noesisnoema}は、
\autoref{fig:cognitivecoupling}の諏訪による\textbf{認知カップリング}\cite{suwa:2016}(思考・知覚・行動の三作用を互いに影響を与えあう関係全体)を改変したものでもある
\footnote{
諏訪がこのように認知カップリングのモデルの図を描いたのは、「情報処理モデル」を批判的に乗り越えるという強いねらいがあったとのだと著者は察する。
情報処理モデルとなるべく対応づけながら認知プロセスを説明しやすくするように描いたのだろう。
いっぽうで、両モデルを丁寧に対応づけようとしすぎているせいなのか、
図の描き方や説明に、3つのはたらきが独立した作用であるかのようなニュアンスを著者は感じる。
\autoref{fig:noesisnoema}の概念モデルは逆にその部分が肝でもある。
}。
諏訪が情報処理モデルを乗り越える認知カップリングの図によって強調したことのひとつは、「思考が知覚を変容させる」という点である。
\autoref{fig:noesisnoema}・\autoref{fig:cognitivecoupling}はどちらも共通して「思考」が知覚(というか認知全体)をつくることを表す図になっている。
この点は、\ref{sec:shintaisei}節後半で指摘したように、身体性認知も必ずしも積極的には扱っていないことである。

\subsection{身体知の学び:環世界を主体的に創りかえゆく営み}
\label{subsec:shintaichinomanabi}
\autoref{fig:noesisnoema}で確認したように、「身体で問う」はすでにプロセスの萌芽でありうる。
身体知はプロセスのさなかに存在する。
身体で問うことで、新たな問いが(身体で問う営みが)生まれるからである。
その小さなプロセスがつむがれることで「身体知の学び」と呼びうるプロセスとなる(\autoref{fig:shintaichinomanabi})。
問いが問いを生むということは、「身体知がプロセスにならざるを得ない」のであると同時に、
「主体的にプロセスを生み出してゆく」ことでもある。
学習者は、学びがい\cite{saeki:1995}を求めて、より望ましい自身のありかたを渇望して、学びたいから、学ぶ。
ぎこちなくても、それは主体的にやっている。
生きながら学ぶ。学びながら生きる。
学びは一般に、終わりなきプロセスなのである。

\begin{figure}[h]
  \centering
  \includegraphics[width=\textwidth]{./images/shintaichinomanabi.pdf}
  \caption{身体知の学びの概念モデル}
  \label{fig:shintaichinomanabi}
\end{figure}

身体知の学びは、「環世界を主体的に創りかえてゆく」営みでありうるだろう。
学習者は、みずからの環世界の内側から、みずからの環世界を揺さぶるのである。
それは少なくとも、「思考」があるからこそ、主体的に「思考」を変えることで、身体で問うこと(思考と知覚と行為とその全体)そのものを変容させられるのであろう。
運動学習者はそのようにして、試行錯誤的に身体で問い続ける(\ref{chapter:hajimeni}章参照)。
主体的に、生きながら学び、学びながら生き、みずからが生き・学ぶ環世界を創り変えてゆく。
そうして「意味」が醸成されてゆく。
その意味こそが、身体知と呼びうるものなのである。
\autoref{fig:shintaichinomanabi}で主張していることは、\ref{subsec:constructionloop}項で述べる中島・諏訪・藤井によるFNSダイヤグラム(\autoref{fig:fnsdiagram})とも本質的には通底していると著者は考える。

本研究の第一部研究は、\autoref{fig:shintaichinomanabi}の意味づくりプロセス全体のありようを探究するものであり、
第二部研究はそれにくらべると\autoref{fig:noesisnoema}の意味づくりの源のありように力点を置きつつ探究するものと言える。


