% --------- 読み方ガイド / 凡例 -------------
\chapter*{本論文の読み方・凡例}
\addcontentsline{toc}{chapter}{本論の読み方・凡例}

\begin{description}
  \item[人名表記]  よく知られた人物は、英語ではなくカタカナで記す。
        初出時に英語表記(括弧)を併記する:
        例)\textit{デカルト}

  \item[人名への(生年-没年)付記]  すでに亡くなっている人物については、適宜、人名に(生年-没年)を付記する。なるべく時代背景からも把握しやすくするためである。

  % \item[引用スタイル]  参考文献は \textsc{Bib\TeX} (スタイル \textsf{apalike})を用いる。
  %       複数文献の同時引用は \verb|\cite{key1,key2}| とする。

  % \item[図表番号]  図は「図\,1.1」、表は「表\,2.3」。
  %       本文中では \verb|\autoref| コマンドを用い、自動で「図」「表」を付す。

  % \item[略号]  AI(Artificial Intelligence)、VR(Virtual Reality)など。
  %       略号リストは付録\,A 参照。

  % \item[強調方法]  概念語は *イタリック*、実験条件名は **ボールド** 。

\end{description}

% \bigskip
% 本論では読者が章ごとに参照できるよう、冒頭に目的を、末尾に要点を示す。
