\chapter{目的と方法論}
\label{chapter:mokutekiandhouhou}
\section{本研究の目的}
本研究の目的は、運動学習を身体知の学びとしてとらえ、一人称研究によってその現実を描き出すことである。
それはつまり、
運動学習者が主体的に生きるなかで、学習対象の運動と自己の関係性を身体で問うことをとおして、意味を醸成してゆく(みずからの意味世界を創り変えてゆく)さまを、一人称研究によって記述することである。

第一部では、陸上十種競技である著者自身が「アスリートとして生きる」さま、すなわち、競技と生活が渾然一体となって、みずからの「走り」をより望ましいありかたへと創り変えてゆく意味づくりのプロセスを、描き出す。

第二部では、動いている身体の「表情」が意味づくりの源たりうるという思想のもと、動いている身体の「表情」の感得をうながす
ツールをデザインし、ツールの制作とツールをもちいた実践をとおして、「表情」から意味づくりするというのがいかなることなのかを探究する。
\section{構成的ということ}
\subsection{構成のループ}
\label{subsec:constructionloop}
身体知の学びを探究する本研究では、研究方法論や本論文の書き方など、さまざまな面にわたり\textbf{構成的}\cite{nakashima_et_al:2008,suwa_fujii:2015,suwa_hori:2015,suwa:2022}という態度を大事にする。
構成的とはどういうことか。
文献\cite{nakashima_et_al:2008,suwa_fujii:2015,suwa_hori:2015,suwa:2022}に拠りつつ説明する。

諏訪・藤井は、
ある知識や情報の意味することを、みずからのからだを使うことをとおしてもしくは生活文脈の実感に照らして(根ざして)わかることの重要性、
そのようなしかたでわかろうと試行錯誤するプロセスの重要性を説く。
そしてそれが、「ものごとを自分ごととしてわかる営み(=学び)」であると同時に「新しいものごとを生み出す営み(=デザイン行為)」
でもあるのだと、諏訪・藤井\cite{suwa_fujii:2015}は説いている。
このことは、本研究で述べてきた「身体で問う」という営み、身体知の学び、とも概して同じ主張である。

現代の生活では私たちはややもすればそれを怠ってしまいがちだと諏訪・藤井は警鐘を鳴らす。
たとえばだが、私が暮らす家の台所のキャビネットのデザイナ-には、それが十分にできていなかったのではないかと私は思う。
キャビネットの扉(\autoref{fig:cabinet})は開け閉め機構がローラーキャッチ式になっており、開け閉めするのに結構力がいる(固い)。
なのに、ツマミが小さく、くぼみもほぼなく、非常につまみづらいのである。
台所は料理をする場所なのだから、手が油や洗剤などでヌルヌルになっていることは「常」である。
料理中、キャビネットに収納した醤油や油を取り出そうとこのツマミを引っ張ろうとするのだが、
うまく引っ張れず、開けられないのである。
台所とはどういう場所なのか、デザイナーが自身の生活文脈に照らして考えていれば、こういうツマミは作らなかったのではないかと思う。
少し吟味すれば気づけたろうに、まるで「\ruby{他人}{ひと}ごと」である。
この事例は余談ではない。
書籍\cite{suwa_fujii:2015}に書かれた「みずからの生活文脈に照らして考える/考えられていない」とはどういうことなのかを、私自身が、みずからの生活文脈に照らして考えてみる、
ということを、私は実践してみたのである。
私は、私の卑近な例を引き出すことによって、その一端を読者に示したのである。

\begin{figure}[h]
  \centering
  \includegraphics[width=0.4\textwidth]{./images/cabinet.pdf}
  \caption{キッチンキャビネットの引っ張りにくいツマミ}
  \label{fig:cabinet}
\end{figure}

中島・諏訪・藤井\cite{nakashima_et_al:2008}は、こうした試行錯誤プロセス、それすなわちプロセスのなかで記号と実体とを共創的に変容させていくことを、\textbf{構成のループ:FNSダイヤグラム}として一般化している(\autoref{fig:fnsdiagram})。
\begin{figure}[h]
  \centering
  \includegraphics[width=\textwidth]{./images/fnsdiagram.pdf}
  \caption{構成のループ:FNSダイヤグラム(図は著者が描き直した)\cite{nakashima_et_al:2008}}
  \label{fig:fnsdiagram}
\end{figure}
\autoref{fig:fnsdiagram}を、\ref{sec:myexample}節で挙げた、私がより良い走りを身につける試行錯誤の事例で説明してみよう。
私は「接地位置を手前に引き寄せて真下接地を実現するために、骨盤を転がす意識で走ろう)」という構想(未来ノエマ)を抱いていた。
実際にそういう意識でもって\textbf{とりあえず}\footnote{
  \cite{suwa_fujii:2015}には、「とりあえず」が重要語句として索引に登録されている。
}現実世界(実体レベル)で走りを試してみた($C_1$)。
実際にやってみると、現実世界では、予期していなかったインタラクションが巻き起こった($ C_{\sqrt{2}} $、図の雲型部分)。
ここで\autoref{fig:fnsdiagram}内の「実体レベル」部分の描き方を説明しておく。
多数の小さな丸がネットワーク上に結びついているが、丸が実体レベルの要素、それらをむすぶ線は要素どうしの関係性である。
中央部にある楕円は、未来ノエマにもとづいてつくられた現象をさす。
雲型に描くことによって、この範囲を$C_1$以前に予め規定・予期することができないことを表している。

事例の説明にもどろう。実際に走ってみたことではじめて巻き起こったインタラクション($ C_{\sqrt{2}} $、図の雲型部分)は、少なくとも次のことを孕んでいた。
接地時のブレーキを減じることができたいっぽうで、「膝が曲がりすぎて」しまっていた。
膝の過剰な曲がりは、速いスピードで走ったときに、スイングする脚が前に出てこずにつんのめって転びそうになってしまう、という事態を招いた。
そしてこのことを認識するはたらきことこそ($C_2$)であり、私はインタラクションから現在ノエマを生成したのである。
図では、$C_2$は雲型のインタラクション全体を出どころとした矢印である。
現在ノエマとは、\autoref{fig:noesisnoema}でいうところの(「身体で問う」ことにともなう)「問い」に相当する。
私はこの現状認識をもとに、「じゃあ、接地時のブレーキを利用して、鞭運動のようにスイングする脚が自然に前へ振り出されるような練習をしよう」
といった構想(未来ノエマ)をつくり出したのだった。これが$C_3$である。
そうしてふたたびやってみて($C_1$)・・・(以下略)といったプロセスをなすのが、FNSダイヤグラムである。

FNSダイヤグラムはこのようにサイクリックなプロセスである。
同時にFNSダイヤグラムは、フラクタル的であることも藤井・中島・諏訪は説く。
たとえば「骨盤を転がす意識で走る」$C_1$というなかでも、より細かなFNSループは存在している。
FNSダイヤグラムは上記したように「身体知の学び」を説明する図にもなる\footnote{
  FNSダイヤグラムのフラクタルであることを考えれば、身体知の学びだけでなく、
  身体知の学びのより細かいプロセス単位である「身体で問う」ことにも相当する。
}。
これが構成的ということである。

研究するという営みも、デザイン=学び、構成的な営みであると諏訪・藤井\cite{suwa_fujii:2015}は論じている。
自然科学をはじめとした多くの学問では「分析」を示したり、分析的な論文の書き方をするが、
研究という営みそのものは構成的なはずだ、と中島は以下のように鋭く指摘する。
\begin{quote}
  一旦理論化してしまえば客観的評価の土壌に乗せますが、理論化自体は研究者の一人称的プロセスです。
  研究者は苦労して理論を構築した後で、それがさも最初からあったかのように演繹的に論文を書くのです、
  科学研究は客観的かつ演繹的になされると思っている人もいるかもしれませんが、実際はそうではありません(後略)
\end{quote}
\subsection{構成的という態度をふまえた本論文の書き方}
こうした構成的という態度を手放さないために、本論文では随所で、著者自身が研究者・生活者・実践者として本研究を進めてゆくようすをも語ることにする。
% 第一部\ref{chapter:monogatarizenshi}章や
第二部\ref{sec:tetsugakujousei}節はその顕著な例である。
第二部\ref{sec:tetsugakujousei}節について言えば、
著者は、第一部の研究を遂行することをとおして、実践者かつ研究者として、運動学習にかんする問題意識を醸成した。
その問題意識を足場にしたからこそ著者は第二部の研究の手がかり(それが哲学概念「表情」\cite{hiromatsu:1989}である)をつかみ、第二部研究着手に至ったわけである。
そういうプロセスを(概要的にではあるが)示したのが第二部\ref{sec:tetsugakujousei}節である。
その他、本論文の随所で、構成的という態度を意識し、通常の論文では省かれるであろうものごとであっても、積極的に書くことがある。
\section{一人称研究}
\label{sec:ichininshokenkyu}
\subsection{一人称研究の思想}
\textbf{一人称研究}は、ひとの知を研究するとき、客観性や普遍性や再現性を重んじる自然科学的な方法論(これが三人称研究)だけに縛られていては漏らされてしまうものごとがある、という問題意識の上に立つ。
2010年代に日本の人工知能学会から起こったムーブメントである\footnote{
ここでいう「かしこさ」とは、80年代の第二次AIブームでAI研究がぶち当たった「フレーム問題」や「記号接地問題」に代表される諸問題である。
2010年代は第三次AIブームの時期にあたり、その中核技術である機械学習が産業界へ続々と応用されはじめた頃であった。
}。
一人称研究とはなにか。
書籍『一人称研究のすすめ』\cite{suwa_hori:2015}では、次のように書かれている。
\begin{quote}
  あるひとが現場で出合った モノゴトを、その個別具体的状況を捨て置かずに、一人称視点で観察・記述し、そのデータを基に知の姿についての新しい仮説を立てようとする研究
\end{quote}
一人称研究は構成的な研究手法と言える。
一人称研究には2種類のスタイルがあると諏訪・藤井\cite{suwa_fujii:2015}は説いている。
\begin{itemize}
  \item 研究者が自分のからだや生活そのものを研究対象にする(研究者と被験者が同一)
  \item 研究対象は自分のからだや生活ではないが、研究対象者の固有名詞としての「顔」が明確に見え、さらにその対象者の知(もしくは知を発揮もしくは獲得するプロセス)を研究者が自分のからだや生活実体に対応づけて研究する
\end{itemize}
本研究第一部研究は、前者のスタイルである。
第一部の研究は、著者自身が対象者であり、陸上十種競技選手として、からだメタ認知を駆使して「走り」の学びを実施した一人称研究である。


第二部研究は、動いている身体が醸し出す「表情」\cite{hiromatsu:1989}を意味づくりの源として位置づけ、表情の感得を促すツールをデザインし、ツールをもちいた学びの実践をおこなう。
実践の対象者は、著者自身ではない他の対象者である。
主たるデータは、対象者がアプリをもちいて身体で問うた記述データである(次項で説明するからだメタ認知記述)。
対象者らが意味づくりしながら残した記述データをもとにして、陸上十種競技の一人称研究をやってきた著者が、その意味するところを解説をする、というやりかたを採る。
こうした入れ子になっている意味の記述は、佐伯がいうところの\textbf{二人称かかわり}\cite{saeki:2017}に符号する。
「園児のみている世界を保育士がみる」というふうに、私たちは他者の世界に「共感」するようにかかわることをとおして、他者の意味世界を解釈することができる\footnote{
  佐伯は、文献\cite{saeki:2017}のタイトルとして「『子供がケアする世界』をケアする」という入れ子になった命名をしている。
}。それが二人称的かかわりである。共感は同感とはちがい、当人の「思ったとおりそのまま」のことを読み取るわけではない。

第二部研究が一人称研究か議論の余地がある(上記した2つ目のスタイルの一人称研究かどうか)が、本研究では、一人称研究としては扱わないものとする。
前掲書\cite{suwa_hori:2015}の編者かつ著者である諏訪は一人称研究を推し進めてきたが\footnote{
  一人称研究は、ここ10数年をかけて、徐々に市民権を得てきている。
}、
近年諏訪が提唱する一人称研究\cite{suwa:2022}は、「構成的」という点を強調しているからである。
すなわち、研究者が研究という学びの営みを、やってみては考え、またやってみて・・・という構成のループをまわしながら、
問いを記述し、記述データを蓄積させてゆく研究が、一人称研究だということである。
これに照らすと、第二部研究は、著者がみずからの研究・デザインのプロセス(ツールをつくるプロセス)を日々一人称視点から問うたデータを扱うわけではない。
この点に重きを置けば、第二部研究は一人称研究ではないということになる。

またもし、第二部研究を構成的な研究と呼びうるとするならば、それは、哲学概念「表情」\cite{hiromatsu:1989}に着目して、それを机上の哲学に終始したわけではなく、
じっさいにアプリケーションにつくり、それをもちいた他者の学びの実践を試みたという意味においてであろう。

\subsection{からだメタ認知}
\label{subsec:embodiedmetacognition}

本研究は、身体知の学びのデータを取得する方法として、認知的手法:\textbf{からだメタ認知}\cite{suwa:2016, suwa:2022}をもちいる。
からだメタ認知は、「身体で問う」ことを促しながら、その内容を記述・記録する。
「メタ」という語句を意識して言い換えるなら、「ことばの力を借りてからだ認知をからだ認知する」手法とも言えよう。
ここでいう「からだ認知」とは、\ref{subsec:noesisnoema}項で述べた「認知カップリング」のことであり、学びにおける「身体で問う」のことである。

からだメタ認知でことばにする対象はからだ認知すなわち「自身がなにをどう思考・知覚・行為しているのか」である。
書かれた内容は、(身体での)「問い」として書かれているわけであるが、
問いにもさまざまありうる。
浦上\cite{uragami:2015}は、問いの種類を、〈感触〉、〈違和感〉、〈疑問〉、〈解釈〉、〈分析〉、〈仮説〉、〈問題点〉、〈問題意識〉、〈目標〉という9種類のカテゴリーに分類できると述べる。
このリストは概して、暗黙的・身体感覚的な問いから明晰的な問いへ、という順番で並んでいる。
からだメタ認知においては、〈問題意識〉や〈仮説〉といったはっきりした思考の形をとる問いだけではなく、
〈違和感〉といった曖昧模糊とした体感的な問いも、(内容の正しさにこだわらずに)積極的にことばとして絞り出して書きつけることが重要である。
「走り」を学ぶアスリートの事例で考えよう。
「走りのフォームの悪い動きは、歩きのフォームに原因がある」は〈問題意識〉に相当しうる問いである。
「接地直前に骨盤をスイッと前へ滑らせれば、ブレーキを減じられるだろう」は〈解釈〉に相当しうる問いである。
「いつもより肩甲骨の引かれかたがネバっこい気がするぞ?」は〈違和感〉に相当しうる問いである。

からだメタ認知の意義は、知覚・思考・行為の記録を残すだけではない。
このようにことばとして記述することで、思考内容(認知内容)は「外的表象化」される。
書き手は外的表象化されたことばと、否応なくインタラクションする(ことばは文字として知覚可能な対象になる)。
ことばにしてみると、推論・連想的にさらに新たなことばが紡がれ(=ことばがことばを生む\cite{suwa:2016})
、知覚をも変容させ、それが原動力になって、さらに新たな問いや行為を生成することができるのだ。
学習者の生きる環世界に、(今までは自覚していなかった)新しい変数が流れこむ。
それは\autoref{fig:shintaichinomanabi}で「?」として描いたなにかである。
上記例で言えば「肩甲骨の動きの粘り気」といった変数がたとえばそれに相当し、ことばを紡ぐなかで生まれ、変数として象られた身体感覚でありうる。
だからこそ、曖昧な問いであっても、その内容が正しいかどうかにこだわらずに、とりあえずことばにしてみることが肝要なのである。

からだメタ認知は、従来型メタ認知\cite{flavell:1979}とは大きく異なる。
従来型メタ認知の目的は「自らの思考を客観的にモニタリングする」ことである。
従来型メタ認知がメタ認知する内容は「思考」だけであり、「客観的に」とあるように、「ことばがことばを生む」ということは、つまり主体がことばに巻き込まれるといったことがねらいには含まれていない。

このようにして、ことばの力に半ば身をあずけながら、
ことばの力を積極的に利用しながら認知を進化(深化)させようとするのが、からだメタ認知である。
からだメタ認知は、生き・学ぶなかで身体で問う様相を観測・記述を残しながら、同時に、生き・学び・身体で問うことそのものを促す、構成的な実践手法ともなる。(\cite{suwa_fujii:2015}, pp.175-190)。
からだメタ認知の身体知の学びへの有効性は、ボウリングやダーツなど、多種の学習ドメインで例証されている\cite{suwa:2016}\cite{suwa:2022}。

\subsection{補足:「ことば」の位置づけ}
最後に、前項でも触れたことだが、本研究の「ことば」の位置づけを述べておく。
「思考」が認知を形づくることは\ref{subsec:noesisnoema}〜\ref{subsec:shintaichinomanabi}節にて述べたが、
それは、学びという営みにおける「ことば」の位置づけに直結する。
つまり、ことばは認知をつくる。
ことばは、知覚をつくり、行為をつくり、
ことばにするという営みそのものが認知であり、新たなことばを生み、認知をうむ。

ここまでで「言語化」と呼ばずに「ことばにする」という言い方をしてきたが、それにはこだわりがある。
言語化と呼ぶと、つい、従来型メタ認知のような、頭のなかでだけ起こるプロセスであるとか、「思考を思考する」といったふうに誤解してしまいかねない。
それはなんとしても避けたいのだ。

あえてこう言い切ってしまってもよいだろう。
暗黙知\cite{polanyi:1966}は「言語化できない」からこそ「ことばにする」のである。

こうした考え方、たとえばことばが知覚を変容させる、という点については、
諏訪\cite{suwa:2016}が挙げるダルメシアンの例を引いておく。
私たちは、曖昧なドットが集合した白黒画像をみせられてもなにもこれといった知覚できないけれど、
そこでさらに「犬がいます」と言われた瞬間(=ということばを念頭に置くと)、白黒のドットパターンに「ダルメシアンの姿」をみてとってしまう。
また、虹は「赤橙黄緑青藍紫」という語彙をもつ日本人には7色にみえるが、「Roygbiv\footnote{
アメリカ人は通常虹は、Red, Orange, Yellow, Green, Blue, Violetの6色だと習う。
}」の語彙をもつアメリカ人には6色に見える、といった言語相対主義的な事例にも端的に表れている。

長滝\cite{nagataki:1999}も知覚とことばの関係性について(文字通り『知覚とことば』と題した文献のなかで)以下のように述べている。
\begin{quote}
  知覚的認識と言語的認識は、それらのあいだにはっきりとした境界線が画定されえないほど、相互に影響しあいまた共働しているとさえいえる。
だからこそ、知覚世界の構造が言語によって再構造化されることもあれば、言語が知覚世界のうながしによってあらたな意味を獲得することもある。
(中略)
知覚的認識と言語的認識との境界が曖昧であるのは、それらがつねに相互構成的な創造的循環のなかに置かれているからである。

(\cite{nagataki:1999}, p.174)
\end{quote}

ことばは、なにかを正確に表し、それを他者に伝えるための媒体、というだけではなく、上記した役割、
いわば、自分の身体を進化させるための媒体でもある。
からだメタ認知は、見過ごされがちなことばのその性質を積極的に利用しようとする認知的手法であり、
本研究は、ことばをそういうものとしても扱う。




