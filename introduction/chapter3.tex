\chapter{運動学習の既往研究}
\label{chapter:motorlearning}
本章では、既往の運動学習研究を概観し、身体知の学びとして運動学習をとらえる本研究の特殊性を示す。

\section{スポーツ科学的アプローチ}
最たるのはスポーツ科学的アプローチであろう。
たとえば小林ら\cite{kobayashi:2009}は、陸上短距離走のexpertとnoviceを比較実験し、スタートから9歩まで(これは加速区間の前半にあたる)は、expert群の方が地面反力の力積(つまり接地中に足裏が地面から受けた力の総和)の水平成分が大きいことを報告している。
このように、身体運動が満たしているべき客観的特徴の正解を示す。
たしかに運動学習者や現場のコーチはこうした知見を得ることで、みずからの客観的身体運動になにが足りないかを知り、それをもとに練習の指針を立てることもできる。
スポーツ科学は、このように、運動学習の一助となる知見を蓄積してきたと言える。

だが、この反面にも目をむけねばなるまい。
諏訪\cite{suwa:2016}は、\textbf{入力変数/出力変数}という用語でもって、およそ以下のような問題意識を表明している。
たとえば上記知見に触れた学習者Aが、「水平成分の力積が大きくなっている」走りを実現するには、
ただそのまま「水平成分の力積を大きくしよう」とか「足裏で地面を後ろ側へ押し込もう」と意識すればいいというわけではない。
学習者Aが「足裏が地面に与える力積水平成分」を大きくなる走りを実現したとしよう。
実はA本人からすれば、その走りを実現するのは(促すのは)「両肩甲骨をクッと寄せる」という独自な意識かもしれないのである。
ある全身運動を実現するために、なにを・どのように意識すればいいのかは、個人によって、状況によって、変わりうるものである。
これを入力変数と呼ぶ。
スポーツ科学の知見が示す運動の特徴は、「結果的に運動がそうなっているべき」という特徴(これが出力変数)なのであって、
必ずしも、あるひとりの学習者にとって、これを入力変数にすればうまくいく、ということまで明らかにするものではない。
一般に、学習者本人にとって、そのときにふさわしい入力変数は、自明なものではない。
入力変数は本人が主体的に試行錯誤的に探らねばならないものである。
だから身体知の学びは険しい道のりとなるのである。
こうした諏訪の主張に、本研究は賛同する。
研究者も現場の学習者も、入力変数と出力変数を混同すべきではないだろう。
なおここでいう「変数」とは、ギブソン夫妻\cite{gibson_gibson_1955}が述べた「着眼点」とでも呼びうる意味の変数である。

スポーツ科学の知見が身体知の学びプロセスに対して有する価値は、主体的な試行錯誤のための材料としてである。
スポーツ科学は「What」の知見を提供するとするならば、身体知の学びの意味づくりを扱う本研究は「How」の知見と言える\cite{suwa:2016,horiuchi_suwa:2020}。
\section{量的アプローチの運動学習研究}
\subsection{情報処理アプローチ}
「機械」としての身体の運動をどう制御するのか、
という視点から運動を探究するのが、Schmidt\cite{schmidt:1991}に代表される情報処理アプローチである。
概要は、前章で述べた情報処理モデルとそう変わらない。
このアプローチがとらえる運動学習とは、情報処理システムが運動制御関数のパラメータを「修正」してゆくことだと言えよう。

しかし、「パラメータを修正」だけでは、身体知の学びは十分にとらえられない。
前節にも登場した諏訪による「入力変数」の着眼・発見、という用語を引用すれば、そのことはわかりやすい\footnote{
  ただし、本研究では「入力変数」という用語そのものは積極的にはもちいないことにする。
  入力変数というと、入力→出力、といった「情報処理的なにおい」がぬぐえないからである。
  諏訪がそれでもなお「入力変数」と呼ぶことのねらいは、やはり、なるべく情報処理モデルに寄り添ったうえで、
  「でもそれだと限界がある」、という論理で身体知論を展開するためであろうと私は察する。
  本研究では、こうした変数は「入力」なのではなく「ノエマ」であるととらえる。
}。
すなわち、誤解を恐れずに言うならば、「変数そのものの発見」のと「ある変数のなかで変数値の発見(これがパラメータ修正)」はまったく異なるのである。
\subsection{ダイナミカルシステムアプローチ}
情報処理モデルだと、機械を制御する中枢(=小人・ホムンクルス)が、筋それぞれに対応する鍵盤を弾いて指令を出すといった構図になる。
ロシアの生理学者・Bernstein(1896-1966)\cite{bernstein:1996}は、
そのような情報処理モデルだと、現実の状況においては、所望の運動を実行するために各筋へのどのように指令すればよいかが、不良設定問題(文脈の多義性や関節の自由度などが絡む)となってしまうことを指摘した。

そうしてBernsteinは、末梢側がなんらかの自律的な組織化(シナジー)をしているという考えを示した。
ここにギブソンの、行為と知覚とは環境のなかでカップリングして成り立つという考えを盛り込みながら発展してきたのが、ダイナミカルシステムアプローチ(DSA)である。
DSAがとらえる運動学習とは、知覚-行為システムが、現在成り立っている全身の協調構造から、(より良い)新たな協調構造を(創発的に)獲得することだと言えよう。

これらは「認知作用」を記述していると言えるが、認知内容(ノエマ)を記述するものではない。
くわえて、これまで述べてきた3アプローチいずれも、「expert-novice間」や「何かの実践の前-後」間を比較するかたちで
学習の時間を止めて「差」を示すスタイルであるが、それは、「意味づくりプロセス」を陽には扱わないことになり、その点が本研究とは異なる。
\section{質的アプローチの運動学習研究}
意味的側面に質的にアプローチする研究群もある。
\subsection{現象学的アプローチ}
体育学の領域では、「意味」の観点から運動の学習と教育を体系化したスポーツ運動学がある。
K.マイネル(1898-1973)\cite{meinel:1960}は、フッサール(1859-1938)現象学やヴァイツゼッカーの仕事\cite{weizsaecker:1950}(\ref{subsec:noesisnoema}節にも登場)にも触れながら、
スポーツ運動を現実におこなわれている姿のままでとらえようとする\textbf{モルフォロギー}的方法(形態学的方法)の重要性を唱える。
この「形態」とはなにか。
マイネルと親交のあった金子(1927-2024)
\footnote{金子は1952年ヘルシンキオリンピックの男子体操に日本代表として出場した選手でもあり、のちに指導者としても活躍した。}
は、マイネルの「形態」を重視する運動学には、同じくドイツのゲーテ(1749-1832)による動植物の形態発生学が礎にあるとみた。
形態(gestalt)は、静的な物的の構造のことではない。
「発生」とあるように、成長の流れのなかにこそ在る「かたち」のことである。
形態は、客観的な観察だけで得られるものでもない。
形態は、具体的な運動のなかで、運動している本人の自己観察や、あるいは運動者に潜入するような他者観察によって、生き生きと感じることのできる次元にある。
このことは第二部で扱う「表情」の考え方にも通ずる。

マイネルは、モルフォロギー的方法によって、全身運動のもちうる「質」を、局面構造、運動リズム、連係、流動、正確性、調和、弾性、先取りといった徴表にカテゴライズした。
スポーツ運動を個別科学(解剖学的・生理学的・心理学的あるいは物理学・力学など)によって分析するにしても、
それらに先立って/それらの方法で分析する出発点として、まずはモルフォロギー的方法が必要不可欠なのだとマイネルは説く。
また金子は、フッサールの現象学をも足場にしつつ、マイネル運動学を理論的に発展させ、それを金子独自の「発生運動学」へと体系化した。
金子は「コツ」や「カン」を身体知であるとし、科学知に対するものとして身体知を考えた。

このように意味づくりの面に質的にアプローチしている点は、本研究との共通点と言える。
いっぽうで、金子らの研究スタンスには「体系性」(これすなわち普遍的な記述)をめざすむきが強い、という指摘もある。
この指摘は、美学者(なかでもプラグマティズムを引き継いだ美学)・教育学者としてスポーツを探究する樋口\cite{higuchi_et_al:2017}によるものである。
著者も樋口の指摘に賛同する。
本研究は、意味づくりのプロセスを質的に記述するが、個別具体性を重視して、可能なかぎり、現実に起きるできごとの複雑さや「手触り」を色濃く残すようにして記述することをめざす。
それは体系性を第一義とするものではない。
\subsection{状況論アプローチ}
学習科学の領域では、状況論的なアプローチからも運動学習をとらえている。
\textbf{状況に埋め込まれた学習}\cite{lave_and_wenger:1991}の考え方は、
「知識の獲得・技能の習得」といった学習観や
「学校教室のような脱文脈化された知識の教授」といった教育観を乗り越え、
学習・教育を、学習者が対象の(文化的)実践の共同体に参加し、そのなかで生きてゆくなかで、
成員としてのアイデンティティを作ってゆくことだと考える。
学習とは本来的に共同体のなかでの学び合いとしてみるのである。

生田\cite{ikuta:2007}は、日本の伝統芸能における学び(これらも運動学習と言える)に着目し、技の
伝統芸能では、師匠は弟子に「このように動きなさい」と知識教授することはあまりなく、
弟子は、師匠とともに生活をするなかで学んでゆく。
生田は、M.モース(1872-1950)による\text{ハビトゥス}の理論を足場に、その学びを説明する。
ハビトゥスとはなにか。
(生田に寄せつつ説明すれば、)歩き方や身振り手振りなどの人々のふるまいが、
運動生理学的な都合でそうなっているというだけでなく、
その人々が生きる文化・社会のありようや価値観に仕立てられた「身体技法」だと考えるとき、
そのふるまいをハビトゥスと呼ぶ。
つまり、弟子は、師匠とともに生活するなかで当のわざの「世界へ潜入」し、その文化や状況に身全体でコミットしながら、生きてゆく。
そうやってわざの表面的な「形」の模倣を超え、\textbf{型}(ハビトゥス)を習得しているのであるとし、
このことは、伝統芸能に限らず学びの本質なのだと生田は説いた。

学びを生きているということのなかで捉えようとする状況論アプローチの基本的態度は、本研究とも響き合うと言ってよい。
そのうえで、状況論アプローチは、社会・文化・共同体という全体で学びをとらえる向きが強く、
いわば「(広義の)他者との関係性」に重点がある。
本研究は\textbf{状況依存的な知}\cite{clancey:1997}の思想を受容しつつも、
むしろ状況とのインタラクションが、本人の環世界の一部にどう取りこまれてゆくのか、という部分に力点を置く。
「学習者本人にとってのみえ」として包み返すのが、身体知の学びとして扱う態度と言えるだろう。

重要なことに、状況論アプローチの「ことば」の位置づけは、のちに述べる本研究の「ことば」の位置づけとかなり近い。
運動学習の現場において、ことばの果たす役割は、知識を「正確に」伝えたり「教授」するためではない。
師匠やコーチといった、同じ学びの場に生きる他者との「対話」をうながすことである。
このような位置づけのことばを、生田らは\textbf{わざ言語}\cite{ikuta:2011}と呼んでいる\footnote{
\cite{ikuta:2011}ではわざ言語の明確な定義は書かれていない。
}。

「他者と対話」と書いたが、状況論では「自己」もまた他者的存在でありうる。
その事例は前掲書\cite{ikuta:2011}に書かれている。
元・陸上短距離選手で、北京オリンピック陸上男子4×100mリレーでアンカーをつとめ日本に銀メダルをもたらした朝原宣治氏の例である。
朝原氏は、前掲書著者の北村との対談において、自身が日々変化してゆく感覚と対話し、感覚を開拓するために「感覚ノート」を書き綴っていたことを述べている。
感覚ノートにはたとえば\autoref{fig:asahara}のようなことが書き綴られている\footnote{
  前掲書\cite{ikuta:2011}には感覚ノートの写真が掲載されており、著者がその内容を文字起こししたものである。
}。

\begin{figure}[H]
  \centering
  \begin{mynote}
    
  体がまっすぐ前を向き、足もまっすぐ前に出るようにキープし、それで力がぬけないように腹筋でおさえられ地面に力が加わる感覚。
  角度を感じ、うでの手のひらのもどりでコントロール。・・・(後略)

  \textbf{[朝原氏5/8の感覚ノートより抜粋]}

  肩を脱力し、自然にダラリとおろし、そのPositionから体が(肩)がブレてねじれないようにうでをふる。
  そのときに手のひらを体の真中線・・・(後略)

  \textbf{[朝原氏5/15の感覚ノートから抜粋]}
  \end{mynote}
  \caption{朝原氏の感覚ノートの一例}
\label{fig:asahara}
\end{figure}


「感覚ノートの意義はなにか」という問い対して、朝原は以下のように答えている。
\begin{quotation}
  時にはびっしりと毎日、感覚ノートを書いて自分の中でどのように動きが変化するのか、
  こうすればどのような結果が出るのかを自分で試して練習していました。
  こうして始めると自然と自分の体としっかり対話ができるようにもなりますし、自分の体調の変化やバランスが崩れてしまうと、何か見逃しがちなひらめきとか、そういうものにキャッチする力というのがなくなってしまうのです。

  ですから、こういうふうに頭の中にあると、何かふとしたことがきっかけで、
  「あ、これは面白いのではないか」とか、「あ、こういう感覚で走ってみようか」とか、
  「次はこういう意識で練習しよう」など、どんどんイメージが湧いたり、ひらめきが出てきたりしますから、感覚ノートに書いてじっくり考えてやるのは効果があるように思います。(\cite{ikuta:2011},p.285 )
  
\end{quotation}

くわえて朝原氏は、感覚ノートに書き残されたことばは、のちのちの自分にとってどういう意義をもつのかについても語っている。
当初は「これさえつかんでおけば、いける」といった「普遍的」なものを期待しつつ書いていた面もあったという。
しかし、のちに見返してみたときの自分は、書いた当時の自分から変化してしまっているから、そう簡単にことは運ばないという。
ノートに残されたことばの果たす意義はむしろ、書いた当時から変化した自分が、「そこから自分でやり直しする」ためのヒントとしてなのだと朝原氏は言う。

このように、状況論アプローチは、学びの場において、つねに変化しゆく感覚を開拓するための「ことば」でありうることを指摘し、
それは本研究の思想とも相通ずることである。
\section{本研究アプローチの位置付け}
以上より、「本研究からみた場合」には、各立場は、およそのところ、それぞれ以下のように運動学習をとらえている、と整理してもよいだろう。
\begin{description}
  \item [古い現場の考え方]学習者が考えることなしに何万回の反復によってフォームを矯正する
  \item [情報処理アプローチ]情報処理システム(≒機械のパイロット)が、機械としての身体を操縦する各種パラメータを、修正する
  \item [ダイナミカルシステムアプローチ]知覚-行為システムが、物的環境のなかで知覚行為の新しい協応パタンを、創発させる。
  \item [現象学的アプローチ]学習者が、意味をつくってゆく。(研究は、その普遍性ある質的モデル・体系を記述する)
  \item [状況論的アプローチ]学習者が、他者ふくむ状況(社会や文化)のなかで生き、学び合う。
  \item [本研究]学習者が、身体で問い、環世界を主体的に創り変える。(研究は、個別具体的なプロセスを記述する)
\end{description}


