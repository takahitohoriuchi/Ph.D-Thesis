% !TeX root = ../data.tex


\part*{序論}
\addcontentsline{toc}{part}{序論}
\chapter{はじめに}
\label{chapter:hajimeni}

\section{本研究の射程:身体知の学び}
\label{sec:embodiedwisdomlearning}

速く走ったり、美しく踊ったり、淀みなく鍋包丁を捌いたり、気分やシーンにあわせて自分らしく衣服を着こなしたり、ウイスキーを豊かに味わったり、抽象画を鑑賞したり、デスクまわりの設えをちょいと調整して居心地を一変させたり、絶妙な押し引きのあるトークを繰り広げたり、統計の分散分析で「2乗」の計算が含まれる理由を腑に落としたり・・・。

どれもが\textbf{身体知(embodied wisdom)}である\cite{suwa:2016, suwa:2022}。
知といっても、頭のなかに格納された知識・情報ではない。
だからといって、物的な身体がなすふるまい(運動)だけを指していうのでもない。
身体をそなえた私たちは、絶えず予測不可能な状況(環境含む)にさらされており、
知覚・行為・思考しながら(\ref{subsec:noesisnoema}項で後述するが、本研究では、これら3つを別個のはたらきではなく、ひとつの認知作用の別側面であると考える)
、なんとか状況をやりくり\cite{decerteau:1980}しながら生きている。
このプロセスのなかで知は創りだされてゆくととらえるのが身体知の考え方である。
身体知は、\ruby{身体}{からだ}をそなえた「わたし」と世界(環境)とのあいだのインタラクションあってこそ成り立ち、
いわば、みずからにとってより望ましい生を生きるための\ruby{術}{すべ}である。
その点で、身体知は、\textbf{個人固有性}と\textbf{状況依存性}という2側面をもちあわせている\cite{suwa:2016,suwa:2022}ものだと考える。
本研究は、ひとの知はすべからく身体知であるととらえる。

% 身体は、メルロ=ポンティが論じたように、自己の、世界の、あらゆる関係性の礎たりうるのである\cite{merleau-ponty:1967}。

ひとは\textbf{学ぶ}いきものである。
多かれ少なかれ「主体的」に身体知を創りながら生きているということである。
「自動的に身体知が生産されてゆく」とか「ただ生きのびるためだけにしょうがなく身体知を創っている」わけではない。
佐伯\cite{saeki:1995}は、ひとは「学びがい」を求めて学ぶのだと説いている\footnote{
  佐伯は、ひとは「学習する」のではなく「学ぶのだ」、という主張のなかで両者の相違について「学びがい」という言葉を持ちだす。
}。
90年代以降、学びの研究は盛んにおこなわれてきた。
学びの肝は「問う」ことにあり、といった考え方が多くの文献で論じられている(たとえば\cite{saeki:1995,suwa:2016,suwa:2022})。
そこでも「身体」の存在が重要視される。
身体知の学びの肝は、物的な身体-環境間インタラクションだけではなく、それにともない\textbf{身体で問う}ということ(\ref{subsec:noesisnoema}節で詳述)にあると考える。

主体的に生きながら学び、主体的に学びながら生きている。
ゆえに身体知の学びは、生きることと学習ドメインとの界面にある。
それは、学習者から切り離されて存在するような対象の「情報」を得ることではない。
ゲシュタルト心理学の語彙を借りて言えば、わたしの生きるということ全体を地にして、浮かび上がってくる図のようなものである。
わたしの生に照らされて浮かび上がるもの、それすなわち、「意味」である。

身体知の学びとは、いわば主体的に\textbf{意味を醸成する}営みである。
私たちはそうやって、自分なりの意味世界を生きている。
そして、みずからの生きる意味世界をより望ましいほうへと作り変えながら生きている。
身体知の学びは、創造のプロセスとも言える。

ここまでわざわざ「身体知とは〜」よりも「知を身体知とらえるとは〜」とまわりくどい書きかたをしてきたのは、次の誤解を避けたいからである。
「知には通常の知とそうではない知(=身体知)とがあって、その後者を研究するのですね」という誤解である。
あるいは「身体知とは、運動を主題とする知のことですね」とか、ひいては「体育に代表される実技的・副教科的な知のことですね」といった誤解である。
そうした領域限定的な知を指して身体知と呼んでいるのではない。
身体知とはそうではない。
「知をすべからく身体知としてみなおす」視点に立つのが、身体知の考え方である。
冒頭の事例からしても、いわゆる「体育」的なものだけが身体知ではないということがわかるだろう。
以降本論文で「身体知」と書けば、それは原則「知を身体知としてとらえなおす」ことを含意する。

本研究は、陸上競技アスリートやダンサーの運動学習を身体知の学びとして捉え、そのプロセスを探究するものである。
運動学習は身体知の学びとして恰好のドメインである。
運動学習者らは主体的に身体知づくりをおこなっているからだ。
彼らはより望ましい運動を目指して日々試行錯誤している\footnote{Simon\cite{simon:1969}によるデザイン(という行為)の定義にも符号するところがある。}。
% 「わたしのより望ましいありかたを目指して学ぶ」と言い換えてもよいだろう。
ダンサーであれば、より独創的であったり、より曲調や場の雰囲気への調和したり、より高難易度な、といった運動をめざすだろうし、
陸上競技アスリートであれば、少なくとも「より好記録」の運動をめざしている。
彼らはそういう目指すべきところに方向づけられながら、主体的に、自分なりの意味世界をより望ましいように創りだし/創り変えてゆく。
運動学習を身体知の学びとしてとらえるとき、当然、学習者は学びがい\cite{saeki:1995}を求めているということも重要になってくる。
学びのプロセスを生み出すことそのものも学びの目的にふくまれる\footnote{
  山登りになぞらえるならば、山頂まで自分の足腰で踏破してこそ山登りなのであって、山頂に至ることだけに目的があるのではない。
  ヘリコプターで山頂まで運搬されても、そこに手応えも満足感もなく、それは山登りとは呼び得ないだろう。
}ということだ(私は元・陸上十種競技のアスリートとしてそう信じている)。
単なる成果主義・実利主義的なプロセスとは異なるのである。
念を押しておくが、身体知研究である本研究が運動学習を扱っている理由は、「身体を使うから運動学習は身体知」と短絡的にとらえているからではない。
\section{運動学習の現場:著者が「走り」を学ぶ事例から}
\label{sec:myexample}
運動学習者は日々、より望ましい運動のありかたを身につけようと研鑽を続ける。
まずは運動学習の現場を見つめてみることから本論文の第一歩を踏みだそう。

陸上十種競技の選手であった私(著者)が走りを学ぶプロセス\cite{horiuchi:2016a}から一事例を紹介しよう。
十種競技は、走・跳・投にわたる計10種目の総合力を競う陸上競技の種目である。
十種競技選手として私は「走る」ことを不得意としており、速く走るための試行錯誤を日々、続けていたのだった。

速く走るためには、接地瞬間に身体重心が接地位置鉛直真上の近傍にあること(=「真下接地」)が重要である。
速い選手の客観的走フォームはこの特徴を満たしている。
私はこれが出来ずにいた。
「へっぴり腰」(身体重心が、接地位置鉛直真上より過度に後方にある状態)で接地し、接地自体がブレーキになってしまっていることを私は自覚した。
これを問題視した私は、\autoref{fig:kostubankorobashi}に示す様に、「接地位置をより手前へ引き寄せれば(\autoref{fig:kostubankorobashi}の局面1)、結果として真下接地に
なるのでは?」という仮説を立てて実践することにした。
\autoref{fig:kostubankorobashi}はその頃の私の走りの動画のキャプチャである。
私は走りを実践してはそれを動画に撮影し、実際の走りの感覚と動画との関係性を反省しながら、反省することを繰り返した。
試行錯誤をつうじて私は、「接地脚の上で骨盤を転がす」という私独自の体感(局面2〜3でそれが生じている)を編み出した。
それを意識しながら走ることで、手前に引き寄せられた接地を走ることで実現され、ブレーキを減じることに成功した。

しかし、すぐに弊害が見つかった。
その方法では、全速力で走ると接地そのものの力強さがなく、次第にうまく走れなくなったのである。
緩やかな走りの時点でもその原因がすでに表れていたことに私は気づいた。
接地直前に、接地位置を手前に引き寄せる動き(\autoref{fig:kostubankorobashi}の局面1)が、引っ掻くような弱い接地を生み(局面2)、地脚の膝が潰れるように曲がりすぎてしまっていた(局面3)のである。
全速力で走る場合には、これらの動きが原因で、スイングする脚が素早く前へ出てこなくなってしまい、うまく走れないのだと考えられる。
はたして私は、スイングする脚がもっと素早く前へ出るように、接地時のブレーキをうまく活用して、
ブレーキにともなって鞭運動の原理でスイングする脚が自然に前方へ放り出されるような練習をするようになったのだった(それはサッカーボールを蹴る、というトレーニングだった)。
このエピソードは、私が、「真下接地」という「情報」を、「接地脚の膝関節角度」や「スイングする脚」との関係で咀嚼し、自分の身体にとっての重要な「意味」として理解してゆく事例である。

\begin{figure}[h]
  \centering
  \includegraphics[width=10cm]{./images/kostubankorobashi.pdf}
  \caption{私が「骨盤を転がす」体感で以てゆったり走る様子}
  \label{fig:kostubankorobashi}
\end{figure}

これが運動学習の現実である。
運動学習プロセスは一朝一夕では成らぬ、険しいものである。
あちら立てばこちら立たずである。
きのうときょうでは、心身の状態はちがう。
元陸城短距離選手・朝原氏\footnote{
  元・陸上短距離選手で、北京オリンピック陸上男子4×100mリレーでアンカーをつとめ日本に銀メダルをもたらした。
  100mの自己ベストは10.02秒、言わずと知れた日本短距離会のパイオニアである。
}は、調子が良いときにはその身体感覚を克明に言葉で記述し、ノートに残していたのだが、その記述を調子が悪い別の日に読んでもそれがそのまま使えるわけではなかった、という実体験をインタビューで語っている\cite{ikuta:2011}。
他のケースもある。
「筋肉がつく」ことによっても、感覚は容易に変わってしまう\cite{ikuta:2011}ことを、
スピードスケートのコーチを務める結城氏\footnote{
  結城は、オリンピック日本代表選手団コーチを複数回務め、清水宏保選手や小平奈緒選手などの金メダリストをコーチングした実績をもつ。
  結城氏自身も、選手元スピードスケートの選手として、ワールドカップ3位の経歴をもち、さらに研究者でもある。
}はインタビューで述べている。


運動学習者らは、実際に運動しながら、「どう身体が動いているのか/どう動かすべきなのか?」、「どんな体感が生じているのか?」といったことについて問う。
まるで自らの身体の「声」に耳を傾けるようにして、彼らは問いを発するのである。
彼らは主体的に、「身体で問う」ているのだ。
そうして、「ああでもない、こうでもない」と、学習者は泥臭く試行錯誤し続ける。

\section{生きているなかで学ぶ}
\label{sec:konzenittai}
そして、運動学習のプロセスは、単に「練習場」だけで起こるわけでもない。
彼らは日常を生きている。
多かれ少なかれ、日常生活をも運動学習者として自覚的に生きうる。
\ref{sec:embodiedwisdomlearning}節で述べたことは、運動学習でも当てはまるのである。
これは本研究の核たる主張である。
例えば、陸上短距離選手であれば、玄関で靴紐を結ぼうと前屈みになったとき、短距離走のクラウチングスタート時の前傾姿勢に通底しそうな、腹筋群の新鮮な収縮感覚に気づいてしまったりする。
湯船に浸かりリラックスしながらその日の練習を振り返っているとき、練習時には混乱していたことがクリアにみえたりする。
そうやって運動学習者は、生活のなかでさえ問いを発し、自分なりの「意味」を醸してゆく。
望ましい運動をその身でつくりあげてゆく。

剣豪・宮本武蔵(1584-1645)は、兵法書『五輪書』\cite{miyamoto}にて次のように書き記している。
\begin{quote}
  常の心に替事なかれ。
  常にも兵法之ときにも、少も替らすして、こゝろを廣く直にして、きつくひつはらす、すこしもたるます、心のかたよらぬやうに、心をまん中に置て、心を静にゆるかせて、其ゆるきのせつなもゆるきやまぬやうに、能々吟味すへし
  (\cite{miyamoto}、 p.139)。
\end{quote}
兵法と生活を線引きせずに、常に「心を真ん中に揺らがせておく」という柔軟な心持ち、すなわち、そういう「生き方」をこの文章は示しているのだと、私は解釈している。

能を大成した世阿弥(1363-1443)は、能芸論『花鏡』\cite{zeami}にて、次のように書き記している\footnote{
  世阿弥の花鏡も宮本武蔵の五輪書も、晩年に執筆されたものである。
  ゆえにそこに書き記されているものごとは、彼らがそれぞれの道を極めて至った境地であろうと著者は解釈する。
}。
\begin{quote}
かへすがへす、心を糸にして人に知らせずして万能をつなぐべし。
かくのごとくならば能の命あるべし。そうじて即座に限るべからず。
日々夜々、行住坐臥にこの心を忘れずして定心につなぐべし
(\cite{zeami})。
\end{quote}

ここには、万能を一心につなぐ(それぞれの技や芸の\ruby{間}{ま}も心を切らさずつなぐ)ことを、行住坐臥(日常生活の所作)において意識すべしという極意が説かれているのだと私は考える。

宮本武蔵と世阿弥の弁に通ずるのは、生活と実践ドメイン(競技)は、本来的に渾然一体だという考えだといえよう。
運動学習者は、生活のなかで試行錯誤を繰り返す。
そうして、学習者は、自分なりの意味に彩られた世界、\textbf{意味世界}を創り、自覚的に生き、学ぶのであろうと本研究は考える。
本論文第一部の研究では、著者自身がアスリートとして自らが生きる「走り」の意味世界を創りだし・創り変えてきたのかを物語る。

\section{動いている身体の「表情」}
\label{sec:ugoiteirushintainohyojo_intro}
しばしば運動学習者は、\ref{sec:myexample}節の事例でも示したように、自身/他者の運動(の動画)を観察する。
観察は学びの重要な一局面である。
身体知の学びとして運動学習をとらえる以上、観察もまた「身体で問う」ことでありえ、身体で問うような観察が重要だと著者は考える。
本論文第二部の研究ではその問題に焦点をおく。

しかし、対象の身体運動を目の前にしたとき、身体で問うように観察することは容易ではない。
たとえば観察は、実際に走ったときの主観的な体感などを踏まえて、それとのずれを確かめるために、映像に映る客観的特徴をみてとろう、といった具合におこなわれがちである。
そういう観察は、明晰的な分析を可能にするという点ではたしかに有効ではありうるが、身体で問うことになるのかといえば、必ずしもそうではないだろう。
「なにかを学びとってやろう」と身構えることは、かえって、心身が居着くことになり、身体で問うことを阻害してしまうのかもしれない。
考えてみれば、そのときの観察も、「実際に走ったときの体感と照らし合わせてズレを確認しよう」といったねらい(=縛り)がはたらいている。
だから「客観的な姿すべてをそのまま」をとらえられているわけでもなく、無数に存在する関係性があるなかからごく一部だけをみてとっている、というのが実情であろう(観察とはそもそもそういう営みである)。

そこで本研究は、動いている身体の\textbf{「表情」}\cite{hiromatsu:1989}に着目する。
動いている身体の「表情」とはなにか。
詳細は第二部\ref{sec:hyojo}節で説明するとして、
ここでは、
「観察対象の動いている身体の生々しくありありと立ち現れてくるその姿(\autoref{fig:hyojointro}のような具合に)のこと」である、とひとまず説明しておこう。
\autoref{fig:hyojointro}の左は、ストリートダンサーが踊る場面であるが、この踊りを観察してみるとき、たとえば「火山噴火」のような情景をそこにみてしまう
\footnote{
  この左画像はストリートダンサー山崎が、自らのダンスの振り付け創作プロセスを探究した一人称研究\cite{yamazaki:2017}から拝借している。
  この振り付けは、音楽の大きく太いドラムの音に、山崎が火山噴火を想起しながら創作したものである。
  なお、画像加工の許可は本人から取得済である。  
}。
ダンスだと「表情」なるものがあろうことは受け容れやすいかもしれないが、実は、表情はあらゆる身体運動にあるのだと著者は考えている
(たとえば\autoref{fig:hyojointro}の右2つは著者自身の陸上競技のパフォーマンスである)。

身体で問うような観察とは、「表情」をみることなのだと本研究では考える。
身体で問うように観察することは、「表情の感得」をすることなのである。
「表情」は意味づくりの「源」たりうる現象である。
補足しておくと、「表情」そのものは運動学習に限る概念ではない。
コミュニケーションする身体(身振り手振り)はもちろん、植物にも、建築物にも、「表情」はある。
「表情」をめぐる問題は、生活全体のドメインに通底するものであるが、それを本研究では運動学習にもちこむ。
第二部研究では、運動学習者に、動いている身体が醸し出す「表情」を感得することを促す。

\begin{figure}[h]
  \centering
  \includegraphics[width=\textwidth]{./images/hyojointro.pdf}
  \caption{動いている身体が醸し出す「表情」}
  \label{fig:hyojointro}
\end{figure}


\section{本論文の構成}
本研究ではこうした身体知の学び(意味づくり)のありように迫る。
上記したような現場があるいっぽう、これまでの運動学習研究は、\ref{chapter:motorlearning}章で後述するように、必ずしも身体知の学びとしての姿十分に描いてこなかった。
本序論では、続く\ref{chapter:embodiedwisdom}章で、身体知の学びの考え方を、知の研究群のなかに位置づける。
\ref{chapter:motorlearning}章では、知の研究と接しながら蓄積されてきた運動学習の既往研究群を眺め、本研究の意義と目的をより明確にする。

本研究は二部からなる。
第一部は、著者自身が対象者となって実施した\textbf{一人称研究}\cite{suwa_hori:2015,suwa_fujii:2015,suwa:2016,suwa:2022}である(一人称研究については\ref{sec:ichininshokenkyu}で説明する)。
陸上十種競技を専門とする著者自身が「走り」の身体知をいかに学ぶのか、
競技のみならず生活をも渾然一体となってなす「アスリートとして生きる」\cite{horiuchi_suwa:2020}姿を、
その一人称研究の成果を\textbf{物語}\cite{horiuchi_suwa:2020}として語り描く。

第二部の研究は、動いている身体が醸し出す「表情」\cite{hiromatsu:1989}を意味づくりの源として位置づけ、表情の感得を促すツールをデザインし、運動学習を支援可能性を探究する。