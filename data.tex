\documentclass[a4paper,11pt]{jsreport}
% jsarticleは短めの論文につかう。jsreportは長めの論文
\usepackage{bm}
\usepackage[dvipdfmx]{graphicx}%図
\usepackage[dvipdfmx]{hyperref}%ハイパーリンク
\usepackage{longtable}
\renewcommand{\figureautorefname}{図}
\renewcommand{\tableautorefname}{表}
\renewcommand{\equationautorefname}{式}
\makeatletter
\@removefromreset{chapter}{part}
\makeatother


\usepackage{tcolorbox}
\tcbuselibrary{skins}


% メタ認知記述はグレー背景
\newtcolorbox{mynote}{
  colframe=black,        % 線の色  
  boxrule=0.5pt,         % 線の太さ
  colback=gray!10, 
  sharp corners,         % 角を直角に
  width=0.9\textwidth,   % 幅
  left=6pt, right=6pt,   % 内側余白
  top=6pt, bottom=6pt,   % 内側余白
  enhanced,              % 高度な描画
}

% 連続写真合成
\newtcolorbox{continuousphoto}{
  colframe=white,        % 線の色
    boxrule=0pt,
    colback=white, % ← 白背景に明示するのが安全!
    sharp corners,
    width=\columnwidth,
    % width=\textwidth,
    % left=4pt, right=4pt, top=4pt, bottom=4pt,
    before skip=1em, after skip=1em
}

% 表情エントリ
\newtcolorbox{hyojoentry}{
    colframe=white,        % 線の色
    boxrule=0pt,
    colback=gray!6, % ← 白背景に明示するのが安全!
    sharp corners,
    width=\columnwidth,
    % width=\textwidth,
    % left=4pt, right=4pt, top=4pt, bottom=4pt,
    before skip=1em, after skip=1em
}

\hypersetup{
    colorlinks=true, % リンクに色を付ける
    linkcolor=black, % 内部リンク(\refコマンドで生成されるものなど)の色
    citecolor=black, % 文献リンク(\citeコマンドで生成されるもの)の色
    urlcolor=blue   % 外部リンク(ウェブサイトのURLなど)の色
}
\usepackage{booktabs}%表を美しく描く
\usepackage{ascmac}
\usepackage{enumitem}%Q&Aを付録でかくための
\usepackage{bm}
\usepackage{amsmath}
\usepackage{todonotes}
\usepackage{okumacro}
\usepackage{listings}% ソースコード用
\usepackage{mdframed}
\usepackage{xcolor}% 色のパッケージ
\usepackage{color}
\usepackage{subcaption}
\usepackage{overpic}  % 図にナンバリング
\usepackage{caption}
\usepackage{tikz}
\usepackage{float} % ← プリアンブルに追加
% \usepackage{multirow} %グルーピングありの表
\usepackage{tabularx}





% ソースコード部分の色設定
\definecolor{bg}{rgb}{0.09,0.09,0.09}
% \newtcblisting{mylisting}{
%   colback=black,    % 背景色を黒に設定
%   colframe=black,   % 枠も背景と同色に
%   listing only,
%   listing options ={    
%     language=java,              % JSはなぜかJavaらしい?
%     backgroundcolor=\color{bg},
%     basicstyle=\color{white}\ttfamily,   % 基本スタイル
%     keywordstyle=\color{blue},           % キーワードスタイル
%     stringstyle=\color{orange},          % 文字列スタイル
%     commentstyle=\color{green},          % コメントスタイル
%     breaklines=true,                     % 長い行の折り返し
%     numbers=left,                        % 行番号の表示
%     numberstyle=\tiny\color{gray},       % 行番号スタイル
%     frame=single                         % 枠線のスタイル
%   }
%   left=1mm, % 内側の余白を調整
% }

\lstset{    
  language=java,              % JSはなぜかJavaらしい?
  backgroundcolor=\color{bg},
  basicstyle=\color{white}\ttfamily,   % 基本スタイル
  keywordstyle=\color{blue},           % キーワードスタイル
  stringstyle=\color{orange},          % 文字列スタイル
  commentstyle=\color{green},          % コメントスタイル
  breaklines=true,                     % 長い行の折り返し
  numbers=left,                        % 行番号の表示
  numberstyle=\tiny\color{gray},       % 行番号スタイル
  frame=single                         % 枠線のスタイル
}


% :::個別ファイル作業時のみ
% このdocumentclassからbeginまでを、プリアンブルと呼ぶ
% NOTE:パスを指定するとき、「./」は書いてはいけない!!!!
% \includeonly{
%   introduction/how2read.tex,  
%   introduction/chapter1.tex,
%   introduction/chapter2.tex,
%   introduction/chapter3.tex,
%   introduction/chapter4.tex,
%   part1/chapter1.tex,  
%   part1/chapter2.tex,
%   part1/chapter3.tex,
%   part1/chapter4.tex,
%   part1/chapter5.tex,
%   part2/chapter1.tex,
%   part2/chapter2.tex,
%   part2/chapter3.tex,
%   part2/chapter4.tex,
%   part2/chapter5.tex,
%   part2/chapter6.tex,    
%   conclusion/chapter1.tex,
%   conclusion/addendum.tex,  
%   conclusion/acknowledgement.tex,
%   } 
% :::
\begin{document}

\begin{titlepage}
  \centering
  {\LARGE 博士論文 2025年度\par}
  \vspace{2cm}
  {\LARGE 身体知の学びとしての運動学習の一人称研究\par}
  \vspace{1cm}
  {\Large -「アスリートとして生きる」ということの物語と、動いている身体の「表情」の感得を促すツールのデザインをとおして-\par}    
  \vfill    
  {\Large 慶應義塾大学大学院政策・メディア研究科\par}
  \vspace{1cm}
  {\Large 堀内 隆仁\par}  
\end{titlepage}

\pagenumbering{roman}  % i, ii, iii, ...
論文題目:
\vspace{0.5cm}
身体知の学びとしての運動学習の一人称研究
-「アスリートとして生きる」ということの物語と、動いている身体の「表情」の感得を促すツールのデザインをとおして-
\vspace{1cm}

要旨(和文):
\vspace{0.5cm}

本研究では、運動学習を、生きるなかで主体的に意味をつくってゆく「身体知の学び」としてとらえ、2種類の一人称研究による実践をとおしてその意味生成プロセスを記述した。
本論文は、構成的という態度をなるべく大事にしながら書き綴った。

アスリートやダンサーらの運動学習の現場をみてみよう。学習者らは、みずから身体を動かしたり、自他の運動を観察したりしながら、「ああでもないこうでもない」と試行錯誤的に、
主体的に問いを展開する実践的プロセスをとおして、自身に固有な(独自的な)「意味」をつくっている。
この試行錯誤的プロセスは、身体知の学びである。

知の科学の歴史をみてみると(序論\ref{chapter:embodiedwisdom}章)、
心と身体を切り離し、心に知を求める認知主義(情報処理モデル)が主流であった。
情報処理モデルでは、日常の現実生活において混沌とした状況から意味を創りだすひとの知をとらえることができない。
身体性認知の考え方は、認知主義に対抗し、知には身体も必要であり、脳・身体・環境の全体が相互作用するシステムによっていかに知が成り立つかを探究する(DSA)。
近年は脳神経科学をも巻き込むかたちで、脳-身体-環境全体からなる「認知作用」の機序や、それによってどう「認知内容」が生まれてくるか、というむきが強く、
認知内容そのものは、必ずしも主題的には扱われていない。

「運動学習研究群」に絞ってみると(序論\ref{chapter:motorlearning}章)、
運動学習を量的モデルで説明しようとするアプローチは、スポーツ科学をふくめ、情報処理アプローチであれ、身体性認知のアプローチであれ、
多くは「expert-novice間」や「何かの実践の前-後」間を比較して、学習の時間を止めて「差」を示すものが多く、
意味づくりプロセスを陽には扱わない。
質的に運動学習にアプローチする研究群もある。
現象学的なアプローチであるスポーツ運動学や、意味の側面に迫ってはいるものの、体系的で普遍的な学習プロセスのありようを説明しようとするむきも強く、個人固有な
状況論的アプローチは、意味生成プロセスは扱うにしろ、他者との関係性をふくんだ状況に力点があるという点で、「本人からみえる状況」に力点をおく本研究とは異なる。

そこで本論文ではまず実践に先立って、序論\ref{chapter:embodiedwisdom}章で「身体知の学び」という考えかたを呈示した。
運動学習者がする「問う」行為は、「頭のなか」だけで進行するプロセスではない。
「問う」ということがすでに身体的な認知作用であり、
「問う」と相互限定するかたちで「問い」という認知内容が生じているのである。
相互限定は、互いに収束させあうだけでなく、むしろ、もつれあい・ずらしあう関係でもありうることに着目した。
このように問うことを、本研究では「身体で問う」と呼んだ。
これの理論的背景には、木村敏の論じる主体と世界との関わりの一般原理--フッサールやヴァイツゼカーやユクスキュルの思想を統合したもの--がある。
また、ユクスキュルの環世界論を参照して、運動学習における認知は、意味づくりの様相は「個人固有的」でありうるという考えにつなげた。
身体で問うこと自体が、主体性の萌芽であり、身体知の学びプロセスの最小単位的プロセスであり、
身体知の学びとは、身体で問うことによって、主体的に環世界を創り変えてゆく営みだとした。

序論\ref{chapter:mokutekiandhouhou}章では、本研究の目的と態度と方法論を論じた。
本研究は全体として「構成的」という態度を大事にする。
認知は構成的であり、構成的な認知を探究するためには、構成的な手法が必要であること、
そして、研究という営みもそもそも構成的な営みであることを指摘した。
本論文は、構成的という態度を大事にして、通常の科学論文では省かれてしまうような研究者の試行錯誤そのものについても、
著者が必要と思えば(平たく言えばストーリーテリングのようなかたちで)説明するように書いている。
本研究が採る方法論は、諏訪の提唱する認知的手法:からだメタ認知である。
からだメタ認知は身体で問う具体的手法であり、
みずからの思考・知覚・行動、すなわち「自分からみた自身と世界との関係性」について、
それがあいまいな違和感的なものであっても、はっきりした問題意識的なものであっても、積極的にことばにしてみて書きつづり問うてみる。
ことばにしてみることで、連想や推論など、自身とことばとの予期せぬインタラクションを起こし、自らの認知を変容させることになる。
つまり、身体知の意味づくりのプロセスを記述する方法として適したデータ記録方法でありながら、同時に、身体で問うことをうながし、構成のループを駆動する実践手法でもある。
本研究での「ことば」の位置づけは、ものごとを正確に記述したり他者に伝えたりというよりも、本源的に自身の身体との共創を起こす媒体である。

第一部研究について述べる。
\ref{chapter:monogatari}章では、私自身の身体知の学び、すなわち、私がひとりのアスリートとして生活と競技を分けずに「走り」を学ぶ(アスリートとして生きる)様を、物語として描き出した。
私は、自らのままならない身体と付き合い、体感に傾聴しながら、問題意識を醸成し、自分にとって納得できる動きの意味を試行錯誤的に探った。
怪我や生活上の出来事をきっかけとして、百均製LEDをもちいたトラッキングをDIY的に実施することで、自身の身体と動きを手触るようにして問うてみたり、
日常生活における、立つ・歩く動きをスキルとみなして根本的な再構築を図ったりした。
そして、日常生活で自身を取り巻く、競技に一見関係ないモノをツールへと転用しながら、
それらを通して、よりよい身体運用スキル、そして根本的な身体のあり方を問うことにすら試みた。
このようにして、私は数々の問題意識を醸成してきた。
\ref{chapter:hashirinohenka}章では、これらの努力の結果として私の走フォームにいかなる変化が生じたかを考察した。
\ref{chapter:yaseika}章では、私が自覚的にアスリートとして生きようとする態度を「学びの野生化」と命名し、その意義を論じた。
\ref{chapter:monogatarinoigi}章では、物語が他者にもたらす意義を議論した。

第一部研究を終えた私は、自身の身体を触発するさまざまな「トイ(おもちゃ)」づくりをしてみたり、
ある種「悟った」ようになった思惟を、「身体知輻輳性」と命名しながら論じてみたりしていた。
そうした問題意識のなかで私は、第二部研究の着想となる哲学概念「表情」に出会った。
そのことを第二部\ref{sec:tetsugakujousei}節で論じた。

第二部研究では、動いている身体の「表情」の感得をうながす運動学習支援webアプリ『HJ-Playground』を制作し、アプリをもちいた身体知の学びの実践をおこなった。
\ref{sec:hyojo}節では、哲学概念「表情」がどういう概念なのかを確認した。
私たち人間は「表情」の満ち満ちた環世界を生きており、
「表情」とは、視覚が主題になるような現象でありながら、行為の契機や情動や感情の契機をも孕みもつかたちで、生々しく立ち現れてくるものである。
そのことから、みずからの環世界を主体的に創り変えながら生き・学ぶ運動学習者にとっては、
とくに動いている身体が醸し出す「表情」は、意味の源たりうる現象であり、「表情」感得できるのが良いという仮説を立てた。
それをもとに、
\ref{chapter:fortoolmaking}章では、
身体知の学びとしての運動学習支援研究を概観したのち、
動いている身体の「表情」に近しいものに迫っているプロジェクト(研究や作品)を調べて
アプリ制作のヒントを探り、
「運動をもとにして、素朴で抽象的で図形的な見た目を表現する」ことが、「表情」の感得をうながすひとつの方法であるという着想を示した。

そうして本研究では、webアプリ『HJ-Playground』を制作し、\ref{chapter:hj-playground}章にてアプリの仕様を説明した。
本アプリは、あらかじめ計測したユーザ自身/他者の運動データ(各部位の三次元時系列位置情報)を、画面内の三次元空間に動く点群として描き、
ユーザに、それら点どうしのあいだに線分や円などの補助線を描きくわえ「表情図形」を作図することを促す。
ユーザには、作図した表情図形を鑑賞しながら、感得している「表情」をオノマトペで命名し、そのさなかで生まれる問いをからだメタ認知で内省記述することを促す。
これらは保存したり、再鑑賞することができる。
これらによって、ひとつの身体運動にさまざまな「表情」を豊かに感得することをうながす。

\ref{chapter:jissenresult}章では、アプリをもちいた学びの実践について記述した。
対象者のストリートダンサーAは、「oldman」というAの専門とは別ジャンル基本動作を対象に、それが醸し出す「表情」をアプリで探った。
身体内部のあるひとつの仮想的四面体が2通りの「表情」として感じられることを発見したり、
点同士左から右へ一筆書きでつなげることで、風に吹かれて移動するような「表情」を得たり、
両肩をむすんだ線の動きに「鉛筆を指先で軽くつまんで振るような情景」の「表情」をみてとって、予想外に不安定な揺れ方をしているのに気付いたりした。
別の対象者の三段跳選手Bは、一本ゲタ対人運動という不思議な動きを対象に、それが醸し出す「表情」をアプリで探った。
肩と手と肘の3点をむすび、「三角形が相手を突く」という攻撃性ある「表情」をみてとり、自分が到達すべき三段跳の力強い接地のビジョンをみたり、
円をつかった図形を作図することで「自身と他者とのあいだの空間が泡立つ」ような「表情」をみてとることで、それまで三段跳選手である自分が重要視していなかった「脱力」という深い概念について、その意味するところを自分なりに納得したり、
ゲタと足を線でむすびそれを延長してみることで「足首で剣を打ち込む」ような「表情」をみて、三段跳びの助走の一歩目の新しい踏み出し方を発想したりした。
このように本アプリの実践をとおして、対象者らは、アプリで遊んで、自らの身体運動をもとに主体的に表情図形を作図することで、ひとつの身体運動にさまざまな表情を感得し、
自身の運動のしかたを問いなおしていた。

\ref{chapter:analysis}章では、実践者らが表情図形の作図をとおして生み出した問いを分析した。
表情図形が、実践者らの問い立てをうながすパタンには、少なくとも2種類あり、
ひとつは、表情図形をみずからの身体に「仮想的図形」として召喚してそこに身体感覚を呼び起こすパタン、
ふたつめは、抽象的な表情図形にたいし、元の身体運動とは異なる日常生活のドメインの情景(できごとやシーン)に見立てる、メタファ的パタンであった。
ふたつめのパタンは、現役の運動学習者であるAとBよりも、すでに現役を引退しているCとBのほうが顕著にみられた。
また、ふたつめのパタンでは、どういうドメインの情景がメタファのソースとしてもちいられているのか、そのバラエティを調べた。
元々は身体運動をもとにした抽象的図形にもかかわらず、実に多様なドメインが、ソースとしてもちいられていた。

結論部では、以上の流れをおさらいしたのち、
身体知の学びの概念モデルを、本研究結果を踏まえて再構成した。
「身体で問うこと(認知作用=ノエシス)と問い(認知内容=ノエマ)との相互限定関係」は、
「主体的に生き・学んできた環世界」と「日常生活全般の野」との「界面」として生じる。
「表情」豊かに感得されるとき/「意味」が含蓄あるものとして醸成されているとき、
それは、野で出くわした対象が、環世界の多ドメインのものごとの輻輳として(輻輳に照らされて)、
多義的・多重的に、立ち現れてくる。
そして、そのように立ち現れてくるよう学習者は身構えることができる、として、本研究のまとめとした。

\vspace{2cm}

\textbf{キーワード: \\身体知の学び, 運動学習, 身体で問う, 生きる, 意味生成プロセス, 環世界, 状況, 「表情」, 「野」, デザイン, 一人称研究}

\vspace{2cm}
慶應義塾大学大学院 政策・メディア研究科 後期博士課程 堀内 隆仁
\clearpage

Title:
Practices of learning embodied wisdom by reconfiguring the way the body should and could be: 
Through designing a toolkit that facilitates perception of “hyojo” and encourages intentionality toward body parts and bodily sensations 

Abstract:

The difficulties of the boundary between universal principles and individual uniqueness in fields of embodied wisdom learning is one of the factors that could be hinderance to sound learning. 
However, previous studies on embodied wisdom in the field of sports science and motor learning have not necessarily addressed the issue.

Suwa (2016) has pointed out that there are at least two difficulties.
The first concerns the question of whether or not there is a single“correct form”of bodily movement that serves as the universal standard.
Bodies move under the laws of physics, and thus certain forms may be more efficient than others. On the other hand, the body of each athlete has its own individual characteristics― such as joint flexibility or muscle strength―, and therefore that makes it difficult to define a universally ideal form. In other words, ideal forms for athletes exist within a certain flexible range. 
Drawing a clear boundary between universality and individual uniqueness is inherently difficult.

The second difficulty lies in the fact that even if the learner has come to understand his or her ideal form that matches his or her characteristics, what body parts and/or variables need to be attended to (we call them “input variables”) for the purpose of bringing the form to realization is dependent on each person; there is no universally applicable way of directing attention. 
The learner in the beginning stage of learning does not know how to do it.
For the learner’s coach, too, since the coach’s body differs from that of the learner, the uniqueness of the learner’s input variables becomes a wicked problem.

Learners of embodied wisdom need to tackle these two challenges, exploring both their own ideal form and the relevant input variables for themselves.
This process involves being engaged in repeated trial-and-error: attempting bodily movements, perceiving subtle differences in bodily sensations, and referring to theories and advice from others. 
During the learning process, the method of embodied meta-cognition―a method of questioning one's own bodily being by verbalizing what one thinks, feels, and how one’s body moves―is found to be effective empirically (Suwa, 2016).

In Research 1 of this dissertation, I conducted a first-person study on how I have learned to run as a decathlete. 
Using embodied meta-cognition, I repeatedly practiced and verbalized my experience, exploring how learning emerged within my own body.

I faced my uncooperative body and kept reshaping my running form, directing my thoughts toward each part of the body and paying attention to the delicate bodily sensations that arose from their movements.
For example, I devised my own unique phrase―"digging into viscous air with my arms"―and experimented with a form that felt slow and wide, yet powerful.
I also implemented a DIY-style tool using LED lights from a 100-yen shop to visualize my bodily movements and questioned my form as if “touching it with my own hands.”

The journey of learning was not smooth; I often found myself in a trade-off situation where improving one aspect would compromise another. Eventually, I also suffered a knee injury.
Reflecting back, I realized it might have been caused by an excessive focus on making my form bigger, which led to overly deep knee flexion upon ground contact.

While undergoing rehabilitation, I came to suspect that the root of the issue lies in my everyday way of walking.
Gradually, I began to see that correcting how I “stand” and “walk” in daily life is also a significant form of training.
That is, training does not exist only in the athletic stadium, but also in reconsidering the bodily use throughout the daily life.
This thought felt entirely natural, given the obvious actual circumstances that all kinds of embodied wisdom—whether in sport or everyday life—“converge” into a single body.

Through these experiences of learning, I regard the expansion of my learning process—from the context of athletic performance to that of everyday life—as a form of the “going sauvage” of learning in Research 1.

The learning processes I experienced in Research 1 was written in the form of a “narrative paper,” which describes the concrete and particular aspects of my body, personality, and daily life. It has been evaluated as meaningful in the sense that it can give inspiration to readers to reflect on and deepen their own embodied learning.

Through the trial and error of Research 1, I have come to deeply realize that it is indeed possible to transform one’s own form through proactive effort, that such transformation can lead to states or ways of being that I have not necessarily imagined at the outset, and that the path of such learning is inevitably steep and demanding.

At the same time, I arrived at a hypothesis: that perceiving the “hyojo”—the subtle, emergent gestalt of a body in motion that cannot be fully captured by words—is essential to the learning process. Here, “hyojo” does not simply refer to the face, but rather to the vivid, palpable presence—the felt gestalt—of the motion of people or things, including oneself (Hiromatsu, 1989). 
I came to believe that sensing such “hyojo” helps maintain intentionality toward bodily sensations which are often neglected, and prompts reflection on unnoticed relationships between different body parts. This hypothesis became the foundation for Research 2.

Research 2 was conducted under the guiding principle of the concept of “hyojo”. In this context, “hyojo” refers to a form in which emotions, intentions, or will are richly present yet remain undifferentiated. 
When I now look back at the expansive and “relaxed” running form I initially aimed to achieve, I perceive in it an “hyojo” that contains a kind of effortful will—something like the intention to lift or push against resistance, as if shouting “dokkoi-sho!” at the moment of ground contact.
This is clearly different from the final ideal I came to pursue: an “hyojo” that feels like gliding smoothly and effortlessly through the air, almost like a quiet “sweeee.”

Based on this realization, Research 2 set out to create a web application—HJ-Playground (HJP)—designed to help athletes perceive the “hyojo” of the body in motion, which is often difficult to notice through verbalization and embodied meta-cognition alone. HJP fosters the perception of bodily “hyojo” through the following mechanism: the application visualizes a point cloud generated from the user’s own movements (captured beforehand via motion capture, as time-series 3D positional data of selected body parts).
Users can freely draw various auxiliary lines between these points to create abstract geometrical shapes, referred to as hyojo figures.
They are then encouraged to name the “hyojo” they perceive from these figures using onomatopoeia, and to describe their experience in words through an embodied meta-cognitive lens.

I observed how a street dancer and a triple jumper engaged with HJP to reflect on their own bodily being. Each participant actively created various hyojo figures using the app, and by playfully interacting with them, they came to perceive hyojos that not only reflected the physical movements of dance or triple jump, but also evoked scenes from their everyday lives connected to those movements.
The dancer discovered the quality of wavering in his own movements, grasped new ways of directing attention during performance, and formed a higher-level hypothesis: that by switching the hierarchy between body parts (which leads, which follows), one could expand their movement repertoire. 
The triple jumper, on the other hand, envisioned a strong and forceful landing form, and linked two elusive concepts often heard in sports—“relaxation” and the idea that “arms steer the motion,” a trait commonly attributed to skilled triple jumpers—into a deeper, unified understanding. 
The triple jumper also discovered a new approach to initiating his run-up. 
These findings suggest that HJP successfully fostered the perception of “hyojo”, and that this perceptual shift triggered critical reflection on their own bodily being and movement.

The significance of this study lies in two main contributions.
First, it has practically demonstrated the significance of reconceptualizing embodied wisdom as convergent wisdom—where wisdom from both athletic performance and everyday life flows together—and of exploring for and reconfiguring the ways the bodies should and could be by regarding athletic performance and bodily usages in everyday life as a whole.
Secondly, through focusing on “hyojo", the dynamic and gestalt-like quality of bodily movement that often escapes verbal articulation, this study has devised a novel tool that facilitates the perception of such “hyojos”, and has confirmed its effectiveness in supporting learners in exploring for and reconfiguring the ways his or her own body should and could be.

\vspace{2cm}


\textbf{Keywords: \\Learning Embodied Wisdom, Motor Learning, Questioning with the Body, Living, Meaning-Making Process, Umwelt, Situation, Hyojo, the Wild, Design, Research from The First-Person's Viewpoint}

\vspace{2cm}

Graduate School of Media and Governance, Keio University  

Doctoral Program

Takato Horiuchi

% 脚注をアスタリスクなしの連番形式にする
\makeatletter
\renewcommand{\thefootnote}{\arabic{footnote}}
\makeatother
% 目次
\tableofcontents
% 図目次
\listoffigures
% 表目次
\listoftables

% --------- 読み方ガイド / 凡例 -------------
\chapter*{本論文の読み方・凡例}
\addcontentsline{toc}{chapter}{本論の読み方・凡例}

\begin{description}
  \item[人名表記]  よく知られた人物は、英語ではなくカタカナで記す。
        初出時に英語表記(括弧)を併記する:
        例)\textit{デカルト}

  \item[人名への(生年-没年)付記]  すでに亡くなっている人物については、適宜、人名に(生年-没年)を付記する。なるべく時代背景からも把握しやすくするためである。

  % \item[引用スタイル]  参考文献は \textsc{Bib\TeX} (スタイル \textsf{apalike})を用いる。
  %       複数文献の同時引用は \verb|\cite{key1,key2}| とする。

  % \item[図表番号]  図は「図\,1.1」、表は「表\,2.3」。
  %       本文中では \verb|\autoref| コマンドを用い、自動で「図」「表」を付す。

  % \item[略号]  AI(Artificial Intelligence)、VR(Virtual Reality)など。
  %       略号リストは付録\,A 参照。

  % \item[強調方法]  概念語は *イタリック*、実験条件名は **ボールド** 。

\end{description}

% \bigskip
% 本論では読者が章ごとに参照できるよう、冒頭に目的を、末尾に要点を示す。

\clearpage
\pagenumbering{arabic}
% !TeX root = ../data.tex


\part*{序論}
\addcontentsline{toc}{part}{序論}
\chapter{はじめに}
\label{chapter:hajimeni}

\section{本研究の射程:身体知の学び}
\label{sec:embodiedwisdomlearning}

速く走ったり、美しく踊ったり、淀みなく鍋包丁を捌いたり、気分やシーンにあわせて自分らしく衣服を着こなしたり、ウイスキーを豊かに味わったり、抽象画を鑑賞したり、デスクまわりの設えをちょいと調整して居心地を一変させたり、絶妙な押し引きのあるトークを繰り広げたり、統計の分散分析で「2乗」の計算が含まれる理由を腑に落としたり・・・。

どれもが\textbf{身体知(embodied wisdom)}である\cite{suwa:2016, suwa:2022}。
知といっても、頭のなかに格納された知識・情報ではない。
だからといって、物的な身体がなすふるまい(運動)だけを指していうのでもない。
身体をそなえた私たちは、絶えず予測不可能な状況(環境含む)にさらされており、
知覚・行為・思考しながら(\ref{subsec:noesisnoema}項で後述するが、本研究では、これら3つを別個のはたらきではなく、ひとつの認知作用の別側面であると考える)
、なんとか状況をやりくり\cite{decerteau:1980}しながら生きている。
このプロセスのなかで知は創りだされてゆくととらえるのが身体知の考え方である。
身体知は、\ruby{身体}{からだ}をそなえた「わたし」と世界(環境)とのあいだのインタラクションあってこそ成り立ち、
いわば、みずからにとってより望ましい生を生きるための\ruby{術}{すべ}である。
その点で、身体知は、\textbf{個人固有性}と\textbf{状況依存性}という2側面をもちあわせている\cite{suwa:2016,suwa:2022}ものだと考える。
本研究は、ひとの知はすべからく身体知であるととらえる。

% 身体は、メルロ=ポンティが論じたように、自己の、世界の、あらゆる関係性の礎たりうるのである\cite{merleau-ponty:1967}。

ひとは\textbf{学ぶ}いきものである。
多かれ少なかれ「主体的」に身体知を創りながら生きているということである。
「自動的に身体知が生産されてゆく」とか「ただ生きのびるためだけにしょうがなく身体知を創っている」わけではない。
佐伯\cite{saeki:1995}は、ひとは「学びがい」を求めて学ぶのだと説いている\footnote{
  佐伯は、ひとは「学習する」のではなく「学ぶのだ」、という主張のなかで両者の相違について「学びがい」という言葉を持ちだす。
}。
90年代以降、学びの研究は盛んにおこなわれてきた。
学びの肝は「問う」ことにあり、といった考え方が多くの文献で論じられている(たとえば\cite{saeki:1995,suwa:2016,suwa:2022})。
そこでも「身体」の存在が重要視される。
身体知の学びの肝は、物的な身体-環境間インタラクションだけではなく、それにともない\textbf{身体で問う}ということ(\ref{subsec:noesisnoema}節で詳述)にあると考える。

主体的に生きながら学び、主体的に学びながら生きている。
ゆえに身体知の学びは、生きることと学習ドメインとの界面にある。
それは、学習者から切り離されて存在するような対象の「情報」を得ることではない。
ゲシュタルト心理学の語彙を借りて言えば、わたしの生きるということ全体を地にして、浮かび上がってくる図のようなものである。
わたしの生に照らされて浮かび上がるもの、それすなわち、「意味」である。

身体知の学びとは、いわば主体的に\textbf{意味を醸成する}営みである。
私たちはそうやって、自分なりの意味世界を生きている。
そして、みずからの生きる意味世界をより望ましいほうへと作り変えながら生きている。
身体知の学びは、創造のプロセスとも言える。

ここまでわざわざ「身体知とは〜」よりも「知を身体知とらえるとは〜」とまわりくどい書きかたをしてきたのは、次の誤解を避けたいからである。
「知には通常の知とそうではない知(=身体知)とがあって、その後者を研究するのですね」という誤解である。
あるいは「身体知とは、運動を主題とする知のことですね」とか、ひいては「体育に代表される実技的・副教科的な知のことですね」といった誤解である。
そうした領域限定的な知を指して身体知と呼んでいるのではない。
身体知とはそうではない。
「知をすべからく身体知としてみなおす」視点に立つのが、身体知の考え方である。
冒頭の事例からしても、いわゆる「体育」的なものだけが身体知ではないということがわかるだろう。
以降本論文で「身体知」と書けば、それは原則「知を身体知としてとらえなおす」ことを含意する。

本研究は、陸上競技アスリートやダンサーの運動学習を身体知の学びとして捉え、そのプロセスを探究するものである。
運動学習は身体知の学びとして恰好のドメインである。
運動学習者らは主体的に身体知づくりをおこなっているからだ。
彼らはより望ましい運動を目指して日々試行錯誤している\footnote{Simon\cite{simon:1969}によるデザイン(という行為)の定義にも符号するところがある。}。
% 「わたしのより望ましいありかたを目指して学ぶ」と言い換えてもよいだろう。
ダンサーであれば、より独創的であったり、より曲調や場の雰囲気への調和したり、より高難易度な、といった運動をめざすだろうし、
陸上競技アスリートであれば、少なくとも「より好記録」の運動をめざしている。
彼らはそういう目指すべきところに方向づけられながら、主体的に、自分なりの意味世界をより望ましいように創りだし/創り変えてゆく。
運動学習を身体知の学びとしてとらえるとき、当然、学習者は学びがい\cite{saeki:1995}を求めているということも重要になってくる。
学びのプロセスを生み出すことそのものも学びの目的にふくまれる\footnote{
  山登りになぞらえるならば、山頂まで自分の足腰で踏破してこそ山登りなのであって、山頂に至ることだけに目的があるのではない。
  ヘリコプターで山頂まで運搬されても、そこに手応えも満足感もなく、それは山登りとは呼び得ないだろう。
}ということだ(私は元・陸上十種競技のアスリートとしてそう信じている)。
単なる成果主義・実利主義的なプロセスとは異なるのである。
念を押しておくが、身体知研究である本研究が運動学習を扱っている理由は、「身体を使うから運動学習は身体知」と短絡的にとらえているからではない。
\section{運動学習の現場:著者が「走り」を学ぶ事例から}
\label{sec:myexample}
運動学習者は日々、より望ましい運動のありかたを身につけようと研鑽を続ける。
まずは運動学習の現場を見つめてみることから本論文の第一歩を踏みだそう。

陸上十種競技の選手であった私(著者)が走りを学ぶプロセス\cite{horiuchi:2016a}から一事例を紹介しよう。
十種競技は、走・跳・投にわたる計10種目の総合力を競う陸上競技の種目である。
十種競技選手として私は「走る」ことを不得意としており、速く走るための試行錯誤を日々、続けていたのだった。

速く走るためには、接地瞬間に身体重心が接地位置鉛直真上の近傍にあること(=「真下接地」)が重要である。
速い選手の客観的走フォームはこの特徴を満たしている。
私はこれが出来ずにいた。
「へっぴり腰」(身体重心が、接地位置鉛直真上より過度に後方にある状態)で接地し、接地自体がブレーキになってしまっていることを私は自覚した。
これを問題視した私は、\autoref{fig:kostubankorobashi}に示す様に、「接地位置をより手前へ引き寄せれば(\autoref{fig:kostubankorobashi}の局面1)、結果として真下接地に
なるのでは?」という仮説を立てて実践することにした。
\autoref{fig:kostubankorobashi}はその頃の私の走りの動画のキャプチャである。
私は走りを実践してはそれを動画に撮影し、実際の走りの感覚と動画との関係性を反省しながら、反省することを繰り返した。
試行錯誤をつうじて私は、「接地脚の上で骨盤を転がす」という私独自の体感(局面2〜3でそれが生じている)を編み出した。
それを意識しながら走ることで、手前に引き寄せられた接地を走ることで実現され、ブレーキを減じることに成功した。

しかし、すぐに弊害が見つかった。
その方法では、全速力で走ると接地そのものの力強さがなく、次第にうまく走れなくなったのである。
緩やかな走りの時点でもその原因がすでに表れていたことに私は気づいた。
接地直前に、接地位置を手前に引き寄せる動き(\autoref{fig:kostubankorobashi}の局面1)が、引っ掻くような弱い接地を生み(局面2)、地脚の膝が潰れるように曲がりすぎてしまっていた(局面3)のである。
全速力で走る場合には、これらの動きが原因で、スイングする脚が素早く前へ出てこなくなってしまい、うまく走れないのだと考えられる。
はたして私は、スイングする脚がもっと素早く前へ出るように、接地時のブレーキをうまく活用して、
ブレーキにともなって鞭運動の原理でスイングする脚が自然に前方へ放り出されるような練習をするようになったのだった(それはサッカーボールを蹴る、というトレーニングだった)。
このエピソードは、私が、「真下接地」という「情報」を、「接地脚の膝関節角度」や「スイングする脚」との関係で咀嚼し、自分の身体にとっての重要な「意味」として理解してゆく事例である。

\begin{figure}[h]
  \centering
  \includegraphics[width=10cm]{./images/kostubankorobashi.pdf}
  \caption{私が「骨盤を転がす」体感で以てゆったり走る様子}
  \label{fig:kostubankorobashi}
\end{figure}

これが運動学習の現実である。
運動学習プロセスは一朝一夕では成らぬ、険しいものである。
あちら立てばこちら立たずである。
きのうときょうでは、心身の状態はちがう。
元陸城短距離選手・朝原氏\footnote{
  元・陸上短距離選手で、北京オリンピック陸上男子4×100mリレーでアンカーをつとめ日本に銀メダルをもたらした。
  100mの自己ベストは10.02秒、言わずと知れた日本短距離会のパイオニアである。
}は、調子が良いときにはその身体感覚を克明に言葉で記述し、ノートに残していたのだが、その記述を調子が悪い別の日に読んでもそれがそのまま使えるわけではなかった、という実体験をインタビューで語っている\cite{ikuta:2011}。
他のケースもある。
「筋肉がつく」ことによっても、感覚は容易に変わってしまう\cite{ikuta:2011}ことを、
スピードスケートのコーチを務める結城氏\footnote{
  結城は、オリンピック日本代表選手団コーチを複数回務め、清水宏保選手や小平奈緒選手などの金メダリストをコーチングした実績をもつ。
  結城氏自身も、選手元スピードスケートの選手として、ワールドカップ3位の経歴をもち、さらに研究者でもある。
}はインタビューで述べている。


運動学習者らは、実際に運動しながら、「どう身体が動いているのか/どう動かすべきなのか?」、「どんな体感が生じているのか?」といったことについて問う。
まるで自らの身体の「声」に耳を傾けるようにして、彼らは問いを発するのである。
彼らは主体的に、「身体で問う」ているのだ。
そうして、「ああでもない、こうでもない」と、学習者は泥臭く試行錯誤し続ける。

\section{生きているなかで学ぶ}
\label{sec:konzenittai}
そして、運動学習のプロセスは、単に「練習場」だけで起こるわけでもない。
彼らは日常を生きている。
多かれ少なかれ、日常生活をも運動学習者として自覚的に生きうる。
\ref{sec:embodiedwisdomlearning}節で述べたことは、運動学習でも当てはまるのである。
これは本研究の核たる主張である。
例えば、陸上短距離選手であれば、玄関で靴紐を結ぼうと前屈みになったとき、短距離走のクラウチングスタート時の前傾姿勢に通底しそうな、腹筋群の新鮮な収縮感覚に気づいてしまったりする。
湯船に浸かりリラックスしながらその日の練習を振り返っているとき、練習時には混乱していたことがクリアにみえたりする。
そうやって運動学習者は、生活のなかでさえ問いを発し、自分なりの「意味」を醸してゆく。
望ましい運動をその身でつくりあげてゆく。

剣豪・宮本武蔵(1584-1645)は、兵法書『五輪書』\cite{miyamoto}にて次のように書き記している。
\begin{quote}
  常の心に替事なかれ。
  常にも兵法之ときにも、少も替らすして、こゝろを廣く直にして、きつくひつはらす、すこしもたるます、心のかたよらぬやうに、心をまん中に置て、心を静にゆるかせて、其ゆるきのせつなもゆるきやまぬやうに、能々吟味すへし
  (\cite{miyamoto}、 p.139)。
\end{quote}
兵法と生活を線引きせずに、常に「心を真ん中に揺らがせておく」という柔軟な心持ち、すなわち、そういう「生き方」をこの文章は示しているのだと、私は解釈している。

能を大成した世阿弥(1363-1443)は、能芸論『花鏡』\cite{zeami}にて、次のように書き記している\footnote{
  世阿弥の花鏡も宮本武蔵の五輪書も、晩年に執筆されたものである。
  ゆえにそこに書き記されているものごとは、彼らがそれぞれの道を極めて至った境地であろうと著者は解釈する。
}。
\begin{quote}
かへすがへす、心を糸にして人に知らせずして万能をつなぐべし。
かくのごとくならば能の命あるべし。そうじて即座に限るべからず。
日々夜々、行住坐臥にこの心を忘れずして定心につなぐべし
(\cite{zeami})。
\end{quote}

ここには、万能を一心につなぐ(それぞれの技や芸の\ruby{間}{ま}も心を切らさずつなぐ)ことを、行住坐臥(日常生活の所作)において意識すべしという極意が説かれているのだと私は考える。

宮本武蔵と世阿弥の弁に通ずるのは、生活と実践ドメイン(競技)は、本来的に渾然一体だという考えだといえよう。
運動学習者は、生活のなかで試行錯誤を繰り返す。
そうして、学習者は、自分なりの意味に彩られた世界、\textbf{意味世界}を創り、自覚的に生き、学ぶのであろうと本研究は考える。
本論文第一部の研究では、著者自身がアスリートとして自らが生きる「走り」の意味世界を創りだし・創り変えてきたのかを物語る。

\section{動いている身体の「表情」}
\label{sec:ugoiteirushintainohyojo_intro}
しばしば運動学習者は、\ref{sec:myexample}節の事例でも示したように、自身/他者の運動(の動画)を観察する。
観察は学びの重要な一局面である。
身体知の学びとして運動学習をとらえる以上、観察もまた「身体で問う」ことでありえ、身体で問うような観察が重要だと著者は考える。
本論文第二部の研究ではその問題に焦点をおく。

しかし、対象の身体運動を目の前にしたとき、身体で問うように観察することは容易ではない。
たとえば観察は、実際に走ったときの主観的な体感などを踏まえて、それとのずれを確かめるために、映像に映る客観的特徴をみてとろう、といった具合におこなわれがちである。
そういう観察は、明晰的な分析を可能にするという点ではたしかに有効ではありうるが、身体で問うことになるのかといえば、必ずしもそうではないだろう。
「なにかを学びとってやろう」と身構えることは、かえって、心身が居着くことになり、身体で問うことを阻害してしまうのかもしれない。
考えてみれば、そのときの観察も、「実際に走ったときの体感と照らし合わせてズレを確認しよう」といったねらい(=縛り)がはたらいている。
だから「客観的な姿すべてをそのまま」をとらえられているわけでもなく、無数に存在する関係性があるなかからごく一部だけをみてとっている、というのが実情であろう(観察とはそもそもそういう営みである)。

そこで本研究は、動いている身体の\textbf{「表情」}\cite{hiromatsu:1989}に着目する。
動いている身体の「表情」とはなにか。
詳細は第二部\ref{sec:hyojo}節で説明するとして、
ここでは、
「観察対象の動いている身体の生々しくありありと立ち現れてくるその姿(\autoref{fig:hyojointro}のような具合に)のこと」である、とひとまず説明しておこう。
\autoref{fig:hyojointro}の左は、ストリートダンサーが踊る場面であるが、この踊りを観察してみるとき、たとえば「火山噴火」のような情景をそこにみてしまう
\footnote{
  この左画像はストリートダンサー山崎が、自らのダンスの振り付け創作プロセスを探究した一人称研究\cite{yamazaki:2017}から拝借している。
  この振り付けは、音楽の大きく太いドラムの音に、山崎が火山噴火を想起しながら創作したものである。
  なお、画像加工の許可は本人から取得済である。  
}。
ダンスだと「表情」なるものがあろうことは受け容れやすいかもしれないが、実は、表情はあらゆる身体運動にあるのだと著者は考えている
(たとえば\autoref{fig:hyojointro}の右2つは著者自身の陸上競技のパフォーマンスである)。

身体で問うような観察とは、「表情」をみることなのだと本研究では考える。
身体で問うように観察することは、「表情の感得」をすることなのである。
「表情」は意味づくりの「源」たりうる現象である。
補足しておくと、「表情」そのものは運動学習に限る概念ではない。
コミュニケーションする身体(身振り手振り)はもちろん、植物にも、建築物にも、「表情」はある。
「表情」をめぐる問題は、生活全体のドメインに通底するものであるが、それを本研究では運動学習にもちこむ。
第二部研究では、運動学習者に、動いている身体が醸し出す「表情」を感得することを促す。

\begin{figure}[h]
  \centering
  \includegraphics[width=\textwidth]{./images/hyojointro.pdf}
  \caption{動いている身体が醸し出す「表情」}
  \label{fig:hyojointro}
\end{figure}


\section{本論文の構成}
本研究ではこうした身体知の学び(意味づくり)のありように迫る。
上記したような現場があるいっぽう、これまでの運動学習研究は、\ref{chapter:motorlearning}章で後述するように、必ずしも身体知の学びとしての姿十分に描いてこなかった。
本序論では、続く\ref{chapter:embodiedwisdom}章で、身体知の学びの考え方を、知の研究群のなかに位置づける。
\ref{chapter:motorlearning}章では、知の研究と接しながら蓄積されてきた運動学習の既往研究群を眺め、本研究の意義と目的をより明確にする。

本研究は二部からなる。
第一部は、著者自身が対象者となって実施した\textbf{一人称研究}\cite{suwa_hori:2015,suwa_fujii:2015,suwa:2016,suwa:2022}である(一人称研究については\ref{sec:ichininshokenkyu}で説明する)。
陸上十種競技を専門とする著者自身が「走り」の身体知をいかに学ぶのか、
競技のみならず生活をも渾然一体となってなす「アスリートとして生きる」\cite{horiuchi_suwa:2020}姿を、
その一人称研究の成果を\textbf{物語}\cite{horiuchi_suwa:2020}として語り描く。

第二部の研究は、動いている身体が醸し出す「表情」\cite{hiromatsu:1989}を意味づくりの源として位置づけ、表情の感得を促すツールをデザインし、運動学習を支援可能性を探究する。
\chapter{知の学問における本研究の位置づけ}
\label{chapter:embodiedwisdom}
本章では、認知科学や人工知能を中心とした、広く知の学問を概観し、
本研究の思想(身体知の学び)の位置づけを明確にする。

\section{情報処理モデルの思想}
\label{sec:aiboom}
認知科学や人工知能が誕生した1950年代以降、知の学問のメインストリームをなしてきたのは、身体と心とを分離し、そのうえで、心のほうに知を求める思想・方法論である
\footnote{17世紀のデカルトによる「我思うゆえに我あり」とした心と身体を分離する考え方\cite{descartes:1637}は、色濃く後世に影響を残している。
}。
いわゆる\textbf{認知主義}である。
認知主義は、\textbf{行動主義}---動物実験のように刺激と反応のセットすなわち客観的に観測可能な物的身体のふるまいだけに知を求める---へ反発する立場である。
認知主義では、知は「コンピュータ」に見立ててモデル化された。
\textbf{情報処理モデル}である。
情報処理モデルは、システム内部(あたまのなか)に「知識」が格納されており、知覚を外界からシステムへの「入力」、行動をシステムから外界への「出力」とみなす。
システムは、入力をもとに、それをもとに、すでに貯蔵済の内部知識のなかから最適と判断されたものを選び計算し、行動として出力する。
出力の結果はシステムにフィードバックされ、内部知識を更新する、といったものだ。
80〜90年代には、医学診断などの専門ドメインにおける膨大な知識をif-then形式で格納・活用するエキスパートシステムが注目され、第二次AIブームが巻き起こった。
こうした情報処理モデル(シンボリックAI)は、well-definedな(かたちで用意された)世界のなかでは威力を発揮した。

しかし、である。
いざ現実世界に放り出されてみれば、事態は異なった。
シンボリックAIは現実世界で出くわす問題に適切に対処することができなかったのだ。
シンボリックAIは、ill-definedな現実世界において、当座の問題にはなにが関係していて/関係していないのか、という思考の枠を準備しておくことができない。
\textbf{状況}に対して適切に対処できないのである(枠があっても、上記の選別に無限の時間がかかってしまう)。
いわゆるフレーム問題\cite{McCHay:69,dennett:1984}である。

いっぽう人間はといえば、状況に応じて、フレームを狭めたり拡げたりしながら、なんとかやっていく\cite{decerteau:1980}ことができている\footnote{もちろん「人間はフレーム問題を完全に解決している」わけではない}。
\ref{sec:embodiedwisdomlearning}節で述べた「状況から知を創りだす」ということに相当する。
それを可能にするのは「身体」であるというのが、有力な見方であり\footnote{
  諏訪\cite{suwa:2018}は、お笑い芸人の大喜利やデザイナーのスケッチを例に、クリエイティブな発想の源は、身体ごと世界に没入して世界に触り、身体や感情の発露にある(さらに、そこに「ことば」を駆使するのがよい)と説いている。
}、本研究もそう考える。
ひとは、身体をそなえて生きており、ひとの知は身体あってこそ成り立つ。
シンボリックAIは「身体」をそなえていないのである。

この問題は、2025年現在第四次AIブームの中心を担う生成AIについても、本研究からみればさして変わらない。
生成AI
\footnote{ChatGPTをはじめ生成AIは非常に便利であり、仕事や学術や芸術といった様々な領域において、私たち人間との共同する時代が到来している。
}は、人間の脳のニューラルネットワークを模したモデル
\footnote{ニューラルネットとそれをビッグデータをつかって深層学習させることは、2010年代に起きた第三次AIブームからの中核技術である。
}であり、
ChatGPTに代表されるように、テキスト・画像・音声などマルチモーダル学習をした大規模言語モデルが実現されている。
そこでの内部知識は明示的表象としてあるわけではなく、ニューラルネット全体へ\textbf{分散的に表象}されている(サブシンボリックAI)という点で、
フレーム問題に対して一歩前進しているとは言えるのかもしれない。
しかしそれでもなお、サブシンボリックAIもまた「身体」をそなえているわけではない。
それに、サブシンボリックAIは「人間の用意したビッグデータ」から「学習させられる」のであって、
主体的に「学ぶ」\cite{saeki:1995}ことはしない。
サブシンボリックAIは「情報」を処理しているだけであって、ものごとの「意味」を知らないのである。
極論を言ってしまえば、サブシンボリックAIは身体をそなえて生きていないから、ひとの知とは異なるのである。

\section{身体性認知の思想}
\label{sec:shintaisei}

知の成立には「身体」が欠かせないと説く研究思想もある。
生態心理学の祖・J.J.ギブソン(1904-1979)\cite{gibson:1979}はそのひとりと言えよう。
認知主体は内的表象の構成というプロセスを介さずとも、「環境から直接に情報を知覚」できるとギブソンはいう。
生態心理学では原則的に行為と知覚は一体であると考え(この関係性は情報処理モデルのフィードバック機構とは異なる)、
そのうえで知覚-行為システムとして認知を記述しようとする。

工学的アプローチ(ロボティクス)から「身体」の重要性を示したのは「\textbf{知能の身体性}」という考え方である。
ロボットは、プログラムとして明示的に表象された命令どおりにふるまうわけではない。
ロボットの物理的な身体と環境との相互作用もあってこそはじめて、(プログラムもふくめた)その全体から何らかのふるまいが\textbf{創発(emergence)}するのである
\footnote{PheiferによるDidabot\cite{pheifer:2001}を例にとろう。
Didabotは小さな車型ロボットであり、前部の左右それぞれに近接センサを搭載している。
「直進せよ。左右どちらかの近接センサに物体を検知したときは、曲がれ(回避)」という命令がプログラムされている。
だが、実際にサイコロ状の物体をたくさん散りばめたフィールドでDidabotを一定時間動かしてみたところ、驚くべきことが起きた。
Didabotはフィールドの物体群を「お掃除」してしまっていたのである。
そんなプログラムはどこにも書いていないのに、なぜなのか?
こういうことだ。
物体aが真正面にあるとき、左右どちらの近接センサにも反応しない。
Didabotは物体aをそのまま押し進める(プログラムの視点からすれば、何事も起きていない)。
その状態で左右どちらかに物体bを検知すると、Didabotは曲がる。
このとき、物体aがその場(物体bのそば)に置き去りにされる。
これを繰り返すと、はじめ散らばっていた物体群がだんだんと集積してゆく、というカラクリである。
「おそうじ」行動は、プログラムの中身のみならず、物体重量や形、Didabotの馬力や形、センサの検知可能範囲やとりつけ位置・・・などの物理的相互作用によって、創発したのである。
}。
哲学者・Clark\cite{clark:1997}は、こうした知見を踏まえて、知は、脳内から身体、世界へと「漏れ出」ているのだと説いた。
Clarkは、脳-身体-環境の相互作用によって脳が計算の負荷を減らせる可能性を指摘している。
Clarkの立場は、表象が必要ないとまで主張するギブソンよりも急進性は低いと言える。

ヴァレラ(1946-2001)らによる\textbf{エナクティヴィズム}の考え方\cite{varela_et_al:1991}もある。
エナクティヴィズムでは、まずなにより「行為」を考え、環境も知覚も「行為」ありきで成り立つ。
みずから行為することによる、感覚運動カップリングがはたらき、
感覚運動カップリングの歴史が、世界を、知覚を、産出する、と考える。
ミツバチと花とが、ミツバチが密を吸いやすように、花が蜜を吸われやすいように、というかたちで共進化する。
また、ヴァレラは、知の問題を探究するときには科学的なアプローチだけでは足りないことを指摘していた。
知を「研究するわたし」から切り離して対象化したうえで、知に「ついて」扱おうとするだけでは、知の本質には迫れない。
そこでヴァレラは、仏教や現象学やプラグマティズム哲学をふまえて、三昧/開かれた反省によって知が現れると説いた。
それは、自己(わたし)や反省プロセスそのものをも巻き込むような、身体的な反省\footnote{ヴァレラは「身体としてある反省」と呼んだ。}である。
ギブソンの考えと、表象主義を乗り越えようとする点では共通しているが、ギブソンは環境に実在すると考えるいっぽう、行為が産出すると考えることがちがう。
また、Clarkらの考え方はヴァレラの考え方よりもいくぶん機能主義的、と言えよう。

近年は、本節で挙げてきた研究思想群(知能の身体性、生態心理学、エナクティヴィズムなど)を、脳神経科学をも巻き込むかたちで「身体性認知」として統合的にとらえようとする動向もある。
知を「脳内」だけに閉じ込めず(脳を単なる中枢とせず)、\textbf{脳-身体-環境のダイナミカルなシステム}(力学系)のなかで知をとらえる。
そのうえで立場はいろいろ分かれている。
ギブソンに似た、認知作用にそもそも表象を認めず脳-身体-環境システムだけで考えるような急進的な立場、
Clarkに似た、認知作用は脳-身体-環境システムが「脳がする計算負荷を減らす」役割を果たすのであって脳の表象・計算そのものはあるとするような立場、
ヴァレラに似た(あるいはLakoff\&Johnson\cite{lakoff_johnson}のように)、認知内容の身体づけられたありようを扱う立場、
などである(本研究の焦点は本文の3つめに挙げた立場に近い)。
Gallagher\cite{gallagher:2023}はこうした流れを、4E(embodied, embedded, extended, enactive)という用語をもちいて展望している。

本研究は身体性認知へ一定の賛意は示しつつ、その焦点は「身体づけられた意味のありよう」にある。
本研究は、認知作用が脳-身体-環境のダイナミカルなシステムがあって成り立つことは認めつつも、
認知作用を客観的かつミクロに記述することはしない。
それにともなう認知内容の側にこそ、本研究の焦点はある。
身体と「意味」とは不可分である。
身体がどのように意味をつくりだし、意味はどのように身体づけられているのか、を真っ向から扱うのが本研究である。
もしここにも、心脳問題で指摘される「説明のギャップ」\cite{levine:1983}を持ち込んで良いのならば、
「脳-身体環境の物的相互作用」を記述しようが「心的体験そのもの」には迫れない、と考えることもできる。
両者のあいだには埋まることのない「ギャップ」が残らざるを得ないのではないか、と著者は考える。

そればかりではない。
本研究は思考(やことば)といった「認知内容」が、(身体性)認知作用が原因となって結果的に生じるものだ、とは考えない。
それだけでなく、思考が認知をかたちづくる、という視点に本研究は立つ(詳細は\ref{subsec:questioningwithbody}、\ref{subsec:constructionloop}、\ref{subsec:embodiedmetacognition}項などを参照)。
重要なことだが、思考(やことば)をそのように位置づけるとき、「意味」とは、単なる内観とか記号的な表象とは異なるものである。
身体性認知の研究群のなかでさえ、科学的にあつかうのがむずかしいからなのか、思考やことばが認知をつくる(知覚や行為をつくる)ありようを迫るアプローチは盛んとは言えないのが現状だろう。
しかし意味の問題を真っ向から扱うためには、この性質こそ重要なことなのである。

以上を鑑みると、意味を主題的に扱う本研究に意義を認められるだろう。
\section{身体知の学びの思想}
本節では、前節で主張したような身体と意味の関係性を理論立てながら、本研究の中核概念である身体知の考え方を打ち立てる。

\subsection{環世界:固有な意味世界を生きる}
\label{subsec:umwelt}
「意味」の次元に迫りうる概念を早くから提唱していたのは、生物学者・ユクスキュル(1864-1944)である。
ユクスキュルの提唱した\textbf{環世界(Umwelt)}\cite{uexkull:1934}論は、本研究が拠って立つ思想家のひとりであるメルロ=ポンティ(1908-1961)をはじめ、現象学・哲学領域にも大きな影響をあたえた。
環世界とはなにか。
ダニの例が手っとりばやい。
ダニは、灌木につかまって待ち構えている。
哺乳類の酪酸のにおいを嗅ぎつけると、
木から手を離し、落下する。
落下場所の衝撃を感じ、
哺乳類の体表の毛を這いまわり、
毛のないところをみつけると、そこに喰いこみ、吸血する。
それだけなのだ。
ダニの生きる世界には、酪酸、温度、接触という3つの刺激だけがあり、
その他の刺激は「存在しない」に等しい。

ダニの例が示すのは、生物が世界をどのように「経験」するかは、その生物の身体(どのような感覚器官をもち、どのような運動をなすか)に深く依存しているということである。
言い換えれば、生物が世界に出会うしかたは、その生物固有の知覚と作用によって象られている。
このように、それぞれの生物種は、固有な知覚と作用の器官を有し、知覚と作用(行動・運動)によって環境の対象とかかわっている(知覚と作用というツメをもつピンセットで対象をはさみこむかのごとく、である)。
これを\textbf{機能環}と呼ぶ。
それぞれの生物は、その生物固有な知覚と作用だけに連関した世界、機能環の集まり/連なりによってなす世界を生きている。
この意味世界を、ユクスキュルは「環世界」と呼んだ。
ダニのばあい環世界は単純で3つの機能環のみからなるが、犬や人間ともなれば環世界はもっと複雑である。
ダニと犬と人間はそれぞれ、客観的には同じ環境(Umgebung)に存在していても、異なる環世界(Umwelt)を生きている。
環世界概念は、私たち一人ひとりが固有な環世界を生きていることを含意している。
本研究の身体知の\textbf{個人固有性}\cite{suwa:2016}の考え方は、一人ひとりが自らの身体に照応した固有の環世界を生きていることとして、環世界とあわせて理解できる。

環世界の考え方は、身体知の学びの実践者が、当人固有の意味世界を生きるということ(\ref{sec:konzenittai}節参照)を下支えする。
先に述べておくと、私たち人間が生きる環世界は「表情」に満ち充ちており\cite{hiromatsu:1989}、そのことが第二部研究の根幹テーマになる。
なお、ユクスキュルの環世界論には、本研究と主張とはなじまない別の解釈にもつながりうる面もあるので、それについては、補論\ref{sec:horon_kansekai}で述べる。

\subsection{ノエシスとノエマの相互限定}
\label{subsec:noesisnoema}
木村敏(1931-2021)はユクスキュルやWeizsäcker(1886-1957)の仕事を踏まえながら、それを現象学の考え方と紐づけることで、人間が生きているということについての原理を論じている。
木村の論から、「身体で問う」ということを説明しよう。

Weizsäcker\cite{weizsaecker:1950}の提唱した\textbf{ゲシュタルトクライス}を確認することからはじめよう。
有機体は、知覚と運動をたえまなく「立て直し」続ける\footnote{Weizsäckerはこれを転機(クリーゼ)と呼んでいる}ことで、つど世界(対象)と出会い続け、みずからの生を維持している。
知覚と運動とでなすひとつの円環に有機体と客体が挟まれており、
同時に、この全体の円環のなかにこそ知覚-運動、有機体-客体の関係は維持される。
この関係性を、Weizsäckerはゲシュタルトクライス\footnote{クライスは円環という意味のドイツ語である。}と呼んだ。
象徴的なフレーズを\cite{weizsaecker:1950}から以下に直接引用しておこう\footnote{
ほかにも\cite{weizsaecker:1950}には以下のような象徴的なフレーズが書かれている。
「私が自分で動くときに私は一つの知覚を感じるという事態として、また私が或るものを知覚するときに私にとって一つの運動が現前するという事態として成立し(p.58)」
「知覚が自らを生じせしめる要因として自己運動を含んでいるというのではない。むしろ、知覚はそれ自体、自己運動なのである(p.59)」
}
\begin{quote}
ゲシュタルトクライスの要点は、一切の生物的行為において知覚と運動が互いに一方を代理しうる2つの状態であること、この両者は常に相互に隠蔽されていること、このからみ合い、代理、隠蔽には主体と客体の両者も関与していることにある(\cite{weizsaecker:1950}, p.17)。
\end{quote}

有機体の知覚-運動の関係性は「フィードバック機構」のそれとはちがう。
情報処理のフィードバックは知覚と運動を互いに独立した入力・出力とするが、クライスには入力も出力もない。
有機体の知覚と運動は、いわば「回転扉の両面」のように相互に隠蔽しあい交互に現出しあいながら回っているような関係をなすとWeizsäckerは言う。
それがゲシュタルトクライスである\footnote{
ゲシュタルトクライスは機能環(\ref{subsec:umwelt}項参照)よりもいっそう知覚と運動が「ひとつのはたらき」であることを強調していると言える。
機能環が強調するのは、知覚と行為の連関が各生物によって「固有」であることであろう。
}。

木村\cite{kimura:2005}はゲシュタルトクライスの考え方を、現象学の祖・フッサール(1859-1938)の考え方と統合する。
フッサールは「思考」という現象は、\textbf{ノエシス}(「思考する」という作用)と\textbf{ノエマ}(思考された内容・対象)という2面から成り立つことを述べた。
これらは\ref{sec:shintaisei}節で登場させていた「認知作用」「認知内容」に対応する語である。
木村は、人間のノエシス(思考する)とは、単に「あたまのなかだけで起こる精神的作用」ではなく
まさしくこのゲシュタルトクライス(知覚する-運動する)なのだと論じた
\footnote{たしかに私たちが日常で思考するとき、ロダンの『考える人』のようにじっとして思いめぐらすだけでなく、動き回りながら思考する\cite{saeki:1990}であろう。}
。
木村は音楽演奏を例にとる。
音楽演奏では、まさしく演奏者は、メロディを聴きとりながら(知覚)メロディを奏でている(運動)。
メロディとは過去・現在・未来をふくみもったなにか(ゲシュタルト)にほかならず、知覚運動カップリング(=ノエシス)のさなかでこそ、演奏者はメロディを感得できるのである(=ノエマ)。
木村はこれについて「演奏するというノエシス的行為が音楽のノエマ的表象を意識に送り込むのではあるけれども、ノエマ的な音形態を知覚しないで演奏行為をおこなうことは不可能(\cite{kimura:2005}, p.52)」だと説明している。
すなわちノエシスとノエマは、ノエシスがノエマを生み出すのと同時に「ノエマ的面がノエシスを限定する」という「相互限定」的な関係にある。

\subsection{身体で問う}
\label{subsec:questioningwithbody}
こうして、思考する、知覚する、行為する、という3種の作用は、「3つ独立した作用がインタラクションしている」
というよりも「本来的に三位一体だが、見かけ上3つの作用になっている」という関係としてとらえることが可能になる
(\ref{sec:embodiedwisdomlearning}節で頭出ししておいたことである)。
これが「認知」である。

本研究では、「生きている」ことにおける「認知」の説明をベースにしながら、「生きている」よりも主体性の高まった「学ぶ」における場合に転用して考えよう。
「認知」は「\textbf{身体で問う}」(という認知)と呼ぶべきものになる。
一点注意しておきたい。
上記した木村の音楽演奏の事例だと、3種の作用が「うまく滑らかに連関している」にみえるかもしれない。
ところが、一般に学習者が「ああでもない、こうでもない」というふうに身体で問うとき、必ずしもそうはならない。
むしろ、3種の作用は「もつれあう」ようにしてはたらき、「ぎこちない」関係でありうる。
そのもつれあいにおいて、新しく認知内容を創りなおし/気づきなおしてゆくのである。
「反省」は頭のなかだけで起こる活動ではなく身体的行為そのものだ、とするヴァレラ\cite{varela_et_al:1991}の考え方にも通底する。
身体で問うことの概念モデルを\autoref{fig:noesisnoema}に示す。

\begin{figure}[h]
  \centering
  \includegraphics[width=\textwidth]{./images/noesisnoema.pdf}
  \caption{「身体で問う」ことの概念モデル}
  \label{fig:noesisnoema}
\end{figure}
\begin{figure}[h]
  \centering
  \includegraphics[width=0.4\textwidth]{./images/cognitivecoupling.pdf}
  \caption{認知カップリングの概念モデル\cite{suwa:2016}}
  \label{fig:cognitivecoupling}
\end{figure}

図の描き方にこめた意味を解説する。
認知は、認知するというノエシス(認知作用)とノエマ(認知内容)からなり\footnote{
  「問エシス」と「問エマ」という用語を定義してもよいのかもしれない。
}
、両者が相互に限定しあっているという関係性を、円(枠線)と円の内部領域との関係によって表現している。
認知するというノエシスは、知覚する-行為する-思考する、という三位一体のひとつの作用であることを、(ゲシュタルトクライスの図に倣って)3分円が3つでひとつの円をなす関係で表現している。
完全な円として閉じきらず「互い違い」になっているのは、もつれあいの関係性を表すためである。
言うなれば、3種の作用のもちつ・もたれつ・もつれあう関係性を図示している。
そして、図に薄いグレーで描いたクエスチョンマーク(?)は、いまその瞬間には本人が窺い知ることのできないなにがしかである。
このあともつれによって生じたすきまから、新しいなにかとして問いへと流れこみうる存在である。
\footnote{
「?」は、郡司\cite{gunji:2019}がいうところの「外部」に相当しうるものだろう。
この図をみて、枠線の外になにかがありうること、それが「外部」たりうることを指摘してくれたのは
山本篤氏や児玉謙太郎氏である。
}。
思考内容や知覚内容と呼びうるものは、\autoref{fig:noesisnoema}内ではノエマに相当する(このことは、第二部の研究で扱う「表情」\cite{hiromatsu:1989}とも本質的にかかわる)。

\autoref{fig:noesisnoema}は、
\autoref{fig:cognitivecoupling}の諏訪による\textbf{認知カップリング}\cite{suwa:2016}(思考・知覚・行動の三作用を互いに影響を与えあう関係全体)を改変したものでもある
\footnote{
諏訪がこのように認知カップリングのモデルの図を描いたのは、「情報処理モデル」を批判的に乗り越えるという強いねらいがあったとのだと著者は察する。
情報処理モデルとなるべく対応づけながら認知プロセスを説明しやすくするように描いたのだろう。
いっぽうで、両モデルを丁寧に対応づけようとしすぎているせいなのか、
図の描き方や説明に、3つのはたらきが独立した作用であるかのようなニュアンスを著者は感じる。
\autoref{fig:noesisnoema}の概念モデルは逆にその部分が肝でもある。
}。
諏訪が情報処理モデルを乗り越える認知カップリングの図によって強調したことのひとつは、「思考が知覚を変容させる」という点である。
\autoref{fig:noesisnoema}・\autoref{fig:cognitivecoupling}はどちらも共通して「思考」が知覚(というか認知全体)をつくることを表す図になっている。
この点は、\ref{sec:shintaisei}節後半で指摘したように、身体性認知も必ずしも積極的には扱っていないことである。

\subsection{身体知の学び:環世界を主体的に創りかえゆく営み}
\label{subsec:shintaichinomanabi}
\autoref{fig:noesisnoema}で確認したように、「身体で問う」はすでにプロセスの萌芽でありうる。
身体知はプロセスのさなかに存在する。
身体で問うことで、新たな問いが(身体で問う営みが)生まれるからである。
その小さなプロセスがつむがれることで「身体知の学び」と呼びうるプロセスとなる(\autoref{fig:shintaichinomanabi})。
問いが問いを生むということは、「身体知がプロセスにならざるを得ない」のであると同時に、
「主体的にプロセスを生み出してゆく」ことでもある。
学習者は、学びがい\cite{saeki:1995}を求めて、より望ましい自身のありかたを渇望して、学びたいから、学ぶ。
ぎこちなくても、それは主体的にやっている。
生きながら学ぶ。学びながら生きる。
学びは一般に、終わりなきプロセスなのである。

\begin{figure}[h]
  \centering
  \includegraphics[width=\textwidth]{./images/shintaichinomanabi.pdf}
  \caption{身体知の学びの概念モデル}
  \label{fig:shintaichinomanabi}
\end{figure}

身体知の学びは、「環世界を主体的に創りかえてゆく」営みでありうるだろう。
学習者は、みずからの環世界の内側から、みずからの環世界を揺さぶるのである。
それは少なくとも、「思考」があるからこそ、主体的に「思考」を変えることで、身体で問うこと(思考と知覚と行為とその全体)そのものを変容させられるのであろう。
運動学習者はそのようにして、試行錯誤的に身体で問い続ける(\ref{chapter:hajimeni}章参照)。
主体的に、生きながら学び、学びながら生き、みずからが生き・学ぶ環世界を創り変えてゆく。
そうして「意味」が醸成されてゆく。
その意味こそが、身体知と呼びうるものなのである。
\autoref{fig:shintaichinomanabi}で主張していることは、\ref{subsec:constructionloop}項で述べる中島・諏訪・藤井によるFNSダイヤグラム(\autoref{fig:fnsdiagram})とも本質的には通底していると著者は考える。

本研究の第一部研究は、\autoref{fig:shintaichinomanabi}の意味づくりプロセス全体のありようを探究するものであり、
第二部研究はそれにくらべると\autoref{fig:noesisnoema}の意味づくりの源のありように力点を置きつつ探究するものと言える。



\chapter{運動学習の既往研究}
\label{chapter:motorlearning}
本章では、既往の運動学習研究を概観し、身体知の学びとして運動学習をとらえる本研究の特殊性を示す。

\section{スポーツ科学的アプローチ}
最たるのはスポーツ科学的アプローチであろう。
たとえば小林ら\cite{kobayashi:2009}は、陸上短距離走のexpertとnoviceを比較実験し、スタートから9歩まで(これは加速区間の前半にあたる)は、expert群の方が地面反力の力積(つまり接地中に足裏が地面から受けた力の総和)の水平成分が大きいことを報告している。
このように、身体運動が満たしているべき客観的特徴の正解を示す。
たしかに運動学習者や現場のコーチはこうした知見を得ることで、みずからの客観的身体運動になにが足りないかを知り、それをもとに練習の指針を立てることもできる。
スポーツ科学は、このように、運動学習の一助となる知見を蓄積してきたと言える。

だが、この反面にも目をむけねばなるまい。
諏訪\cite{suwa:2016}は、\textbf{入力変数/出力変数}という用語でもって、およそ以下のような問題意識を表明している。
たとえば上記知見に触れた学習者Aが、「水平成分の力積が大きくなっている」走りを実現するには、
ただそのまま「水平成分の力積を大きくしよう」とか「足裏で地面を後ろ側へ押し込もう」と意識すればいいというわけではない。
学習者Aが「足裏が地面に与える力積水平成分」を大きくなる走りを実現したとしよう。
実はA本人からすれば、その走りを実現するのは(促すのは)「両肩甲骨をクッと寄せる」という独自な意識かもしれないのである。
ある全身運動を実現するために、なにを・どのように意識すればいいのかは、個人によって、状況によって、変わりうるものである。
これを入力変数と呼ぶ。
スポーツ科学の知見が示す運動の特徴は、「結果的に運動がそうなっているべき」という特徴(これが出力変数)なのであって、
必ずしも、あるひとりの学習者にとって、これを入力変数にすればうまくいく、ということまで明らかにするものではない。
一般に、学習者本人にとって、そのときにふさわしい入力変数は、自明なものではない。
入力変数は本人が主体的に試行錯誤的に探らねばならないものである。
だから身体知の学びは険しい道のりとなるのである。
こうした諏訪の主張に、本研究は賛同する。
研究者も現場の学習者も、入力変数と出力変数を混同すべきではないだろう。
なおここでいう「変数」とは、ギブソン夫妻\cite{gibson_gibson_1955}が述べた「着眼点」とでも呼びうる意味の変数である。

スポーツ科学の知見が身体知の学びプロセスに対して有する価値は、主体的な試行錯誤のための材料としてである。
スポーツ科学は「What」の知見を提供するとするならば、身体知の学びの意味づくりを扱う本研究は「How」の知見と言える\cite{suwa:2016,horiuchi_suwa:2020}。
\section{量的アプローチの運動学習研究}
\subsection{情報処理アプローチ}
「機械」としての身体の運動をどう制御するのか、
という視点から運動を探究するのが、Schmidt\cite{schmidt:1991}に代表される情報処理アプローチである。
概要は、前章で述べた情報処理モデルとそう変わらない。
このアプローチがとらえる運動学習とは、情報処理システムが運動制御関数のパラメータを「修正」してゆくことだと言えよう。

しかし、「パラメータを修正」だけでは、身体知の学びは十分にとらえられない。
前節にも登場した諏訪による「入力変数」の着眼・発見、という用語を引用すれば、そのことはわかりやすい\footnote{
  ただし、本研究では「入力変数」という用語そのものは積極的にはもちいないことにする。
  入力変数というと、入力→出力、といった「情報処理的なにおい」がぬぐえないからである。
  諏訪がそれでもなお「入力変数」と呼ぶことのねらいは、やはり、なるべく情報処理モデルに寄り添ったうえで、
  「でもそれだと限界がある」、という論理で身体知論を展開するためであろうと私は察する。
  本研究では、こうした変数は「入力」なのではなく「ノエマ」であるととらえる。
}。
すなわち、誤解を恐れずに言うならば、「変数そのものの発見」のと「ある変数のなかで変数値の発見(これがパラメータ修正)」はまったく異なるのである。
\subsection{ダイナミカルシステムアプローチ}
情報処理モデルだと、機械を制御する中枢(=小人・ホムンクルス)が、筋それぞれに対応する鍵盤を弾いて指令を出すといった構図になる。
ロシアの生理学者・Bernstein(1896-1966)\cite{bernstein:1996}は、
そのような情報処理モデルだと、現実の状況においては、所望の運動を実行するために各筋へのどのように指令すればよいかが、不良設定問題(文脈の多義性や関節の自由度などが絡む)となってしまうことを指摘した。

そうしてBernsteinは、末梢側がなんらかの自律的な組織化(シナジー)をしているという考えを示した。
ここにギブソンの、行為と知覚とは環境のなかでカップリングして成り立つという考えを盛り込みながら発展してきたのが、ダイナミカルシステムアプローチ(DSA)である。
DSAがとらえる運動学習とは、知覚-行為システムが、現在成り立っている全身の協調構造から、(より良い)新たな協調構造を(創発的に)獲得することだと言えよう。

これらは「認知作用」を記述していると言えるが、認知内容(ノエマ)を記述するものではない。
くわえて、これまで述べてきた3アプローチいずれも、「expert-novice間」や「何かの実践の前-後」間を比較するかたちで
学習の時間を止めて「差」を示すスタイルであるが、それは、「意味づくりプロセス」を陽には扱わないことになり、その点が本研究とは異なる。
\section{質的アプローチの運動学習研究}
意味的側面に質的にアプローチする研究群もある。
\subsection{現象学的アプローチ}
体育学の領域では、「意味」の観点から運動の学習と教育を体系化したスポーツ運動学がある。
K.マイネル(1898-1973)\cite{meinel:1960}は、フッサール(1859-1938)現象学やヴァイツゼッカーの仕事\cite{weizsaecker:1950}(\ref{subsec:noesisnoema}節にも登場)にも触れながら、
スポーツ運動を現実におこなわれている姿のままでとらえようとする\textbf{モルフォロギー}的方法(形態学的方法)の重要性を唱える。
この「形態」とはなにか。
マイネルと親交のあった金子(1927-2024)
\footnote{金子は1952年ヘルシンキオリンピックの男子体操に日本代表として出場した選手でもあり、のちに指導者としても活躍した。}
は、マイネルの「形態」を重視する運動学には、同じくドイツのゲーテ(1749-1832)による動植物の形態発生学が礎にあるとみた。
形態(gestalt)は、静的な物的の構造のことではない。
「発生」とあるように、成長の流れのなかにこそ在る「かたち」のことである。
形態は、客観的な観察だけで得られるものでもない。
形態は、具体的な運動のなかで、運動している本人の自己観察や、あるいは運動者に潜入するような他者観察によって、生き生きと感じることのできる次元にある。
このことは第二部で扱う「表情」の考え方にも通ずる。

マイネルは、モルフォロギー的方法によって、全身運動のもちうる「質」を、局面構造、運動リズム、連係、流動、正確性、調和、弾性、先取りといった徴表にカテゴライズした。
スポーツ運動を個別科学(解剖学的・生理学的・心理学的あるいは物理学・力学など)によって分析するにしても、
それらに先立って/それらの方法で分析する出発点として、まずはモルフォロギー的方法が必要不可欠なのだとマイネルは説く。
また金子は、フッサールの現象学をも足場にしつつ、マイネル運動学を理論的に発展させ、それを金子独自の「発生運動学」へと体系化した。
金子は「コツ」や「カン」を身体知であるとし、科学知に対するものとして身体知を考えた。

このように意味づくりの面に質的にアプローチしている点は、本研究との共通点と言える。
いっぽうで、金子らの研究スタンスには「体系性」(これすなわち普遍的な記述)をめざすむきが強い、という指摘もある。
この指摘は、美学者(なかでもプラグマティズムを引き継いだ美学)・教育学者としてスポーツを探究する樋口\cite{higuchi_et_al:2017}によるものである。
著者も樋口の指摘に賛同する。
本研究は、意味づくりのプロセスを質的に記述するが、個別具体性を重視して、可能なかぎり、現実に起きるできごとの複雑さや「手触り」を色濃く残すようにして記述することをめざす。
それは体系性を第一義とするものではない。
\subsection{状況論アプローチ}
学習科学の領域では、状況論的なアプローチからも運動学習をとらえている。
\textbf{状況に埋め込まれた学習}\cite{lave_and_wenger:1991}の考え方は、
「知識の獲得・技能の習得」といった学習観や
「学校教室のような脱文脈化された知識の教授」といった教育観を乗り越え、
学習・教育を、学習者が対象の(文化的)実践の共同体に参加し、そのなかで生きてゆくなかで、
成員としてのアイデンティティを作ってゆくことだと考える。
学習とは本来的に共同体のなかでの学び合いとしてみるのである。

生田\cite{ikuta:2007}は、日本の伝統芸能における学び(これらも運動学習と言える)に着目し、技の
伝統芸能では、師匠は弟子に「このように動きなさい」と知識教授することはあまりなく、
弟子は、師匠とともに生活をするなかで学んでゆく。
生田は、M.モース(1872-1950)による\text{ハビトゥス}の理論を足場に、その学びを説明する。
ハビトゥスとはなにか。
(生田に寄せつつ説明すれば、)歩き方や身振り手振りなどの人々のふるまいが、
運動生理学的な都合でそうなっているというだけでなく、
その人々が生きる文化・社会のありようや価値観に仕立てられた「身体技法」だと考えるとき、
そのふるまいをハビトゥスと呼ぶ。
つまり、弟子は、師匠とともに生活するなかで当のわざの「世界へ潜入」し、その文化や状況に身全体でコミットしながら、生きてゆく。
そうやってわざの表面的な「形」の模倣を超え、\textbf{型}(ハビトゥス)を習得しているのであるとし、
このことは、伝統芸能に限らず学びの本質なのだと生田は説いた。

学びを生きているということのなかで捉えようとする状況論アプローチの基本的態度は、本研究とも響き合うと言ってよい。
そのうえで、状況論アプローチは、社会・文化・共同体という全体で学びをとらえる向きが強く、
いわば「(広義の)他者との関係性」に重点がある。
本研究は\textbf{状況依存的な知}\cite{clancey:1997}の思想を受容しつつも、
むしろ状況とのインタラクションが、本人の環世界の一部にどう取りこまれてゆくのか、という部分に力点を置く。
「学習者本人にとってのみえ」として包み返すのが、身体知の学びとして扱う態度と言えるだろう。

重要なことに、状況論アプローチの「ことば」の位置づけは、のちに述べる本研究の「ことば」の位置づけとかなり近い。
運動学習の現場において、ことばの果たす役割は、知識を「正確に」伝えたり「教授」するためではない。
師匠やコーチといった、同じ学びの場に生きる他者との「対話」をうながすことである。
このような位置づけのことばを、生田らは\textbf{わざ言語}\cite{ikuta:2011}と呼んでいる\footnote{
\cite{ikuta:2011}ではわざ言語の明確な定義は書かれていない。
}。

「他者と対話」と書いたが、状況論では「自己」もまた他者的存在でありうる。
その事例は前掲書\cite{ikuta:2011}に書かれている。
元・陸上短距離選手で、北京オリンピック陸上男子4×100mリレーでアンカーをつとめ日本に銀メダルをもたらした朝原宣治氏の例である。
朝原氏は、前掲書著者の北村との対談において、自身が日々変化してゆく感覚と対話し、感覚を開拓するために「感覚ノート」を書き綴っていたことを述べている。
感覚ノートにはたとえば\autoref{fig:asahara}のようなことが書き綴られている\footnote{
  前掲書\cite{ikuta:2011}には感覚ノートの写真が掲載されており、著者がその内容を文字起こししたものである。
}。

\begin{figure}[H]
  \centering
  \begin{mynote}
    
  体がまっすぐ前を向き、足もまっすぐ前に出るようにキープし、それで力がぬけないように腹筋でおさえられ地面に力が加わる感覚。
  角度を感じ、うでの手のひらのもどりでコントロール。・・・(後略)

  \textbf{[朝原氏5/8の感覚ノートより抜粋]}

  肩を脱力し、自然にダラリとおろし、そのPositionから体が(肩)がブレてねじれないようにうでをふる。
  そのときに手のひらを体の真中線・・・(後略)

  \textbf{[朝原氏5/15の感覚ノートから抜粋]}
  \end{mynote}
  \caption{朝原氏の感覚ノートの一例}
\label{fig:asahara}
\end{figure}


「感覚ノートの意義はなにか」という問い対して、朝原は以下のように答えている。
\begin{quotation}
  時にはびっしりと毎日、感覚ノートを書いて自分の中でどのように動きが変化するのか、
  こうすればどのような結果が出るのかを自分で試して練習していました。
  こうして始めると自然と自分の体としっかり対話ができるようにもなりますし、自分の体調の変化やバランスが崩れてしまうと、何か見逃しがちなひらめきとか、そういうものにキャッチする力というのがなくなってしまうのです。

  ですから、こういうふうに頭の中にあると、何かふとしたことがきっかけで、
  「あ、これは面白いのではないか」とか、「あ、こういう感覚で走ってみようか」とか、
  「次はこういう意識で練習しよう」など、どんどんイメージが湧いたり、ひらめきが出てきたりしますから、感覚ノートに書いてじっくり考えてやるのは効果があるように思います。(\cite{ikuta:2011},p.285 )
  
\end{quotation}

くわえて朝原氏は、感覚ノートに書き残されたことばは、のちのちの自分にとってどういう意義をもつのかについても語っている。
当初は「これさえつかんでおけば、いける」といった「普遍的」なものを期待しつつ書いていた面もあったという。
しかし、のちに見返してみたときの自分は、書いた当時の自分から変化してしまっているから、そう簡単にことは運ばないという。
ノートに残されたことばの果たす意義はむしろ、書いた当時から変化した自分が、「そこから自分でやり直しする」ためのヒントとしてなのだと朝原氏は言う。

このように、状況論アプローチは、学びの場において、つねに変化しゆく感覚を開拓するための「ことば」でありうることを指摘し、
それは本研究の思想とも相通ずることである。
\section{本研究アプローチの位置付け}
以上より、「本研究からみた場合」には、各立場は、およそのところ、それぞれ以下のように運動学習をとらえている、と整理してもよいだろう。
\begin{description}
  \item [古い現場の考え方]学習者が考えることなしに何万回の反復によってフォームを矯正する
  \item [情報処理アプローチ]情報処理システム(≒機械のパイロット)が、機械としての身体を操縦する各種パラメータを、修正する
  \item [ダイナミカルシステムアプローチ]知覚-行為システムが、物的環境のなかで知覚行為の新しい協応パタンを、創発させる。
  \item [現象学的アプローチ]学習者が、意味をつくってゆく。(研究は、その普遍性ある質的モデル・体系を記述する)
  \item [状況論的アプローチ]学習者が、他者ふくむ状況(社会や文化)のなかで生き、学び合う。
  \item [本研究]学習者が、身体で問い、環世界を主体的に創り変える。(研究は、個別具体的なプロセスを記述する)
\end{description}



% \chapter{目的と方法論}
\label{chapter:mokutekiandhouhou}
\section{本研究の目的}
本研究の目的は、運動学習を身体知の学びとしてとらえ、一人称研究によってその現実を描き出すことである。
それはつまり、
運動学習者が主体的に生きるなかで、学習対象の運動と自己の関係性を身体で問うことをとおして、意味を醸成してゆく(みずからの意味世界を創り変えてゆく)さまを、一人称研究によって記述することである。

第一部では、陸上十種競技である著者自身が「アスリートとして生きる」さま、すなわち、競技と生活が渾然一体となって、みずからの「走り」をより望ましいありかたへと創り変えてゆく意味づくりのプロセスを、描き出す。

第二部では、動いている身体の「表情」が意味づくりの源たりうるという思想のもと、動いている身体の「表情」の感得をうながす
ツールをデザインし、ツールの制作とツールをもちいた実践をとおして、「表情」から意味づくりするというのがいかなることなのかを探究する。
\section{構成的ということ}
\subsection{構成のループ}
\label{subsec:constructionloop}
身体知の学びを探究する本研究では、研究方法論や本論文の書き方など、さまざまな面にわたり\textbf{構成的}\cite{nakashima_et_al:2008,suwa_fujii:2015,suwa_hori:2015,suwa:2022}という態度を大事にする。
構成的とはどういうことか。
文献\cite{nakashima_et_al:2008,suwa_fujii:2015,suwa_hori:2015,suwa:2022}に拠りつつ説明する。

諏訪・藤井は、
ある知識や情報の意味することを、みずからのからだを使うことをとおしてもしくは生活文脈の実感に照らして(根ざして)わかることの重要性、
そのようなしかたでわかろうと試行錯誤するプロセスの重要性を説く。
そしてそれが、「ものごとを自分ごととしてわかる営み(=学び)」であると同時に「新しいものごとを生み出す営み(=デザイン行為)」
でもあるのだと、諏訪・藤井\cite{suwa_fujii:2015}は説いている。
このことは、本研究で述べてきた「身体で問う」という営み、身体知の学び、とも概して同じ主張である。

現代の生活では私たちはややもすればそれを怠ってしまいがちだと諏訪・藤井は警鐘を鳴らす。
たとえばだが、私が暮らす家の台所のキャビネットのデザイナ-には、それが十分にできていなかったのではないかと私は思う。
キャビネットの扉(\autoref{fig:cabinet})は開け閉め機構がローラーキャッチ式になっており、開け閉めするのに結構力がいる(固い)。
なのに、ツマミが小さく、くぼみもほぼなく、非常につまみづらいのである。
台所は料理をする場所なのだから、手が油や洗剤などでヌルヌルになっていることは「常」である。
料理中、キャビネットに収納した醤油や油を取り出そうとこのツマミを引っ張ろうとするのだが、
うまく引っ張れず、開けられないのである。
台所とはどういう場所なのか、デザイナーが自身の生活文脈に照らして考えていれば、こういうツマミは作らなかったのではないかと思う。
少し吟味すれば気づけたろうに、まるで「\ruby{他人}{ひと}ごと」である。
この事例は余談ではない。
書籍\cite{suwa_fujii:2015}に書かれた「みずからの生活文脈に照らして考える/考えられていない」とはどういうことなのかを、私自身が、みずからの生活文脈に照らして考えてみる、
ということを、私は実践してみたのである。
私は、私の卑近な例を引き出すことによって、その一端を読者に示したのである。

\begin{figure}[h]
  \centering
  \includegraphics[width=0.4\textwidth]{./images/cabinet.pdf}
  \caption{キッチンキャビネットの引っ張りにくいツマミ}
  \label{fig:cabinet}
\end{figure}

中島・諏訪・藤井\cite{nakashima_et_al:2008}は、こうした試行錯誤プロセス、それすなわちプロセスのなかで記号と実体とを共創的に変容させていくことを、\textbf{構成のループ:FNSダイヤグラム}として一般化している(\autoref{fig:fnsdiagram})。
\begin{figure}[h]
  \centering
  \includegraphics[width=\textwidth]{./images/fnsdiagram.pdf}
  \caption{構成のループ:FNSダイヤグラム(図は著者が描き直した)\cite{nakashima_et_al:2008}}
  \label{fig:fnsdiagram}
\end{figure}
\autoref{fig:fnsdiagram}を、\ref{sec:myexample}節で挙げた、私がより良い走りを身につける試行錯誤の事例で説明してみよう。
私は「接地位置を手前に引き寄せて真下接地を実現するために、骨盤を転がす意識で走ろう)」という構想(未来ノエマ)を抱いていた。
実際にそういう意識でもって\textbf{とりあえず}\footnote{
  \cite{suwa_fujii:2015}には、「とりあえず」が重要語句として索引に登録されている。
}現実世界(実体レベル)で走りを試してみた($C_1$)。
実際にやってみると、現実世界では、予期していなかったインタラクションが巻き起こった($ C_{\sqrt{2}} $、図の雲型部分)。
ここで\autoref{fig:fnsdiagram}内の「実体レベル」部分の描き方を説明しておく。
多数の小さな丸がネットワーク上に結びついているが、丸が実体レベルの要素、それらをむすぶ線は要素どうしの関係性である。
中央部にある楕円は、未来ノエマにもとづいてつくられた現象をさす。
雲型に描くことによって、この範囲を$C_1$以前に予め規定・予期することができないことを表している。

事例の説明にもどろう。実際に走ってみたことではじめて巻き起こったインタラクション($ C_{\sqrt{2}} $、図の雲型部分)は、少なくとも次のことを孕んでいた。
接地時のブレーキを減じることができたいっぽうで、「膝が曲がりすぎて」しまっていた。
膝の過剰な曲がりは、速いスピードで走ったときに、スイングする脚が前に出てこずにつんのめって転びそうになってしまう、という事態を招いた。
そしてこのことを認識するはたらきことこそ($C_2$)であり、私はインタラクションから現在ノエマを生成したのである。
図では、$C_2$は雲型のインタラクション全体を出どころとした矢印である。
現在ノエマとは、\autoref{fig:noesisnoema}でいうところの(「身体で問う」ことにともなう)「問い」に相当する。
私はこの現状認識をもとに、「じゃあ、接地時のブレーキを利用して、鞭運動のようにスイングする脚が自然に前へ振り出されるような練習をしよう」
といった構想(未来ノエマ)をつくり出したのだった。これが$C_3$である。
そうしてふたたびやってみて($C_1$)・・・(以下略)といったプロセスをなすのが、FNSダイヤグラムである。

FNSダイヤグラムはこのようにサイクリックなプロセスである。
同時にFNSダイヤグラムは、フラクタル的であることも藤井・中島・諏訪は説く。
たとえば「骨盤を転がす意識で走る」$C_1$というなかでも、より細かなFNSループは存在している。
FNSダイヤグラムは上記したように「身体知の学び」を説明する図にもなる\footnote{
  FNSダイヤグラムのフラクタルであることを考えれば、身体知の学びだけでなく、
  身体知の学びのより細かいプロセス単位である「身体で問う」ことにも相当する。
}。
これが構成的ということである。

研究するという営みも、デザイン=学び、構成的な営みであると諏訪・藤井\cite{suwa_fujii:2015}は論じている。
自然科学をはじめとした多くの学問では「分析」を示したり、分析的な論文の書き方をするが、
研究という営みそのものは構成的なはずだ、と中島は以下のように鋭く指摘する。
\begin{quote}
  一旦理論化してしまえば客観的評価の土壌に乗せますが、理論化自体は研究者の一人称的プロセスです。
  研究者は苦労して理論を構築した後で、それがさも最初からあったかのように演繹的に論文を書くのです、
  科学研究は客観的かつ演繹的になされると思っている人もいるかもしれませんが、実際はそうではありません(後略)
\end{quote}
\subsection{構成的という態度をふまえた本論文の書き方}
こうした構成的という態度を手放さないために、本論文では随所で、著者自身が研究者・生活者・実践者として本研究を進めてゆくようすをも語ることにする。
% 第一部\ref{chapter:monogatarizenshi}章や
第二部\ref{sec:tetsugakujousei}節はその顕著な例である。
第二部\ref{sec:tetsugakujousei}節について言えば、
著者は、第一部の研究を遂行することをとおして、実践者かつ研究者として、運動学習にかんする問題意識を醸成した。
その問題意識を足場にしたからこそ著者は第二部の研究の手がかり(それが哲学概念「表情」\cite{hiromatsu:1989}である)をつかみ、第二部研究着手に至ったわけである。
そういうプロセスを(概要的にではあるが)示したのが第二部\ref{sec:tetsugakujousei}節である。
その他、本論文の随所で、構成的という態度を意識し、通常の論文では省かれるであろうものごとであっても、積極的に書くことがある。
\section{一人称研究}
\label{sec:ichininshokenkyu}
\subsection{一人称研究の思想}
\textbf{一人称研究}は、ひとの知を研究するとき、客観性や普遍性や再現性を重んじる自然科学的な方法論(これが三人称研究)だけに縛られていては漏らされてしまうものごとがある、という問題意識の上に立つ。
2010年代に日本の人工知能学会から起こったムーブメントである\footnote{
ここでいう「かしこさ」とは、80年代の第二次AIブームでAI研究がぶち当たった「フレーム問題」や「記号接地問題」に代表される諸問題である。
2010年代は第三次AIブームの時期にあたり、その中核技術である機械学習が産業界へ続々と応用されはじめた頃であった。
}。
一人称研究とはなにか。
書籍『一人称研究のすすめ』\cite{suwa_hori:2015}では、次のように書かれている。
\begin{quote}
  あるひとが現場で出合った モノゴトを、その個別具体的状況を捨て置かずに、一人称視点で観察・記述し、そのデータを基に知の姿についての新しい仮説を立てようとする研究
\end{quote}
一人称研究は構成的な研究手法と言える。
一人称研究には2種類のスタイルがあると諏訪・藤井\cite{suwa_fujii:2015}は説いている。
\begin{itemize}
  \item 研究者が自分のからだや生活そのものを研究対象にする(研究者と被験者が同一)
  \item 研究対象は自分のからだや生活ではないが、研究対象者の固有名詞としての「顔」が明確に見え、さらにその対象者の知(もしくは知を発揮もしくは獲得するプロセス)を研究者が自分のからだや生活実体に対応づけて研究する
\end{itemize}
本研究第一部研究は、前者のスタイルである。
第一部の研究は、著者自身が対象者であり、陸上十種競技選手として、からだメタ認知を駆使して「走り」の学びを実施した一人称研究である。


第二部研究は、動いている身体が醸し出す「表情」\cite{hiromatsu:1989}を意味づくりの源として位置づけ、表情の感得を促すツールをデザインし、ツールをもちいた学びの実践をおこなう。
実践の対象者は、著者自身ではない他の対象者である。
主たるデータは、対象者がアプリをもちいて身体で問うた記述データである(次項で説明するからだメタ認知記述)。
対象者らが意味づくりしながら残した記述データをもとにして、陸上十種競技の一人称研究をやってきた著者が、その意味するところを解説をする、というやりかたを採る。
こうした入れ子になっている意味の記述は、佐伯がいうところの\textbf{二人称かかわり}\cite{saeki:2017}に符号する。
「園児のみている世界を保育士がみる」というふうに、私たちは他者の世界に「共感」するようにかかわることをとおして、他者の意味世界を解釈することができる\footnote{
  佐伯は、文献\cite{saeki:2017}のタイトルとして「『子供がケアする世界』をケアする」という入れ子になった命名をしている。
}。それが二人称的かかわりである。共感は同感とはちがい、当人の「思ったとおりそのまま」のことを読み取るわけではない。

第二部研究が一人称研究か議論の余地がある(上記した2つ目のスタイルの一人称研究かどうか)が、本研究では、一人称研究としては扱わないものとする。
前掲書\cite{suwa_hori:2015}の編者かつ著者である諏訪は一人称研究を推し進めてきたが\footnote{
  一人称研究は、ここ10数年をかけて、徐々に市民権を得てきている。
}、
近年諏訪が提唱する一人称研究\cite{suwa:2022}は、「構成的」という点を強調しているからである。
すなわち、研究者が研究という学びの営みを、やってみては考え、またやってみて・・・という構成のループをまわしながら、
問いを記述し、記述データを蓄積させてゆく研究が、一人称研究だということである。
これに照らすと、第二部研究は、著者がみずからの研究・デザインのプロセス(ツールをつくるプロセス)を日々一人称視点から問うたデータを扱うわけではない。
この点に重きを置けば、第二部研究は一人称研究ではないということになる。

またもし、第二部研究を構成的な研究と呼びうるとするならば、それは、哲学概念「表情」\cite{hiromatsu:1989}に着目して、それを机上の哲学に終始したわけではなく、
じっさいにアプリケーションにつくり、それをもちいた他者の学びの実践を試みたという意味においてであろう。

\subsection{からだメタ認知}
\label{subsec:embodiedmetacognition}

本研究は、身体知の学びのデータを取得する方法として、認知的手法:\textbf{からだメタ認知}\cite{suwa:2016, suwa:2022}をもちいる。
からだメタ認知は、「身体で問う」ことを促しながら、その内容を記述・記録する。
「メタ」という語句を意識して言い換えるなら、「ことばの力を借りてからだ認知をからだ認知する」手法とも言えよう。
ここでいう「からだ認知」とは、\ref{subsec:noesisnoema}項で述べた「認知カップリング」のことであり、学びにおける「身体で問う」のことである。

からだメタ認知でことばにする対象はからだ認知すなわち「自身がなにをどう思考・知覚・行為しているのか」である。
書かれた内容は、(身体での)「問い」として書かれているわけであるが、
問いにもさまざまありうる。
浦上\cite{uragami:2015}は、問いの種類を、〈感触〉、〈違和感〉、〈疑問〉、〈解釈〉、〈分析〉、〈仮説〉、〈問題点〉、〈問題意識〉、〈目標〉という9種類のカテゴリーに分類できると述べる。
このリストは概して、暗黙的・身体感覚的な問いから明晰的な問いへ、という順番で並んでいる。
からだメタ認知においては、〈問題意識〉や〈仮説〉といったはっきりした思考の形をとる問いだけではなく、
〈違和感〉といった曖昧模糊とした体感的な問いも、(内容の正しさにこだわらずに)積極的にことばとして絞り出して書きつけることが重要である。
「走り」を学ぶアスリートの事例で考えよう。
「走りのフォームの悪い動きは、歩きのフォームに原因がある」は〈問題意識〉に相当しうる問いである。
「接地直前に骨盤をスイッと前へ滑らせれば、ブレーキを減じられるだろう」は〈解釈〉に相当しうる問いである。
「いつもより肩甲骨の引かれかたがネバっこい気がするぞ?」は〈違和感〉に相当しうる問いである。

からだメタ認知の意義は、知覚・思考・行為の記録を残すだけではない。
このようにことばとして記述することで、思考内容(認知内容)は「外的表象化」される。
書き手は外的表象化されたことばと、否応なくインタラクションする(ことばは文字として知覚可能な対象になる)。
ことばにしてみると、推論・連想的にさらに新たなことばが紡がれ(=ことばがことばを生む\cite{suwa:2016})
、知覚をも変容させ、それが原動力になって、さらに新たな問いや行為を生成することができるのだ。
学習者の生きる環世界に、(今までは自覚していなかった)新しい変数が流れこむ。
それは\autoref{fig:shintaichinomanabi}で「?」として描いたなにかである。
上記例で言えば「肩甲骨の動きの粘り気」といった変数がたとえばそれに相当し、ことばを紡ぐなかで生まれ、変数として象られた身体感覚でありうる。
だからこそ、曖昧な問いであっても、その内容が正しいかどうかにこだわらずに、とりあえずことばにしてみることが肝要なのである。

からだメタ認知は、従来型メタ認知\cite{flavell:1979}とは大きく異なる。
従来型メタ認知の目的は「自らの思考を客観的にモニタリングする」ことである。
従来型メタ認知がメタ認知する内容は「思考」だけであり、「客観的に」とあるように、「ことばがことばを生む」ということは、つまり主体がことばに巻き込まれるといったことがねらいには含まれていない。

このようにして、ことばの力に半ば身をあずけながら、
ことばの力を積極的に利用しながら認知を進化(深化)させようとするのが、からだメタ認知である。
からだメタ認知は、生き・学ぶなかで身体で問う様相を観測・記述を残しながら、同時に、生き・学び・身体で問うことそのものを促す、構成的な実践手法ともなる。(\cite{suwa_fujii:2015}, pp.175-190)。
からだメタ認知の身体知の学びへの有効性は、ボウリングやダーツなど、多種の学習ドメインで例証されている\cite{suwa:2016}\cite{suwa:2022}。

\subsection{補足:「ことば」の位置づけ}
最後に、前項でも触れたことだが、本研究の「ことば」の位置づけを述べておく。
「思考」が認知を形づくることは\ref{subsec:noesisnoema}〜\ref{subsec:shintaichinomanabi}節にて述べたが、
それは、学びという営みにおける「ことば」の位置づけに直結する。
つまり、ことばは認知をつくる。
ことばは、知覚をつくり、行為をつくり、
ことばにするという営みそのものが認知であり、新たなことばを生み、認知をうむ。

ここまでで「言語化」と呼ばずに「ことばにする」という言い方をしてきたが、それにはこだわりがある。
言語化と呼ぶと、つい、従来型メタ認知のような、頭のなかでだけ起こるプロセスであるとか、「思考を思考する」といったふうに誤解してしまいかねない。
それはなんとしても避けたいのだ。

あえてこう言い切ってしまってもよいだろう。
暗黙知\cite{polanyi:1966}は「言語化できない」からこそ「ことばにする」のである。

こうした考え方、たとえばことばが知覚を変容させる、という点については、
諏訪\cite{suwa:2016}が挙げるダルメシアンの例を引いておく。
私たちは、曖昧なドットが集合した白黒画像をみせられてもなにもこれといった知覚できないけれど、
そこでさらに「犬がいます」と言われた瞬間(=ということばを念頭に置くと)、白黒のドットパターンに「ダルメシアンの姿」をみてとってしまう。
また、虹は「赤橙黄緑青藍紫」という語彙をもつ日本人には7色にみえるが、「Roygbiv\footnote{
アメリカ人は通常虹は、Red, Orange, Yellow, Green, Blue, Violetの6色だと習う。
}」の語彙をもつアメリカ人には6色に見える、といった言語相対主義的な事例にも端的に表れている。

長滝\cite{nagataki:1999}も知覚とことばの関係性について(文字通り『知覚とことば』と題した文献のなかで)以下のように述べている。
\begin{quote}
  知覚的認識と言語的認識は、それらのあいだにはっきりとした境界線が画定されえないほど、相互に影響しあいまた共働しているとさえいえる。
だからこそ、知覚世界の構造が言語によって再構造化されることもあれば、言語が知覚世界のうながしによってあらたな意味を獲得することもある。
(中略)
知覚的認識と言語的認識との境界が曖昧であるのは、それらがつねに相互構成的な創造的循環のなかに置かれているからである。

(\cite{nagataki:1999}, p.174)
\end{quote}

ことばは、なにかを正確に表し、それを他者に伝えるための媒体、というだけではなく、上記した役割、
いわば、自分の身体を進化させるための媒体でもある。
からだメタ認知は、見過ごされがちなことばのその性質を積極的に利用しようとする認知的手法であり、
本研究は、ことばをそういうものとしても扱う。





% \part{「アスリートとして生きる」学びの物語}
\chapter{物語をはじめる前に}

\section{物語の概要}
\label{sec:deca}
これまで書いてきたように私は、
研究者かつ陸上十種競技の実践者として、
からだメタ認知を駆使しながら「走り」の学びを実践してきた。
私は十種競技選手として、「走り」に十種競技の基礎があると信じている(この考えは、物語執筆中の現在も変わっていない)。
陸上競技の(十種競技の)文化に生きてきた私からすると、十種競技選手として走りに基礎を求める態度は決して珍しくはない。
十種競技得点における走種目得点の重要性を示す報告(e.g.\cite{yasuda:2013})もある。
しかも私は「走り」を不得意としていたのだ。
第一部のメインパート\ref{chapter:monogatari}章において、陸上十種競技選手である私が、生きていることと競技が渾然一体となりながら「走り」をより望ましいように創りかえてゆく意味生成プロセスを、物語として描く。


なぜ「物語」なのか?
アスリートである私が試行錯誤しながら、問題意識が変遷する様を描こうとするならば、私本人の一人称視点から見える世界と自身の関係性を時系列的に、そして、個別具体性を色濃く残したまま語る他ないからである。
それは必然的に「物語」になる。
物語であることの意義は、\ref{chapter:monogatarinoigi}章で説明する。

物語本編の\ref{chapter:monogatari}章では、私が記録し続けたからだメタ認知の記述を拠り所としながら、私が語り手となって、スキルの学びの物語を描く。
物語において、実践を私が生み出した問題意識は、文中でハイライトして表記する。
文中で鉤括弧「」で括った文言は、私にとって重要な変数や意識である。
先にも述べたが、新しい変数に着眼することが、一般に学びの始まり(必要条件)である。

2015年4月23日から2019年10月9日までの間に、623 件(総文字数 496,900 字)の記述を書き溜めてきた。
これを拠りどころとしながら物語として構成した。
623件の記述から、「生きていることと渾然一体であるアスリートの学び」の物語の骨子になりうるものを、私が厳選した(2015年12月5日から2017年9月23日の範囲であり、全623件のうちほんの一部であることは述べておく)。

身体知の学びのプロセスが、生きているなかで営まれ、生きていることと渾然一体であることを主張してきたが、実は、物語開始時点では「主人公の私」は渾然一体がよいことを未だ自覚できていない。
主人公である私は、アスリートおよび自身の学びを探究する研究者として、学びを進める中で様々な出来事をきっかけとして、2016年11月頃(\ref{sec:kegagenin}節)から次第に、「生活と競技の渾然一体」を自覚的に為すようになる。
「物語の語り手である私」は、「主人公である私」の学びが生活の上にあることを強調するために、可能な限り、学びに起因・関連したと思われる生活上の出来事にも随所で触れながら、物語ることにする。





% \chapter{学びの物語本編}
\label{chapter:monogatari}

\section{接地面から体幹を介して、肩甲骨へと繋がる意識}
\label{sec:denpa}
学部4年シーズンが終了した2015年11〜12月、私は「接地」に留意していた。
「フラット接地」(足裏全面が同時に接地)によって、「臀筋\footnote{
お尻の筋肉のこと。
大腿骨と骨盤をつなぎ、股関節の動きを制御する。
人体のうち最大の筋肉であり、陸上競技において重要な筋肉である。
}
で地面を押し、高出力を発揮しよう」と試みていた。
「身体が固く力みやすい」という癖をもつ私は、足首が曲がりにくく、つま先から接地しがちである。
自らの癖を強く自覚してからは、むしろ「カカトを振り下ろす」ことを意識することによって、フラット接地に近づけるという考えに至っていた(問題意識1:カカトを振り下ろすことによって足裏前面で接地し、臀筋を使って地面を押す)。

そんな中、私は学士卒業論文を執筆しなければならなかった。
執筆作業に追われ、練習時間はあまり確保できなくなった。
本格的な運動ができない状況で私が優先的に取り組んだのは、自室でできる「体幹トレーニング」である。
体幹トレーニングとは、腹筋群・背筋群を鍛える運動をさす。
その頃、体幹トレーニングには、自分なりに価値を見出していたのだ。
それが垣間見える記述を\autoref{fig:20151205}に示す。
「ボディバランスのトレーニング」が、体幹トレーニングのことである。
下線部で語られているのは、身体を意識的にあれこれ動かす以前に、まず「体幹部のコンディション(=ボディバランス)」を整えておく(問題意識2)ことの重要性である。
こうして私は毎日30分の体幹トレーニングに取り組むことを習慣づけた。

\begin{figure}[h]
\centering
\begin{mynote}
ボディバランスを重視したトレーニングは毎日のように欠かさないべきだということ。
(中略)久しぶりの走りだというのに、普通に過去最高級の走りだった。
芝での流しだが、重心の高い位置がキープしたまま、下に落ちずに進んでいく。
(中略)間違いなくここ数日間のボディバランスのトレーニングの成果だと思っている。
身体の、いろいろな部位の、いろいろなインナーマッスルに力が入る状態になっていると、力を使わずに脚が上がったりする。

\textbf{[2015年12月5日、からだメタ認知記述から抜粋]}
\end{mynote}
\caption{体幹トレーニングに自分なりの価値を見出した、私の記述}
\label{fig:20151205}
\end{figure}

やがて卒論執筆が終了し、2016年2月からは冬季練習に移った。
すると、前述のフラット接地による地面反力が、肩甲骨あたりにまで伝わってくる体感を掴んだ。
卒論執筆期に注力的に鍛えた体幹部が、臀部から肩甲骨付近へ橋渡ししたのだろうか。
私は接地面から肩甲骨までを「ひとつながり」に感じるという実感を得た。
走りにおいて肩甲骨の重要性が叫ばれる意味を、このとき初めて自分なりに納得できたのだ。
\autoref{fig:dempa}は、当時の私がその体感を探る様子である。
接地直前に踵にアクセントを置きつつ、肩甲骨から腕を振り込み(局面1)、接地瞬間に臀筋と体幹部を締めることで(局面2)、直後に地面反力が肩甲骨あたりまで伝播するのを感じる(局面3)のだ。

\begin{figure}[h]
\centering
\includegraphics[width=10cm]{./images/dempa.pdf}
\caption{地面反力が右踵から肩甲骨に伝播する体感」を探る私の様子(2016年2月)}
\label{fig:dempa}
\end{figure}

その発見以降は、肩甲骨を大きく動かすことを鍵とした。
2016年11月上旬まで、
% すなわち修士一年シーズンを通して、
すなわち2016年シーズンを通して、
私は肩甲骨の意識を拠点としながらスキルを模索した。
すると、私がこれまで気にかけてこなかった「腕振り」が自ずと意識の前面に浮上した(\autoref{fig:20160831})。
記述後半の「後ろ寄り」とは、腕の動きが、両肩を結んだラインより後方に広い可動域をもった状態をさす。
それまで無自覚的に「腕振り」として受け容れていた動きを、肩甲骨を大きく動かす体感をもとにして、「腕掘り(下線部)」という独特な表現へ再編成したということである(問題意識3:接地面から肩甲骨までのつながりの結果としての「腕掘り」)。

本節で登場した私の問題意識を列挙すると、以下のようになる。

\begin{itemize}
\item 問題意識1:カカトを振り下ろすことによって、足裏前面で接地し、臀筋を使って地面を押す
\item 問題意識2:体幹部のコンディションを整えておく
\item 問題意識3:接地面から肩甲骨までのつながりの結果としての「腕掘り」
\end{itemize}

\begin{figure}[h!]
\centering
\begin{mynote}
S(※私の練習パートナー)と話しながら、腕振り、肩甲骨の使い方について考える。
\underline{「腕掘る」という表現について、(中略)掘る位置は、自分のかなり手前なのだと。}
最\underline{初の加速のときは、もしかしたら、大きく大きく、遠くから掘ってこないといけな}
い\underline{のかもしれない。}
(手前を掘ることによって、自然に腕振りのレンジが、後ろ寄りになるということだ。
自分の身体より後ろの可動域が拡がる。
後ろで腕がしっかり動く。
すなわち、肩甲骨がよく動いているということも満たされる。

\textbf{[2016年8月31日、からだメタ認知記述から抜粋]}

「※」は本稿執筆時に著者が加えた注釈を指す。
以降の記述でも、適宜注釈を付してある。
\end{mynote}
\caption{体振りを「腕掘り」と解釈する、私の記述}
\label{fig:20160831}
\end{figure}

\section{「百均LEDトラッキング」の実践}
\label{sec:led}
% 修士課程に進んで
2016年シーズンに入って
まもなく(2016年4月)、私は、研究プロジェクト型授業のメンバーであるX氏(Tが在籍する大学の教員)に出会うことになる。
Xはインタラクション研究者及びメディアアーティストである、
それからというもの、私は、私自身の学びと研究について、Xと時折議論するようになった。

私は、自身のスキルを言葉以外の方法で外化する手法の可能性をXに持ちかけた。
私がおぼろげながらに想定していた外化手法は、スポーツ科学で常套的な「動きの可視化」である。
この観点で、Xとの議論から、「LEDライトを使えば、動きの軌跡がとれる」という案が湧き出て、これに着手した。
名付けて「百均LEDトラッキング」である。
私はさっそく百円ショップでLEDライトを購入した(\autoref{fig:ledlight})。
\begin{figure}[h]
\centering
\includegraphics[width=5cm]{./images/LED.pdf}
\caption{私が百円ショップで購入したLEDライト}
\label{fig:ledlight}
\end{figure}


LEDライトを身体に(以下の事例では右手と右膝)装着した状態で、暗所で走る。
運動の様子を固定点から長時間露光撮影すると、装着部位の動きが光の軌跡として表れる(\autoref{fig:ledpath}は、ある日の撮影結果である)。
自宅最寄りの百円ショップで発見したドーム型LED(直径3.5cm)を使用し、撮影はスマートフォンカメラアプリケーション「MagicShutter(当時数百円)」をもちいた擬似露光撮影効果で賄った。
実践は、ナイター照明の点く普段の陸上競技場ではなく、夜陰に包まれる土グラウンドで行なった。

\begin{figure}[h]
\centering
\includegraphics[width=10cm]{./images/ledpath.pdf}
\caption{百均LEDトラッキングの光軌跡(画像右手前に向かって、6歩走っている。
上部の軌跡は右手、下部の軌跡は右膝。
)}
\label{fig:ledpath}
\end{figure}

「動きの可視化」といえば、本実践よりも精密なテクノロジーを用いた「計測」(例えば、モーションキャプチャ手法など)が一般的だが、なぜそれを選ばなかったのか。
それらの計測には、使いこなすための知識が必要であるし、実際の計測に際しては多大な労力がいる。
私は、アスリートとして学ぶ実践を最優先にしていたため、練習時間を削ってまで、
テクノロジーを使いこなすための一連のものごとに労力を割くわけにはいかなかったのだ。

スポーツ科学では計測に時間をかけがちであるが、計測にかかる労力が、実は、学び手本人の問題意識の醸成プロセスの継続性を阻害する致命的要因になってしまうことを、諏訪・矢島・筧・仰木\cite{suwa_et_al:2012}は指摘している。
問題意識が生じたら、それをこのようなツール制作にすぐ反映させ、実践して可視化する。
早いサイクルで、つくり、いじることを通して、新しい変数を見出したり、問題意識を生み出すことは、学びにおいて本質的である。
「百均LEDトラッキング」実践のような「手軽さ」は鍵なのだ。
こういう意図からも、私は本実践に取り組んだわけである。

右手と右膝にライトを装着した。
右手を選んだのは、私にとって、上述した問題意識の「腕掘り」が軌跡としてはどうなっているのかを観察して、深く考えることが切に必要であると考えたからである。
右膝は、股関節の動きを表す重要部位として選んだ。

テクノロジー(モーションキャプチャ)によって
可視化する場合、一般に、身体は「可視化の対象」として、グラフ化されて描かれる。
これに対し、本実践で得た\autoref{fig:ledpath}は、その意味での「グラフ」ではない。
自分が主体として、自身の身体を空間に塗りつけるように描いた軌跡なのである。
それ故であろうが、私自身が図6の軌跡を見たとき、単に「グラフ」を眺める以上の、自分が自分を塗りつけたような感覚が迫ってきた。
すなわち、前節の「腕掘り」という体感が、観察される軌跡と符合したのだ。

すると興味深いことに、私は、「右手が掘った対象」(それは空気である)をも、この軌跡の中に見出すことになった。
そして、その空気は、「粘性のある空気」であるという解釈も生まれた(問題意識4:「粘性のある空気」を腕で掘る)。
以下に、該当する記述を掲載する(\autoref{fig:20160915})。

もし、テクノロジーによって可視化していたら、軌跡は「右手」以上のものを表さないので、それを「グラフ」として見るだけで、空気やその粘性にまで意識を及ばせることはできなかったであろう。
百均LEDトラッキングの実践は、「モノ(この場合はLED)を通して、自分の身体を手触るように問う」ことの有効性を、身を以て知った経験であった。

別の日にも、右手の百均LEDトラッキングをX氏(Xは陸上未経験者)とともに実施した(\autoref{fig:ledpath2})。
Xの軌跡と比較してみると、「粘性のある空気を腕で掘る」感覚(問題意識4)はまた少し違うしかたで増長される。
丸で囲った頂点近傍にに着目してほしい。
これは腕振りの腕を振り上げから振り下ろしに切り替わるフェーズである。
私の軌跡は滑らかだが、X氏のは尖っている。
これは、語り手の私には以下のようにみえてくる。
ふつう「腕振り」とは、「振り上げ」と「振り下ろし」2フェーズの繰り返す振り子のようにとしてとらえられ、そういう動きになりがちなのだろう。
すると、振り上げ切った直後に振り下ろしという別フェーズへ切り替わり、それが頂点の「とんがり」として表れるのではないか、と私は考える。
いっぽう「腕掘り」の意識は、腕の振り上げから振り下ろしのフェーズを「途切れ」させることなく、滑らかにつなげる意識でもある。
粘性のある空気を「掘る」感覚(問題意識4)のはこの振り下ろしへの切り替わりと振り下ろしフェーズに生じる感覚である。

\begin{figure}[h]
\centering
\includegraphics[width=\textwidth]{./images/ledpath2.pdf}
\caption{右手の百均LEDトラッキング:私(左)とX氏(右)の比較}
\label{fig:ledpath2}
\end{figure}


\begin{figure}[h]
\centering
\begin{mynote}
この光の軌跡のカーブ(※\autoref{fig:20160915}内でポインティングした部分)が、なんとなく大きくゆるやかになっているのがいい走りなのだ。
(中略腕を振り下ろす、振り下ろすというか、腕を掻くように掘るようにするときの「質感」が重要なのだと!!S(※私の練習パートナー)
も「『ぐ〜っっぐ〜っっ』っていう感じなんですよね」と前に言っていた。
そのことと同じことだと思うのだが、何も抵抗がなく一瞬で軽く振り下ろすように腕は動いてはいけない。
(中略)\underline{空気よりも密度が高い、もう少し粘性もあるような、そういうものをま}
さ
\underline{に「掘る」感じで腕は動くべきなのだ。}
(中略)一歩一歩確実に加速していくためには、この軽すぎない質感が大事なのである。

\textbf{[2016年9月15日、からだメタ認知記述から抜粋]}
\end{mynote}
\caption{「腕で掘る」対象は「粘性のある空気」と解釈する、私の記述}
\label{fig:20160915}
\end{figure}

本節で登場した私の問題意識は、以下である。
\begin{itemize}
\item 問題意識4:「粘性のある空気」を腕で掘る
\end{itemize}

\section{「軸」の意味を納得する}
\label{sec:jikunoimi}
まもなく私は、「粘性のある空気を腕で掘る体感」を「軸」という表現にも紐付けた。
実は、私はそれまで「軸」への着眼を敢えて避けてきた。
なぜなら、軸という定型句を安易に使ってわかった気になると、思考停止に陥ると考えていたからである。
「粘性のある空気を腕で掘る体感」という問題意識を持つに至って、ようやく、「軸」の概念が私の腑に落ちた。
どう腑に落ちたのだろうか?当時のからだメタ認知記述をみてみよう(\autoref{fig:20160917})。
「体幹まわりの筋肉(\ref{sec:denpa}節)が、全身連動を生むための拘束条件として機能すれば軸が形成される(問題意識5)」という解釈に至ったのである(下線部参照)。
詳しく説明する。
腕掘りが起こる「接地面から肩甲骨までの『伝播する体感』(問題意識1)」や、「掘られる空気の『粘性』(問題意識4)」は、動きの「重み」を表していると、私は解釈する。
それが生じるとき、腕(四肢)を「拘束」できている状態にある(拘束されていない四肢は、各々独立して「軽々しく」動いてしまう)。
そして、拘束を生むのは「体幹部の筋群(問題意識2)」であり、それこそが軸なのだ、という解釈である。

\begin{figure}[h]
\centering
\begin{mynote}
やはり「軸」が大事なのだと。
(中略)\underline{体軸は保たれたまま、その上でうねうね}し
て
\underline{いることが重要。}
(中略)自分から軸を形成するもの(骨か?)を動かしていってはいけない。
あくまで連動。
\underline{連動するためには、自由度があまりに高すぎる状態で}は
\underline{だめで、それなりの束縛条件をつくっておく必要があるのだ。そのひとつが軸}。
(中略)
\underline{最近ホットな腹横筋腹斜筋は、軸を意識したときにちょっと使われる感覚があっ}
\underline{た}。
\textbf{[2016年9月17日、からだメタ認知記述から抜粋]}
\end{mynote}
\caption{「軸」の概念が腑に落ちた、私の記述}
\label{fig:20160917}
\end{figure}

\begin{figure}[h]
\centering
\includegraphics[width=10cm]{./images/boltrun.pdf}
\caption{ウサイン・ボルトの走りに私が付す解釈}
\label{fig:boltrun}
\end{figure}

100mと200m走の世界記録保持者であるウサイン・ボルトの走り\footnote{https://youtube/89J4pgVVsQcより引用。ボルトが現世界記録9.58を記録した世界陸上2009年ベルリン大会時の走りである}には、
その性質が顕著である(\autoref{fig:boltrun})。
白丸部を四肢の連結部(程よい拘束条件)として機能させた上で、四肢を脱力させると、四肢が自動的に連動し、うねうねした走り
\footnote{ボルトが「脊柱側弯症(脊柱が身体の左右方向に湾曲している)」という持病を抱えることも、この走りには関係していると思われる。
}
になるのだと、私は解釈している。

本節で登場した私の問題意識は、以下である。
\begin{itemize}
\item 問題意識5:体幹まわりの筋肉が、「全身連動を生むための拘束条件」として機能すれば軸が形成される
\end{itemize}

\section{怪我の原因を「歩き」に見出す}
\label{sec:kegagenin}
こうした気付き(\ref{sec:denpa}〜\ref{sec:jikunoimi}節)の連鎖にあった
% 修士1年シーズン(2016年4月〜10月下旬)は、
2016年シーズン(2016年4月〜10月下旬)は、
十種競技全体、そして構成する多くの種目で自己ベストを更新する飛躍のシーズンとなった。
しかし、実は、学部4年時に受傷した右膝の怪我(膝蓋靭帯炎)が完治せず、ごまかしながら競技に出続けたシーズンでもあった。

シーズンを終え、きたる冬季鍛練期に備え、怪我の根本的な治療を決意した。
丹念にリハビリする中で、怪我を引きずり続けた根本原因をようやく痛感した。
それは、腰を落としながら、ゆっくりと前に大きく1歩踏み出すリハビリをした時のことである。
怪我の右足を踏み出すときに、痛みを恐れ不自然な動きになったのだ。その瞬間踏み出す時に「下腿(ひざ下)が振り出される」癖を感知した。
この癖は左右ともにあった(痛みは右だけだが)。

すぐに、走りでもその癖が悪さをしていると気づいた。
\autoref{fig:kiryu}に、その分析的な解釈を示す。

\begin{figure}[h]
\centering
\includegraphics[width=10cm]{./images/kiryu.pdf}
\caption{膝蓋靱帯を痛める私の走り(上段)。桐生選手の走り(下段と比較する形で示す。}
\label{fig:kiryu}
\end{figure}

「腕で掘ろう」(問題意識3)と、右腕を肩甲骨(体軸に近い部位)から大きく動かそうとするあまり、右下腿が大きく振り出され(局面1)、身体重心(白丸)より過度な前方接地となってしまうのだ(局面2)。
「真下接地」が理想なのに比べ、前方接地はブレーキをかけることになる。
更に、膝に、曲がる向きに力が加わり(「a」の矢印)、膝蓋靱帯に激しくダメージを与える。
そしてその直後(局面3)、接地脚の膝関節が屈曲し(「潰れた接地」)、接地脚である右脚側の膝関節を伸展させるように離地(局面4)しながら左脚のスイングが行われる。
つまり、非効率的かつ、膝蓋靱帯にダメージを与えるフォームだったから怪我をしたのだということに気づいた(問題意識6:怪我の原因は「前方接地」にある)。

下段の桐生選手(100m走で日本歴代3位\footnote{
本物語と同時期の2017年9月9日、桐生選手は日本人史上初となる9秒台、9.98秒(+1.8)を記録し、日本陸上界の新しい扉をこじあけた。
福井県でおこなわれた全日本インカレの決勝レースでのことである。
\autoref{fig:kiryu}の動画のURLは以下である。 https://youtube/Sg1fyPh294w
})の走りと比べれば、違いは明らかである。
桐生選手も下腿が大きく振り出される特徴を有しているが(局面1)、接地は重心真下で(局面2:スイングする脚の膝が、私と比べてすでに接地脚(右脚)の膝よりも前方に出ていることに着目)、右膝が潰れることなくスムーズに前に進めている(局面3、4)。
畳んだ左脚の素早いスイング(局面2、3)がこの接地に関係しているのだろう(なお、スイング脚に着目したのは本物語執筆時である)。
私のこの動き(膝屈伸)は、皮肉なことに、力みすら生んでいた。

本節で登場した私の問題意識は、以下である。
\begin{itemize}
\item 問題意識6:怪我の原因は「前方接地」にある
\end{itemize}

\section{「立つ・歩く」を見つめ直す}
\label{sec:tatuaruku}
この経験(リハビリ動作で前方接地という欠点を自覚)を通して、ある仮説が私に芽生えた。
前方接地の悪癖は、日常生活における何気ない「歩き方」に根付いているのではないか?
だからこそ走りにも表出してしまうのではないか?という仮説である。

そうであるならば、歩きを改造せねばなるまい
(問題意識7:走りのフォームの悪い動きは、歩きのフォームに原因がある)。
私は、より良い歩きを探究せんと模索を始め、数多くの新しい変数を見出し始めた。
それが如実に表れた記述を以下に掲載する(\autoref{fig:20161125})。
この記述には、数多くの変数への言及がみられるが、記述中最後の変数「インパクトの瞬間に一番ヒットして、そのあとは『軽く引っ掻く?』ような感じでスカッと」以外は、どれもほぼ同様のことを表す体感や解釈である。
色々な角度からサーチライトを当てるようにして、アクセントの微妙な差異を確かめているのだ。

\begin{figure}[h]
\centering
\begin{mynote}
歩きが練習なのだと胸を張って言える。
靴によって感覚が全然異なるが。「スイっとグイッと接地中最後の最
後まで力が加わっている」必要がある。
それが「膝抜き」であり「ハムウォーク」であり、効率よく力を加
えている歩き方なのである。
そして、別名「体重移動だけで歩く」という動きなのである。
新たなチェックポイントとして、離地した直後に、足裏が後ろから見えてはいけない。
なるべく「足裏を見せずに歩く」のだ。中学生への指導として「足の裏を見せないように歩くんだよ」というもの、それの意味も今は「ナンバ歩き」、この歩き方の結果としての話だという理解になっている。
「接地中の地面への力の加わり方は、あまりアクセントが無く一定な感じ」。
普通の今までのダメな歩き方だと、「インパクトの瞬間に一番ヒットして、そのあとは『軽く引っ掻く?』ような感じでスカッと」というような、力のかかり方の時系列変化なのだ。

\textbf{[2016年11月25日、からだメタ認知記述から抜粋]}
\end{mynote}
\caption{「歩き方」に多くの変数を見出している、私の記述}
\label{fig:20161125}
\end{figure}

私はさらに、日常生活において「立つ」ことを、走りを改善するための基礎スキルとして位置付けた。
この思想は、原初的な身体のあり様を説く野口\cite{noguchi:2003}に多大な影響を受けている。
野口は卵が立つ様子から、筋力に頼らずに骨だけで立つことこそ良い立ち方であると洞察した。
卵はほんの一点だけで支えられ、その状態から少しでも外れると、すぐに倒れる。
しかし、すぐ倒れるということはわずかの力で動き出せることでもある、という逆説めいた説を、野口は以下のように説明する。
\begin{quote}
運動能力が高いということは、その動きに必要
な状態の差異を、自分のからだの中に、自由に
創り出すことができることである。
(\cite{noguchi:2003}、p.22)

\end{quote}

私は、以前から野口の思想を知りながらも、実は軽視してきた。
怪我と向き合って初めて、野口の弁が身に染みるように理解できるようになった。
私はこう悟った。
自分は、本質的には日常生活で「立つ」こと(自然に立つ)すらできていないのだ、と(問題意識8:自分は「立つ(骨で立つ)」ことすらできていない)。
それまでの私は(身体障害や身体機能不全を抱えていないということもあり)、「立つ」という運動に「立つことの達成具合」といった見方をしたことなど一度もなかったし、
「日本語を喋れる」とか「自転車に乗れる」とかと同様に、「立つ」ことを、当たり前にできていることとして不問に付していたが、
その「立てている」とはしょせん、「日常生活に不自由なく立つということができている」という意味にすぎなかったわけである。
私にとって「立つ」という運動の意味が大きく変革された。
このように「立つ」を「自然に立つ・骨で立つ」という意味で捉え直したとき、力んで走ってしまっている自分は、まともに「立つ」ことすらできていないこと思い知ったのである。
走っている時に「真下接地」を達成できていないのが何よりの証拠であろう。
よく考えてみると、真下接地は、「自然に立つ」ことの必要条件そのものである。

それからというもの、私は、「無駄な筋力を使わず、重力を最大限利用した高効率な動き」を生む身体(問題意識9)のあり方を目指すようになる。
「立つ」ことはその基礎である。
これに則した形で、Tが重視してきた体幹トレーニング(\ref{sec:denpa}節)の意味も更新される。
以下記述を掲載しよう(\autoref{fig:20161203})。

\begin{figure}[h]
\centering
\begin{mynote}
体幹にいい具合に力とか刺激が入っている状態というのは、何も力を入れなくても、「骨で立つ」状態が自然に維持できるような状態である。
そうなって初めて「椎骨(※連結して脊椎をなす骨)を積み上げる」感覚が芽生える。
(中略)つまり、「バランスの良い体幹トレーニングは、骨で立つため」に行っているのである。
力をいれずとも正しい姿勢になるようなトレーニングなのである。

\textbf{[2016年12月3日、からだメタ認知記述から抜粋]}
\end{mynote}
\caption{「骨で立つ」ことから、体幹トレーニングの意味を見直した、私の記述}
\label{fig:20161203}
\end{figure}

こうして私は、立つことを基礎として、変数を続々と開拓しながら歩きを改造し、それらを走りや他の動きと関連付けることを続けた。
代表的な変数を以下に挙げよう。

\begin{itemize}
\item「上から吊られている」(\autoref{fig:walk}の全局面)
\item「接地直後に一瞬『ふっ』と膝を抜く、崩す」(\autoref{fig:walk}の局面1)
\item「接地位置中心に転がっていく(全身が)」(\autoref{fig:walk}の局面2)
\item「足裏が地面から『剥がれる』」(\autoref{fig:walk}の局面3)
\end{itemize}

\autoref{fig:walk}は、私がこれらの体感と向き合いながら「歩き」を試行錯誤するようすである。
\begin{figure}[h]
\centering
\includegraphics[width=10cm]{./images/walk.pdf}
\caption{私の歩きに生じる体感}
\label{fig:walk}
\end{figure}

「上から吊られている」ような重心の高い姿勢によって、接地が潰れるのを防ぐ。
その上で、「接地直後に『ふっ』と膝を抜き・崩す」ことにより、膝関節屈伸による「跳ねる動作」が抑えられ、自然な倒れこみとして次の1歩が繰り出される。
その結果、離地では「足裏が剥がれる」感覚が生じるのだ。これらのことの総体として、全身に「転がる」体感が生じる。
「転がる」ことは、走りでは素早い股関節回転となる。
なお、図内には体感の効果線や注釈を描きくわえているが、この図の歩きにおいて、確実にこれらの体感が生じているとは言い難い。
まだこの時点では、「歩き」を「できている」とは言えない状況であり、あくまで試行錯誤のさなかにあり、上記の体感が生じたり消えたりしているところである。
日常動作を、根本からとらえなおして、創りなおしてゆくのは難しいものである。
上述したように、なんら疑いなく当たり前に「できている」と思って年月を生きてきてしまっていたため、慣れてしまっており、
歩きを試行錯誤的に試すなかで生じる体感は非常に繊細で微妙なもので、ある。
このように、歩きかたを無駄な筋力をつかわないしかたに創りなおすのはとても微妙な現象であり、

残念ながら、
% 修士2年シーズンの
2017年シーズンの
幕開け直前(2017年3月)に、私は大怪我を負ってしまう(右足舟状骨の疲労骨折)。
コンディションに気遣っていたつもりでも、蓄積した疲労に十分にケアしきれなかったのだろう。
スキルを学ぶこととは、こうも儘ならないのかと肩を落とした。
% 修士2年シーズンにあたる、
2017年シーズンにあたる、
2017年3月〜2017年9月(本物語終点)は、怪我からの復帰が叶わなかった。
ジョギングすら出来ず、足への衝撃がない穏やかな運動のみ許された(2017年8月に1週間だけ一時復帰するが、痛みが再発した)。
しかし、今や、立つ・歩くことも練習と化している。
したがって、学びの道は決して閉ざされていないと、私は自らをそう信じることに決めた。

本節で登場した私の問題意識を列挙すると、以下のようになる。

\begin{itemize}
\item 問題意識7:走りのフォームの悪い動きは、歩きのフォームに原因がある
\item 問題意識8:自分は「立つ(骨で立つ)」ことすらできていない
\item 問題意識9:「無駄な筋力を使わず、重力を最大限利用した高効率な動き」を生む身体
\end{itemize}

\section{日常生活にあるモノをツールに転じて、身体を問う}
\label{sec:monowotool}
怪我によって、運動が厳しく制限されると、競技場での練習時間が短くなり、生活の時間に新たな余白が生まれた。
% こうした修士2年開始のタイミング(2017年4月)で、X(\ref{sec:led}節に登場)が私の修士研究副査となった。

2017年シーズンでも、X(\ref{sec:led}節に登場)から、私の取り組みに対し定期的にアドバイスをもらう機会をもらえるようになった。
これらの事態が重なったことは、さらなる学びの開拓を目論む私にとって、「モノを通して自分の身体を手触るように問う(\ref{sec:led}節)」態度を加速させた。

まず、私は、「コンピュテーショナルツールの試作」に取り組むようになる。
自らの身体に新しい気づきを得るためのツールである。
より本研究に即したかたちで言い換えれば、自らの身体とことばの関係性を進化させるためのツールである。
電子工作やプログラミングのスキルが皆無に等しかった私にとって、制作はかなり大変だった。
だが、その大変さはむしろ、ふだんの練習や百均トラッキング(\ref{sec:led}節)のような手軽い取り組みとは異なった新鮮なしかたで自らの身体のありかたに迫ろうとするゆえでもあり、私は楽しく取り組むことができた。

各ツールからアスリートとして価値高い問題意識を得られたわけではないのだが、
アスリートとしてツールを作ってみる経験そのものが、少なくとも、私のアスリートとして「モノ」とじっくり向き合う鍛錬になったろうと私は考える。
ツール試作の取り組みについて以下に簡単に述べておく。
2017年4月〜5月、はじめに、上下の加速度を音の高さにリアルタイムに変換する「加速度可聴化ベルト」を制作した。
2017年7〜8月(\ref{sec:jikunoimi}節と\ref{sec:tatuaruku}節の間の時期)には、
PCのwebカメラからの映像を入力とし、常に各ピクセルにおけるフレーム差分を計算して、それをもとにして効果を付与したドット映像を出力し続けるインタラクティブな鏡型の「FusionMirror」を試作した(\autoref{fig:fusionmirror})。
\begin{figure}[h]
\centering
\includegraphics[width=10cm]{./images/fusionmirror.pdf}
\caption{FusionMirrorの画面キャプチャ(私が片脚で立ってバランスをとるようす)}
\label{fig:fusionmirror}
\end{figure}
%\footenote{
%制作にはProcessing言語をもちいた。ProcessingはJavaベースのプログラミング言語であり、画面上に動的なスケッチを生成するのに優れている。
%}。
FusionMirrorの制作にはProcessing言語をもちいている。

Processing言語をもちいたスケッチ生成や画像処理の可能性をみた私は、簡易的な2次元モーションキャプチャリングシステム「DIYモーションキャプチャキット(以下DIYモーキャプキット)」にも取り組んだ(\autoref{fig:diymocapkit}〜\autoref{fig:diymocapgraph2})。
DIYモーキャプキットは、映像をProcessingで画像処理して動きを可視化するソフト面だけでなく、発泡スチロール球に色を塗った手作りマーカや撮影背景用の黒布も自作している(\autoref{fig:diymocapkit})。
なるべくささっと手軽に撮影環境を準備できるようなキットを作ろうと工夫した(\autoref{fig:diymocapenv})。
実際にDIYモーキャプキットをもちいて、私と陸上未経験者とを比較するかたちで「歩き」の撮影や独自の可視化にも試みた(\autoref{fig:diymocapgraph1}、\autoref{fig:diymocapgraph2})。
歩きという運動を扱っているのはもちろん当時の私が問題意識7・8・9(\ref{sec:tatuaruku}節)を抱いていたからである。
例えば\autoref{fig:diymocapgraph1}は、左肩・左肘・左膝・左足首の4点にマーカを装着して撮影したものを、
その4点を上から順に線で結び、その軌跡を映像に重ね描きするようにしたものである。
当時の私にとって、黄色プロット(膝)の軌跡が密なエリアは「接地中」なのだが、著者(右側)と陸上未経験者(左側)では、
その疎密変化の具合が異なり、著者のほうが疎密変化が緩慢であることがみてとれる。
私にとってこのことは、\ref{sec:tatuaruku}節で登場した「接地直後に『ふっ』と膝を抜く、崩す」という変数とむすびつくものであった。

\begin{figure}[h]
\centering
\begin{minipage}[b]{0.45\linewidth}
\centering
\includegraphics[width=\linewidth]{./images/diymocapkit.pdf}
\caption{DIYモーキャプキットの手作りマーカと黒布}
\label{fig:diymocapkit}
\end{minipage}
\hspace{0.04\linewidth}%画像間の余白
\begin{minipage}[b]{0.45\linewidth}
\centering
\includegraphics[width=\linewidth]{./images/diymocapenv.pdf}
\caption{DIYモーキャプキットの撮影環境}
\label{fig:diymocapenv}
\end{minipage}
\end{figure}

\begin{figure}[h]
\centering
\includegraphics[width=\textwidth]{./images/diymocapwalk.pdf}
\caption{DIYモーキャプで撮影した歩き(上段:私、下段:陸上未経験者、右肩・右肘・右膝・右外踝の4点に自作マーカを取り付けている。}
\label{fig:diymocapwalk}
\end{figure}

\begin{figure}[H]
\centering
\begin{minipage}[b]{0.45\linewidth}
\centering
\includegraphics[width=\linewidth]{./images/mocapwalkplot.pdf}
\caption{DIYモーキャプキットで試した可視化1}
\label{fig:diymocapgraph1}
\end{minipage}
\hspace{0.04\linewidth}%画像間の余白
\begin{minipage}[b]{0.45\linewidth}
\centering
\includegraphics[width=\linewidth]{./images/mocapwalkphase.pdf}
\caption{DIYモーキャプキットで試した可視化2(相空間のグラフを映像に重ねる)}
\label{fig:diymocapgraph2}
\end{minipage}
\end{figure}


このように、ツールの制作には、
% 修士2年シーズン
2017年シーズン
を通して断続的に取り組んだ。
上述のように、各ツールからアスリートとして価値高い問題意識を得られた訳ではない。
しかし、アスリートとしてツールを作ってみる経験そのものが、私の「モノへのまなざし」を鋭敏にした可能性がある。

そして、興味深いことに、この態度が日常生活にも表れ出たのである。
私は、日常生活で、一見競技に関係ないモノと自身の身体の関係を積極的に取り結び、ツールとして転用することによって身体を問い、スキルを学ぼうとした(問題意識10:日常生活で身を取り巻くモノは、身体を問うツールになる)。
生活における些細な気づきの瞬間を逃さずに、アスリートとして生きようとしたのである。
その片鱗を以下に物語ろう。
\subsection{「石花」にみる、「骨で立つ身体」}

2017年5月、とあるコワーキングスペースに、友人と赴く機会があった。
イベント等に積極的な場所だからか、様々な展示や広告が陳列されている。
その場で、「石花(石を絶妙な形に積み上げる遊び)」の活動団体ポスターに目線が吸い込まれた。
積み上がった石に「骨で立つ身体(問題意識8)」を見出してしまったのだ。
私は、怪我で走れない状態だが、スキルの学びを止める気は一切なかった。
その押し込められた「学び欲」が、漏れ出た瞬間であったかもしれない。

早速後日、自宅から程近い相模川へ赴き、石花を実践してみた。
石の選定から始める必要がある。
選定段階からすでに、朧げながら、自分の身体(骨)と石の関係を探っていたのかもしれない。
積み上げるのにも苦戦しながら、ようやく作品が1つできた(\autoref{fig:ishihana})。
この作品は、安定したDの上に、下から順にC、B、Aと積み上げるのでは成立しない。
BとCを手で支持しながら、同時にAを「真ん中を貫く」ように置くことで初めて、安定が生まれる。
私は作りながらそれを実感した。

\begin{figure}[h]
\centering
\includegraphics[width=10cm]{./images/ishihana.pdf}
\caption{私の石花作品}
\label{fig:ishihana}
\end{figure}

石同士を結ぶ「筋肉」は一切ない。
4つの石の間に素朴に成立する絶妙なバランスを見て、筋の緊張能力に頼らず「骨だけで立つ」ような自らの体感を体内に湧き起こらせたのだ。
石花作品からは、「軸(問題意識5)の姿」すら見えてくる。
軸の姿とは、「そのモノの見た目が棒状」であることよりも、「どこから眺めても『スッ』と1本通っている感じ」なのだと、私は見抜いたのである(\autoref{fig:ishihana}の右側写真)。

\subsection{リュックを「腹負う」}

バランスの意識は、重心の意識と密に関係するものである。
春雨の降る新宿で、私は建物を出た。
傘をささねばならず、それを開いて歩き出した時、愛用のリュックが濡れるのを回避するために、とっさに前に抱えてみた。
この瞬間、新鮮な体感が舞い降りたのである。
その時の私の意識が、\autoref{fig:20170513}のメタ認知記述に表れている。

\begin{figure}[h]
\centering
\begin{mynote}
雨の新宿を歩いていて、リュックを前に背負う。
いや、「『腹』負う」。
ここで気づく。
リュックを前に下げると、「正しい歩き」の感覚がおりてきやすくなる。
\underline{リュック}-
\underline{身体系という1つのモノの重心は、体ひとつよりも全然前に移動するからだろう}。
\underline{ふつうに接地したときの感覚が、リュック-身体系の真ん中を貫いてくれる}感覚がある。
末端だけで歩くような感覚にはなかなかならない。

\textbf{[2017年5月13日、からだメタ認知記述から抜粋]}
\end{mynote}
\caption{リュックを「『腹』負う」体感について解釈する、私の記述}
\label{fig:20170513}
\end{figure}

身体重心よりも前方に位置する「リュックと身体を合わせた重心」を、あたかも身体重心のように感知することで、接地位置に乗り込むという、良い歩きの感覚(問題意識9)に迫ることができる。
これを「リュックを『腹』負う」と命名した。
動作だけでなく、日常生活のモノと動作をセットで括った、
この命名行為は価値高い。
このエピソードは、「生きること自体を、スキルの学びの前面に出す態度」の顕れであると私は解釈している。

\subsection{洗濯機を一人で運ぶ、体幹トレーニング術}

夏休みのある日のこと、私は兄から引っ越しの手伝いを頼まれた。
大きく重く持ち手もない洗濯機を、玄関から所定の位置へと運び入れる必要があった。
その経路(廊下)は狭く、複数人で運搬するのは至難の業である。

このように「問題」が発生した時こそ、一般に、人は自身の置かれた状況に思索を巡らすものである。
私は次のようなことを考えた。
「一人で運ぶしかない。
運び得るのはこの場には自分しかいない。
洗濯機には持ち手もない。
鍛えてきた体幹の筋群(\ref{sec:denpa}節)をうまく使って抱えれば、良質なトレーニングになるし、結果的に一人で運び切れるのではないか?」。
そう決断し、それを成し遂げた。
この試行にはいかなる問いが生じていたのだろうか(\autoref{fig:20170825})。

\begin{figure}[h]
\centering
\begin{mynote}
R(※兄)には運べないのになぜ自分に運べるのか。
これは間違いなく、体幹部の使い方であろう。
もちろん少なくとも体幹の筋肉が発達していない限りは、「体幹で支える」ことはできない。
腕などではない。
「体幹で持っている」のだ。「ものの重心」と、「自分の重心」と、「そのあわせた重心」と。
これらをすべて身体で感じて、体幹を用いてうまく操作することが肝心なのだ。

\textbf{[2017年8月25日、からだメタ認知記述から抜粋]}
\end{mynote}
\caption{洗濯機を体幹で運ぶことを解釈する、私の記述}
\label{fig:20170825}
\end{figure}

身体とモノが一体としたときの重心を見極めんとする態度(下線部)は、前節で語ったリュックの事例と相通じており、興味深い。
私にとって、これらの変数関係は「引き出せる術」になっていると考えられる。
「ウエイトトレーニング(高重量ウエイトを挙上する筋トレ)」では、ややもすると、その見た目の雄々しさや豪快さに酔いしれたり、挙上できる最大重量値に囚われる。
私も過去に、そのマインドに嵌まり、自身の「力み癖」を助長し、身体の連動性を減じてしまったことがあった。
そんな私は洗濯機を、「連動性を生む体幹(問題意識5)」を注力的にトレーニングするための、いわば「数値なきウエイト」と見立てたのだ。

本節で登場した問題意識は以下である。
\begin{itemize}
\item 問題意識10:日常生活で身を取り巻くモノは、身体を問うツールになる
\end{itemize}

\section{醸成された問題意識群と、それらの相互関係性}
以上で物語を終える。
物語のまとめとして、私の物語で登場した全10個の問題意識の関係を、\autoref{fig:zentaizou}に示す。
私が自身の走りのスキルを学んだ「howのプロセス」は、こうした問題意識(とその関係性)に支えられているのだと私は解釈している。

\begin{figure}[h]
\centering
\includegraphics[angle=90, height=\textheight]{./images/mondaiishikizentaizou.pdf}
\caption{私の物語に登場した問題意識の全体像}
\label{fig:zentaizou}
\end{figure}

問題意識1〜6において、走りにおける身体を自分なりに分節・焦点化し、それらや地面の関係を考えた。
1〜6は、立つこと歩くことと分離しているわけではない。
例えば6は、走る以前の、「歩くリハビリ動作」に発見した問題意識であった(\ref{sec:tatuaruku}節)。

「粘性」や「拘束(軸)」といった仮想的・比喩的な体感(4や5)さえも問題意識として浮上した。
そののちに、「歩く」こと(7)、「立つ」こと(8、9)や重力が走りと重要な関係を有することに考察が及び、遂に10では、「生活におけるモノ」を、「身体」
を問うためのツールとして転用した。
次第に走るスキルについての学びを私が「生活」と交えながら、拡張する様子が見てとれる。



% \chapter{走りはどう変わったのか?連続写真からの考察}
\label{chapter:hashirinohenka}

前章で物語った実践とそこで得た問題意識群は、私のパフォーマンスに変化を及ぼした。
本章では私の「走り」に着目し、4つの時期の連続写真(\autoref{fig:runtransformation})を用いてそれを考察する。
全て30fpsで撮影された連続写真を、右脚接地瞬間(局面3)で4つの走りを揃えて表示した。
私の身体に書き込んだ線は、「上体の傾き」「右大腿骨」「右下腿」の代表線である。
カメラは固定カメラではなく、カメラの場所や角度は4つで統一していない。
その上でなお読み取り得る変化を、本章では考察する。


\begin{figure}[h]
  \centering
  \includegraphics[width=10cm]{./images/runtransformation.pdf}
  \caption{走りに表れた変化}
  \label{fig:runtransformation}
\end{figure}

\begin{description}
\item[A:2016年7月24日(\ref{sec:led}節に相当)]
\end{description}
「腕で掘る」意識がまだ濃かった時期である。局面3で、真下接地できていない。
\begin{description}
\item[B:2016年9月18日(\ref{sec:jikunoimi}節に相当)]
\end{description}
「軸」の意味を、再解釈した時期である。
Aに比べて、局面1〜2で背筋が伸びているのに加え、局面3で、より接地位置真上に近い位置に腰があり、軸の意識が表れているように見える。
いっぽうで、右接地位置を手前にしようと、右腰がより引けてしまって左側の骨盤が前傾してしまい、左脚の「膝のたたみ」が甘くなり、左脚スイングが右膝関節の屈曲を助長してしまっている(局面4)。
Bの局面4は、接地脚膝関節の大きな屈曲に伴って、踵も浮いており、Aの局面4は踵が付いている。
局面5でAよりも深く右膝が潰れ、局面6でAよりも上に伸び上がってしまっている。

\begin{description}
\item[C:2017年1月17日(\ref{sec:kegagenin}節・\ref{sec:tatuaruku}節に相当)]
\end{description}
立つ・歩くを作り直していた冬季である。全局面を通して、AやBよりも前傾姿勢気味である。局面3〜5で、AとBは右腰が引けてしまっている(真上から見て骨盤が右回転)のに対し、Cは骨盤が進行方向に正対している。
すなわち、接地において、AとBよりも重心が乗り込めている。
局面5では、右膝は屈曲しているが「潰れて」いるのではない。
乗り込めており、意識的に「膝を抜いて」(\ref{sec:tatuaruku}節)いるのだ。
局面6では、上に伸び上がらず、前に進めている。
また、全局面にかけて、AとBに比べて、つま先があがっており。踵から接地に入り、接地位置の上を身体が「転がる」ようにみてとれる。
\begin{description}
\item[D:2017年7月29日(\ref{sec:monowotool}節に相当)]
\end{description}
大怪我から1週間だけの一時復帰時である。
Cと比較して、全局面にかけて上体はより直立に近い。
Cよりも、右膝の屈曲がない、すなわち、局面4・6での右下腿が前傾しすぎていない。
より最後まで踵が浮かずに、地面に力を加えられている可能性があり、これは「足裏が剥がれる」(\ref{sec:tatuaruku}節)感覚と符合する。
また、Cよりも膝をたたんで左脚スイングできている。
局面5〜6でそれが顕著である。
局面4〜5でCよりも腰が高く、局面6で、上半身と右大腿のなす角の大きさ(局面6の図内に線で表した)が、Cより小さい。
すなわち、離地で右股関節が開きすぎておらず、次なる右大腿のスイングが早まる。
石花や洗濯機の実践を経て、「軸」という点で、動きがより研ぎ澄まされたかもしれない。
ただし、Dの地面のみ、「タータン(合成ゴム)」ではなく「硬い人工芝」である。
後者の方が、滑りやすくて私にとっては走りにくい。

以上のように、物語の期間において、私はさまざまに身体での問いを発し問題意識を醸成してきたのにともなうかたちで、
私の走りは変化してきたのである。


% \chapter{物語の考察:学びの野生化}
\label{chapter:yaseika}

私が、生活と競技を切り離さず、生活の中にも様々なヒントを見出しながら、スキルを学ぶ態度は、野生の思考\cite{levi-strauss:1962}と解釈できる現象なのではなかろうか。
レヴィ=ストロースら人類学者は、世界各地の未開文化をフィールドワークによって詳らかにしてきた。
未開民族は、彼らの生きる環境をなす自然種を精緻な関心によって弁別し、それらの関係を隠喩的な思考媒体として活用することで社会文化を作り上げていた。
例えば、南ボルネオのイバン族は、各種の鳥の性質の差異を解釈する。
アジアキヌバネドリ属の一種の警戒の鳴き声は、喉を切られた動物の喘ぎに似ていることから豊猟の前兆であり、同属の別種は、その「笑声」で交易旅行がうまくゆく前兆とされたり、その派手な赤い頸毛が戦勝や遠征に伴う威光の印だという。

このように、身の回りにある事物を組み合わせて、そこに生じる関係から象徴的に秩序や法則をつくり出す態度を、レヴィ=ストロースは「野生の思考」と呼ぶ。
いっぽう、法則や秩序から事物をトップダウンに(一意的に)位置付ける、近代科学的態度を「栽培的思考\cite{levi-strauss:1962}」と呼ぶ。
レヴィ=ストロースは、野生の思考という基礎の上に栽培的思考があるべきだと説く。
また、ありあわせの資材の断片から、なんとかやりくりする器用仕事を「ブリコラージュ」と呼び、現代の我々に表れる野生の思考だとレヴィ=ストロースは言う。

私の一連の学びを俯瞰すると、まさに、レヴィ=ストロースが言うところの「野生」に符合する。
当初の私は、競技場やトレーニングルームだけで学んでいた。
特に陸上競技場は、競技用に洗練されたタータン(合成ゴムの地面)に、均等かつ精密に引かれたライン等によって、秩序づけられた空間である。
その空間に則する形で距離やタイムを計ったり陸上競技用の道具を用いたりしながらスキルを学ぶのは、レヴィ=ストロースのいう「栽培的」な学びである。

私はそこから脱皮し、自らの学びの場を日常生活へと開放したのだと解釈できる。
例えば、立つ・歩くことをスキルとして問うたり(\ref{sec:tatuaruku}節)、競技に一見関係ない石ころや洗濯機やリュックを、私自身の身体を問うツールへと再解釈した(\ref{sec:monowotool}節)のである。
\ref{sec:monowotool}節の各試行は、ブリコラージュだと捉えることができる。

このように、私は自らの「生」において、持ち合わせの日常生活動作や、身を取り巻くモノと、私自身の身体との関係を取り結び、その関係を通して身体を問い、競技スキルを形作らんとしてきた。
これを、スキルの学びに「野生」の思考が浮上してきた形として、本稿では\textbf{野生化}と命名する。
野生化と命名する積極的な意義は、その学び方を、奇抜というより、むしろ、狭い場からより広い場への拡張したのだと理解するためである。
野生/栽培のアナロジーを強調して表現するならば、守られた環境で栄養をもらう(栽培的な学び)だけではなく、主体的に野に身を晒しながら栄養を獲得する(野生的)ようになったということである。

本稿では、あらゆるアスリートが自らの学びを野生化するべきであると主張するつもりはない。
しかし、私の過去数年間においては、走りが苦手であったが故に、そのスキルを根本から見直そうと、体感の微妙な差異を感じ取り、そこから様々な問題意識を立て、日常生活にもヒントを見出そうとしてきた軌跡が、次第に私の学びを「野生化」に導くことになった。
少なくとも私にとっては、「野生化」は、走りのスキルを磨く上で大きな学びをもたらしてきたと断言できる。

学びの野生化という、学びの実践が拡張していったことこそ、認知が構成的な姿そのものと言える。
そして、狭義の一人称研究で構成的手法をとったからこそ、野生化が観測できた(仮説を立てて検証する、という1サイクルを研究とみなすスタイルでは、こうした野生化の現象は観測できなかったであろう)


% \chapter{物語の意義}
\label{chapter:monogatarinoigi}
本稿では、Tがアスリートとして、生活と競技を分けずにスキルを学ぶ様を物語として描いた。
研究論文を物語の形で執筆することは、どういう意義を持つのだろうか?
研究分野に何をもたらすのであろうか?

物語は、一人称視点に立つからだメタ認知の記述を拠り所にして、構成されたものである。
従来の科学観に則るならば、「一人称研究による物語は、普遍的に検証された知見ではないから、知の蓄積としての価値はない」という言説も成り立つかもしれない。
しかしながら、本稿に示した学びの物語は、普遍的に証明された「情報」の提示ではなく、一人のアスリートがスキルを学ぶプロセスで醸成した「意味」の例示である。
一人の人間に生じた事例であり、個人固有性やそのアスリートTが置かれた状況に色濃く依存した物語であるが、Tがどのような意味を醸成しながら「生きた」のか、そのリアリティを描き出したものである。
そこには、アスリートがスキルを学ぶとはどういうものごとなのか、どんな問題意識を育まれ、生活と競技がどのように一体となるのか、という知の姿が描かれている。
それを論文として伝えることは、今後のスキル研究に新たな視座を与えるものである。

科学論文は、普遍性・客観性・論理性という三大原則に則って、普遍的な知見の獲得を目論むものである。
しかし、一方で、それらの原則は「生きる」姿を削ぎ落としてしまうことに、つまり、科学的方法論は人が「生きる」ことから乖離してしまっていることに、哲学者・中村雄二郎\cite{nakamura:1979}は懸念を示している。
人が生きる上で醸成した「意味」を記述した物語には、人が生きる姿が色濃く描かれている。

作家・小川は、臨床心理学者・河合が言うところの「物語(物語るという行為)」に、「事実を否定する絵空事ではなく、『いのち』や『たましい』を手触りあるものとして刻みつける」(\cite{ogawa_el_al:2011}、p.129)意義があると解説する。
すなわち、「科学論文になくて物語にあり」得るものは、手触りである。物語には手触りがあるからこそ、読者に「感触」\cite{uragami:2015,suwa:2016}を届けることができるのではないだろうか?
そして、読者は、その物語を自分ごととして咀嚼し、自分自身の生きる身体で以って「問い」、自らの生に活かすことができるはずだ。

FolkPsychologyという分野を提言したブルーナー\cite{bruner:1990}も、物語は「本当らしさ」や「実生活らしさ」を伝える媒体であると説く。
(物語に内在する)時系列性を以ってプロットを語り、通常から逸脱したもの(「必ずしも普遍的ではない知の姿」のことであると著者は理解している)を読者に理解可能な形で提示するものである。
ブルーナーは物語を、「人のコミュニケーションにおいて、もっとも身近にあり、もっとも力強い談話形式の一つ」(p.108)だと明言している。
つまり、個別具体的で、状況依存的で、必ずしも普遍的ではない(一人の人間に生じた)ものごとであっても、十分他者に伝わる媒体であると著者は解釈している。
研究論文として物語を提示する意義は、そこに描かれた知の姿が内包する「意味」を(本論文の場合は、アスリートが生活の中から、様々な問題意識の変遷を経て、スキルを学ぶプロセスにおける知の立ち上がりの姿)、読者である各々の研究者が「感触」を伴って納得する場を提供し、今後のスキル研究に新たな視座を提示することにあるのではないか。


% エピソード記述では、以下三種類の語りが含まれることを論じる
% \begin{description}
% \item[背景]体験が生じた舞台にまず読み手を招じ入れる
% \item[エピソード]書き手の心揺さぶられた様を語る
% \item[考察(メタ観察)]書き手が心揺さぶられた理由を添える
% \end{description}









% \part{動いている身体の「表情」の感得をうながすアプリのデザイン実践}

\chapter{背景と問題意識}
第二部は、運動学習の身体知創造プロセスを支援するツールをデザインする研究である。
動いている身体が醸し出す「表情」を感得することを運動学習者に促すツールである。
まずは本章にて、私が「表情」に出会ういきさつ、「表情」とはなんなのか、「表情」が身体知創造プロセスとどう関係するものなのかを述べることで、
第二部研究の問題意識を展開する。


\section{問題意識の醸成}
\label{sec:tetsugakujousei}
本節では、
\textcolor{red}{第一部の実践を終えた私が(修士論文としてまとめた私が)、}
第二部の研究に至るようになったのかを示す。

\subsection{身体感覚を触発するおもちゃを試作する}
\label{subsec:toy}
私は博士過程に入ってからも、自分の身体感覚をつつき、ことばを触発するようなメディアの試作を続けていた。
% たとえば、MRIをつかって、生成()Ultra-Regeneration of MRI\cite{horiuchi_suwa:2018}は、MRIで撮影した人体の横断面\footnote{人体を仮にきゅうりに見立てたときの「輪切り」の面である}
% の断面図動画を素材にもちいる映像型のおもちゃである\footnote{
%   制作には、openFrameworksをもちいた。openFrameworksはProcessingに似たプログラミング言語(C++言語のソフトウェアフレームワーク)であり
%   グラフィックス、オーディオ、PC外部の各種センサやアクチュエータなど、様々な技術どうしをつなげたものを制作することができる。
%   Processingよりも高速な処理が可能であり、主にメディアアートの分野でもちいられている。  
% }。
% PC画面内で、MRI動画をふたたび横断面方向に重ねることで「人体を再生」したうえで、画面上

たとえば、骨盤の「傾き」と「回転軸」\footnote{
  「軸」という言葉そのものは、武術やスポーツでしきりに意味深げに語られる概念でもある。
}をリアルタイムに映像可視化するインスタレーション型ツールを、他の陸上競技実践者とともにデザインすることに試みたりもした\cite{horiuchi_suwa:2020b}。
ユーザは加速度センサ搭載のベルト(\autoref{fig:jikubelt})を腰に巻いて運動すると、
(それがPCにUDP通信でリアルタイムに送信され、)
目の前のスクリーンに、骨盤の3Dモデルによって骨盤の傾きと回転軸が可視化される(\autoref{fig:jikuscreen})。
\autoref{fig:jikuplay}はユーザがプレイするようすである。
立った状態で右脚をゆっくりももあげする動作で試したとき、2種類の意識のしかたでは、
回転軸のふるまいがまったく異なることを私たちは発見した(\autoref{fig:jikuplay})。

\begin{figure}[H]
\centering
\begin{minipage}[b]{0.45\linewidth}
\centering
\includegraphics[width=\linewidth]{./images/JIKU/belt.pdf}
\caption{加速度センサ搭載ベルト}
\label{fig:jikubelt}
\end{minipage}
\hspace{0.01\linewidth}%画像間の余白
\begin{minipage}[b]{0.53\linewidth}
\centering
\includegraphics[width=\linewidth]{./images/JIKU/jikuscreen.pdf}
\caption{映像の内容(黄色が回転軸)}
\label{fig:jikuscreen}
\end{minipage}
\end{figure}

\begin{figure}[h]
  \centering
  \includegraphics[width=\textwidth]{./images/JIKU/junkun.pdf}
  \caption{ユーザがプレイするようす(上段と下段は2種類の意識で動きを比較した)}
  \label{fig:jikuplay}
\end{figure}

ほかにも私は、自らの「足」の模型を作り、それとともに暮らしてみることもした。
模型は、足を3Dスキャンしたのち、それを3Dプリンタで出力した硬質樹脂製のもの(\autoref{fig:foot_tate}・\autoref{fig:foot_yoko})
と、樹脂製モデルで型取りしてそこに軟質樹脂を流し込んで固めたもの(\autoref{fig:foot_yawa})を作った。
扁平足である自身の足裏だが、
通常の生活では、自分の足裏が天を向いているのを眺める、ということもあまりない。
足裏が扁平足なりにちゃんと「地形」的になっており、それを握ってみたり、指の腹で実際になぞってみたりして、
特異なしかたで、自らの足裏の感覚をつつくような経験であった。

\begin{figure}[H]

\centering
  \begin{minipage}[b]{0.3\linewidth}
    \centering
    \includegraphics[width=\linewidth]{./images/footmake/foot_tate.pdf}
    \caption{硬質樹脂性の私の足模型を握ってみる}
    \label{fig:foot_tate}
  \end{minipage}
  \hspace{0.01\linewidth}%画像間の余白
  \begin{minipage}[b]{0.3\linewidth}
    \centering
    \includegraphics[width=\linewidth]{./images/footmake/foot_yoko.pdf}
    \caption{硬質樹脂性の私の足模型を上向きに机の上に置く}
    \label{fig:foot_yoko}
  \end{minipage}
  \hspace{0.01\linewidth}%画像間の余白
  \begin{minipage}[b]{0.3\linewidth}
    \centering
    \includegraphics[width=\linewidth]{./images/footmake/foot_yawa.pdf}
    \caption{軟質樹脂性の私の足模型を指でなぞる}
    \label{fig:foot_yawa}
  \end{minipage}
\end{figure}

なお、
% 修士終了後の
\textcolor{red}{2017年シーズン終了後の}
私は、十種競技からは引退し\footnote{
  より精密にいえば、博士課程1年目にふたたび大学陸上部に所属して現役復帰したのだが、1シーズンで競技者としては引退することになった。
}、競技中心の生活は送らなくなり、たっぷりと時間をつかって練習したり、ベスト記録をめざして競技会に出場する機会はほぼなくなったことは述べておく。
だが相変わらず、空いた時間をつかって走ったり身体運動をしてより望ましい動きの試行錯誤は続けていた。
そんななかで私はこうした「おもちゃ」によって、身体運動の「なにか」に迫ろうとしていたのである。

当時の私はこうしたメディアを「トイ(おもちゃ)」と仮称していた\footnote{
  「トイ」はで本研究のキーワードである「身体で問う」の「問い」とダブルミーニングである(つまり、「\ruby{トイ}{問い}」)。
}\cite{horiuchi_suwa:2018b}。
呼び方にはこだわりがあった。
「ツール」と呼ぶとどこか合目的的な響きをもってしまうと、当時の私は感じていた。
あたかも、身体運動のみるべき変数があらかじめ決まっていてその変数の状態を厳密に評価するためのメディアである、というようにである。
私がもくろんでいたのはそういうメディアのありかたではなく、
とりあえずメディアと「戯れてみる」ことによって、打算を超えたしかたで身体を触発して、新しい感覚やことばを生み出そうとするものだった。
それが「トイ」である。

\subsection{哲学的な思惟を抱く:身体知輻輳性の提唱}
\label{subsec:fukusousei}
身体運動をめぐる私の思惟は、以前よりも哲学的(現象学的)な傾向をもつようになっていた。
こうしたは傾向は、
% 修士2年時に
\textcolor{red}{2017年シーズンに}
積極的・自覚的に「アスリートとして生きる」ようになっていったころ(第一部の\ref{sec:tatuaruku}あたり)からであった。
そんな私は、学習者が積極的に「アスリートとして生きる」のが良いのは、身体知に\textbf{輻輳性(confluential structure)}があるからだ、と提唱した\cite{horiuchi_suwa:2019b}。

輻輳性とはなにか、説明しよう。
\textcolor{red}{
そもそも「\ruby{輻湊}{ふくそう}」とはどういう意味のことばか。
輻輳の語源は、車輪において\ruby{輻}{や}(=スポーク)が、中心軸の(\ruby{轂}{こしき})に集中する様子からきており、ものごとが四方からひとところに寄り集まっているようすをいう。
現代でも、交通や通信が一部へ集中し混雑することを輻輳と呼ぶ。
}

「身体運動の名前」と「その身体運動の意味」という対応関係を考えたときに、私たちはつい技名と意味との一対一対応で考えてしまう。
野球のバッティングとサッカーのシュートは「異なる」とみなしがちだろうし、また、AさんのバッティングとBさんのバッティングは運動名が同じだから「同じ」運動だとしがちだろう。
2つのバッティングが「同じ」運動だとしたうえで、両運動のあいだにどういう差異があるのかを見極めようとするのが、
運動学習者であれ、スポーツ科学の研究者であれ、人々の自然的な傾向であろうと私は考える。
輻輳性はその考え方を否定する。

たしかに、「名前」のほうは体系的に整理されていたり、技どうしの区分けもはっきりしている。
しかし「意味」のほうもそうなのか?そんなに意味どうしがきっちりと棲み分けられているものだろうか?
そんなことはないだろう。
それが証拠に、スキルの「転移」という現象がある。
バッティングを例にとれば、バッティングで学んだことがゴルフスイングにも影響する、というのがスキルの転移である。
上記の一対一対応関係が否定されるのである。


各種運動名は分かれていても、そのそれぞれに対応する「意味」はそんなに分かれていないのだとすると、それは、
運動の名前レベルでは運動名(あるいは名指されたひとつ一つの運動体験)に整理整頓され象られているそれらが、
意味レベルではひとつに「輻輳」している、という構図になる。
運動学習者はたったひとつの身体でもって人生を生き、生活も競技もひっくるめたさまざまな身体運動をその身でこなしてきている。
いろいろなシーンで体験してきた多種多様なスキルは、言うなれば「ひとつの身体(という場)」に輻輳して、身体知をなしているのである。
もともとスキルが輻輳してあるのならば、「転移」という表現も実は奇妙である。
転移という表現をもちだすことになるのは、「名前ありき」でとらえる見方をしてしまっているせいである。
「輻輳ありき」ならば、その「輻輳したひとつの塊」に、運動名によって分節化しただけである。
分節化によって際立つものごとはあるにせよ、ひとつの塊をそれぞれ別角度から眺めたようなスキルAとBに影響関係がみてとれるのは当たり前である。
これが輻輳性の考え方である。

輻輳性の考え方に拠れば、以下のことも成り立ちうる。
Aさんにとって、Aさんの「走り」とBさんの「走り」とでは、同じ動作名がついていても、現実的にはまったく異なる意味をもつ。
あるいは、Aさんにとって、Aさんの「歩き」とAさんの「走り」は、動作名は異なっても、ほぼ同じ意味をもったりもする。

私が実践者として日常生活と競技を積極的に融合させていったのは(第一部\ref{sec:tatuaruku}〜\ref{sec:monowotool}節)、
「意味」の輻輳的なありかたを認めたうえで、
運動名(ラベル)にはとらわれず(だが参考にはしながら)に、
身体の声に耳を傾けながらスキルを根本的なとらえなおしをはかったということになる。
私が「リュックを腹負う」(第一部\ref{sec:monowotool}節参照)とわざわざ技として命名していたのも、そういう態度の表れとも解釈できる。
そうやって、生きる全体から「意味」を「図」として浮かび上がらせようとしたのだろうと私は解釈する。

私にとって輻輳性の考え方は、運動名とその意味という対応関係の構図でとらえるときに、その構図の内側から構図そのものを超克しうるような概念装置となった。
見てわかるとおり、輻輳性は現象学の考え方と類似している。
現象学をよく知らなかった当時の私は、
自らの運動学習の実践からなかば自力で現象学の基本的な考え方を醸成したのである。
トイによって身体を触発しようとしていた(\ref{subsec:toy}項)のも、
方向性としては、こういう現象学的なありかたを欲していたのだと、私は確信している。

\subsection{「表情」と出会う}
そうして、哲学書をも読み漁るようになっていた私は、
ある書物との衝撃的な出会いを果たす。
哲学者・廣松渉の著作「表情」\cite{hiromatsu:1989}である。
「表情」は(次節で詳述するが)意味の源たる現象である。
私が実践者かつ研究者としてつかもうとしていた現象の正体こそ「表情」なのではないか?
そう思えてならなかった。
というのも、自分や他者の身体運動を観ながらなにかを学びとろうとするときに、
私はそこ(観測対象の身体運動)に、まずもって「生々しいなにか」を感得しかけるのである。
だが「ことば」の力だけを借りてそれと関わってみても、なかなかしっくりくる共創が巻き起こせていない、という実感があった。
運動学習者は、自他の動いている身体が醸し出す「表情」を豊かに感得できるのが良いのだと、
私は仮説を生成した。


こうして、「表情」という概念を手に入れた私は、
運動学習者が、動いている身体の「表情」を感得することを促すツールを制作することにした。
このように、私が実践者かつ研究者として一人称研究を遂行し、学んできたからこそ、
「表情」に着眼することになったわけである。
本節ではそのいきさつを簡単に示した。

\section{「表情」}
\label{sec:hyojo}
\subsection{表情論導入}
「表情」とはなんなのか。
E.カッシーラー\cite{cassirer:1929}(1874-1945)が記号論の文脈で論じ、
M.メルロ=ポンティ\cite{merleau-ponty:1945}(1908-1961)はそれを身体的な現象学に引き継ぐ形で、
Ausdruck(ドイツ語)・Expression(英語)を論じた。
Expressionは日本語だと「表情/表現」という二重の意味をもつ。
廣松渉(1933-1994)はExpressionに「表情」という日本語訳のほうをあて、それらを引き継ぎながら、独自の表情論を展開した。

廣松は以下のように書いている。
\begin{quotation}
  風景に眼を向けて見よう。われわれの日常如実の体験相においては、いま例えば、
  「いま裏山の松の樹はガッシリとしているが大枝はノタウッテいる。
  崖にかけて淡竹がスクスクと伸びており、刃先はピンと張っている。
  ・・・小川はサラサラと流れ、魚はスイスイと泳いでいる。
  雪がヒラヒラと舞い始め、やがてシズシズと降りしきる。
  松はコンモリと雪帽子を被り、いよいよドッシリと落ち付いて見える。
  一陣の風がサッと捲き起こり、雪がパッと散る。が、松はカタクナに立っている。
  竹はタワワに軋み、雀がピョンピョンと枝渡りすると、ドタドタと雪が零れる。
  夕陽がノンビリと傾き、月影がソッと忍び寄って来る・・・。」

  環界的情景は、表情性に満ち充ちている。\\
  (\cite{hiromatsu:1989}, p.9)
\end{quotation}
表情におよそ相当するのは、カタカナで表記された、オノマトペを主とした述語的部分である。
廣松はこのように、「表情」を言葉の問題にむすびつけながら考究している。
末尾にある「環界的情景」とは、(私たちひとの生きる)環世界\cite{uexküll:1934}のことである。
ユクスキュルの環世界\cite{uexkull:1934}は、それぞれの生物が客観的には同じ環境に存在していても独自な意味世界を生きていることを主張するものである(\ref{subsec:umwelt}節)。
廣松の「表情」は、とくに人間の環世界に着目し、人間の環世界は無味乾燥で殺風景なものではなく、
意味に満ち充ちた生々しいものである、ということを強調するものだと解釈できよう。
生々しさとは、オノマトペに代表されるように「身体感覚的」に把握されるものである。
廣松は続ける。

\begin{quotation}
  直接的な体験意識に即するとき、事物(というものが在って、それ)が表情性を帯びている、 
  という表現方式は実態には合わない。
  右の文章では、松がグネグネしているとか、淡竹がスクスク伸びているとか、
  事物的分節体が表情性を呈するかのような表現方式になっているが、
  原基的にはむしろ、グネグネしているあれ、スクスク伸びているこれの覚知が先であって、
  その覚知与件が松・竹として事物的に認知・命名されるというのが実情であろう。 \\
  (\cite{hiromatsu:1989}, pp.9-10)
\end{quotation}
つまり、私たちに生きられた世界(環世界)は、あらかじめすでに、「表情」に満ちている。
私たちの常識からすると、(1)「竹」という物がまずあって、(2)私がそれを認識する、という二段構えで理解される。
いわゆる物心二元論的な構図である。
二元論的構図で描くかぎり、世界は殺風景になってしまう。
「グネグネ」は、この構図には描けない体験である。
グネグネは、殺風景な世界にはないし、「各自が心に生成した、二次的で私秘的なもの」かといえば、そうではない。
廣松は「表情」という概念をもちこむことで、この構図そのものを覆すのである。
私たちは(1)「グネグネ」といった表情こそをまず先に感得しており、
そのあと事後的に、(2)その表情を「竹」といった事物的なものとして捉えている(捉えなおす)だけだ、
と廣松はいうのである。
表情感得こそが、私たちの「生まの体験」だと廣松は説いている。

\begin{quote}
  「純然たる知覚現相」などというものは如実には存在せず、如実の現相はその都度すでに“情意的な\ruby{契機}{モメント}を孕んで”おり、本源的に表情的である。
  より正確に言えば、如実の\ruby{環境世界}{ウムヴェルト}的現相は本源的に情動的価値性を“懐胎”せる表情性現相である。
  従って、表情性現相は汎通的である。(p.17)
\end{quote}


菅野\cite{sugeno:1998}も、表情を環世界に絡めて説明する。
「記号の精神からの音楽の誕生」と題する論考のなかで菅野は、
「生物は環世界の事物に対して問いを発し、これに事物が応答するという、問いと答えの応酬のさなかから、事物の意味が立ち現れる(\cite{sugeno:1998}, p.134)」
と説明したうえで、環世界の事物がもつ「意味のトーン」が「表情」なのだと論じている。
こうしたことからも著者は、「表情」感得が、意味づくりの源になるのではないか?と考える。
菅野が考察の対象にする「トーン」という概念は、ユクスキュルの環世界論に由来する。
ユクスキュルは、トーンという概念\cite{uexkull:1934}を、主体の気分によって対象のもつ意味(どういう行為でかかわるか)が変わる、というニュアンスでもちいている。
ヤドカリにとってのイソギンチャクの知覚像は、気分に応じて変わる(異なるトーンを帯びる)。
自分の家にイソギンチャクをつけているヤドカリにとっては「保護」のトーン、
家を失ったときには「居住」のトーン、
腹が減ったときには「餌食」のトーンを、イソギンチャクは帯びるのである。 
「表情」は、観察主体の気分に応じて変わりうるものであると考えられる。


「表情」は顔のそれ(表情)を万物のそれへと拡張した概念だ、ととらえてみるとわかりやすい。
私たちは、相手の顔面に表情をみずにはいられない。
「表情抜きの物的顔面の動き」だけをみてとることは逆にむずかしい。
顔面は物的パーツの集合体であるとわかっていてもなお、
その動きに表情をみてしまう。
よもや、相手の目鼻口といったパーツのひとつひとつを別々に認識したのちそれらを「合算」して「ということは、このひとは喜んでいるのだな」と計算しているわけでもあるまい。
私たちはじかに、相手の動く顔の全体としての表情を感得しまえているだろう。
それが、身ぶり手振りを交わす滑らかなコミュニケーションの成立にも寄与しているのだろうし、
その逆も然りであろう。
「表情」はそういう刺激の単純な総和ではなく、
「ゲシュタルト的全一態(\cite{hiromatsu:1989}, p.75)」で感得されるものである。
廣松がExpressionの訳語に「表情」という日本語を採用したのも、「表現」よりも「表情」と呼ぶほうが、上記してきた表情論の問題を顕著にできるからである。

\subsection{表情の中身}
表情の中身(構成)をもう少しくわしくみておこう。
廣松\cite{hiromatsu:1989}は表情感得がゲシュタルト的であることを前提しつつも、だがそのうえであえて下位的に分解的に説明できるともいう。
そのように「表情」を説明した文言も、\cite{hiromatsu:1989}には散見される。
以下に示す。
\begin{quote}
  知覚相・情緒価・即応価の“融合態”(p.78。太字は著者による)
\end{quote}
\begin{quote}
  表情性感得は、知覚的認知と感情的興発と反応的態勢との融合的感受である(p.77)
\end{quote}
\begin{quote}
  表情感得とは、情緒価と即応価とを\ruby{内自化}{イン・ジッヒ}\footnote{}せる知覚的現認(p.79)
\end{quote}

\textbf{知覚相}とは、「事実知覚相」とも表記ゆれしている。事物的・物的なあらわれである。
「相(phase)」とは、物質でいう「固体相・液体相・気体相」のように、表層的にあらわれ出ている状態であると考えられる。

\textbf{情緒価}は、「感情価」「情動価」とも表記ゆれしうるという。末尾の「価(value・potential)」とは、知覚「相」とくらべて、「こもっている」「帯びている」ようなありようを示しているのだと著者は解釈している。
色彩心理学が示すように、赤に「情熱」を・青に「冷静」を見てとってしまうというのが簡単な例になるだろう。
ただ廣松によれば、感情価はこうした単純な「喜怒哀楽」だけに限らない。

\textbf{即応価}は、廣松は「行動価」または「信号価」とも呼んでいる。
認知科学領域でも、即応価に相当するさまざまな概念は述べられている。
まずは、廣松もとりあげているように、認知主体が環境から直接に知覚しうる「行為の可能性」としての\textbf{アフォーダンス(affordance)}\cite{gibson:1979}の考え方は最たるものである。
また、郵便受けが「入れてください」と言っているように知覚されるという、ゲシュタルト心理学でいわれる\textbf{要求特性}\cite{kofka:1935}も、即応価の契機のわかりやすい例であろう。
「運動共感」という考え方\cite{kaneko:2005}も、即応価に相当するだろう。
ミラーニューロン\cite{Rizzolatti:2004}やカノニカルニューロンと呼ばれるニューロン群の存在は、即応価の存在の根拠ともなりうる神経的基盤であろう。

知覚相・情緒価・即応価をさらに、下位的に分析してみるならば、
\begin{quote}
  知覚相・情緒価・即応価という三契機の各々が、質態値・度量値・趨勢値を内自化せしめた相で現前する(p.79)
\end{quote}
のだという。
質態値とは質である。
度量値とは量である。「大小・強弱・濃淡といった度合的(p.76)」であり「スカラー的に配位され」るものだという。
趨勢値とは、質と量がつねに変化のさなかにあるということを示し、「生滅的であれ変様的であれ移動的であれ、ともあれ予期的に覚知されるディスポジショナルな変化様態値(p.77)」であり、
「ヴェクトル的に描出される(p.77)」ものだという。

ここで再び、廣松がとりあげる「表情」の例をみてみよう。
「ゾクッ」という表情である。
「ゾクッ」の体験は、(1)敵の姿の対象認知が起こり、(2)恐怖の感情が興発し、(3)逃走の行動が誘発される
という情報処理的順番では描けない。
「ゾクッ」は(2)だけに相当するものではないのだ。
そうではなく、「ゾクッ」がまず生じて、それがそののち、敵の姿となり、恐怖感情となり、逃走行動になる、という説明のほうが、
ありのままの体験を表せているのではないだろうか。
いわば、「ゾクッ」という表情が、これら知覚・感情(思考)・行動という3つのはたらきに分凝\cite{hiromatsu:1989}するのである
(ゾクッの時点ではまだはっきりとした的の姿をとらえ終えていない。)。
ありのままの体験は、対象の知覚は、感情価と行動価をはらんだ知覚相でもって生じる。

表情は、ただ私たちに\textbf{立ち現れ}てくるのである。
「立ち現れる」という動詞は、現象学者・大森荘蔵\cite{ohmori:1976}(1921-1997)がしきりにもちいる動詞であり、本研究ではれっきとした用語としてもちいている。
大森は「立ち現れ一元論」を唱導し、物vs心(世界vs私)という構図をまるごと脱した一元論的な見方である\footnote{余談だが、\ruby{惣}{すべて}ということばがあることがここで思い出される}。
「立ち現れる」とは、そういう一元論的な構図がよく表れた動詞なのである。
補論(\ref{sec:horon_tathiaraware}節)にて、大森による象徴的なフレーズを掲載する。
「表情」がどういう現象かの理解の助けにもなるはずである。

ここまでの議論を踏まえ、「表情」という概念を図示するならば、\autoref{fig:hyojo}のように描ける。

\begin{figure}[h]
  \centering
  \includegraphics[width=0.5\textwidth]{./images/hyojo.pdf}
  \caption{「表情」の概念の図示}
  \label{fig:hyojo}
\end{figure}

\autoref{fig:hyojo}は、
知覚相・行動価・情緒価の3つを、それぞれ「まがたま形」にすることで、「3つでひとつの○になる」ように描いた\footnote{
  「三つ巴」の図をヒントにしている(が、3者が争っているというよりは、本質的に融合し一体的であることを表現している)。
}。
これは、全体性を表現するのと同時に、「つねに動きつづけてこそ在る」ことを表現している。

このようにして「表情」の構成をみてみると、表情は、表面的には視覚という知覚体験としてありながら、
その内実は、すでにして「行為・行動」の契機、「思考」の契機が溶け込んでいる。
要するに「表情」は身体知の「源泉」である、と考えることができる。
\autoref{fig:hyojoandshintaichinomanabi}に、「身体知の学び」の概念モデル(\autoref{fig:shintaichinomanabi})と「表情」(\autoref{fig:hyojo})
とをあわせた図として、両者の関係性を示す。
「表情」は、身体で問うことにともなって立ち現れる問い(ノエマ)の、
原基的なありようである。
問うというノエシスは、知覚・行為・思考の三位一体の作用である。
問いというノエマは、(原基的には、)知覚相・行動価・情緒価の未分的融合としての「表情」である。
両者は相互限定関係にある。
また、上記の大森の思想も踏まえ、この図は、まだ世界vs私といった構図の生まれる「以前」を表すものでもあるから、
この図には、「身体をもったひと」「対象」「ひとの頭から出る吹き出し」といった要素は登場し得ないのである。
図では、赤は思考、青は知覚、緑は行為、というふうに色付けを対応させている。


\begin{figure}[h]
  \centering
  \includegraphics[width=\textwidth]{./images/hyojoandshintaichinomanabi.pdf}
  \caption{「表情」と「身体で問う」ことの関係性}
  \label{fig:hyojoandshintaichinomanabi}
\end{figure}

\section{動いている身体の「表情」を感得することのむずかしさ}
\label{sec:学びにおける、動く身体を鑑賞するという体験}
序論の\ref{sec:ugoiteirushintainohyojo_intro}節でも頭出ししておいた問題を再度述べよう。
運動学習者はしばしば、自身の運動を撮影した動画や、他者の運動をみて観察して、そこからさまざまに内省する。
このとき、その動いている身体が醸し出す「表情」はつい、認識や内省から取りこぼされてしまう。
「表情」は、観察者の身構え(要するにノエシス)に応じて立ち現れてくる。
「学び取ってやろう」と身構えるほど、その志向は、物的な身体運動の一部の特徴に絞られてしまいがちである。
そこには豊かな「表情」があるのに、観察者が取りこぼしているとするならば、それは意味づくりの可能性を狭めていることにもなってしまう。
そういう意味では、実は、眼の前の全身運動を「そのままの物的事実としてフラットにみる」、ということも、とてもむずかしいことである。
観察における私たちの志向は、物的なありかたになりがちなのにくわえ、そのなかでも「ごく一部」にむきがちなのである。

このことは、中村\cite{nakamura:1979}の指摘を参照すれば、
視覚という知覚の「性格」のせいであると考えられる。
視覚は「明晰的な意識」と結びついてはたらきやすい。
自己(みる主体)と世界とを引き離し、そうすることで世界を「みられる対象」として定位する、という形ではたらくのである。
それはさらに、視覚がほかの諸知覚から「独走」してしまうことでもある、中村は指摘する。

\textcolor{red}{
からだメタ認知\cite{suwa:2016}によって「ことば」をうまくつかうことは、「表情」を感得するためのひとつのアプローチであると本研究は考える。
}
だが、それだけでは必ずしも十分ではない、という問題意識も、私は抱いている。
じっさい、廣松は「表情」をことばの問題とむすびつけながら論じているが、
「表情」が生々しく豊かに立ち現れてくるのにくらべ、私たちはそれらを表現する確かな語彙や分類体系を持ちあわせていないことを指摘している。
少なくとも日本語においては、表情の種別を各一語であらわすような語彙は十分ではない。
知覚相に相当する部分で言えば、色や音に限定されれば、語彙はよくよく配備されているが、香や形ともなると、とたんに語彙は貧困になる。
感情価や即応価についても、語彙は貧困である
\footnote{感情については、漢語では語彙は豊かだと廣松は指摘する。}。
だから廣松はそこで「オノマトペ」に着目したのであろう。

ことばも合わせ技でつかいながら、表情感得を促す工夫がしたい。
その工夫によって、からだメタ認知による内省を促したい。
豊かな「表情」感得をうながし、より良い意味づくりを支援するしたい。
そういう問題意識のもと、私は、動いている身体の「表情」の感得をうながすツールを制作した。
% 
\chapter{ツールの制作にむけて}
\label{chapter:fortoolmaking}
では、どのようなツールをデザインするのがよいのか?
本章で議論する。

\section{身体知の学びとしての運動学習を支援する研究}
身体知の学びとしての運動学習の支援研究には、どのようなものがあるかみておく。

スポーツ科学的視座からみれば、学習者に「良い運動が有する客観的・量的な特徴を知らせる」といったかたちの支援がありうる。
DSAの視座からみれば、この潮流では近年、(コーチが)学習者の運動に相転移を促す「環境」をデザインする、という支援のありかた\cite{ueda:2023}も起こっている。
本研究はこれらの支援を否定するものでは決してないが、「表情」の問題へアプローチするには、少なくとも運動の「質的な」側面に直接アプローチすることが必要であろうと考えられる。

運動の質的側面を重視するアプローチをみてみよう。
モダン・ダンスのパイオニアであるR.Laban\cite{laban:1971}は、運動をうみだす内的なはたらき(Labanは"Effort"と呼んでいる)
\footnote{Labanは、運動の外的な形や様式を重視するクラシックバレエの考え方に反発した。}
に着目し、あらゆる運動のEffortを、Space(直線的/曲線的)・Time(速い/遅い)・Weight(強い/弱い)・Flow(コントロールされた/自由な)の4側面の組み合わせによって描画する記法"Effort Graph"を編み出した。
Effort Graphは、自らのパフォーマンスを分節したり、新しい動き(振り付け)を創造することにおいて有用であろう。

からだメタ認知を促すことは身体知の学びの支援となる。
前述したように、からだメタ認知のひとつの肝は、体性感覚的で曖昧模糊としたものごとについても積極的にことばにすることであった。
「ことばに正解はない」というのは大前提だとしても、そうだとしても、感覚的なものごとをメタ認知することはそう簡単なことではない(「表情」もこの類の体験である)。
\textbf{創作オノマトペ}\cite{otsuka:2015}は、そうした曖昧模糊とした感性的体験に対して、独自のオノマトペ(例:ぬぴゃふー)で表現してみるからだメタ認知促進メソッドである。
オノマトペにすると、音節という分節構造が自然と生まれるが(例えば、ぬ/ぴゃ/ふー)、
各音節ごとに、どういうニュアンスがこもった/をこめたのか、をからだメタ認知によってことばにするのである。
運動学習ではないが、日本酒の味わいという感性的体験に対して、創作オノマトペによってからだメタ認知感性開拓を促した事例がある\cite{otsuka:2015}。
創作オノマトペは「原初音韻論遊び\cite{noguchi:2003}」という、色々な言葉をとりあげてそれを構成するひとつひとつの音韻に対し音韻そのものがもつ感触やニュアンスを言葉で表現してみる営みをヒントにした手法である。
廣松が「表情」をオノマトペとむすびつけた\cite{hiromatsu:1989}こともふまえつつ、
本研究でもオノマトペは「表情」から問いを立てるための一媒体であると考える。

運動の量的特徴をユーザに提示してユーザにからだメタ認知を促すことを目論んだツールもある。
いわば量と質の両側面から攻めるアプローチである。
Nishiyamaら\cite{nishiyama:2010}が開発した『Motion Prism』は、
時系列データである身体運動の各フレームを「姿勢情報」によりクラスタリングし、各クラスタに色を割り当てる。
各フレームに対して「セル」をつくり、各クラスタ傘下のセルにクラスタの色を塗る。
ふたたび元の時系列でセルを積み上げることで「カラーバー」として姿勢変化を可視化し、ユーザにフィードバックする。
ユーザには能動的に時間的分節を見出したりその意味を自分なりに解釈することを促す。
「身体の物的事実をもとにした意味ありげな運動の分節」をフィードバックすることで、ユーザに「能動的に問うことを促す」というやりかたは、本研究のヒントになる。
\section{動いている身体の「表情」に近しいものを扱うプロジェクト}
\label{sec:design}
\todo{ここは修正なの赤くする}
動いている身体の「表情」の感得を促そうとするときには、
また、スポーツ科学的なアプローチでよく示される「グラフ」や、Labanの「記譜」、あるいは、西山の
といった、表象性質、情報の表象、指標性だけでは、どこかが足りない。
これは、記号論でいうところの「指標性(index)」なものでは、記号論・記号学的議論には深くは立ち入らないことにする。
どこかかゆいところに手が届かない感覚が、私にはある。
それは、廣松がことばのなかでも「オノマトペ」に着目したということにも関連するだろう(もっとも、オノマトペをもってしても、表情にたいしては無力)。
そこで本研究は、オノマトペは重要なものとして残しながら、
もっと豊かにしたい。

動いている身体の「表情」がどういうものなのか、関連するであろう研究や作品群を眺めながら考え、
その感得を促すための方法を議論する。

ヒューマンコンピュータインタラクション(HCI)領域では、質的な側面からダンス学習を支援する研究もある。
Labanのアプローチを足場にしつつもより実践的に磨いたダンス理論参照した、
Fdili Alaouiらによる『THE DOUBLE SKIN / DOUBLE MIND INTERACTIVE INSTALLATION』\cite{fdilialaoui_et_al:2015}は、
巨大スクリーンに映し出された抽象的なバネモデルの映像の前で動くことを体験者に促すインスタレーション作品である。
体験者の動きの速度・加速度・躍度がリアルタイムに検知・計算され、それをバネモデルに対応づけている。
ダンサーを対象にしたユーザ実験では、動きの探索を促し、内省することを助けたという。

Konnoらによる『RAM Dance Toolkit』\cite{konno_et_al:2016}では、
ダンスのためのインスタレーション環境(ダンサーの動きに応じてインタラクティブに変化するさまざまな映像)を、クリエイティブコーディングによってシンプルな手順で作り出すことができるツールキットである。
ユーザ(ダンサー)は、各部に慣性センサが搭載された専用スーツを着用して、スクリーンの前で踊る。
ユーザは踊りながら、「少し先の未来を予測したや関節角度や回転を表示」したり「身体外部に描画する立方体を表示」したりなどの動きに応じるインスタレーション環境(映像効果)を、GUIをつうじて設定・選択したりすることができる。
動きのデータ自体を解析・利用するための機能も備えている。
\footnote{
  世界的なコンテンポラリダンサー安藤洋子氏と、プログラマーやメディアアーティストらが協同して制作したプロジェクトである。
}。
ただし、『THE DOUBLE SKIN / DOUBLE MIND INTERACTIVE INSTALLATION』\cite{fdilialaoui_et_al:2015}も『RAM Dance Toolkit』\cite{konno_et_al:2016}も、
リアルタイムなインタラクションに重きが置かれ、それと関連してか、長期的な運動学習の支援に対して必ずしも直接的な工夫や検討がされているわけではない。
身体知の学びの支援としては、それも重要な点である\footnote{
私は、2017年1月にX氏(第一部\ref{chapter:monogatari}章の物語でも何度か登場。2017年1月は物語の\ref{sec:tatuaruku}節の時期にあたる)から紹介されたイベント:
「Perception Engineeringキックオフ―つなげる身体」@山口情報芸術センター、2017年1月21日)に参加したのだが、
上記『RAM Dance Toolkit』\cite{konno_et_al:2016}は、そのイベントにて紹介されていたプロジェクトである。
思えばそれに触れた当初から、研究者・実践者として生きている私には、動いている身体の「表情」の感得を促す工夫についての「伏線」が貼られていたように思う。
}。

% バイオロジカルモーション
ひとは、点群が動く単純な映像をみるだけで、それが「人体」であり、なにをしているところか、さらに人物の性別・感情・意図さえも感じとることができる\cite{johansson:1973}。
バイオロジカルモーション(以下、BM)として知られるこの認知現象は、動いている身体の表情の一種だと言えそうではある。
しかしオノマトペと結びつけてとらえる「表情」、それも運動学習者が動いている身体に感じとるべき「表情」は、必ずしもBMに限定されるものでもないと著者は考える。
というのも、動いている身体が醸し出す「表情」は、必ずしも「人体の形」にみているとは限らないのではないか、と著者は感あげる。

% 『ISSEY MIYAKE A-POC INSIDE』
そこで短編映像『ISSEY MIYAKE A-POC INSIDE』(2006)\cite{euphrates:online}にも着目しよう。
この作品では、モデルウォークする人物(や駆ける動物の姿)から作られた動く白点群が黒背景に描かれ、
これら点群を頂点とするシンプルな図形、次々と表示される
\footnote{
本作品は、New York ADC Gold Prize in 2007、第11回文化庁メディア芸術祭における優秀賞受賞作品である。
}。
棒人間とは異なる形だが、「ファッションモデルたちの、生き生きとした動きなどを如実に表現する\cite{euphrates:online}」と述べられている。

% 階段を降りる裸体
M.デュシャン(1887-1968)による絵画『階段を降りる裸体 No.2』(1912)は、表題のモチーフで純粋な動きを抽象画として描いたものである\footnote{
  E.マイブリッジ(1830-1904)による連続写真『Woman Walking Downstairs』(1887)に影響を受けた作品である。}。
それにインスパイアされたQuayolaとM.Aktenによる映像作品『Forms』(2011)\cite{akten:online}は、トップアスリートのパフォーマンス映像をもとに計算・生成した3Dの抽象的なCG映像である
\footnote{
本作品は、2013年度Ars Electronicaにおいて、Golden Nica賞(アニメーション部門最高賞)を受賞作品である。
}。
美的な観点から、身体と環境との見えない関係性(力やバランスや優雅さや葛藤)を「抽象的形態として彫刻するかような外挿的な可視化」技法を探究したものだと作者Aktenは説明する。

% 『Strandbeest』
『Strandbeest』(1990〜)\cite{jansen:online}は、複数の木製のリンク機構が軸方向に並列的に連なったキネティックアートであり、風を動力にして作動し砂浜を移動する。
むきだしの木組みが蠢くさまを観て著者は、(あえてことばで描写するならば)負傷した多脚生物が何者かから逃げているかのような表情を感得せずにはいられない。
上記Aktenの作品を踏まえて言えば、リンク機構のふるまいそのものが、逃げ歩く表情をありありと浮かび上がらせる「外挿」なのかもしれない。
キネティックアートは「動きそのもの\cite{miyoshi:2022}」がアートの主題になっているからか、
概して、作品は動きを生む内部構造や部材が隠されたりせず、むしろ「むきだし」になっている。
作品の動きと形とが必然的なむすびつきをもっており、形の動きと動きの形がひとつになっている。
動いている身体の「表情」とは、そういうむきだしな姿において\ruby{顕}{あらわ}になりやすいのではないかと著者は考えた。

\section{どういうツールを作るのがよいか}
本章の議論や「表情」の考え方(\ref{sec:hyojo}項)や「身体知の学びの性質」(\ref{subsec:shintaichinomanabi}項)を総合して、動いている身体の「表情」の感得をうながす身体知学習支援ツールのヒントになる項目は以下である。
\begin{itemize}
  \item 実際の動きをもとにした「かたち」を描くこと
  \item かたちは、抽象的で、素朴で、動きがむきだしになったような見た目であること。
  \item 「人に似て非なる形」や「人ならざる形」のような、人体形をあえて保留した形であること
  \item からだメタ認知(ことば)を促し、かつ、長期的な学習を支援できるようになっていること
  \item 身体知の学びの主体性をうながすこと
\end{itemize}
私はこれらのヒントをふまえてツールを制作した。
次章で説明する。












% \chapter{HJ-Playground}
\label{chapter:hj-playground}

\section{アプリ概要}
本研究では、動いている身体の表情の感得を促すwebアプリ「HJ-Playground」を制作した。
webブラウザでプレイするアプリであり、主たるユーザは身体運動学習者である。
本アプリは、あらかじめ計測したユーザ自身/他者の運動データ(各部位の三次元時系列位置情報)を、画面内の三次元空間に動く点群として描き、
ユーザに、それら点どうしのあいだに線分や円などの「補助線を描きくわえて図形を作図」することを促す(\autoref{fig:playfull_menu})。
これを\textbf{表情図形}と呼ぶ。
ユーザには、作図した表情図形を鑑賞しながら、感得している表情をオノマトペで命名し、そのさなかで生まれる問いをからだメタ認知で内省記述することを促す。
本章各節で、各種仕様とその意義を説明する。

\begin{figure}[htbp]
  \centering
  \includegraphics[width=\textwidth]{./images/hjplayground/playview.pdf}
  \caption{メイン画面(左右ドロワメニュー展開時)}          %和文 cap tion  
  \label{fig:playfull_menu}
\end{figure}

\section{アプリのメイン画面の構成}
\label{sec:mainpage}
画面全体に拡がるプレイスペースで表情図形を作図・鑑賞するのを基本として、
この画面の手前側に覆い被さるかたちで、出し入れ可能な3つのメニューがある(\autoref{fig:screenconfiguration})。
% 画面構成(スクショ)
\begin{figure}[htbp]
  \centering
  \includegraphics[width=\textwidth]{./images/screenconfiguration.pdf}
  \caption{メイン画面の画面構成}          %和文 cap tion  
  \label{fig:screenconfiguration}
\end{figure}

% 画面構成の説明
\begin{description}
  \item[a.プレイスペース]\mbox{}\\
  アプリ画面全体に拡がる抽象的な三次元空間である。
  この空間内にて表情図形を鑑賞したり作図したりする。
  プレイスペースには、デフォルトモードと\textbf{編集モード}とがある。
  デフォルトモードでは背景が薄いグレーとなり、表情図形の鑑賞や、下記のb〜cの設定をしたりする。
  編集モードでは背景が濃いグレーとなり、点群のあいだに補助線を引いたり、再生制御やカメラ操作したりする。
  cmdキーを押しっぱなしにしているあいだのみ、編集モードになる(\autoref{fig:toggleedittingmode})。
  

  \item[b.上バーメニュー](画面上下スクロールで出し入れ)\mbox{}\\
  データ選択画面に遷移したり、言語設定(英語/日本語)・サウンド設定などをおこなう。
  なお、プレイ中はあまり頻繁にはいじらないだろう。    
  
  \item[c.左ドロワメニュー]\mbox{(\textbf{tab}キーでトグル形式出し入れ。\autoref{fig:drawermenu}参照。)}\\  
  各種セッティングをおこなう。
  全4段からなるアコーディオン形式で開閉可能なUIになっている。
  1段目には「備え付け補助線パタン(後述)」の選択肢チップ群が並び、
  2段目には「補助線タイプ(後述)」の選択肢チップ群が並び、
  3段目には表情図形のパラメータ「軌跡」を調整するスライダが並ぶ。
  これらについては後述する。
  4段目にはその他の表示設定用のスイッチが5つ並ぶ。
  スイッチの内容は以下である。()内部はデフォルト値を示す。
  \begin{itemize}
    \item 点の大きさに遠近感をつけるか?(ON)
    \item XYZ座標軸を表示するか?(OFF)
    \item 世界球を表示するか?(ON)
    \item マーカーのラベルをみせるか?(OFF)
    \item 地面のプレートをみせるか?(OFF)
  \end{itemize}
  なお\autoref{fig:screenconfiguration}において4段目は閉じている。

  \item[d.右ドロワメニュー]\mbox{(\textbf{\texttt{]}}キーで出し入れ。\autoref{fig:drawermenu}参照。)}\\
  「表情エントリ」(後述)を、編集したり、データベースへ保存したり、データベースからロードしたりする。
  上段の「マイ表情コレクション」は、現在プレイ中の身体運動データに対してユーザ自身が作成・登録した表情エントリ群の一覧である。
  それらを選択して呼び出せる。
  中段の「みんなの表情ギャラリー」は、現在プレイ中の身体運動データに対して他ユーザが作成した表情エントリ群の一覧である。
  中段以降には、ユーザが現在作成・編集中の表情エントリの詳細編集エリアである。
  その表情エントリの、「表情オノマトペ」と「内省記述」(後述)と、その表情エントリの開始/終了フレームを設定するスライダと、その表情エントリの保存ボタン・コピーボタン・削除ボタンが並ぶ。

\end{description}

% ドロワメニュー展開
\begin{figure}[H]
  \centering
  \includegraphics[width=0.9\textwidth]{./images/drawermenu.pdf}
  \caption{左右ドロワメニュー展開のようす}          %和文 cap tion  
  \label{fig:drawermenu}
\end{figure}

% 編集モード
\begin{figure}[H]
  \centering
  \includegraphics[width=0.9\textwidth]{./images/edittingmode_1.pdf}
  \caption{編集モード}          %和文 cap tion  
  \label{fig:toggleedittingmode}
\end{figure}


\section{プレイ方法}
\subsection{プレイ対象身体運動データを選択する}
ユーザはホーム画面(\autoref{fig:home})またはマイページ画面(\autoref{fig:mypage})から、プレイする身体運動データを選択する。
ホーム画面は、アプリのデータベースに登録されたすべての身体運動データが選択肢として表示され、
マイページ画面は、自分が過去にプレイした身体運動データのみが選択肢に表示されている\footnote{
なお、\autoref{fig:home}と\autoref{fig:mypage}内のグレーの短冊形の部分は、実際のデータ名をモザイクで隠している。  
}。
どちらの画面でも、身体運動データをクリックして選択すると、\ref{sec:mainpage}節で述べたメイン画面に遷移する。
次項以降の説明はすべてメイン画面においての説明である。
\begin{figure}[H]
  \centering
  \includegraphics[width=\textwidth]{./images/home.pdf}
  \caption{ホーム画面(プレイする身体運動データを選択する)}          %和文 cap tion  
  \label{fig:home}
\end{figure}

\begin{figure}[H]
  \centering
  \includegraphics[width=\textwidth]{./images/mypage.pdf}
  \caption{マイページ画面(プレイする身体運動データを選択する)}          %和文 ion  
  \label{fig:mypage}
\end{figure}
\subsection{再生制御・カメラ制御(表情図形作図)}
本アプリでは、一般的な動画再生ソフトと同様のキー割り当てで直感的な再生制御を促す(\autoref{table:play})。
% 再生制御
\begin{table}[htbp]
  \begin{center}
  \caption{再生制御方法}                %和文 caption  
  \label{table:play}
  \begin{tabular}[hbt]{c c c}
  \hline
  \bf パラメータ & \bf 操作方法 \\
  \hline
  再生/一時停止 & スペースキー\\
  コマ送り & 右矢印キーまたはプレイヤースライダ \\
  コマ戻し & 左矢印キーまたはプレイヤースライダ \\  
  \hline
  \end{tabular}
  \end{center}
\end{table}

プレイスペースでは、3D空間内で身体運動を取り囲むようなカメラから眺めることができる。
カメラの位置は「極座標形式の世界座標系」で定義しており、
ユーザは編集モードでカメラを「半径可変の球面上」を移動させるように制御できる(\autoref{table:camera}・図\ref{fig:camera})。
編集モード時、カメラの注視点が赤いバツ印で表示される(\autoref{fig:toggleedittingmode}の2枚目や\autoref{fig:editting}の2・3)。
カメラの注視点は常に$(x,y,z) = (0,100cm,0)$である(y座標が運動データの高さ方向になるように撮影することを想定している)。
カメラの上方向は常にカメラ位置の経線北極方向である。

\begin{table}[htbp]
  \centering
  \begin{minipage}[t]{0.45\textwidth}
    \raggedright
    \vspace{0pt}
    \caption{カメラ制御方法}
    \label{table:camera}
    \begin{tabular}{c p{0.55\textwidth}} % 折り返し対応
      \hline
      \bf パラメータ & \bf 操作方法 \\
      \hline
      動径$r$ & cmd + 上下スワイプまたはピンチインアウト \\
      緯度$\phi$ & cmd + 上下ドラッグ \\
      経度$\theta$ & cmd + 左右ドラッグ \\
      \hline
    \end{tabular}
  \end{minipage}
  \hfill
  \begin{minipage}[t]{0.52\textwidth}
    \centering
    \vspace{0pt}
    \includegraphics[width=\linewidth]{./images/camControl_page1.pdf}
    \captionof{figure}{カメラ制御方法の図示} % ←ここがポイント
    \label{fig:camera}
  \end{minipage}
\end{table}

\subsection{点の表示/非表示を切り替える(表情図形作図)}
身体運動データを選択してメイン画面に遷移してきたとき、
はじめの状態では、身体運動データの点だけが表示されている(つまり、撮影時の関節点が空間上に表示されている)。
\ref{subsec:sakuzu}項で説明するように点のあいだに補助線を引くことが本アプリの主要な特徴であるが、
ユーザは、これら各点の表示/非表示を切り替えることができる。
編集モード時に以下の操作をすることで、各点の表示/非表示を切り替えることができる。
\begin{itemize}
  \item \textbf{cmd + d}: マウスオンしている点の表示/非表示を切り替える(トグル形式)
  \item \textbf{cmd + v}: 複数点の表示を一括で切り替える。全点表示→全点非表示→孤立点のみ非表示→全点表示・・・と3状態でスイッチする。孤立点とは、ほかのどの点とのあいだにも補助線が引かれていない点である。
\end{itemize}

複数点の表示の一括切り替えのようすを\autoref{fig:cmdv}に示した。

\begin{figure}[H]
  \centering
  \includegraphics[width=\textwidth]{./images/cmd+v.pdf}
  \caption{複数点の表示の一括切り替え(cmd+v)}
  \label{fig:cmdv}
\end{figure}


\subsection{手作業で補助線を引く(表情図形作図)}
\label{subsec:sakuzu}
動く点群の布置に補助線を引き、図形を作図することを促す(\autoref{fig:editting})。
補助線は、「手作業で引く」のを基本とし、「備え付け補助線パタン(次項)」をもちいて引くこともユーザに促す。

\textbf{手作業で補助線を引く}やりかたについてまず述べる。
手作業で補助線を引くには、cmdキーを押しっぱなしにしたままおこなう(編集モード)。
\autoref{fig:editting}は、映像内人物がこちら側を向いて自身の右側に重心を傾けているシーンである。
局面1ですでに「頭頂-胸骨下端」「胸骨下端-右手首」に補助線が引かれている状態である。
ここで、cmdキーを押しながら局面2で右手首をクリックし局面3で左膝外側(大腿骨外顆)をクリックすることで、局面4で新たに「右手首-左膝外側」間に補助線を引いている。

ユーザ自ら補助線を引くという仕様にしたのは、ユーザに主体的に問うことを促すためである。
「補助線」とは幾何学でもそういう概念である。
問うために自ら引くのであり、
それまで潜在していた関係性を図形として顕在化させながら、問いを深めたり前進させる意義がある。
ユーザは「人型」にとらわれた図形にする必要はない。

cmd+Eキーのキーボードショートカットにより、マウスオンしている点からその他のすべての表示中の点と補助線を引くことができる。

% 補助線を引く様子
\begin{figure}[htbp]
  \begin{continuousphoto}

  \begin{center}
    \begin{overpic}[width=0.2\columnwidth]{images/hjplayground/edit-1.pdf}
      \put(2,90){ 1}
    \end{overpic}
    \hspace{0.05em}
    \begin{overpic}[width=0.2\columnwidth]{images/hjplayground/edit-2.pdf}
      \put(2,90){ 2}
    \end{overpic}
    \hspace{0.05em}
    \begin{overpic}[width=0.2\columnwidth]{images/hjplayground/edit-3.pdf}
      \put(2,90){ 3}
    \end{overpic}
    \hspace{0.05em}
    \begin{overpic}[width=0.2\columnwidth]{images/hjplayground/edit-4.pdf}
      \put(2,90){ 4}
    \end{overpic}
  \end{center}

  \end{continuousphoto}

  \caption{補助線を引く様子}            
  \label{fig:editting}
\end{figure}


補助線パタンは、ユーザみずから手作業で引いたものであれ、備え付けパタン(次項)を適用したものであれ、それが再生中に動的に変わるといった仕様(速度といった運動学的情報からリアルタイム計算して、それに応じて補助線パタンが変わるといった仕様)
は本アプリでは組み込んでいない。

\subsection{備え付け補助線パタンを適用して補助線を引く(表情図形作図)}
\textbf{備え付け補助線パタン}は左ドロワメニューに設えてある。
チップボタンが並んでおり(\autoref{fig:playfull_menu})、
それぞれのチップボタンを押すと、パタンにしたがって点どうしのあいだに補助線が引かれる(\autoref{fig:playfull_menu}左側1段目)。
「備え付け補助線パタン」の一覧を、\autoref{table:presetpatterns}に示す。
それぞれのパタンの実例を\autoref{fig:presetpatterns1}・\autoref{fig:presetpatterns2}に示す。


% 備え付け補助線パタン一覧実例(1)
\begin{figure}[htbp]
  \centering
  \includegraphics[width=\textwidth]{./images/presetpatterns1.pdf}
  \caption{備え付け補助線パタン一覧(その1)}          %和文 cap tion  
  \label{fig:presetpatterns1}
\end{figure}

% 備え付け補助線パタン一覧(2)
\begin{figure}[htbp]
  \centering
  \includegraphics[width=\textwidth]{./images/presetpatterns2.pdf}
  \caption{備え付け補助線パタン一覧(その2)}
  \label{fig:presetpatterns2}
\end{figure}

% 備え付け補助線パタン一覧表
\begin{table}[htbp]
  \caption{備え付け補助線パタンの一覧}
  \label{table:presetpatterns}
  \begin{center}
  \begin{tabularx}{\textwidth}{>{\bfseries}lX}
    \hline
    パタン名 & 説明 \\
    \hline
    三角形で埋める & 点群のあいだが三角形で埋め尽くされるようにする。三角形群は、なるべく多くが鋭角三角形になるように、かつ、線分どうしが交わらないようにする。ドロネー三角形のアルゴリズムと同様である。 \\
    \hline
    全体の輪郭 & 表示中の全点の重心からある点からスタートして結果的に全点をむすんでできる閉領域(図形)生成する。 \\
    \hline
    輪ゴムかける & 表示中の全点を内部に含む最小の凸多角形(つまり凸包)を生成する \\
    \hline
    うずまき & 表示中の全点の重心から最も遠い点から、渦をまくように点同士を一筆書きしてゆく。 \\
    \hline
    きざみのり & 表示中の各点において自身から最も近い点とペアをなすよう結ぶ。ペア形成の順序は、すべての2点の組み合わせのうちもっとも距離の近い2点同士から結んでゆき、必ずどの点も1ペアのみ形成するようにする。\\
    \hline
    横雨 & 表示中の各点において自身ともっとも画面上での縦の位置が近い点同士とペアをなすよう結ぶ。ペア形成の順序は、すべての2点の組み合わせのうちもっとも縦位置の近い2点同士から結んでゆき、必ずどの点も1ペアのみ形成するようにする。\\
    \hline
    雨 & 表示中の各点において自身ともっとも画面上での横の位置が近い点同士とペアをなすよう結ぶ。ペア形成の順序は、すべての2点の組み合わせのうちもっとも横位置の近い2点同士から結んでゆき、必ずどの点も1ペアのみ形成するようにする。 \\
    \hline
    ランダム一筆がき(近) & 表示中全点からランダムに一点選んで開始点とし、そこからもっとも近い点と結び、次にその点からもっとも近い点をむすび、というふうに一筆書きしてゆく。 \\
    \hline
    ランダム一筆がき(遠) & 表示中全点からランダムに一点選んで開始点とし、そこからもっとも遠い点と結び、次にその点からもっとも遠い点をむすび、というふうに一筆書きしてゆく。\\
    \hline
    川 & 表示中全点からランダムで一点を選び、そこから本流から支流へと分岐してゆく川のように、すべての孤立点がなくなるまで結ぶ。 \\
    \hline
    左から右に一筆がき & 表示中全点からもっとも左にある点から、画面上で横位置がもっとも近い点を順に一筆書きでむすんでゆく。 \\
    \hline
    上から下に一筆がき & もっとも上にある点から、画面上で縦位置がもっとも近い点を順に一筆書きでむすんでゆく。 \\
    \hline
  \end{tabularx}
  \end{center}
\end{table}

\autoref{table:presetpatterns}の補助線を引くアルゴリズムの引数になっているのは、基本的には、
それらのチップボタンをクリックしたときの再生フレームにおける、表示中各点の画面上での二次元位置である。
すなわち、あるチップボタンを押したとき、そのときの、カメラの位置(どこから運動データを眺めるか)と再生フレームによって、実際にどう連結されるかは変わる。
くわえて、\autoref{table:presetpatterns}の説明にある「表示中の全点」という部分も重要である。
ボタンクリック時に非表示になっている点については、連結はされないし、連結パタンの計算アルゴリズムからは除外されるようにしてある。

\autoref{fig:presetpatterns1}・\autoref{fig:presetpatterns2}は、
(図内最上部に施した説明のように)ダンスの運動データを例に、備え付け補助線パタン各種を適用するとどうなるかを示している。
なおこの運動データは続く\ref{chapter:jissenmethod}・\ref{chapter:jissenresult}章で説明する対象者Aのoldmanという動作である。
いずれも、一コマ目(1列目のコマ)で備え付け補助線パタンを適用しており、
それが再生するとどのように補助線によってできた図形がふるまうのか(変形するのか)を示している。
比較のため、どのパタンについてもカメラ位置とコマの位置などは条件はそろえてある。
補助線パタンがちがえば再生中に動きのなかでみえてくるもの(つまり「表情」)が異なりそうだ、ということが実感できるであろう。

備え付け補助線パタンは、ただそのまま適用してその引かれた補助線パタンをそのまま受け入れる必要はない。
むしろ奨励するのは、備え付け補助線パタンの適用したをヒントにしながら、
そこからさらにユーザが手作業で作図して補助線パタンを変更したり、
新しい作図の手がかりを得ることである。

\subsection{補助線スタイルをえらぶ(表情図形作図)}
\label{sec:hojosentype}
\textbf{補助線スタイル}を多種類用意した。
左ドロワメニューに並ぶ各チップ(\autoref{fig:playfull_menu}左側2段目)をクリックすると、補助線のスタイルが変わる。
デフォルトのスタイルは2点間の「線分」である。
補助線スタイルの一覧を\autoref{table:hojosen_type}に示す。
B.ムナーリのデザイン教本『空想旅行』\cite{munari:1992}では、あるランダムな点群の布置に、線や円など素朴な描画要素を描きくわえて「ふたりひとくみ」や「音符」や「摩天楼」や「発芽」など、実に様々に点群の関係づけや見立てができることを示している。
\ref{sec:design}で述べた先行研究\cite{konno_et_al:2016}でも様々なインスタレーション環境の条件(映像効果の種類)から選ぶことを促している。
これらに関連して、補助線のスタイルによってみえる「表情」は変わりうると考えた。
\begin{table}[htbp]
  \begin{center}
  \caption{補助線スタイル一覧}                % 和文 caption  
  \label{table:hojosen_type}
  \begin{tabular}[hbt]{c l}
  \hline
  \bf スタイル名 & \bf 説明 \\
  \hline
  デフォルト & 各線分IJ \\
  線延長 & 各線分IJを両端に1倍ずつ延長 \\
  線延長10倍 & 各線分IJを両端に5倍ずつ延長\\
  外接円 & 各△IJKの外接円 \\
  内接円 & 各△IJKの内接円 \\
  外/内接円 & 各△IJKの外接円と内接円 \\
  注視点からの円 & 各△IJOの外接円(Oはカメラ注視点) \\    
  大三角 & 各△IJKを3倍拡大し重心をOに固定で描く \\    
  \hline
  \end{tabular}
  \end{center}
\end{table}

\subsection{オノマトペを付与する}
本アプリでは廣松の表情論\cite{hiromatsu:1989}を受け、ユーザには作図した表情図形をとおして感得した「表情」をオノマトペで表現することを促す。
オノマトペはオノマトペ入力欄(\autoref{fig:playfull_menu}右ドロワニュー内の中段)に書く。
オノマトペを吟味しやすくするために、入力したオノマトペは自動的に表情図形の背後に重ね描かれるようにした。
ユーザは既存のオノマトペだけに囚われず、創作オノマトペ\cite{otsuka:2015}を与えてもよい。

% \todo{応用編の音節機能}
% 応用編の機能として、ユーザはオノマトペを「ぺっちゃ-んこ」のように「音節」に分けて書くこともできる。
% 音節に分けて書くと、「音節ごとの再生」「音節ごとの内省記述」の機能も自動で開放される。
\subsection{内省記述:からだメタ認知をもちいて}

ユーザには、鑑賞体験をからだメタ認知し、内省記述欄に記述することを促す。
たとえば、身体感覚、情動、想起されるものごと、
表情図形、身体運動、それらの音韻で表象した理由
といった関係性について問い、書きつけることが重要である。
記述欄には、このことをガイドするプレースホルダを記載しているが、
表情が問いを生み出すという考えのもと、書く内容に強い制約は設けていない。

\subsection{表情エントリの保存・読み込み・コピー}
表情図形の構成要素を示す(\autoref{table:expression_elements})。
表情図形を構成する要素は、どの点が表示されているかや補助線のパタンとスタイルといった「空間的情報」だけではない。
フレーム範囲や軌跡の長さと軌跡最小単位などの「時間的情報」も、表情図形を構成する要素とした。
さらに、表情図形の構成要素には、身体運動(図形)をどこから眺めるかというカメラパラメータさえもふくむようにした。
科学的な態度からすれば、観察する主体であるユーザ(つまりカメラ側)は、観察対象である身体運動から切り離すことで、
なるべき客観的な観察を促すのが通常であろう。
しかし、動いている身体の表情とは、観察主体と観察対象との「あいだ」\cite{kimura:2005}に生じる現象である。
本アプリでは、その思想を優先的に反映するために、カメラパラメータも表情図形の構成要素の一部とした。

% 表情図形の構成要素
\begin{table}[htbp]
  \begin{center}
  \caption{表情図形の構成要素}                % 和文 caption  
  \label{table:expression_elements}
  \begin{tabular}[hbt]{c l l}
  \hline
  \bf 構成要素 & \bf 説明 \\
  \hline
  再生コマ範囲 & どこからどのコマか \\
  補助線パタン & どの点どうしの関係の補助線か \\
  補助線スタイル & 線分、延長線、内接円など \\
  構成する点群 & どの点が表示/非表示か \\
  軌跡の長さ & 点や補助線の軌跡の長さ \\
  軌跡の最小単位 & 軌跡の幅(1軌跡の何コマ差か) \\
  カメラ位置 & どこから眺めているか & \\
  遠近感 & 点の大きさが距離によらず同じか/違うか \\  
  \hline
  \end{tabular}
  \end{center}
\end{table}

ユーザには表情図形・オノマトペ・内省記述の3つを1つの\textbf{表情エントリ}としてセットで保存することを促す。
表情エントリを保存すると、右ドロワメニューの「マイ表情コレクション」(\autoref{fig:playfull_menu}右上段)にオノマトペが記載されたチップとして追加される。
チップをクリックすると、その表情エントリが読み込まれ、その表情エントリを再鑑賞できる(\autoref{fig:loadhyojoentry})。
\autoref{fig:loadhyojoentry}では、2枚目から3枚目にかけて、異なる表情エントリを選択したのに応じて、カメラの位置(表情エントリの構成要素である。\autoref{table:expression_elements}参照。)
も切り変わっていることが確認できる。
% my表情エントリのロード
\begin{figure}[htbp]
  \centering
  \includegraphics[height=\textheight]{./images/hyojoentryloading.pdf}
  \caption{保存済の表情エントリをDBから読み込む}          %和文 cap tion  
  \label{fig:loadhyojoentry}
\end{figure}

同様に「みんなの表情ギャラリー」(\autoref{fig:playfull_menu}参照)からは、他ユーザがその身体運動に対して作成した表情エントリを鑑賞することができる。
任意の表情エントリは、右下の「コピー」ボタンを押すと、表情エントリをコピーし再編集し、元の表情エントリとは別の表情エントリとして保存することができる(マイ表情コレクションに追加される)。
これを組み合わせた一連のシナリオを、以下\autoref{fig:senario1}〜\autoref{fig:senario2}に示す。
\begin{figure}[H]
  \centering
  \includegraphics[height=\textheight]{./images/senario1.pdf}
  \caption{シナリオ(局面1):他ユーザの表情エントリを鑑賞}          %和文 cap tion  
  \label{fig:senario1}
\end{figure}

\begin{figure}[H]
  \centering
  \includegraphics[height=\textheight]{./images/senario2.pdf}
  \caption{シナリオ(局面2):他ユーザの表情エントリを自分の手元にコピー}          %和文 cap tion  
  \label{fig:senario2}
\end{figure}

\begin{figure}[H]
  \centering
  \includegraphics[height=\textheight]{./images/senario3.pdf}
  \caption{シナリオ(局面3):コピーした表情エントリを再編集}          %和文 cap tion  
  \label{fig:senario3}
\end{figure}

\begin{figure}[H]
  \centering
  \includegraphics[width=0.9\textwidth]{./images/senario4.pdf}
  \caption{シナリオ(局面4):再編集した表情エントリをマイ表情コレクションへ登録}          %和文 cap tion  
  \label{fig:senario4}
\end{figure}


% 表情エントリの構成とユーザと身体運動データの関係性
\begin{figure}[H]
  \centering  
  \includegraphics[width=0.9\textwidth]{./images/hyojoentrymovementconfiguration@2x.pdf}
  \caption{表情エントリの構成とユーザと身体運動データの関係性}          %和文 cap tion  
  \label{fig:hyojoentryconfiguration}
\end{figure}

表情エントリ(とその構成)とユーザと身体運動データとの一般的な関係性を示す(\autoref{fig:hyojoentryconfiguration})。
一般に、ひとつの身体運動データに対して、複数のユーザがそれぞれ複数の表情エントリを作成できる。
このようにして、ひとつの身体運動に多彩な表情を感得し、採集することを促す。

\subsection{作図と編集操作のキーボードショートカット}
作図と編集操作のキーボードショートカット一覧を\autoref{table:keyboardshortcut}に示す。
ユーザにはこの一覧は、編集モード時にポップアップで表示される(\autoref{fig:toggleedittingmode}の2枚目参照)
\begin{table}[htbp]
  \caption{キーボードショートカット}
  \label{table:keyboardshortcut}
  \begin{center}
  \begin{tabularx}{\textwidth}{>{\bfseries}lX}
    \hline
    キーコマンド & 説明 \\
    \hline
    cmd + V & 複数点の表示を一括で切り替える。全点表示→孤立点のみ非表示→全点非表示→全点表示・・・と3状態でスイッチする。 \\
    \hline
    cmd + D & マウスオンしている点の表示/非表示を切り替える \\
    \hline
    cmd + E & マウスオンしている点と、表示中他全点とむすぶ/外す \\
    \hline
    cmd + R & 表情図形を初期化(リセット) \\    
    \hline
    cmd + S & 表情エントリを保存(または更新) \\    
    \hline
    cmd + Z & 作図の操作をひとつ前に戻す \\    
    \hline
    cmd + shift + Z & やりなおす(作図の操作をひとつ最新状態へ進める) \\    
    \hline
  \end{tabularx}
  \end{center}
\end{table}

% \subsection{その他の表示設定}
% \todo{座標とかのやつ。画像を示す。}



\section{システム構成と運用にもちいている技術}

本アプリは、プログラミング言語JavaScript、HTML、CSSによって制作している。
主にもちいたJavaScriptライブラリを\autoref{table:libraries}に示す\footnote{
  第一部の物語の\ref{sec:monowotool}節で登場したProcessingのJavascript版である。
}。
\begin{table}[t]
  \begin{center}
  \caption{使用ライブラリ。()内はバージョン情報を示す}  
  \label{table:libraries}
  \begin{tabularx}{\textwidth}{l X}
    \hline
    \textbf{ライブラリ} & \textbf{用途} \\
    \hline
    Vue.js (2.6.14) &
    各ページ(メイン画面、データ選択画面など)や、それらのページの子モジュールを構成するために用いた。 \\
    Vuetify (2.6.0) &
    Vue.jsと連携するかたちで、webの基本的なUIパーツ表現を簡易に作成できる。各画面をはじめ、メイン画面では左右ドロワメニューや上部バーメニューのUIを構成するために用いた。 \\
    p5.js (1.4.1) &
    インタラクティブなスケッチを描くのに優れている。メイン画面のプレイスペースの構成に用いた。ただし、点をマウスで選択する機能を実現するにはプレイスペースの3D空間と画面上2D空間の紐づける必要があるが、そうした情報を取得するメソッドは本ライブラリには搭載されていなかったので、自前で計算した。 \\
    \hline
  \end{tabularx}
  \end{center}
\end{table}

本アプリは、Firebase
\footnote{Google社が提供するwebアプリ開発プラットフォーム(BaaS)である。https://firebase.google.com/}
にて運用しており、
データベース(以下、DBと記載)にはFirestore、ホスティングにはFirebase Hostingをもちいている。
Firestoreは階層構造型のDBである。
いわゆるNoSQL型と呼ばれるもので、
SQL型(テーブル型)だと、各行にレコードが並び、各列にレコードのもつ項目変数が並ぶ、という行列構造だが、
NoSQL型のFirestoreはそれとは異なる。
Firestoreでは、コレクションのなかに複数のドキュメント(これが各レコードに相当)が格納されており、
各ドキュメントが項目変数の集合として表現されている。
本アプリでは以下に示す4種類のコレクションで構成している(\autoref{table:user}〜\autoref{table:plots})。
\begin{description}
  \item[usersコレクション]\mbox{傘下の各ドキュメントは各ユーザの情報をもつ}  
  \item[entriesコレクション]\mbox{傘下の各ドキュメントは各表情エントリの情報をもつ}
  \item[plotsコレクション]\mbox{傘下の各ドキュメントは各身体運動データの情報をもつ}  
  \item [plotMirrorsコレクション]\mbox{傘下の各ドキュメントは各身体運動データの概要的情報やメタ情報をもつ}      
\end{description}
なお、斜体で記したフィールド名(例:markerID)は、実際のドキュメントにおいてはこのフィールド名をそのままもっているのではなく、
任意のフィールド名が入り、同様のフィールドがたくさん並ぶことを表す。
紙面節約のため、それらの一般形たる仮のフィールド名をつけ、ひとつのフィールドとして仮記載している。

% users
\begin{table}[H]
  \centering  
  \caption{users コレクションのドキュメント構造}                %和文 caption
  \label{table:user}
  \begin{tabular}{|l|l|p{7cm}|}
    \hline
    \bf フィールド名 & \bf 型 & \bf 説明 \\
    \hline
    accessToken & string & ユーザの認証トークン \\
    % displayName & null/string & 表示名(未設定の場合は null) \\
    email & string & ユーザのメールアドレス \\
    exist & boolean & ドキュメント存在のフラグ \\
    userID & string & ユーザID \\
    entries & array of map & ユーザが作成したエントリ群の概要(entryID、plotID、表情オノマトペ) \\
    % photoURL & null/string & プロフィール画像のURL(未設定なら null) \\
    plots & array of map & ユーザがプレイしたplotID群 \\
    timestamp & number & ドキュメント作成日時 \\
    \hline
  \end{tabular}
\end{table}
% entries
\begin{table}[H]
  \centering
  \caption{entries コレクションのドキュメント構造}  
  \label{table:entries}
  \begin{tabular}{|l|l|p{8cm}|}
    \hline
    \bf フィールド名 & \bf 型 & \bf 説明 \\
    \hline
    exist & boolean & エントリが有効かどうかのフラグ \\
    frameRange & array [int, int] & エントリが対応するフレーム範囲(開始フレームと終了フレーム) \\
    hyojo & string & エントリに対応する表情やオノマトペのテキスト(例:「ぐぃーーーん」) \\
    hyojoSyllables & array of map & 表情の音節情報。各要素が syllable, frameRange, text を持つ \\
    \quad └ syllable & string & 音節単位の表現(例:「ぐぃーー」) \\
    \quad └ frameRange & array [int, int] & その音節が対応するフレーム範囲 \\
    \quad └ text & string & 補足テキスト(未入力の場合もあり) \\
    hyojoFigure & map & モーションや形状の描画に関するパラメータ群 \\
    \quad └ activeMarkers & array of string & 使用中のマーカ名一覧 \\
    \quad └ camParams & map & カメラの設定(位置) \\
    \quad └ dotSize & number & 点のサイズ \\
    \quad └ edges & array of map & 線分情報(例:どのマーカ同士を結ぶか) \\
    \quad └ endFrame & int & 終了フレーム \\
    \quad └ frameCount & int & 現在フレーム \\
    \quad └ hojosenType & string & 補助線の種類(線分、内接円、など) \\
    \quad └ isEnkinkan & boolean & 遠近感の有無 \\    
    \quad └ startFrame & int & 開始フレーム \\
    \quad └ trailDiff & int & 軌跡の間隔 \\
    \quad └ trailLength & int & 軌跡の長さ \\
    \quad └ triangles & array & 三角形の情報(互いに結ばれた3点関係) \\    
    markerIDs & array of string & 使用されたマーカIDの一覧 \\
    userID & string & このエントリを作成したユーザのID \\
    entryID & string & このエントリ自体の一意なID \\
    plotID & string & このエントリが作成された身体運動データのID \\
    text & string & ユーザーが記述したメタ認知記述 \\
    timestamp & number & ドキュメント作成日時 \\
    \hline
  \end{tabular}
\end{table}
% plots
\begin{table}[H]
  \centering
  \caption{plotMirrors コレクションのドキュメント構造}  
  \label{table:plotMirrors}
  \begin{tabular}{|l|l|p{8cm}|}
    \hline
    \bf フィールド名 & \bf 型 & \bf 説明 \\
    \hline
    exist & boolean & このデータが有効かどうかのフラグ \\
    fileName & string & 元のplotファイル名(例:人物名+日付など) \\
    plotID & string & 対応するplot本体のドキュメントID(plotsコレクションへの参照用) \\
    timestamp & number (Unix時間) & 登録または更新された時刻(ミリ秒単位) \\
    \hline
  \end{tabular}
\end{table}
% plotMirrors
\begin{table}[H]
  \centering
  \caption{plots コレクションのドキュメント構造}    
  \label{table:plots}
  \begin{tabular}{|l|l|p{8cm}|}
    \hline
    \bf フィールド名 & \bf 型 & \bf 説明 \\
    \hline
    exist & boolean & このplotデータが有効かどうかのフラグ \\
    fileName & string & 対象者や日付などを含むデータ名 \\
    fps & number & 1秒あたりのフレーム数(例:120fps) \\
    hyojoIDs & array of string & 関連づけられた表情データのID群(空の場合もあり) \\
    index & number & 表示順やデフォルト選択に用いる番号 \\
    krkrIDs & array of string & 関連する補助線などのID群 \\
    plotID & string & このplotの一意なID(外部参照用) \\
    plots & map & 各マーカの記録(斜体で書かれたキーは実際には各マーカのidや名前が入る \\
    \quad └ $markerID$ & map & マーカのID \\
    \quad\quad └ $id$ & string & マーカのID \\
    \quad\quad └ $name$ & string & マーカの日本語名 \\
    \quad\quad └ plots & array of map & 各フレームに対応する座標(x, y, z)情報(例:723点) \\
    \hline
  \end{tabular}
\end{table}

\section{開発環境}
本アプリの開発にもちいたPCは、Apple社のiMac2020(Retina 5K, 27-inch)である。
このハードウェア詳細は以下である。

\begin{itemize}
  \item {プロセッサ: 3.8 GHz 8コアIntel Core i7}
  \item {グラフィックス: AMD Radeon Pro 5500 XT 8 GB + }
  \item {メモリ: 40 GB 2667 MHz DDR4}  
\end{itemize}
プログラミングのエディタには、Microsoft社のVisual Studio Code\cite{vscode}をもちいた。

% \chapter{実践方法}
\label{chapter:jissenmethod}


\section{対象者}
4名(A〜D)を対象にアプリをもちいた実践をおこなった(\autoref{table:subjects})。
A〜Dいずれもからだメタ認知をもちいて学ぶ経験を1年以上有している。
主たる実践対象者(以下、対象者と呼ぶ)は現役の運動学習者であるAとBである。
CとDは副対象者とした。
副対象者として参加してもらう目的は、主対象者であるAやBの学習を触発することにある。
前章で述べたとおり、HJ-Playgroundでは他者の表情コレクションも閲覧できる。
AとBがCやDの表情エントリに影響を受けて自身の表情エントリを作成するなど、学びの可能性をひらいておくため、というのがCとDの参加の第一目的である。
また、対象者の属性によってHJ-Playgroundの効果を比較しやすくするためでもある。

後述するが、Aは自身が踊る「oldman」というダンス技をプレイ対象動作とし、
BとCは「一本ゲタ対人運動」をペアでおこない、これをプレイ対象動作とし、
Dは自身の運動を撮影せずに、上記A〜Cの動作をもちいてプレイした。

\begin{table}[htbp]
  \centering
  \caption{実践対象者の概要}
  \begin{tabular}{c|clll}
    \hline
    \bf ID & \bf 年齢(歳) & \bf 実践ドメイン(現役/引退) & \bf プレイ対象動作 & \bf 学習対象者区分 \\
    \hline
    A  & 24 & ストリートダンサー(現役) & oldman                     & 主たる対象者 \\
    B  & 21 & 陸上・三段跳(現役)         & 一本ゲタ対人運動(with C) & 主たる対象者 \\
    C  & 24 & 陸上・棒高跳&走高跳(引退) & 一本ゲタ対人運動(with B) & 副対象者     \\
    D  & 23 & サッカー(引退) & oldman\&一本ゲタ対人運動  & 副対象者     \\
    \hline
  \end{tabular}
\label{table:subjects}
\end{table}


\section{実践手順}
本実践の実施にあたっては、SFC研究倫理審査委員会の承認を得ている(2023年10月17日承認)。
対象者に実践について説明し、同意書へのサインを得たのち、以下の手順で実践した。
\begin{enumerate}
  \item プレイ対象動作を考案・選定する(著者と学習者とでおこなう)
  \item 対象者のプレイ対象動作をモーキャプで撮影する
  \begin{itemize}
    \item 撮影データをアプリに読み込む
  \end{itemize}  
  \item 対象者がHJ-Playgroundでプレイしながら日々過ごす(実践期間)
  \item 学びを解釈するインタビューの実施
\end{enumerate}

それぞれ説明する。
実践の説明ののち、対象者自身のどういう動作をアプリでプレイするかを著者と共同で考案・選定した。

対象者の動作をモーションキャプチャOptiTrack社製 V120: Trio\footnote{
解像度30万画素、レンズ視野角47°
}をもちいて120Hzで撮影した。
撮影データをアプリDBに登録したのち、
対象者らにwebアプリのURLとアプリ使用説明書を共有し、
この時点をアプリでの「実践期間開始」とした。

実践期間中には、Zoomをもちいたオンラインセッションも2度おこない、不明点の聞き取りや対象者をエンカレッジしたりした。

実践終了後、AとBに2回ずつ、学びを解釈するための半構造化インタビューを実施した。
オンライン会議アプリZoomで著者と1対1で、実際に対象者が作成した表情エントリを鑑賞しながら進めた。
1回目インタビュー(Aは約84分、Bは約70分、ともに2025年4月19日)の予め用意した質問項目は、「実践期間に運動学習者として抱いていた問題意識」、「どのような表情図形を作図し、いかなる問いを生んだか?」、「アプリでの表情感得体験が、身体知の学びになにをもたらしたか?」の3点である。
2回目インタビュー(Aは2025年8月26日に、Bは2025年8月29日、ABともに50分)の予め用意した質問項目は、「表情エントリ相互の派生関係」「その後の学びの活動はどうか」「HJPの経験とその後の学びの関連」の3点である。

インタビューの目的を補足する。あくまで実践の主要データはAとBが作成した表情エントリである。
だが、とくにからだメタ認知記述は、AとBが「自分のため」に問いながら書きつけた言葉であって、必ずしも他者にわかる説明の言葉になっているわけではないし、上記質問項目のような、自身の学びを俯瞰した記述も必ずしもなされない。
本研究では、それを著者が解釈し、個別具体的な学びの様相を読者になるべく了解可能なかたちで解説するという手立てをとる(それが次章である)。
そういう解釈の補助としてインタビューを実施した。
2回目インタビューは、身体知の学びは時間をかけて醸されるという考えから、時間をおいてもう一度実施した。

AとB(インタビュイー)は著者(インタビュワー) と、同研究室で学びながら、からだメタ認知を駆使して身体知の学びに取り組んだ仲間であり、かつ、そのことを相互理解していた。
半構造化インタビューに臨むためのラポール形成の観点からも、このことは重要であろう。
半構造化インタビューでは、対象者は自身の表情エントリと自身の学びとを架橋しうるもっともらしい解釈を語りだし、自身も身体知の学び手である著者は、対象者からそれを引き出すように深掘る質問を投げかけた。
そのように両者のあいだで「折り合いをつける」ようにして、解釈を紡いだ。


\section{対象動作の身体運動データの取得について}

対象者Aは、以下のようにマーカを貼り付けた(\autoref{fig:markers_a_f}〜\autoref{fig:markers_a_b})。

% サトゥー
\begin{figure}[h]
\centering
  \begin{minipage}[b]{0.3\linewidth}
  \centering
  \includegraphics[width=\linewidth]{./images/jikken/sato_f.pdf}
  \caption{対象者Aのマーカ貼り付け位置(正面から)}
  \label{fig:markers_a_f}
  \end{minipage}
\hspace{0.02\linewidth}%画像間の余白
  \begin{minipage}[b]{0.3\linewidth}
  \centering
  \includegraphics[width=\linewidth]{./images/jikken/sato_l.pdf}
  \caption{対象者Aのマーカ貼り付け位置(左から)}
  \label{fig:markers_a_l}
  \end{minipage}
\hspace{0.02\linewidth}%画像間の余白
  \begin{minipage}[b]{0.3\linewidth}
  \centering
  \includegraphics[width=\linewidth]{./images/jikken/sato_b.pdf}
  \caption{対象者Aのマーカ貼り付け位置(後ろから)}
  \label{fig:markers_a_b}
  \end{minipage}
\end{figure}

対象者B・Cは、以下のようにマーカを貼り付けた(\autoref{fig:markers_bc_f}〜\autoref{fig:geta_ushiro})。
% キサラとサラ一緒に
\begin{figure}[H]
\centering
  \begin{minipage}[b]{0.3\linewidth}
  \centering
  \includegraphics[width=\linewidth]{./images/jikken/sarakisara_s.pdf}
  \caption{対象者B・Cのマーカ貼り付け位置(真横から)}
  \label{fig:markers_bc_f}
  \end{minipage}
\hspace{0.02\linewidth}%画像間の余白
  \begin{minipage}[b]{0.3\linewidth}
  \centering
  \includegraphics[width=\linewidth]{./images/jikken/sarakisara_f.pdf}
  \caption{対象者B・Cマーカ貼り付け位置(Bの後ろから)}
  \label{fig:markers_bc_b}
  \end{minipage}
\hspace{0.02\linewidth}%画像間の余白
  \begin{minipage}[b]{0.3\linewidth}
  \centering
  \includegraphics[width=\linewidth]{./images/jikken/sarakisara_k.pdf}
  \caption{対象者B・Cのマーカ貼り付け位置(Cの後ろから)}
  \label{fig:markers_bc_c}
  \end{minipage}
\end{figure}

% キサラとサラ別々
\begin{figure}[H]
\centering
  \begin{minipage}[b]{0.21\linewidth}
  \centering
  \includegraphics[width=\linewidth]{./images/jikken/sara_f.pdf}
  \caption{対象者Bのマーカ貼り付け位置(正面から)}
  \label{fig:markers_b_f}
  \end{minipage}
\hspace{0.01\linewidth}%画像間の余白
  \begin{minipage}[b]{0.21\linewidth}
  \centering
  \includegraphics[width=\linewidth]{./images/jikken/sara_r.pdf}
  \caption{対象者Bのマーカ貼り付け位置(真横から)}
  \label{fig:markers_b_l}
  \end{minipage}
\hspace{0.01\linewidth}%画像間の余白
  \begin{minipage}[b]{0.21\linewidth}
  \centering
  \includegraphics[width=\linewidth]{./images/jikken/kisara_l.pdf}
  \caption{対象者Cのマーカ貼り付け位置(真横から)}
  \label{fig:markers_c_r}
  \end{minipage}
\hspace{0.01\linewidth}%画像間の余白
  \begin{minipage}[b]{0.21\linewidth}
  \centering
  \includegraphics[width=\linewidth]{./images/jikken/kisara_f.pdf}
  \caption{対象者BCのマーカ貼り付け位置(正面から)}
  \label{fig:markers_c_f}
  \end{minipage}
\end{figure}

% ゲタ
\begin{figure}[H]
\centering
  \begin{minipage}[b]{0.21\linewidth}
  \centering
  \includegraphics[width=\linewidth]{./images/jikken/geta_chuo.pdf}
  \caption{中央ゲタのマーカ貼り付け位置}
  \label{fig:geta_chuo}
  \end{minipage}
\hspace{0.01\linewidth}%画像間の余白
  \begin{minipage}[b]{0.21\linewidth}
  \centering
  \includegraphics[width=\linewidth]{./images/jikken/geta_ushiro.pdf}
  \caption{後ろゲタのマーカ貼り付け位置}
  \label{fig:geta_ushiro}
  \end{minipage}
\end{figure}

% \chapter{実践結果}
\label{chapter:jissenresult}
本アプリをもちいて、いかにして「表情」を感得し、いかなる問いを生んだのか?
次節において、対象者AとBがおこなったその意味づくりの様相の一端を、AとBが生成した表情エントリ(と補助的インタビュー)を拠りどころにしながら、
著者の視点から記述する。
前述したように、これは著者がAとBに二人称的にかかわる\cite{saeki:2017}ことによって可能になる。

以降では、対象者が作成した表情エントリは《A4》や《B13》のように《対象者+番号》の形でID表記し、《》の中身には適宜当該エントリのオノマトペを併記する。
対象者がインタビューで語った言葉の直接引用は『』でくくって表記する。

\section{対象者Aの学び}
Aは、流れてくる曲にあわせて即興的に踊るヒップホップダンサーである。
Aがめざしているのは『豊かな技の引き出しをもち、それらを様々に組み合わせて自分らしく楽しく踊れる』ようになることである。
本研究以前からAには、時おり練習しているがいまいち『しっくり』こず、まだ技の引き出しに収められていない技があった。
上半身を下半身より先行させて左右に移動する基本技「oldman」である(\autoref{fig:oldman1})。
oldmanはポップダンスの基本技であるが、『各部位を独立的に動かすアイソレーションあるいは連動や身体軸の制御』など、ヒップホップのエッセンスを多く含んでいるとAは考える。
そこで、oldmanをAの撮影対象動作(プレイ対象動作)として選定した。
% oldman1
\begin{figure}[tb]
  \begin{continuousphoto}                      
  \begin{center}
    \begin{overpic}[width=0.18\columnwidth]{./images/hjplayground/oldman1c-1_page1.pdf}
      \put(2,90){1:始動}
    \end{overpic}
    \hspace{-0.5em}
    \begin{overpic}[width=0.18\columnwidth]{./images/hjplayground/oldman1c-2_page1.pdf}
      \put(2,90){2:上体先行}
    \end{overpic}
    \hspace{-0.5em}
    \begin{overpic}[width=0.18\columnwidth]{./images/hjplayground/oldman1c-3_page1.pdf}
      \put(2,90){3:右足接地}
    \end{overpic}
    \hspace{-0.5em}
    \begin{overpic}[width=0.18\columnwidth]{./images/hjplayground/oldman1c-4_page1.pdf}
      \put(2,90){4:左足接地}
    \end{overpic}
    \hspace{-0.5em}
    \begin{overpic}[width=0.18\columnwidth]{./images/hjplayground/oldman1c-5_page1.pdf}
      \put(2,90){5:上体先行}
    \end{overpic}
  \end{center}
  \end{continuousphoto}
  \caption{Aがoldmanを踊る様子(oldman1)}            
  \label{fig:oldman1}
\end{figure}

Aは2025年3月25日〜4月19日の25日間の期間で、全19個の表情エントリ(《A1》〜《A19》)を作成した(\autoref{table:hyojoentries_a})。
実践開始前に撮影したデータを「oldman1」と記載する(\autoref{fig:oldman1}は実際のoldman1の映像である)。
\begin{longtable}{llllr}
\caption{Aの作成した表情エントリ一覧} \\
\hline
表情エントリID & 作成日付 & 運動データ & 表情オノマトペ & 文字数\\
\hline
\endfirsthead
\hline
表情エントリID & 作成日付 & 運動データ & 表情オノマトペ & 文字数\\
\hline
\endhead
《A1》 & 03/25/25 & oldman1 & フリフリ & 393\\
《A2》 & 03/25/25 & oldman1 & ぐういペ & 512\\
《A3》 & 04/01/25 & oldman1 & ブふぁああーーー & 420\\
《A4》 & 04/01/25 & oldman1 & ガチョベチョガキョ & 494\\
《A5》 & 04/01/25 & oldman1 & バラらららららら & 212\\
《A6》 & 04/02/25 & oldman1 & なぬななぬな & 348 \\
《A7》 & 04/08/25 & oldman1 & ンーコツコツンー & 228 \\
《A8》 & 04/08/25 & oldman1 & グリグリ & 184 \\
《A9》 & 04/08/25 & oldman1 & フラフラぶんフラフラ& 328 \\
《A10》 & 04/13/25 & oldman1 & フサササフサ & 273\\
《A11》 & 04/13/25 & oldman1 & ブふぁああーーーペ & 23\\
《A12》 & 04/17/25 & oldman2 & うー、、わっダラララ & 122\\
《A13》 & 04/17/25 & oldman2 & ねーねねねね... & 200\\
《A14》 & 04/18/25 & oldman3 & mm & 18\\
《A15》 & 04/18/25 & oldman3 & 擬音が思いつかない、、 & 63\\
《A16》 & 04/18/25 & oldman3 & パラララららら & 11\\
《A17》 & 04/18/25 & oldman3 & 惑星 & 9\\
《A18》 & 04/19/25 & oldman1 & バラバラ & 60\\
《A19》 & 04/19/25 & oldman3 & へふいへ & 303\\
\hline
\label{table:hyojoentries_a}
\end{longtable}


\subsection{《フリフリ(A1)》}
\label{subsec:a1}
Aのプレイのしかたには、本実践以前からもっていた「身体内の仮想的な線」を表情図形へ具現してみるというものがあった。
たとえば《フリフリ(A1)》や《ンーコツコツンー(A7)》である。
《フリフリ(A1)》は「oldman1」で、両肩を結ぶ補助線を引き、補助線スタイル「10倍延長」で延長し、それを斜め前から眺める、という図形に感得した表情である(\autoref{fig:hyojonotea1})。
当初の作図のねらいは「動作全局面をとおした両肩線の水平具合」を評価するためであった。
だが実際に鑑賞してみると、「鉛筆を優しくつまんで振った」ような表情が立ち現れた。
その表情は、動作の最後の瞬間(\autoref{fig:oldman1}局面5)をきわだて、
「両肩線が上下に微妙に揺れている」という新事実への着眼をAにもたらしたのだった。
% 《フリフリ(A1)》
\begin{figure}[H]
  \begin{hyojoentry}    
  \textbf{《フリフリ(A1)》} (oldman1)

  \vspace{0.5em} \hrule height 0.5pt \vspace{0.5em}

  \begin{center}
    \begin{overpic}[width=0.3\columnwidth]{./images/hjplayground/A1-1.pdf}
      \put(2,90){1}
    \end{overpic}
    \hspace{0.05em}
    \begin{overpic}[width=0.3\columnwidth]{./images/hjplayground/A1-2.pdf}
      \put(2,90){2}
    \end{overpic}
    \hspace{0.05em}
    \begin{overpic}[width=0.3\columnwidth]{./images/hjplayground/A1-3.pdf}
      \put(2,90){3}
    \end{overpic}
  \end{center}

  \vspace{0.5em}

  \small
  \textit{
    % 鎖骨の二点を結び、線延長することで鎖骨をぶっさすような形を作ってみた。
    (前略)
    最後の数コマの間、棒がフリフリと揺れる様子を見つけた。
    この結び方をすることで鎖骨の動きを見てみようと思って結んでみた。
    鎖骨はなるべく水平を維持できると見た目が綺麗になると思っており、この線を見ることでその平行具合を見ようとしたが、傾きよりも 最後の一瞬に存在する棒が上下に揺れる姿が発見的だった。
    左右の点が交互に上下運動していて、それによってその点を貫く棒が上下に揺れる。
    振れ幅は小さく、周波数は高い。小刻みに細い棒が揺れる姿が、鉛筆を指で持って振るやつに似ている。
    あれは結構ブンブンする感じだけど、これはもっと優しげ。フリフリといった感じ。悪く言えばブレと言える。
    良く言えば、、あんまり良く言えないな、、。
    (後略)
    % 傾きのようなもっと大きなブレを気にしていたが、こんな小さなブレも起きているのか、、これを治すのは骨が折れそう
  }

  % \vspace{0.5em} \hrule height 0.5pt \vspace{0.5em}

  \end{hyojoentry}

  \caption{表情エントリ《A1》}            
  \label{fig:hyojonotea1}
\end{figure}

\subsection{《ンーコツコツンー(A7)》}
《ンーコツコツンー(A7)》(\autoref{fig:hyojonotea7})は、「線分」スタイルの補助線を「右肩-左つま先」「左肩-右つま先」の各2点間に引き、全点を非表示にすることで作図した表情図形から得た表情である。
斜め方向に伸び切る線のようす(局面1)を「限界まで倒れそうになる軸」とし、それを支えるために慌てて直後の足の踏み出しが起こっているように「見えるような見えないような」、
そういう絶妙な違和感をAは生々しく書き綴っている(\autoref{fig:hyojonotea7})。
人体形に似て非なる図形に移入しようとすると、どこかもつれた身体感覚が芽生える。
そうやって違和感(問い)を触発するという、表情感得の事例だと著者は考える。
% 《ンーコツコツンー(A7)》
\begin{figure}[H]
  \begin{hyojoentry}

  \textbf{《ンーコツコツンー(A7)》} (oldman1)

  \vspace{0.5em} \hrule height 0.5pt \vspace{0.5em}

  \begin{center}
    \begin{overpic}[width=0.18\columnwidth]{./images/hjplayground/A7b-1.pdf}
      \put(2,90){1(始動)}
    \end{overpic}
    \hspace{-0.5em}
    \begin{overpic}[width=0.18\columnwidth]{./images/hjplayground/A7b-2.pdf}
      \put(2,90){2(ンー)}
    \end{overpic}
    \hspace{-0.5em}
    \begin{overpic}[width=0.18\columnwidth]{./images/hjplayground/A7b-3.pdf}
      \put(2,90){3(コツ)}
    \end{overpic}
    \hspace{-0.5em}
    \begin{overpic}[width=0.18\columnwidth]{./images/hjplayground/A7b-4.pdf}
      \put(2,90){4(コツ)}
    \end{overpic}
    \hspace{-0.5em}
    \begin{overpic}[width=0.18\columnwidth]{./images/hjplayground/A7b-5.pdf}
      \put(2,90){5(ンー)}
    \end{overpic}
  \end{center}

  \vspace{0.5em}

  \small
  \textit{
    限界まで倒れる軸を支えるために慌てて足が出てるように見えるような気がする。
    実際にはそんなバランス崩れるような動きではないが、この棒二本だとそう見える。
    と思ったが、そうでもない感じする。
    なにかというと、斜めが伸びきっている時、線が伸び縮みしているからだ。と思う。
    (後略)
    % 立体だからか。
    % 後ろにそれば、長さが変わるんだな。奥行きのあるダンス?
    % 真正面から見たら平面的?奥行きのない身体などないな。
    % 地面表示してみたら、この棒だけで、もうオールドマン"
  }

  % \vspace{0.5em} \hrule height 0.5pt \vspace{0.5em}

  \end{hyojoentry}

  \caption{表情エントリ《A7》}            
  \label{fig:hyojonotea7}
\end{figure}

\subsection{《ブふぁああーーー(A3)》}
《ブふぁああーーー(A3)》(\autoref{fig:hyojonotea3anda19}上段)は、
胸骨1点と骨盤4点とに補助線を引いて「四角錐」を作り、それが動くようすから得た表情である。
Aはこの表情図形を、四角錐の胸骨頂点を中心に底面を振り動かす感覚としてとらえた。
同時にAは「固さ」を感じとった。
普段、体内には「ゴム」や「風船」といった弾性体モチーフでアプローチしてきたAにとって、
体内で四角錐を動かす感覚は、いっそう新しく感じられたのである。
(\autoref{fig:hyojonotea3anda19}の内省記述参照)。
「底面の浮き上がる感じ(\autoref{fig:hyojonotea3anda19}上段、局面1〜2)」など、Aはおもしろさを感じていた。
Aは通勤中さえも、歩く動きのなかでこの表情図形の咀嚼を続けていた。
(しかしそう簡単に体得できるものではないという素直な現実が語られている)。
%《ブふぁああーーー(A3)》と《へふいへ(A19)》
\begin{figure}[H]
  \begin{hyojoentry}

  上段:\textbf{《ブふぁああーーー(A3)》} (oldman1)\\
  下段:\textbf{《へふいへ(A19)》} (oldman3)

  \vspace{0.5em} \hrule height 0.5pt \vspace{0.5em}

  
  \setlength{\tabcolsep}{1pt} % ← 横のスキマを詰める!
  \renewcommand{\arraystretch}{0.95} % ← 縦のスキマ(ラベルと画像間)
  % 上段
  \begin{center}
    \begin{overpic}[width=0.18\columnwidth]{./images/hjplayground/A3b-1.pdf}
      \put(2,90){1}
    \end{overpic}
    \hspace{-0.5em}
    \begin{overpic}[width=0.18\columnwidth]{./images/hjplayground/A3b-2.pdf}
      \put(2,90){2}
    \end{overpic}
    \hspace{-0.5em}
    \begin{overpic}[width=0.18\columnwidth]{./images/hjplayground/A3b-3.pdf}
      \put(2,90){3}
    \end{overpic}
    \hspace{-0.5em}
    \begin{overpic}[width=0.18\columnwidth]{./images/hjplayground/A3b-4.pdf}
      \put(2,90){4}
    \end{overpic}
    \hspace{-0.5em}
    \begin{overpic}[width=0.18\columnwidth]{./images/hjplayground/A3b-5.pdf}
      \put(2,90){5}
    \end{overpic}
  \end{center}

  \vspace{0em}

  % 下段
  \begin{center}
    \begin{overpic}[width=0.18\columnwidth]{./images/hjplayground/A19b-1.pdf}
      \put(2,90){1(へ)}
    \end{overpic}
    \hspace{-0.5em}
    \begin{overpic}[width=0.18\columnwidth]{./images/hjplayground/A19b-2.pdf}
      \put(2,90){2(ふい)}
    \end{overpic}
    \hspace{-0.5em}
    \begin{overpic}[width=0.18\columnwidth]{./images/hjplayground/A19b-3.pdf}
      \put(2,90){3(へ)}
    \end{overpic}
    \hspace{-0.5em}
    \begin{overpic}[width=0.18\columnwidth]{./images/hjplayground/A19b-4.pdf}
      \put(2,90){4}
    \end{overpic}
    \hspace{-0.5em}
    \begin{overpic}[width=0.18\columnwidth]{./images/hjplayground/A19b-5.pdf}
      \put(2,90){5}
    \end{overpic}
  \end{center}  
  
  
  \small  
  \textbf{《ブふぁああーーー(A3)》の内省記述}\\
  \vspace{-1.0em} \hrule height 0.5pt \vspace{0.5em}  
    骨盤を底面、胸骨の真ん中を頂点とする四角錐。
    底面がこちら側に浮き上がるように持ち上がる。
    浮き上がる最初の勢いは結構あって、ブワってくるかと思いきや。
    そのあとはふぁあああーーって感じでふわって持ち上がる。
    これおもろそう。
 
    20250401追記。
    今朝、駅で考えながら歩いてみた。
    これ、結構面白い。あんまりまだ意識しこなせないけど。
    手と足が同時に出そうになったりした。
    身体の中に知らない関係図を採用したためじゃない?
    体内がゴムとか風船とかそういう比喩意識で動くことはちょいちょいあったけど、幾何学立体ってのは初めての感覚だな。
    柔らかく動かないせいで、扱いが難しい感じ。
    なぜかこのあと、猫の捻れ問題に思いを馳せた。
    上手いダンサーは割と身体単体でエネルギー収支合わせてそうな気がする。
    仕事量0のダンス?疲れなさそーー。ダンス初めて、ハウスとかhiphopとか、だんだん疲れなくなっていくのを感じる時期があったことを思い出す。  

  % \vspace{0.5em} \hrule height 0.5pt \vspace{0.5em}
  \vspace{0.5em}
  \small  
  \textbf{《へふいへ(A19)》の内省記述}\\
  \vspace{-1.0em} \hrule height 0.5pt \vspace{0.5em}  
    前回は、四角錐の頂点を持って四角錐を振るという想像をしていたが、これ、底面が頂点を振るというふうに見える。
    頂点が思ったよりくるくるブンブン動いていて、底面は底面として移動している感じ。
    底面の角度が頂点を動かしている。
    へ で底面が浮く。 ふい で真ん中を次の位置に向かって移動。へ で次の位置に着地。これは底面の動きに対する擬音。
    この、連動におけるどっちが主導?というはなし、結構大事な要素な気がする。
    同じステップでも、頂点を主とするか、底面を主とするかで2通りの捉え方ができる。
    他の動きにおいても、主導する部位を入れ替えればそれだけで動きのパートリーを倍にできる。
    可能性がある。  

  \end{hyojoentry}

  \caption{表情エントリ《ブふぁああーーー(A3)》と《へふいへ(A19)》}            
  \label{fig:hyojonotea3anda19}
\end{figure}

\subsection{《フサササフサ(A10)》}
% 《フサササフサ(A10)》
《フサササフサ(A10)》は、Aが備えつけ補助線パタン「左から右に一筆がき」をもちいて作図した表情図形から感得した表情である。
くしくもAはStrandbeest\cite{jansen:online}を想起しており、風に吹かれて動くような身体感覚を得た(\autoref{fig:hyojonotea10})。
Aは内省記述内で、以前作成した《ンーコツコツンー(A7)》(\autoref{fig:hyojonotea7})の表情図形についても言及している。
類似の表情図形をも引き込みつつ、両者に通底する表情として感じようとすることは、意味に厚みを与えることでもあると著者は解釈する。
\begin{figure}[H]
  \begin{hyojoentry}

  \textbf{《フサササフサ(A10)》} (oldman1)

  \vspace{0.5em} \hrule height 0.5pt \vspace{0.5em}

  \begin{center}
    \begin{overpic}[width=0.18\columnwidth]{./images/hjplayground/A10-1.pdf}
      \put(2,90){1}
    \end{overpic}
    \hspace{-0.5em}
    \begin{overpic}[width=0.18\columnwidth]{./images/hjplayground/A10-2.pdf}
      \put(2,90){2(フサ)}
    \end{overpic}
    \hspace{-0.5em}
    \begin{overpic}[width=0.18\columnwidth]{./images/hjplayground/A10-3.pdf}
      \put(2,90){3(サ)}
    \end{overpic}
    \hspace{-0.5em}
    \begin{overpic}[width=0.18\columnwidth]{./images/hjplayground/A10-4.pdf}
      \put(2,90){4(サ)}
    \end{overpic}
    \hspace{-0.5em}
    \begin{overpic}[width=0.18\columnwidth]{./images/hjplayground/A10-5.pdf}
      \put(2,90){5(フサ)}
    \end{overpic}
  \end{center}

  \vspace{0.5em}

  \small
  \textit{
    "https://www.youtube.com/watch?v=Pj-NqWDH2qE
    これを感じる。風で動く木組みのモンスターみたいなやつ。 
    風が吹くと連動して全体が動く様
    無駄のない動きってやつはこのように、一つの力が全体に連動していって、勝手に身体が動いていくって感じなんじゃないか。
    軸棒クロスだけのノートを前に作ったけど、あれも可動域の限界がきてトントンって足が出る感じだった。そういうことか。 
    擬音の話で言うと、これはもはやあの木組のモンスターの動きの擬音である。風のフと、軽くて静かな動きササの合わせ技。
  }

  \end{hyojoentry}

  \caption{表情エントリ《フサササフサ(A10)》}            
  \label{fig:hyojonotea10}
\end{figure}

\subsection{《へふいへ(A19)》と《ブふぁああーーー(A3)》}
Aは、《フサササフサ(A10)》を作成した同日、2回目の撮影を行った。
2通りの意識のしかたでoldmanを踊った。
その2通りとは、Aがこれまで作成した10個の表情エントリのなかで、A自身がとくに気になっている表情であった。
それをもとに意識をつくって踊ってみるとどうなるのかをAは試してみたくなったのである。
Aは、ひとつは、《フササフサ(A10)》にもとづき「フササフサ」というオノマトペを意識して踊り(運動データ「oldman2」)
もうひとつは、《ブふぁああーーー(A3)》の表情図形を動かすことをイメージして踊る(運動データ「oldman3」)
ことで、撮影をおこなった。

oldman2に対しては、《うー、わっダラララ(A12)》や《ねーねねねね…(A13)》などの表情を感得したが、あまり新しい問いを生み出せなかったとAはインタビューで語った。
oldman3に対しAは、《へふいへ(A19)》(\autoref{fig:hyojonotea3anda19})を作図した。
これは、oldman1で作図した《ブふぁああーーー(A3)》と同じく胸骨と骨盤からなる四角錐だが、そこに軌跡(120fpsで10フレーム一単位とした15単位=1.25秒ぶん)をつけた図形である。

Aはここで大きな気づきを得る。
両図形とも同じ部位からなる四角錐であるいっぽうで、その表情には大きな違いがあることをAは見出した。
《ブふぁああーーー(A3)》では局面1-2にかけて頂点が右に移動し(ぶ)、
局面2-4にかけて底面が頂点に振られる(ふぁー)。
いっぽう《へふいへ(A19)》では、局面1で底面が浮き(へ)、
局面2-3にかけて頂点が底面に振られ(ふい)(3で底面は自身が振り動かした頂点に追従し)、
局面3-4で底面が着地する(へ)。
つまり《A19》は、《A3》とは底面-頂点の連動主従関係が逆転した別の表情である。
そうしてAは「他の動きにおいても、主導する部位を入れ替えればそれだけで動きのレパートリーを倍にできる」という、踊りの引き出しについての一段階メタな仮説へと昇華したのである(\autoref{fig:hyojonotea3anda19}の内省記述)。

\subsection{アプリの実践がAのダンスにもたらしたこと}
\label{subsec:caseAsummay}

\textcolor{red}{
  以上、Aはoldmanでのアプリ使用をとおして、自身のダンスに以下のような問いを展開した。
\begin{itemize}
  \item 両肩のラインに意外な揺れ動きが生じているという事実に着眼した(《フリフリ(A1)》)
  \item oldman自体のしっくりくるよう踊る意識のしかたを発見した(《フサササフサ(A10)》)
  \item 左右移動するときの新しい感覚を獲得した(《ンーコツコツンー(A7)》)、
  \item 部位間の主従を転換しようと意識するだけで動きのレパートリーを倍増しうるという一段メタな仮説を形成した(《へふいへ(A19)》と《ブふぁああーーー(A3)》)
\end{itemize}
}


本アプリで感得した「表情」が、Aが日常で踊るダンスに確かな影響があったことを、Aはインタビューで語った。
2025年4月16日に踊ったダンスのなかで、Aは《ブふぁああーーー(A3)》を思い浮かべた動きを繰り出した(\autoref{fig:a3i})。
なお、この時点では《A11》までを作成していた。
BPM=約88の曲のフォーカウントからフォーエンカウントの動きである。
局面1で「骨盤底面」が「ブ」と浮き、局面2-4で胸骨頂点を中心に骨盤底面を「ふぁああーーー」と振っている。
この動きはoldmanとは右足と左足を着く順番が逆転しているが、
実際の即興ダンスのなかでは、基本技は、状況(音楽や環境)に即応するよう変形して繰り出される。
そのように踊るためには、状況のもつ表情を感得するのはもちろんのこと、
自らの踊りそのものの表情をも感得することも重要だろうと著者は考える。
% 《ブふぁああーーー(A3)》の実践編
\begin{figure}[H]
  \begin{continuousphoto}

  \begin{center}
    \begin{overpic}[width=0.240\columnwidth]{./images/hjplayground/A3i-1.pdf}
      \put(2,90){1}
    \end{overpic}
    \hspace{-0.5em}
    \begin{overpic}[width=0.240\columnwidth]{./images/hjplayground/A3i-2.pdf}
      \put(2,90){2}
    \end{overpic}
    \hspace{-0.5em}
    \begin{overpic}[width=0.240\columnwidth]{./images/hjplayground/A3i-3.pdf}
      \put(2,90){3}
    \end{overpic}
    \hspace{-0.5em}
    \begin{overpic}[width=0.240\columnwidth]{./images/hjplayground/A3i-4.pdf}
      \put(2,90){4}
    \end{overpic}    
  \end{center}

  \end{continuousphoto}

  \caption{《ブふぁああーーー(A3)》を発動しているAのダンス}
  \label{fig:a3i}
\end{figure}

実践終了直後の2025年4月19日に踊ったダンスで、Aは《へふいへ(A19)》から影響を受けた動きを試してみた(\autoref{fig:a19i})。
BPM=約82の曲のツーカウント(2)からツーエンカウント(2.5)の動きである。
右足に体重をかけて踏み締め、その地面反力を利用して左後ろに身体を放る(局面4の直後)。
骨盤を右足にしっかりと乗せたまま傾け、それを伝播させるように、胸骨をわずかに前を通るようにしながら(局面2-3)、左後ろにむかって(局面4)小さく振り回している。
図からもただちには見てとれないほど微妙で繊細である。
Aはインタビューにて、これを動画で観ると、踊っている最中に自分が感じていたほど《へふいへ》の動きにはなっていないと反省をした。
動いているさなかで本人に立ち現れる感覚(あるいは醸し出そうとする表情)と、それを外からみて感得しうる表情は同じとは限らない。
しかし、動いている身体に感得した表情から問いを展開し、動きかたの引き出しを増やした(=意味をつくった)ことは、
本アプリがAにもたらしたひとつの意義であると考える。
% 《へふいへ(A19)》の実践編
\begin{figure}[H]
  \begin{continuousphoto}
  \begin{center}
    \begin{overpic}[width=0.240\columnwidth]{./images/hjplayground/A19i-1.pdf}
      \put(2,90){\color{white}1}
    \end{overpic}
    \hspace{-0.5em}
    \begin{overpic}[width=0.240\columnwidth]{./images/hjplayground/A19i-2.pdf}
      \put(2,90){\color{white}2}
    \end{overpic}
    \hspace{-0.5em}
    \begin{overpic}[width=0.240\columnwidth]{./images/hjplayground/A19i-3.pdf}
      \put(2,90){\color{white}3}
    \end{overpic}
    \hspace{-0.5em}
    \begin{overpic}[width=0.240\columnwidth]{./images/hjplayground/A19i-4.pdf}
      \put(2,90){\color{white}4}
    \end{overpic}    
  \end{center}

  \end{continuousphoto}

  \caption{《へふいへ(A19)》を発動しているAのダンス}          
  \label{fig:a19i}
\end{figure}

その他Aは、日常の身体運用においても、表情図形を自らの身体に召喚していた。
いつも20分徒歩で通勤しているAは、歩いているときに《ガチョベチョカキョ(A4)》を試して、金属的な足音をまざまざと感じたという。
また表情《ブふぁああーーーペ(A11)》は、《ブふぁああーーー(A3)》と《ぐういぺ(A2)》の2つの表情図形を足し引きして作った表情図形である。
実際に踊りに踊るときの感覚に近づけるためのAのこのやりくりは、本アプリの仕様が促したものとも考えられる。


\clearpage
\section{対象者Bの学び}
\label{sec:caseB}

Bは陸上三段跳選手である。
三段跳は、助走からホップ→ステップ→ジャンプの3歩で跳ぶ距離を競う種目である(\autoref{fig:triplejump})。
\begin{figure}[H]
  \centering
  \includegraphics[width=\textwidth]{./images/hjplayground/triplejump.pdf}
  \caption{三段跳(Bのパフォーマンス動画から著者作成)}          %和文 cap tion
  \label{fig:triplejump}
\end{figure}
三段跳びでは、\autoref{fig:triplejump}のような複雑で激しい動きのなかで、巧みに姿勢を調整することが重要である。
本実践では、Bの姿勢調整技術を研ぎ澄ますために、2種類の「対人一本ゲタ運動」すなわち、BとCが一本ゲタを履いて立ち、対面状態で「互いに手を握り合って姿勢を保持しあう運動(以下、協力)」と「手押し相撲(以下、相撲)」を対象動作として考案した。

一本ゲタではまっすぐ立つことさえ難しい。
Bは本実践以前から、一本ゲタを履いてさまざまな動きを試し、姿勢調整技術を磨かんとしていた。
本実践ではここにさらに「対人運動」という予測不可能性を盛り込むことにしたわけである。
予測不可能にすることで「表情」を漏れ出させよう、とするねらいもあった。

本実践では、「協力 or 相撲」の条件と、Bが履く一本ゲタの種類「歯の位置が中央寄り or 後ろ寄り」という条件(CはBと異なるほうの一本ゲタを履く)とを合わせた
計4パタンの運動データ「協力B後ろゲタ」「協力B中央ゲタ」「相撲B後ろゲタ」「相撲B中央ゲタ」を撮影した。
4条件設けたねらいは、比較実験をするためというよりも、多様な状況をつくり多様な表情から姿勢調整を学習する可能性を拡げるためである。

\begin{figure}[H]
  \centering
  \includegraphics[width=\textwidth]{./images/hjplayground/interpersonal_geta.pdf}
  \caption{対人一本ゲタ運動4条件と、Bが作成した表情エントリ群}          %和文 cap tion
  \label{fig:ippongeta}
\end{figure}

Bは2025年3月25日〜4月9日の15日間で、全23個の表情エントリ(《B1》〜《B23》)を作成した(\autoref{table:hyojoentries_b})。

% Bの表情エントリ一覧
\begin{longtable}{lllp{5cm}r}
\caption{Bの作成した表情エントリ一覧} \\ 
\toprule
表情エントリID & 作成日付 & 運動データ & 表情オノマトペ & 文字数 \\
\midrule
\endfirsthead
\toprule
表情エントリID & 作成日付 & 運動データ & 表情オノマトペ & 文字数 \\
\midrule
\endhead
《B1》 & 03/25/25 & 相撲B後ろゲタ & ほー・ふぐっ & 22 \\
《B2》 & 03/25/25 & 相撲B後ろゲタ & びよびよーー・ふぐっ & 353 \\
《B3》 & 03/25/25 & 相撲B後ろゲタ & ツン(+攻撃) & 332 \\
《B4》 & 03/25/25 & 相撲B後ろゲタ & お局orませた女の子(気が強そう) & 121 \\
《B5》 & 03/25/25 & 相撲B中央ゲタ & すぽーーん & 159 \\
《B6》 & 03/26/25 & 協力B後ろゲタ & ふん・ふん & 56 \\
《B7》 & 03/26/25 & 協力B後ろゲタ & びょーん・がちゃぐん & 86 \\
《B8》 & 04/08/25 & 相撲B後ろゲタ & んぽん(ビリヤード) & 167 \\
《B9》 & 04/08/25 & 相撲B後ろゲタ & ブンブン(んぽんのコピー) & 182 \\
《B10》 & 04/08/25 & 相撲B後ろゲタ & & 35 \\
《B11》 & 04/08/25 & 協力B中央ゲタ & プシュプシュぷわっ(気泡) & 215 \\
《B12》 & 04/08/25 & 協力B中央ゲタ & 横隔膜or綱渡 & 223 \\
《B13》 & 04/08/25 & 協力B中央ゲタ & ぬぬぬんんふわふわふわ & 197 \\
《B14》 & 04/08/25 & 協力B後ろゲタ & ぬぐむぐ(イモムシ はいはい) & 187 \\
《B15》 & 04/08/25 & 相撲B中央ゲタ & ギーコギコ & 270 \\
《B16》 & 04/08/25 & 相撲B中央ゲタ & うぉーーっとっとごめん & 268 \\
《B17》 & 04/08/25 & 相撲B中央ゲタ & ずんずんずんズコーー & 221 \\
《B18》 & 04/08/25 & 協力B後ろゲタ & わん、つ & 163 \\
《B19》 & 04/08/25 & 協力B中央ゲタ & ぱきっつつ & 100 \\
《B20》 & 04/09/25 & 協力B後ろゲタ & あいやっ & 185 \\
《B21》 & 04/09/25 & 協力B後ろゲタ & ぼぶわっ! & 428 \\
《B22》 & 04/09/25 & 協力B後ろゲタ & ディフェンス & 8 \\
《B23》 & 04/09/25 & 協力B後ろゲタ & うーみょううみょうんぶん & 450 \\
\bottomrule
\label{table:hyojoentries_b}
\end{longtable}

Bがこれらの表情エントリを対人一本ゲタ運動のどのシーンに対して作ったのか、時間的対応関係を\autoref{fig:ippongeta}に示す。
Bの実践のダイジェストを、インタビューにてBが語った4つの表情《B3》《B11》《B20》《B21》を取り上げながら説明する。
4つの表情は\autoref{fig:ippongeta}にて赤で示したものであり、図内写真に番号づけた各局面は、本節の\autoref{fig:hyojonoteb3}・\ref{fig:hyojonoteb11}・\ref{fig:hyojonoteb20andb21}の同番号局面と同じである。

\subsection{《ツン(+攻撃)(B3)》}
《ツン(+攻撃)(B3)》(\autoref{fig:hyojonoteb3})は
運動データ「相撲B後ろゲタ」における
「BがCを手押すのをCが受け流す」
0.7秒間のシーンに対し、
備え付け補助線パタン「全体の輪郭」を適用し、
補助線スタイル「線分」で図形を作図し、そこに感得した表情である。
Bは図形の「三角形の鋭角(Bの肩-手首-肘)のふるまい」に、「照準を定めて攻撃する意志をもって突く」ようすを見てとった。
興味深いことに、これは三段跳の各踏切時に「地面に足を衝突させるさま」としても立ち現れた。
それまでBが抱いていた接地のモチーフは「ボールが弾む感覚」であった
(奇しくも、Aが《ブふぁああーーー(A3)》(\autoref{fig:a3i})で語ったのと同じく弾性体のモチーフであることも興味深い)。
そんなBにとってこの「三角形で攻撃的に打突」という新しい接地感覚となった。
同時にそれは「現在の自分はできていない、ひとつの望ましい踏切のありかた」だとして、一種の「指針」として解釈した
\footnote{三段跳のステップ踏切時は、接地直前から着く脚の主動筋群を収縮させ、自ら地面に着きにいくような「積極的着地」が良いともされている。}
。
このようにBは、抽象的な三角形の動きを媒介に、《ツン(+攻撃)》という、「手で押す動作」とも「三段跳の接地動作」ともなりうるような表情を感得したのである。
% 《ツン(+攻撃)(B3)》
\begin{figure}[H]
  \begin{hyojoentry}
  \textbf{《ツン(+攻撃)(B3)》} (相撲B後ろゲタ)

  \vspace{0.5em} \hrule height 0.5pt \vspace{0.5em}

  \begin{center}
    \begin{overpic}[width=0.3\columnwidth]{./images/hjplayground/B3-1.pdf}
      \put(2,90){1}
    \end{overpic}
    \hspace{0.05em}
    \begin{overpic}[width=0.3\columnwidth]{./images/hjplayground/B3-2.pdf}
      \put(2,90){2}
    \end{overpic}
    \hspace{0.05em}
    \begin{overpic}[width=0.3\columnwidth]{./images/hjplayground/B3-3.pdf}
      \put(2,90){3}
    \end{overpic}
  \end{center}

  \vspace{0.5em}

  \small
  \textit{
    ツン とはしたけど表現したいものはもっと攻撃的。
    上体の部分でできている三角形が相手に攻撃していく、刺さる部分を表現したいんだけどいい言葉が思いつかない。
    三角形は相手を攻撃する意志がみえる。
    「ツ」の音がしっくりくる。ただツンって優しく指しているから微妙。
    最初見えなかった三角形がつくられる。
    照準をしぼる、的をねらう。
    三段でいうなら踏切板の近くだろうか。
    そういえば三段跳び、三角形、両方3だな。
    これまで体の中には球体しか作ってこなかったけど(弾ませたくて)案外角のある三角形もやくだつのかなぁ。
    三角形の方が球体よりも能動的に動かさないとはずまない。
    最近の三段の課題はあおられる、後傾することで、もっと突っ込みたい。
    三角の意思つくってみようかな。
  }  

  \end{hyojoentry}

  \caption{表情エントリ《B3》}          
  \label{fig:hyojonoteb3}
\end{figure}

\subsection{《プシュプシュぷわっ(B11)》}
《プシュプシュぷわっ(B11)》(\autoref{fig:hyojonoteb11})は
運動データ「協力B中央ゲタ」における
「BがCが前にバランスをくずして前に体重をかけてきたのをのけぞりつつ耐えたのち、押し返して直立状態に復帰する」
3.4秒間のシーンに対し、
Bが備えつけ補助線パタンの「三角形で埋める」を適用し、
補助線スタイル「内接円+線分」で図形を作図し、そこに感得した表情である。
BC二人の身体の間を満たす円群が大小関係を様々に変化させながら蠢く。
両者のあいだが「泡立つ」かのようである。
Bはここに「全体のエネルギーをある程度保ちつつ、内部のエネルギー疎密を柔軟に移動させる」ことを感じた。
これはBの三段跳と太くつながった。

説明しよう。
Bには「全身が力んでしまいやすい」癖があった。
それを防ぐためにBは三段跳において、体幹部に注力し『常に腹圧をかける』ようにしていた。
しかし空中にて四肢のほうをうまくつかえずにいた
(これをBは『手足に神経が通わない感じ』だと表現する)。
だからといって腕ばかり注力しても、今度は要である体幹部とのつながりが途切れてしまう。
あるいはただ「全身を脱力させる」意識が良いというわけでもない。
三段跳ではステップ・ジャンプ踏切時の衝撃負荷に耐えねばならず
\footnote{ステップの瞬間は体重の7〜12倍もの負荷がかかるという報告\cite{ramey_et_al:1985}もある。}
、完全な脱力はそれになじまないからである。

こうした脱力と緊張との関係性に悩んでいたところ、
Bは《B11》をつうじて「全体を保ちつつ、中身の疎密だけを移動をさせる」といった新しいとらえかたを獲得したのであった。
インタビューでBは、このことについて語るなかで『よく、上手な選手は「腕で舵を取る」と言われる』と言及した。
船で舵を取る動作には、重さや手応えを伴う。
《B11》をふまえて言い換えるならば、上手な選手は「腕から体幹部のほうへと蜜を流し込む」かのように腕を使う、ということだろう。
《B11》は、Bが三段跳の文脈で耳にしていた「意味深な言葉」の意味を自分なりに咀嚼するそのきっかけを与えたのではないかと著者は考える。

% 《プシュプシュぷわっ(気泡)(B11)》
\begin{figure}[H]
  \begin{hyojoentry}

  \textbf{《プシュプシュぷわっ(気泡)(B11)》} (協力B中央ゲタ)

  \vspace{0.5em} \hrule height 0.5pt \vspace{0.5em}

  \begin{center}
    \begin{overpic}[width=0.18\columnwidth]{./images/hjplayground/B11-1.pdf}
      \put(2,90){1}
    \end{overpic}
    \hspace{-0.5em}
    \begin{overpic}[width=0.18\columnwidth]{./images/hjplayground/B11-2.pdf}
      \put(2,90){2}
    \end{overpic}
    \hspace{-0.5em}
    \begin{overpic}[width=0.18\columnwidth]{./images/hjplayground/B11-3.pdf}
      \put(2,90){3}
    \end{overpic}
    \hspace{-0.5em}
    \begin{overpic}[width=0.18\columnwidth]{./images/hjplayground/B11-4.pdf}
      \put(2,90){4}
    \end{overpic}
    \hspace{-0.5em}
    \begin{overpic}[width=0.18\columnwidth]{./images/hjplayground/B11-5.pdf}
      \put(2,90){5}
    \end{overpic}
  \end{center}

  \vspace{0.5em}

  \small
  \textit{
    泡、気泡がこの膜の中にある。
    この気泡はなるべく多く存在していたい。
    でもどちらかに引っ張られれば、無駄に抵抗せずに一旦気泡を潰す。
    そうすると弾けた気泡の勢いでそちら側に体勢を立て直すことができる。 
    変に形を保とうと力んだりせず、柔軟に対応していけばいい、一旦なくなる部分があっていい、そういう肩肘張らない大事さ、抜重、力みと脱力の関係見たいのが見える気がする。
    中心部の気泡は潰れていないのを見ると、守るべき場所もあるようだ。"
  }  

  \end{hyojoentry}

  \caption{表情《B11》}          
  \label{fig:hyojonoteb11}
\end{figure}

\subsection{《あいやっ(B20)》と《ぼぶわっ!(B21)》}
《あいやっ(B20)》(\autoref{fig:hyojonoteb20andb21}上段)は
運動データ「協力B後ろゲタ」における
「Cが前方にバランスをくずして左足を踏み出したのをBがわずかに後傾しつつ耐えた」
1.6秒間のシーンに対して、
「Bの右ゲタの歯先と前端」と「Cの左ゲタの歯先と前端」をそれぞれ結び、
補助線スタイル「10倍延長」で図形を作図し、そこに感得した表情である。
Bはここに、剣道の打ち込みの情景をみた(「あいやっ」は打ち込み時の発声に近い)。

Bはすかさず《あいやっ(B20)》から点群を表示状態に変えた表情図形をつくり、
《ぼぶわっ!(B21)》(\autoref{fig:hyojonoteb20andb21}下段)というオノマトペを名付けた。
表示した点群が、竹刀を握る「手首」として現れた。
打つさいの迷いや、淡々と待ち構える相手すらも見えてくることが内省記述には表れている。
そのさまにBは「テコの原理」を思った。
Bは、元の動作では「足首」に相当する部分なのだという事実知覚相\cite{hiromatsu:1989}へと持ち帰り、
遂に「足首で剣を打ち込むような接地」を発想するに至ったのである。
作図した表情図形に(偶然にも)「線どうしの交差」という幾何学的関係性が生じたからこそ、
剣を切り結ぶ情景が立ち現れたのだろうと推察できる。
% あいやとぼぶわ
\begin{figure}[H]
  \begin{hyojoentry}
  上段:\textbf{《あいやっ(B20)》} (協力B後ろゲタ)\\
  下段:\textbf{《ぼぶわっ!(B21)》}(協力B後ろゲタ)

  \vspace{0.5em} \hrule height 0.5pt \vspace{0.5em}

  
  \setlength{\tabcolsep}{2pt} % ← 横のスキマを詰める!
  \renewcommand{\arraystretch}{0.95} % ← 縦のスキマ(ラベルと画像間)
  % 上段
  \begin{center}
    \begin{tabular}{cccc}
      初動 & 剣を振り上げ & 打ち込む \\
      \begin{overpic}[width=0.3\linewidth]{./images/hjplayground/B20-1.pdf}
        \put(2,90){}
      \end{overpic} &
      \begin{overpic}[width=0.3\linewidth]{./images/hjplayground/B20-2.pdf}
        \put(2,90){}
      \end{overpic} &
      \begin{overpic}[width=0.3\linewidth]{./images/hjplayground/B20-3.pdf}
        \put(2,90){}
      \end{overpic} &      
    \end{tabular}
  \end{center}
  
  \vspace{-0.5em}
  
  % 下段
  \begin{center}
  \begin{tabular}{cccc}
    \begin{overpic}[width=0.3\linewidth]{./images/hjplayground/B21-1.pdf}
      \put(2,90){}
    \end{overpic} &
    \begin{overpic}[width=0.3\linewidth]{./images/hjplayground/B21-2.pdf}
      \put(2,90){}
    \end{overpic} &
    \begin{overpic}[width=0.3\linewidth]{./images/hjplayground/B21-3.pdf}
      \put(2,90){}
    \end{overpic} &    
  \end{tabular}
  \end{center}
  

  
  \small  
  \textbf{《あいやっ(B20)》の内省記述}\\
  \vspace{-1.0em} \hrule height 0.5pt \vspace{0.5em}  
    剣道的な。剣で打ち込んでいる。
    左の打ち込む剣はぶれている。剣に迷いがあるようだ。
    打ち込んだ後引いているところを見ると成功しなかったらしい。
    一方で右の剣は穏やかだ。
    淡々と構えて受けている。
    右の剣の方が長さもある。
    そういえば、剣を打ち込んで止まるとき剣どうしは割と中心付近で交わるけど中心よりは手前で止まる。
    これが両者の間合いだろうか
    近すぎたくはないのか?    

  \vspace{0.5em}

  \small  

    \textbf{《ぼぶわっ!(B21)》の内省記述}\\
    \vspace{-1.0em} \hrule height 0.5pt \vspace{0.5em}  
    あいやっのコピー。点をつけてみた。
    点をつけると棒を持つ手が見える(5つの点で構成)
    そして左の人の手首(棒をもたない3つの点)が震えていることに気づく。 
    はじめは、こんなに震えているようでは渾身の一撃は加えられないぞと批判的に感じた。
    だが、打ち込むスピードに注目してみると、結構はやく動いているのでは?と。
    手首側の2つの点が激しく運動(上下に?回転もかかっている?)することによってその力は剣に効率よく伝わる。
    ちょうどテコの原理のように。
    そう言えば下駄は、というか人間の足はテコの原理を使っていることを思い出す。
    踵が力点で、支点である歯、つま先が作用点といった具合のはず。
    踵にグンと力をほんの一瞬かけることで素早い推進力を得る。
    これまで身体操作をする際に「足」としてテコを意識しても素早く動かせる感覚が生まれにくかったが、手首であれば神経が鮮明なので思い浮かべやすい。
    足の延長線上に剣があると思ってそれを打ち込むのやってみよう。  
  

  \end{hyojoentry}

  \caption{《あいやっ(B20)》と《ぼぶわっ!(B21)》}          
  \label{fig:hyojonoteb20andb21}
\end{figure}

《あいやっ(B20)》と《ぼぶわっ!(B21)》では、もうひとつ興味深いことが起きていた。
\autoref{fig:hyojonoteb20andb21}で左側から打ち込んでいる人物はCであるが、Bは作図中にこれがB自身だと思い込んでいた。
Bはアプリ内カメラ位置の緯度・経度をまわして操作するうち、抽象的な映像内でBとCどちらがどちらかわからなくなり(気にならなくなり)、知らずうちにCへ移入してしまったのだろう。
そうならば本アプリには、自他のきびしい区別が取り払ってより自由な移入をうながす、という効果もあると考えうる。

\subsection{アプリの実践がBの三段跳にもたらしたこと}
\label{subsec:caseBsummary}
\textcolor{red}{
以上、Bは一本ゲタ対人運動でのアプリ使用をとおして、自身の三段跳に以下のような問いを展開した。
\begin{itemize}
  \item (今はまだできないが)めざすべき力強い接地はこうあるべきというビジョンを思い描いた(《ツン(+攻撃)(B3)》)、  
  \item 「脱力」と「腕で舵をとる」という、よく耳にしていた「意味深な言葉」の意味を、ひとつなぎに自分なりに納得した(《プシュプシュぷわっ(B11)》)、
  \item 助走一歩目の新しい踏み込み方(竹刀を打ち込むような踏み込み)を発想した(《《あいやっ(B20)》と《ぼぶわっ!(B21)》)。
\end{itemize}
}

Bは《あいやっ(B20)》と《ぼぶわっ!(B21)》をもとにして、三段跳の「助走」における新しい意識のしかたを創りだした。
「踵を踏み込むテコの原理で、すばやく打ち込むような次の一歩を生む」というものである。
Bはその意識でもって助走を探ってみた。
しかし結果は『無駄な動きを作るだけになってしま』い、『しっくりこなかった』とBは補助的インタビューにて語った。
その理由として、足首は手首にくらべ『神経が通っていない』ことにある、とBなりに解釈を紡いだ。
これについてBは、『(本アプリで得た剣道性を)「ものにする」には、はっきりと意識しながらやるというより像として浮かべながらなんとなくで取り入れられるといいんだろうな』
とインタビューにて反省した。

その他Bは、実践期間において、三段跳の練習中に妄想が起きたという。
「踏切板の両端と自分の腰を結んだパチンコ武器(スリングショット)のような長い鋭角二等辺三角形があり、引っ張ったゴムから指を放しバチンと球が放たれるように、助走からホップ踏切をする」というものだ。
この発想元は《B3》にあるという。
本アプリで補助線を引くという行為をした経験そのものも、Bのアスリートとしての実践に影響したことが伺える。

% \begin{figure}[H]
%   \centering
%   \includegraphics[width=\textwidth]{./images/hjplayground/triplejump.pdf}
%   \caption{三段跳(Bのパフォーマンス動画から著者作成)(再掲)}          %和文 cap tion
%   \label{fig:triplejump}
% \end{figure}
% \chapter{分析}
\label{chapter:analysis}
\section{データ全貌}

前章では、AとBによるアプリをもちいた運動学習について、
すなわちAとBがどのような表情図形を作図し、それを媒介にいかなる表情を感得し、問いを深めたのかを語った。
本章では、アプリをもちいた実践の全体像を、A〜Dの表情エントリのデータから分析する。

まずはA〜D4名が実践で残した表情エントリの全一覧を以下に示す(\autoref{table:hyojoentries_a}〜\autoref{table:hyojoentries_d})。
なお表内の「文字数」とは、からだメタ認知によるからだメタ認知の文字数であり、
AとBのものについては、前章で載せた表の再掲である。

% Aの表情エントリ一覧
\begin{longtable}{llllr}
\caption{Aの作成した表情エントリ一覧} \\
\hline
表情エントリID & 作成日付 & 運動データ & 表情オノマトペ & からだメタ認知文字数\\
\hline
\endfirsthead
\hline
表情エントリID & 作成日付 & 運動データ & 表情オノマトペ & からだメタ認知文字数\\
\hline
\endhead
《A1》 & 03/25/25 & oldman1 & フリフリ & 393\\
《A2》 & 03/25/25 & oldman1 & ぐういペ & 512\\
《A3》 & 04/01/25 & oldman1 & ブふぁああーーー & 420\\
《A4》 & 04/01/25 & oldman1 & ガチョベチョガキョ & 494\\
《A5》 & 04/01/25 & oldman1 & バラらららららら & 212\\
《A6》 & 04/02/25 & oldman1 & なぬななぬな & 348 \\
《A7》 & 04/08/25 & oldman1 & ンーコツコツンー & 228 \\
《A8》 & 04/08/25 & oldman1 & グリグリ & 184 \\
《A9》 & 04/08/25 & oldman1 & フラフラぶんフラフラ& 328 \\
《A10》 & 04/13/25 & oldman1 & フサササフサ & 273\\
《A11》 & 04/13/25 & oldman1 & ブふぁああーーーペ & 23\\
《A12》 & 04/17/25 & oldman2 & うー、、わっダラララ & 122\\
《A13》 & 04/17/25 & oldman2 & ねーねねねね... & 200\\
《A14》 & 04/18/25 & oldman3 & mm & 18\\
《A15》 & 04/18/25 & oldman3 & 擬音が思いつかない、、 & 63\\
《A16》 & 04/18/25 & oldman3 & パラララららら & 11\\
《A17》 & 04/18/25 & oldman3 & 惑星 & 9\\
《A18》 & 04/19/25 & oldman1 & バラバラ & 60\\
《A19》 & 04/19/25 & oldman3 & へふいへ & 303\\
\hline
\label{table:hyojoentries_a}
\end{longtable}

% Bの表情エントリ一覧
\begin{longtable}{lllp{5cm}r}
\caption{Bの作成した表情エントリ一覧} \\ 
\toprule
表情エントリID & 作成日付 & 運動データ & 表情オノマトペ & 文字数 \\
\midrule
\endfirsthead
\toprule
表情エントリID & 作成日付 & 運動データ & 表情オノマトペ & 文字数 \\
\midrule
\endhead
《B1》 & 03/25/25 & 相撲B後ろゲタ & ほー・ふぐっ & 22 \\
《B2》 & 03/25/25 & 相撲B後ろゲタ & びよびよーー・ふぐっ & 353 \\
《B3》 & 03/25/25 & 相撲B後ろゲタ & ツン(+攻撃) & 332 \\
《B4》 & 03/25/25 & 相撲B後ろゲタ & お局orませた女の子(気が強そう) & 121 \\
《B5》 & 03/25/25 & 相撲B中央ゲタ & すぽーーん & 159 \\
《B6》 & 03/26/25 & 協力B後ろゲタ & ふん・ふん & 56 \\
《B7》 & 03/26/25 & 協力B後ろゲタ & びょーん・がちゃぐん & 86 \\
《B8》 & 04/08/25 & 相撲B後ろゲタ & んぽん(ビリヤード) & 167 \\
《B9》 & 04/08/25 & 相撲B後ろゲタ & ブンブン(んぽんのコピー) & 182 \\
《B10》 & 04/08/25 & 相撲B後ろゲタ & & 35 \\
《B11》 & 04/08/25 & 協力B中央ゲタ & プシュプシュぷわっ(気泡) & 215 \\
《B12》 & 04/08/25 & 協力B中央ゲタ & 横隔膜or綱渡 & 223 \\
《B13》 & 04/08/25 & 協力B中央ゲタ & ぬぬぬんんふわふわふわ & 197 \\
《B14》 & 04/08/25 & 協力B後ろゲタ & ぬぐむぐ(イモムシ はいはい) & 187 \\
《B15》 & 04/08/25 & 相撲B中央ゲタ & ギーコギコ & 270 \\
《B16》 & 04/08/25 & 相撲B中央ゲタ & うぉーーっとっとごめん & 268 \\
《B17》 & 04/08/25 & 相撲B中央ゲタ & ずんずんずんズコーー & 221 \\
《B18》 & 04/08/25 & 協力B後ろゲタ & わん、つ & 163 \\
《B19》 & 04/08/25 & 協力B中央ゲタ & ぱきっつつ & 100 \\
《B20》 & 04/09/25 & 協力B後ろゲタ & あいやっ & 185 \\
《B21》 & 04/09/25 & 協力B後ろゲタ & ぼぶわっ! & 428 \\
《B22》 & 04/09/25 & 協力B後ろゲタ & ディフェンス & 8 \\
《B23》 & 04/09/25 & 協力B後ろゲタ & うーみょううみょうんぶん & 450 \\
\bottomrule
\label{table:hyojoentries_b}
\end{longtable}

% Cの表情エントリ一覧
\begin{longtable}{llllr}
\caption{Cの作成した表情エントリ一覧} \\ 
\toprule
表情エントリID & 作成日付 & 運動データ & 表情オノマトペ & 文字数 \\
\midrule
\endfirsthead
\toprule
表情エントリID & 作成日付 & 運動データ & 表情オノマトペ & 文字数 \\
\midrule
\endhead
《C1》 & 03/25/25 & 相撲B後ろゲタ & ぐいーーーん & 313 \\
《C2》 & 03/25/25 & 相撲B中央ゲタ & ふえーん & 280 \\
《C3》 & 03/25/25 & 協力B後ろゲタ & すろりーす & 190 \\
《C4》 & 03/29/25 & 協力B中央ゲタ & ぼわあ 〜ん & 304 \\
《C5》 & 03/29/25 & 相撲B後ろゲタ & キリキリ & 296 \\
《C6》 & 04/02/25 & 相撲B後ろゲタ & グギーーーぎっぐ & 67 \\
《C7》 & 04/02/25 & 相撲B後ろゲタ & うーーーーーんしょ & 12 \\
《C8》 & 04/03/25 & 協力B中央ゲタ & すしゅんすシュンスゥイーーーん & 76 \\
《C9》 & 04/08/25 & 相撲B中央ゲタ & 何も生まれなかった & 24 \\
《C10》 & 04/08/25 & 相撲B中央ゲタ & 何も生まれなかったのコピー & 24 \\
《C11》 & 04/08/25 & 協力B後ろゲタ & よーいよぃよう & 181 \\
《C12》 & 04/08/25 & 協力B後ろゲタ & んがんがあ〜〜 & 101 \\
《C13》 & 04/09/25 & 協力B中央ゲタ & オフフォーっ & 495 \\
《C14》 & 04/09/25 & 協力B後ろゲタ & 一旦 & 0 \\
《C15》 & 04/09/25 & 協力B後ろゲタ & うぅーーーいおぉーーい & 279 \\
《C16》 & 04/09/25 & 協力B後ろゲタ & こそこそ & 208 \\
《C17》 & 04/09/25 & 協力B後ろゲタ & ういおい & 215 \\
《C18》 & 04/09/25 & 協力B後ろゲタ & ゆるゆる & 115 \\
《C19》 & 04/09/25 & 協力B後ろゲタ & ふんふんほいふんふんふんほい & 36 \\
《C20》 & 04/16/25 & 協力B中央ゲタ & ウェーーーい & 228 \\
《C21》 & 04/16/25 & 相撲B中央ゲタ & う〜〜〜〜〜 & 199 \\
\bottomrule
\label{table:hyojoentries_c}
\end{longtable}

% Dの表情エントリ一覧
\begin{longtable}{lllp{5cm}r}
\caption{Dの作成した表情エントリ一覧} \\
\toprule
表情エントリID & 作成日付 & 運動データ & 表情オノマトペ & 文字数 \\
\midrule
\endfirsthead
\toprule
表情エントリID & 作成日付 & 運動データ & 表情オノマトペ & 文字数 \\
\midrule
\endhead
《D1》 & 03/21/25 & oldman1 & するんりゅ & 51 \\
《D2》 & 03/21/25 & oldman1 & つゔぁーんグ & 85 \\
《D3》 & 03/21/25 & oldman1 & ひょいった & 168 \\
《D4》 & 03/21/25 & oldman1 & ひょいった のコピー & 168 \\
《D5》 & 03/21/25 & oldman1 & にゃるヴーんぱ & 98 \\
《D6》 & 03/21/25 & 相撲B後ろゲタ & はっふっは & 81 \\
《D7》 & 03/21/25 & 相撲B後ろゲタ & ギーコっとギコくわぁあ & 130 \\
《D8》 & 04/02/25 & oldman1 & フルマラソンのゴール直後、倒れそうになるも踏ん張る膝 & 250 \\
《D9》 & 04/02/25 & oldman1 & ふぶるんふぶるん & 225 \\
《D10》 & 04/03/25 & 相撲B後ろゲタ & ぐぐっグ & 166 \\
《D11》 & 04/03/25 & 相撲B中央ゲタ & ふぅおっと & 162 \\
《D12》 & 04/05/25 & oldman1 & くわうんくわ & 151 \\
《D13》 & 04/06/25 & 協力B後ろゲタ & おっと危ない! & 154 \\
《D14》 & 04/06/25 & 協力B後ろゲタ & クンらりらんと & 204 \\
《D15》 & 04/07/25 & 協力B中央ゲタ & ドゥルン & 154 \\
《D16》 & 04/09/25 & 相撲B後ろゲタ & ふぅわああ & 125 \\
《D17》 & 04/11/25 & 相撲B中央ゲタ & アタタタタタ! & 133 \\
《D18》 & 04/13/25 & 協力B後ろゲタ & 逃げるな!こっちだ! & 99 \\
《D19》 & 04/13/25 & 協力B中央ゲタ & 引っ張ってドン! & 98 \\
《D20》 & 04/17/25 & oldman2 & んふあ!あんフ & 91 \\
\bottomrule
\label{table:hyojoentries_d}
\end{longtable}



A〜D4名の対象者が実践で残した表情エントリの量を以下に示す(\autoref{table:hyojoentrysnumber}・\autoref{table:naiseitoukei_subject})。
Aはoldman1・2・3の3つ、、BとCは対人運動4つ、Dは「oldman3以外」の6つの運動データでプレイをしている(\autoref{table:hyojoentrysnumber})。
なお、繰り返すが、D自身の動作は撮影していない。


\begin{table}[htbp]
  \centering
  \begin{minipage}[t]{0.52\textwidth}
    \raggedright
    \vspace{0pt}
    \caption{運動データごとの表情エントリの数}
    \label{table:hyojoentrysnumber} 
      \begin{tabular}{lrrrrr}
        \hline
        \bf 運動データ & \bf A & \bf B & \bf C & \bf D & \bf 合計 \\
        \hline
        oldman1 & 12 & 0 & 0 & 8 & 20 \\
        oldman2 & 2 & 0 & 0 & 1 & 3 \\
        oldman3 & 5 & 0 & 0 & 0 & 5 \\
        協力B後ろゲタ & 0 & 8 & 9 & 3 & 20 \\
        協力B中央ゲタ & 0 & 4 & 4 & 2 & 10 \\
        相撲B後ろゲタ & 0 & 7 & 4 & 4 & 15 \\
        相撲B中央ゲタ & 0 & 4 & 4 & 2 & 10 \\
        \hline
        合計 & 19 & 23 & 21 & 20 &  \\
        \hline
      \end{tabular}
  \end{minipage}  
  \hfill
  \begin{minipage}[t]{0.45\textwidth}
    \raggedright
    \vspace{0pt}
    \caption{対象者別のからだメタ認知の統計}
    \label{table:naiseitoukei_subject}    
  % \ecaption{Statistics of self-reflection description by subject}  
    \begin{tabular}{lrrr}
      \hline
      \bf 対象者 & \bf エントリ数 & \bf 平均文字数 & \bf SD \\
      \hline
      A & 19 & 221.2 & 166.4 \\
      B & 23 & 192.5 & 119.7 \\
      C & 21 & 173.5 & 129.1 \\
      D & 20 & 139.7 & 50.8 \\
      \hline
      合計 & 83 &  &  \\
      \hline
    \end{tabular}
  \end{minipage}
  
\end{table}




% \begin{table}[htbp]
%   \centering
%   \caption{対象運動データ別のからだメタ認知の統計}
%   \ecaption{Statistics of self-reflection description by target movement data}
%   \label{table:naiseitoukei}
%   \begin{tabular}{lrrr}
%     \hline
%     \bf 運動データ & \bf エントリ数 & \bf からだメタ認知平均文字数 & \bf SD \\
%     \hline
%     oldman1 & 20 & 236 & 145 \\
%     oldman2 &  3 & 163 &  56 \\
%     oldman3 &  5 &  81 & 126 \\
%     協力B後ろゲタ & 20 & 151 & 119 \\
%     協力B中央ゲタ & 10 & 200 & 123 \\
%     相撲B後ろゲタ & 15 & 173 & 115 \\
%     相撲B中央ゲタ & 10 & 183 &  94 \\
%     \hline
%     合計 & 83 & 182 & 124 \\
%     \hline
%   \end{tabular}
% \end{table}



\section{表情エントリの派生関係}
自他の表情エントリ相互の影響・派生関係はあったか?
  厳密な判別は不可能だが、エントリ内の記述やインタビューから判明した範囲で述べる\footnote{
    HJPでは、エントリ相互の参照情報は保管されない。  
  }。
  Aの19エントリ中2つ以上は派生である。
  《ブふぁああーーーペ(A11)》は、《ブふぁああーーー(A3)》と《ぐういぺ(A2)》の2つの図形をAなりに足し引きして作った表情図形だという(図\ref{fig:hasei}左上)。
  このようにAは、実際に踊るときの感覚に近づけるためにHJPの仕様でやりくりしていた。
  また、《へふいへ(A19)》は《ブふぁああーーー(A3)》に軌跡をつけた図形である(図\ref{fig:hyojonotea3anda19})。
  その他Aは、自身の一作図方略(\ref{subsubsec:hefuihe}項)からして、
  それまで作った図形のバリエーションを意識するという点で、
  % \ridX{X4}
  エントリどうしの影響関係はあるだろうことを、インタビューで語った。
  Aは他者(D)のエントリからの影響は受けなかったという。

  Bは23エントリ中2つ以上は派生である。
  《ブンブン(んぽんのコピー)(B9)》が《んぽん(ビリヤード)(B8)》から(図\ref{fig:hasei}右上)、
  《ぼぶわっ!(B21)》が《あいやっ(B20)》から派生した(図\ref{fig:hyojonoteb20andb21})。
  《んがんがあ〜〜(C12)》は、Bの《ぬぐむぐ(イモムシ はいはい)(B14)》をコピーして編集した。
  その他、BはCのエントリ全体からの影響を語った。
  Bは当初、全身の「軸」を重要視するあまり、縦方向を意識しつつ複雑な図形をつくる態度でいた。
  いっぽうCは点を非表示にしたり、線が少なかったりなど、比較的シンプルな表情図形をつくっていた(例:《よーいよぃよう(C11)》)。
  それをみてBは、『そういうのもアリなんだな』と、作図のしかたのヒントを得たという。
  《ぬぐむぐ(イモムシ はいはい)(B14)》は、その発想から作ったものである。
  それを受けさらにCは《B14》をコピーしてその図形のまま《んがんがあ〜〜(C12)》を作成したのだった(図\ref{fig:hasei})。
  「学び合い」の一端と言える。

  \begin{figure}[H]
    \centering
    \includegraphics[width=\columnwidth]{./images/hjplayground/hasei.pdf}
    \caption{表情エントリどうしの顕著な派生事例}          %和文 caption    
    \label{fig:hasei}
  \end{figure}

\section{表情図形を情景に見立てることへの着目}
実践で得られた表情エントリを眺めてみると、抽象的な表情図形を、元の身体運動とは異なる情景として見立てているケースが散見される。
前章の例で言えば、《A1》(図\ref{fig:hyojonotea1})で「鉛筆を指先でつまんで振る」情景や、《B20》《B21》(図\ref{fig:hyojonoteb20andb21})では足首に「剣道での打ち込み」の情景を立ち上げたのである。
《A10》《B11》《B20》《B21》もそうである。

ここでメタファ理論\cite{lakoff_1993}\cite{lakoff_johnson}を参照しよう。
メタファでは「議論は戦争だ」というふうに喩える。
「議論」という抽象的な概念に対して、より具体的・身体経験的な「戦争」というドメインをあてがうことで、喩え元のドメイン(=戦争)に成り立つ諸関係性(敵陣への攻撃、防御不可能性、戦略、勝利、領地獲得など)が喩え先のドメイン(=議論)に転写され、「議論」のなんたるかを身体的に把握するのである。
このようにメタファは(単なる修辞技法ではなく)身体をそなえた私たちの原基的な認知様式なのだとLakoffらは論じる。

メタファ理論は、抽象的な表情図形に対して具体的なドメインを動員して表情感得しようとすることを説明しうるものである。
B.Munariのデザイン教本『空想旅行』\cite{munari:1992}でも、あるランダムな点群の布置に、線を様々なスタイルで描きくわえて「ふたりひとくみ」や「音符」や「摩天楼」や「発芽」など、様々に点群の関係づけて見立てることの重要性を示している。

そこで、見立てという観点から本実践のデータを眺め、表情図形はどの程度/どのようなドメインからの見立てを促したのか、その全体像を調査する。

\section{モチーフ語を抽出する}
からだメタ認知記述から、以下のルールで、見立ての元である名詞的概念:「モチーフ語」を抽出した。
  \begin{itemize}
  \item Xみたいに/のように/に似ている/に見えるなどの比喩表現にともなってもしくは暗喩表現で登場する語はモチーフ語として抽出する。
  \item あるソースドメインに属する複数の名詞的概念をもちいひとつの情景を詳細に描写している場合も、ひとつ一つをモチーフ語として抽出する。
  % \item 情景が語られながらもモチーフ語が記述のなかに不足している場合は、モチーフ語を補完する。
  \end{itemize}
  事例で説明する。
  《A1》の場合(図\ref{fig:hyojonotea1})「鉛筆を指で持って振るやつに似ている」という記述が情景描写の箇所である。
  ここから「鉛筆」を抽出する。
  《A1》では鉛筆を奇抜な使い方をする情景ではあるが、使い方にかかわらず「鉛筆」を抽出する
  \footnote{    
    《B1》の例を抽出ルールを補足しておく。
    仮に、「〇〇に似ている」の〇〇が    
    「棒」ならば棒がモチーフ、
    「野球」ならば野球(という娯楽・文化)がモチーフとなる。
  }。
  % 《B11》の場合(図\ref{fig:Hyojonoteb11})、「泡」と「細胞」
  《B12》の場合、図形を「横隔膜」「綱渡をする人」「バネ」という三重に喩えているのだが、これら3つともモチーフ語として抽出する。
  このようにして、すべての表情エントリに対しモチーフ語を抽出した。

  \section{モチーフ語の抽出結果}
  モチーフ語抽出の結果、A〜Dが作成した全83エントリ中57エントリがモチーフ語を有し、57エントリから計87語のモチーフ語を抽出した。
  対象者それぞれの「モチーフありエントリ数/全エントリ数」の内訳は、Aは9/19、Bは15/23、Cは16/21、Dは17/20、となった。
  これを運動学習者(AとB)と非運動学習(CとD)にまとめると、
  運動学習者群では42エントリ中24エントリ(57.1\%)で、非運動学習者群では41エントリ中33エントリ(80.5\%)でモチーフ語が使用された。
  非運動学習者群でより高いモチーフ語使用傾向が観察されたが、被験者数が限られているため(N=4)、統計的検定は行わず記述統計として報告した。
  非運動学習者のほうが「学びに活かそう」という強い目的をもたぬゆえに、「比較的自由に」妄想をはたらかせやすいのかもしれない。

  \section{モチーフ語をドメインツリーに位置付ける}
  表情図形のモチーフが属するドメインを調査するために、比喩辞典\cite{nakamura:2023}をもちいる方法を考案した。
この辞典では、約1600語に及ぶ比喩の喩え元となる語が、【大分類(自然/人間/文化の3種)】・\{中分類(全17種)\}・(小分類(全93種))の3階層からなるドメイン分類ツリーに体系的に位置付けられている。
各エントリから抽出した各モチーフ語をこの比喩辞典から探し、その語が属する【大】-\{中\}-(小)ドメインに位置付けた。
例えば、《A4》の「歯軋り」は【人間】-\{生理\}-(生理現象)、《B2》の「納豆」は 【文化】-\{産物・製品\}-(食)に位置付けた。
すべてのモチーフ語を必ずひとつの【大】-\{中\}-(小)ドメインに位置付けた。
モチーフ語が比喩辞典に存在しない場合は、そのモチーフ語がありうるツリーの枝に位置付けた。

\section{ドメインツリーにおけるモチーフ語分布}
% sunburst_all 2
全モチーフ語をツリーに位置付けた結果を、網羅性を把握しやすいようサンバースト図(図\ref{fig:sunburst_A})で示す。
  図は、中心から順に大・中の2階層を示し、各階層内に全ドメインを並べ、
  各ドメインは度数が0ならグレー、1以上なら青で示した(度数が高いほど濃い青)。
  本分析では詳細な度数よりも、そのドメインのモチーフがもちいられたか否か(青かグレーか)を重視する。
  表情感得は本質的に一回性のある現象だという前提に立ち、表情図形のもつ多様性を積極的に拾うためである。
  ドメインの占める角度に意味はない。
  該当ドメインのモチーフ語を有する《エントリID》をドメイン内に付置した(Aは赤、Bは緑、Cは青、Dは黄)。
  ドメイン内に付置されたエントリの数とモチーフ語数が一致しないケースは、
  そのエントリが当該ドメインのモチーフ語を複数有することを意味する。

  \begin{figure}[tb]
    \centering
    \includegraphics[width=\columnwidth]{./images/hjplayground/sunburst_all2.pdf}
    \caption{比喩ドメインツリーにおけるモチーフ語の分布}          %和文 caption
    \ecaption{Distribution of motif words in the metaphor domain tree} %英文 caption
    \label{fig:sunburst_A}
  \end{figure}

  % \begin{figure}[tb]
  %   \centering
  %   \includegraphics[width=\columnwidth]{./images/sunburst_abcd.pdf}
  %   \caption{比喩ドメインツリーにおけるモチーフ語の分布(対象者毎)}          %和文 caption
  %   \ecaption{Distribution of motif words in the domain tree (by participant)} %英文 caption
  %   \label{fig:sunburst_A2}
  % \end{figure}

  全体としては(図\ref{fig:sunburst_A})、モチーフ語87語でツリーの大分類3種全て・中分類13種(17種中)・小分類33種(93種中)をカバーした。
  【大分類】でみると、【文化】が突出しており【人間】と【自然】はほぼ同数である。
  \{中分類\}では不足ドメインが\{物象\}\{感性\}\{社会\}\{抽象\}の4種あったが、
  \{抽象\}は本実践ではむしろ「喩え先」に相当するからかモチーフ語としては登場しなかった。
  少なくとも\{社会\}ついては小分類(政治・経済・法律)と(教育)が属する)身体運動から遠すぎるということも関係していよう。

  エントリの付置にも着目しよう。  
  A〜D全員、【大分類】では3種すべてを網羅しているが、その度数や、
  \{中分類\}でのカバーするドメインには、それぞれ違いが表れる。  
  Aのエントリは【自然】と【文化】に偏り(【人間】は《A4》のみ)、
  Bは【文化】に偏る。
  C・Dでは大分類3種が満遍なく分布する。
  A〜Dの個人特性とHJPとの関係性、プレイした対象動作の違いが、こうした違いを生んでいるのだろう。

  \section{考察}
表情図形は、身体部位どうしの関係性や身体感覚への志向を促しただけでなく、
「見立て」をも促した。
2種の身体運動から4人が創った計83エントリ中57個で見立てが起きた。
見立て元のドメインは、メタファ喩え元辞典の中分類17種中13種をカバーした。
興味深い結果である。

身体知の学びの基本思想とメタファ理論とに照らせば、
この結果は、表情図形作図の果たすひとつの役割を示唆する。
元の身体運動というドメインに縛られずに、
本人の日常を生きる全体経験から一部を動員し、その異ドメインに成り立つ関係性を図形に転写しながら、身体運動を問うことを促す、という役割である。
この結果は、身体知の学びが本人の「生きている」ことと地続きであるという本研究の基本思想とも符号する。
「だからこそ抽象的な表情図形は、通常の映像よりもかえって『ありあり』と立ち現れてくるのではないか?」という考えも著者には浮かんでくる。

AもBも、実践後の学びでも「図形的な意識」が癖づいていた。
ここにもHJPの価値が垣間見える。
ツールは使用中に便利なだけではなく、
使うなかでひとの身体知は進化する。
進化したならば、同ツールに頼る必要はない。
「卒業」である。
ぬか床ロボット「Nukabot」をデザインしたドミニクはインタビュー\cite{nakanishi_2024}にて「卒業するツール」の考えを展開する。
HJPも卒業まで伴走してくれるツールなのかもしれない。

AとBは、それぞれの問題意識や方略に基づいて、自分なりにHJPを駆使し(図形を作図)、問いを紡いでいた。
では、HJPはどういう学習者にとって有用なのか?
著者は、万人に有用であるとまでは主張しない。
諏訪による「見立て(構成的知覚)」能力の向上の研究\cite{suwa_2004}を参照しよう。
対象者ら(大学生)は9ヶ月間、写真や空間や場に対してそこにある要素や位置関係についてからだメタ認知する研鑽を続けた。
鍛錬の前後で、曖昧図形解釈課題(与えられた曖昧図形を何通りに見立てられるかという解釈数をテスト)を実施すると、解釈数は1.6倍に増加し、この増加率は、研鑽をしなかった統制群と比較して有意に高かったという。
この結果が支持するのは、からだメタ認知の鍛錬が見立ての能力向上に寄与するということである。

本実践で観測された多彩な「見立て」による表情感得は、表情図形(の作図)が可能にしたことなのか、それともからだメタ認知だけでも可能だったのか?
本研究の結果からは、明確な線引きはできない。
しかしながら、AとBがその両方を駆使し、見立てにおいては「抽象的図形を」見立て、表情感得して問いを紡いだことは揺るぎない事実である。
少なくともからだメタ認知をする学習者にとってはHJPが有用であることを支持する結果が得られたと、著者は考える。

本研究は、表情図形の詳細な作図プロセスを扱わなかったが、
作図プロセスを細かく調べれば、さらなる表情感得の様相や図形と言葉の共創様態に迫れる可能性がある。
また、HJPでのユーザどうしでの「学び合い」についても、本実践では十分に促せなかったが、実践的検討の余地は残されている。
% \part{結論}
% \addcontentsline{toc}{part}{結論}

\chapter{考察}

\section{本論文の議論の振り返り}
本研究では、運動学習を、生きるなかで主体的に意味をつくってゆく「\textbf{身体知の学び}」としてとらえ、\textbf{一人称研究}による実践をとおしてその\textbf{意味生成プロセスを記述}した。
イントロダクションからの流れを振り返る。

知の科学の歴史は、心と身体を切り離し、心に知を求める認知主義(情報処理モデル)が主流であった。
しかし情報処理モデルでは、日常の現実生活において混沌とした状況から意味を創りだすひとの知をとらえることができない。
身体性認知の考え方は、認知主義に対抗し、知には身体も必要であり、脳・身体・環境の全体が相互作用するシステムによっていかに知が成り立つかを探究する。
脳神経科学をも巻き込むかたちで、脳-身体-環境全体からなる「認知作用」の機序や、それによってどう「認知内容」が生まれてくるか、というむきが強く、
認知内容そのものは、必ずしも主題的には扱われていない。

本研究はそのどちらとも異なる「身体知の学び」という考え方を推し進めている。
木村敏の論じる主体と世界との関わりの一般原理--フッサールやヴァイツゼカーやユクスキュルの思想を統合したもの--を足場にして、
\textbf{認知とは「認知作用(ノエシス)と認知内容(ノエマ)との相互限定」であり、
認知作用(ノエシス)とは、知覚する-行為する-思考するが三位一体の作用である}、という見方を呈示した。
また、ユクスキュルの環世界論を参照して、ひとは(学習者は)「個人に固有な意味世界」を生きうることを述べ、
運動学習における認知も\textbf{個人固有}でありうるという考えにつなげた。
これを踏まえ、「生きるなかで学ぶ」ということに焦点を置く本研究では、
身体的な認知作用と認知内容との相互限定は、互いに収束させあうだけでなく、むしろ、もつれあい・ずらしあう関係でもありうることに着目した。
それを巻き起こすことを「\textbf{身体で問う}」と呼んだ。
身体で問うこと自体が、主体性の萌芽であり、身体知の学びプロセスの最小単位的プロセスであり、
\textbf{身体知の学びとは、身体で問うことによって、主体的に環世界を創り変えてゆく営み}だとした。

従来の運動学習研究群は、
スポーツ科学は物体的身体の客観的特徴を解明することで、運動学習の一助となる知見を蓄積してきた。
学習プロセスをモデル化しようとする試みでも量的アプローチが盛んである。
運動学習を、
「情報処理システムが機械としての身体の制御パラメータを修正する」としてみる情報処理的アプローチ、
「知覚-行為システムが身体に成り立つ協調的構造を相転移する」としてみるダイナミカルシステムアプローチがある。
これらの多くは客観的な観測・記述であるのにくわえ、「expert-novice間」や「何かの実践の前-後」間を比較して、学習の時間を止めて「差」を示すものが多く、
意味づくりプロセスを陽には扱わない。
質的アプローチは意味の領域を扱うものではある。
現象学的なアプローチであるスポーツ運動学は、機械運動として身体運動を記述することの限界を指摘し、運動の「形態(ゲシュタルト)」を扱った。
形態は、植物の形態発生のごとく成長のなかにあるかたちであり、自己観察や他者観察によって内的に感得されるべきものである。
本研究の「表情」とも通ずるものであるが、金子らのアプローチは運動学習を体系的に記述することをめざすという点では、本研究とは異なる。
状況論アプローチでは、伝統的な教える-教えられるといった「脱分脈化された学校教室での知識教授の構図」を乗り越え、
学びを、状況・他者との関係性のなかで生きながら学び合うことだとしてとらえてきた。
意味づくりを「環世界の変容」すなわち「本人からみた自己と世界(環境や他者)との関係性」でとらえる本研究とは、その点では力点が異なる。
そのうえで状況論アプローチは、「意味は生きているなかの状況から創りだされる」といった思想や、ことばの位置づけなど、本研究の思想と近しいところも多い。

本研究全体に通底する態度である「\textbf{構成的}」ということについても論じた。
認知は構成的であり、構成的な認知を探究するためには、構成的な手法が必要であること、
そして、研究という営みもそもそも構成的な営みであることを指摘した。
本論文は、構成的という態度を大事にして、通常の科学論文では省かれてしまうような研究者の試行錯誤そのものについても、
(平たく言えばストーリーテリングのようなかたちで)説明してきた。

身体知の意味づくりのプロセスを記述する方法として、本研究では諏訪の提唱する\textbf{認知的手法:からだメタ認知}を採用した。
からだメタ認知は身体で問う具体的手法であり、
みずからの思考・知覚・行動、すなわち「自分からみた自身と世界との関係性」について、
それがあいまいな違和感的なものであっても、はっきりした問題意識的なものであっても、積極的にことばにしてみて書きつづり問うてみる。
ことばにしてみることで、連想や推論など、自身とことばとの予期せぬインタラクションを起こし、自らの認知を変容させることになる。
データ記録方法でありながら同時に、身体で問うことをうながし、構成のループを駆動する実践手法でもある。
本研究での「ことば」の位置づけは、ものごとを正確に記述したり他者に伝えたりというよりも、本源的に自身の身体との共創を起こす媒体である。

第一部研究では、
本研究では、私自身の身体知の学び、すなわち、私がひとりの\textbf{アスリートとして生活と競技を分けずに「走り」を学ぶ(アスリートとして生きる)様}を、\textbf{物語}として描き出した。
私は、自らのままならない身体と付き合い、体感に傾聴しながら、問題意識を醸成し、自分にとって納得できる動きの意味を試行錯誤的に探った。
怪我や生活上の出来事をきっかけとして、百均製LEDをもちいたトラッキングをDIY的に実施することで、自身の身体と動きを手触るようにして問うてみたり、
日常生活における、立つ・歩く動きをスキルとみなして根本的な再構築を図ったりした。
そして、日常生活で自身を取り巻く、競技に一見関係ないモノをツールへと転用しながら、
それらを通して、よりよい身体運用スキル、そして根本的な身体のあり方を問うことにすら試みた。
このようにして、私は数々の問題意識を醸成してきた。
これらの努力の結果として私の走フォームにいかなる変化が生じたかを考察した。
私が自覚的にアスリートとして生きようとする態度を「\textbf{学びの野生化}」と命名し、その意義を論じた。
物語が他者にもたらす意義を議論した。

第一部研究を終えた私は、自身の身体を触発するさまざまな「トイ」づくりをしてみたり、
ある種「悟った」ようになった思惟を、「身体知輻輳性」と命名しながら論じてみたりしていた。
そうした問題意識のなかで私は、第二部研究の着想となる哲学概念\textbf{「表情」}に出会った。

第二部研究では、\textbf{動いている身体の「表情」の感得をうながす運動学習支援webアプリ『HJ-Playground』を制作}し、アプリをもちいた身体知の学びの実践をおこなった。
私たち人間は「表情」の満ち満ちた環世界を生きている。
「表情」とは、視覚が主題になるような現象でありながら、
行為の契機や情動や感情の契機をも孕みもつかたちで、生々しい現象である。
みずからの環世界を主体的に創り変えながら生き・学ぶ運動学習者にとって、
\textbf{動いている身体が醸し出す「表情」は、意味づくりプロセスの源}になるという仮説を立てた。
それをもとに、アプリの制作と実践をおこなった。
身体知の学びとしての運動学習支援研究を概観したのち、
動いている身体の「表情」に近しいものに迫っているプロジェクト(研究や作品)をみやり、
アプリ制作のヒントを探り、
\textbf{「運動をもとにして、素朴で抽象的で図形的な見た目を表現する」}ことが、「表情」の感得をうながすひとつの方法であると考えた。

制作したwebアプリ『HJ-Playground』は、あらかじめ計測したユーザ自身/他者の運動データ(各部位の三次元時系列位置情報)を、画面内の三次元空間に動く点群として描き、
ユーザに、それら点どうしのあいだに線分や円などの\textbf{補助線を描きくわえ「表情図形」を作図する}ことを促す。
ユーザには、作図した表情図形を鑑賞しながら、感得している「表情」をオノマトペで命名し、そのさなかで生まれる問いをからだメタ認知で内省記述することを促す。
これらは保存したり、再鑑賞することができる。
これらによって、ひとつの身体運動にさまざまな「表情」を感得することをうながす。

アプリをもちいた実践をおこなった。
対象者のストリートダンサーAは、「oldman」というAの専門とは別ジャンル基本動作を対象に、それが醸し出す「表情」をアプリで探った。
身体内部のあるひとつの仮想的四面体が2通りの「表情」として感じられることを発見したり、
点同士左から右へ一筆書きでつなげることで、風に吹かれて移動するような「表情」を得たり、
両肩をむすんだ線の動きに「鉛筆を指先で軽くつまんで振るような情景」の「表情」をみてとって、予想外に不安定な揺れ方をしているのに気付いたりした。

別の対象者の三段跳選手Bは、一本ゲタ対人運動という不思議な動きを対象に、それが醸し出す「表情」をアプリで探った。
肩と手と肘の3点をむすび、「三角形が相手を突く」という攻撃性ある「表情」をみてとり、自分が到達すべき三段跳の力強い接地のビジョンをみたり、
円をつかった図形を作図することで「自身と他者とのあいだの空間が泡立つ」ような「表情」をみてとることで、それまで三段跳選手である自分が重要視していなかった「脱力」という深い概念について、その意味するところを自分なりに納得したり、
ゲタと足を線でむすびそれを延長してみることで「足首で剣を打ち込む」ような「表情」をみて、三段跳びの助走の一歩目の新しい踏み出し方を発想したりした。
このように対象者らは、自身の運動を新規なかたちで身体で問い、

そののち、実践者らが表情図形の作図をとおして生み出した問いを分析した。
表情図形が、実践者らの問い立てをうながすパタンには、少なくとも2種類あり、
ひとつは、表情図形をみずからの身体に「仮想的図形」として召喚してそこに身体感覚を呼び起こすパタン、
ふたつめは、抽象的な表情図形にたいし、元の身体運動とは異なる日常生活のドメインの情景(できごとやシーン)に見立てる、メタファ的パタンであった。
ふたつめのパタンは、現役の運動学習者であるAとBよりも、すでに現役を引退しているCとBのほうが顕著にみられた。
また、ふたつめのパタンでは、どういうドメインの情景がメタファのソースとしてもちいられているのか、そのバラエティを調べた。
元々は身体運動をもとにした抽象的図形にもかかわらず、実に多様なドメインが、ソースとしてもちいられていた。


\section{身体知の学びの概念モデルの再構成}
以上のように、本研究では、運動学習を身体知の学びとしてとらえ、一人称研究によって実践をとおしてその意味生成プロセスを記述してきた。

\chapter{身体知の学びの概念モデルを再構成する}
\begin{figure}[H]
  \centering
  \includegraphics[width=\textwidth]{./images/zentai.pdf}
  \caption{身体知の学び:輻輳性}          %和文 caption
  % \ecaption{Number of Hyojo entries with/without metaphor (A \& B vs. C \& D)} %英文 caption
  \label{fig:zentai}
\end{figure}

第一部でも第二部でも、本一人称研究がつまびらかにした身体知の学びのすがたには、
運動学習者は、表面的な学習ドメイン内部に縛られず、学習者が生きている全体のなかの他ドメインのものごとをも巻きこみながら問いを立てる、という共通点がみられた。
そのうえで両部にみられた学びのプロセスのちがいには、
第一部は、日常生活の運動やモノを、積極的に運動学習ドメインの文脈へ取り込むありよう、
第二部は、自身の身体運動をもとに作図した抽象的な図形を、日常生活のできごとに見立てること、さらにふたたび運動学習ドメインの文脈に取り込もうとするありよう、
があった。
本論文の最後のまとめとして、この2点を組み込めるかたちで、身体知の学びの概念モデルを再構成する(\autoref{fig:zentai})。
この終わりかたは、終わりなき構成のループ\cite{suwa_fujii:2015}を意識したものである。

\subsection{日常生活全体の「野」(薄グレー平面)}
\autoref{fig:zentai}は三次元の図である。
最上部に位置する横-奥行き方向に拡がる薄グレー平面について説明する。
これは、日常生活全体の場そのものである。
第一部研究の考察(\ref{chapter:yaseika}章)で言う「\textbf{\ruby{野}{の}}」に相当するものである。
薄グレー平面上に、色で塗りつぶされた歪形が配置されている。
それぞれの歪形は、なんらかの「ものごと」を指す。
歪形の色が重要で、その歪形の「ドメイン」を表している。
\autoref{fig:zentai}では、黄色い歪形は、その学習者にとっての実践ドメインのものごとである。
第一部の私の事例なら「陸上競技ドメイン」、第二部の対象者Aの事例ならば「ダンスドメイン」とも呼びうるものである。
「ドメイン\ruby{?}{はてな}」については後述する。
薄グレー平面上に歪形は4つしか描いていないが、実際には無数にありうる。
日常生活の場には、さまざまなドメインのものごとがある、ということを表している。
ドメインとは相対的なものであり、「運動学習ドメイン」や「家事ドメイン」というレベルもあれば、
「陸上競技ドメイン」や「ダンスドメイン」というレベル、
「走りというドメイン」といったひとつひとつのスキルのレベルまで、さまざまありうるし、
さまざまなレベルのドメインは混在しうる(それが「野」である)。


薄グレー平面領域のうち、赤青緑の三色円環(これまで\autoref{fig:noesisnoema}として登場してきた「身体で問う」)と、
その「内部」と「外部」についてくわしくは後述するが、
「外部」に上述の歪形群は位置している、ということはポイントだと述べておく。

\subsection{学習者が生き・学び・創り変えてきた「環世界」(薄グレー円柱)}
\autoref{fig:zentai}の下から上まで伸びる薄グレー円柱について説明する。
これは、身体知の学びのプロセスである。
つまり、これまで登場してきた\autoref{fig:shintaichinomanabi}と同じである。
だがそれを下から上にむかって積み重なってゆくように描くことで、これまでの\autoref{fig:shintaichinomanabi}版よりも、「歴史的」なありようを強調している。
薄グレー円柱は、学習者が主体的に生き・学び、創り変えてきた環世界そのものなのである。

薄グレー円柱の内部に注目してほしい。
歪形群が縦方向に重なるように配置されている。
この各歪形の色は、前項で述べた薄グレー平面上の各歪形の色と対応している。

\subsection{「野」と「環世界」の界面=身体で問う(薄グレー平面と薄グレー円柱の関係性)}
\label{subsec:grayheimenuandgrayenchu}
赤青緑の三色円環は、\autoref{fig:noesisnoema}と同じく「身体で問う」ということを表している。
三色円環とその内部は、薄グレー円柱と薄グレー平面との接面にひとしい。
これはとても重要な事態を表現している。
\textbf{
  「身体で問うこと(認知作用=ノエシス)と問い(認知内容=ノエマ)との相互限定関係」
  は、
  「主体的に生き・学んできた環世界」
  と
  「日常生活全般の野」
  との「界面」として生じる
  }
ということである。
だとすると、\textbf{身体知の学びにおける意味の源としての「表情」とは、}単に環世界(薄グレー円柱)の事物が帯びるもの、というだけでなく、
\textbf{「環世界」と「野」の界面において生じている}(薄グレー円柱と薄グレー平面の接面)ことになる。
そして\textbf{「意味」とは、}創り変えられていく環世界そのもの(薄グレー円柱そのもの)なのだが、
それすなわち、\textbf{「環世界」と「野」とが出会い続けた歴史に宿っている}のである。

「野」と「環世界」の関係性について、さらに、「ドメイン」の観点も盛り込むと、第一部・第二部の結果をより統合できるようにある。
まずは、\autoref{fig:zentai}のとおりに、図を三人称視点から眺めよう。
薄グレー平面(日常生活全体の野)には、さまざまなドメインの歪形群が広く散らばっている。
いっぽう薄グレー円柱(学習者がひとつの身体でもって生き・学び・創ってきた環世界)においては、薄グレー円柱断面と歪形の大きさはそう違わず、
歪形群は、縦方向に積み重なるように、所狭しと存在している。
さまざまなドメインごとがギュッと寄り集まっている。
\autoref{fig:zentai}を真上から眺めることを考えればわかりやすいが、
積み重なった歴史(環世界)を見通すようにすると、
「輻輳」している、という構図になるのである。
\autoref{fig:zentai}は「\textbf{輻輳性}」(第二部\ref{subsec:fukusousei}項で述べた)を表しているのである。



\subsection{事例で考える}
これをふまえて、いまこのとき、身体知の学びの実践者が、ある状況に出くわしたとしよう。
このとき生まれる「問い」は、上述したように、野と環世界との出会いである。
だからこの問いの「色」は、あらかじめ決まるものではない。
たとえば、\autoref{fig:zentai}の「ものごとb」はブルーグレイだが、
学習者がこのものごとbについて問うたからといって、
その問いがブルーグレイになるとは限らない。
野と環世界とが接面を保っているかぎりにおいて、問いは生まれるのだから、
問いの色は、環世界に輻輳したものごと群のいろいろな色からも照らされてこそ決まるものだからである。
すなわち、野で出くわしているものごと群のドメイン群(歪形の色群)と、
環世界をなしているものごとのドメイン群との、
かけあわせによって問いのドメインは決まる。

本研究の第一・二部の結果を事例に考えよう。
まずは第一部前半のケースである。
私が陸上競技場で、陸上競技の練習を文字通りしていた段階である(第一部物語考察\ref{chapter:yaseika}章でいう、「栽培的」\cite{levi-strauss:1962}な学び)。
陸上競技場という場には黄色い歪形ばかりがあり、
私の環世界にも黄色い歪形ばかりが降り積もっていたのだろう。
このときに生まれる問いは、
野と環世界との接面で生じる歪形は、黄色くなるのだろうと考えられる。

第一部後半のケースを考えよう。
野生的\cite{levi-strauss:1962}に学ぶようになった段階である。
日常生活の野において、さまざまな色の歪形群と出くわす。
私の環世界には黄色い歪形ばかりではなく、いろいろな色の歪形が降り積もっていたかもしれない。
だが重要なのは、このときに生まれる問いは、
野と環世界との接面で生じる歪形が黄色くすることができるということだ。
つまり、野にある状態の色と、問いの内容になるときに色を塗り替えることができるのである。

第二部のケースを考えよう。
第二部の結果で見られたのは、直接に運動学習の実践に紐づけるのではなく、
運動学習ドメインではない、日常生活の別のドメインのできごとに喩え(抽象的な表情図形にそういうドメインの情景を召喚し)ていた。
さらには明示的にそれを介することで運動学習の実践につなげているような学びかたもあった。
ではこの図のとおりこの図を三人称視点から眺めてみよう。
HJ-Playgroundで作成する表情図形は、抽象的な見た目をしている。
いわば、もとの身体運動というドメインを離れて、抜け殻のように「ドメインが未定」とでも言いうる状態になっている。
\autoref{fig:zentai}のものごとxは黒く塗りつぶしてあるのが「ドメイン未定」を表している。
野にあるほうの対象のものごとxはドメイン未定(だが抽象的な図形)、としてみる。
環世界のほうはどうなっているのか?
環世界のほうでは、いろいろな色の歪形と、実践ドメインの歪形がありうる。
このときに生まれる問いの色(つまり「表情」)は、いろいろな色の歪形群の「輻輳」したそれに照らされて決まってくるものである。
つまり、見立ての構造になっている。

ものごとが「表情」豊かに感得されるとき、
あるいは、
「意味」が含蓄あるものとして醸成されているとき、
それは、
野で出くわした対象が、
環世界の多ドメインのものごとの輻輳として(輻輳に照らされて)、
多義的・多重的に、立ち現れてくるということなのだろう。
そしてやはり、主体的な身体知の学びなのだから、
そのように立ち現れてくるというのも、そのように主体的に身構える、ということなのであろう。
ひとは、「世界に身を挺した主体」なのである\cite{merleau-ponty:1945}。
以上が、研究をとおして獲得した身体知の学びについての視座である。


% \section{MEMO}}

% インゴルドも言っている。
% 客観性・普遍性・再現性・論理性。
% 客観性とはなんなのか。
% 「アトラス」(生物や物理現象の図鑑)を例に、客観性の歴史を紐解く。
% R.ダストンは、
% 佐伯はおもしろい研究とは・・・
% ロレインダストン


% プラグマティズム哲学の流れ。
% ヴァレラでもそうだった。
% Shustermanでもそうだった。
% 樋口でもそうだった。
% HCI領域では、近年、「自伝的デザイン」の機運が高まっている。
% K.Hookがそうである。
% 本研究では、かすってはいたし、におわせてはいたのだが、そのプロセスを丹念に扱うことはしなかった。
% 彼らの

% 「守られた予定調和な場」に収束してゆくと、そこに「意味」はつくられない。
% 意味とは、生活と実践の界面にある。
% 「学び」には
% 日常生活で出くわす「状況」は混沌としており、予測もできない。
% そうした状況に居合わせたとき、なんとか、やりくりする。
% そうやって、なんとかやっていかざるを得ない。
% 守られた学びから、そういうままならない、生きている場に学びの場を転じた。
% スキルの学びに、そのように意味をもたせたのである。
% 思い通りになる場から、思い通りにならない場へと、身を転じた。
% AIにはできない学び方である。

% \todo{質的研究の成果呈示:「一式」をつくりみせる}

% \todo{
%   ひとは、身体的なノエシスとそれにともなうノエマとが、相互限定的な関係性を維持することで、当人の身体に連関した環世界を生きている。
% ひとは世界に身を挺した主体\cite{merleau-ponty:1967}として、世界にむかって身の構えをとっている。
% 生きていることの全体から「意味」を浮かびあげる。
% 新奇の状況に居合わせたとき、身体でもって問いかえす。
% すると状況から「意味」が、図として浮かび上がってくる。
% なんとかやりくりする。
% 一時的にあえて滑らかな協応を崩すことさえある。
% それが意味づくりであり、意味を作りながら生きている。
% }

% (身体性認知でもたびたび引用されるヴァレラは、仏教や現象学を援用しながら、研究者自身が身体としてある反省を実践することで認知を研究すべきことを唱導していた)。

% 自己矛盾のポイント
% \begin{itemize}
%   \item 知には、身体知という知と非身体知があり、研究対象は身体知だが、研究という営みは非身体知だと考えている。
%   \item 知はすべからく身体知であり、研究対象が身体知だし、研究という営みも非身体知なのだから、そういう研究せよ
% \end{itemize}

% \cite{neustaedter:2012}
% 「みえ」を呈示(これこれこんな着眼点たちがこう関係しています、という一覧の図示)
% ツールキットの呈示。
% パタンランゲージの提出
% KJ法のA型図解化の提出。
% 概念モデル。

% どれもが、「一式」をみせている。
% ひとつの着眼点やパタンだけを取り出しても、それは成果としての意義はそう大きくない。
% そういう実利主義的な考え方ではなく、
% 「全体のセット」にこそ意味がある。


% \todo{Somaesthetics:「机上の哲学」から「現場での実践」へ
% 美学の流れでは、伝統的には西洋哲学で形而上での論述にあふれていた。

% カントしかり、バウムガルデンしかりである。

% そこから、シュスタマンは、プラグマティズムの流れを合流させることで、
% 学者自身が身体(soma)をつかって実践することの重要性を説き、
% 「Somaesthetics」を提唱した。

% 実践するということは、その自らの身体を研ぎ澄まし、より望ましいありかたへと「鍛錬」してゆくことにほかならない。

% この鍛錬をとおして、さまざまな問いを生み出し、そこから洞察を得て、「哲学」へと醸成してゆく。

% こうしたボトムアップなアプローチこそが健全なのである。

% 本研究はまったく同意する。}

\section{補足}
\subsection{環世界概念の別解釈について}
\label{sec:horon_kansekai}

ユクスキュルは環世界の考え方によって、「生物は固有な世界を生きている主体である」ということを「機械操作係である」という言い方でも書いている。
2つの言い方はいずれも、「生物が客観的世界に組み込まれた機械である(生理学ではそのように生物を記述する)」ことを批判的に飛び越えようとしている。
しかし、本研究からすればこの2つは互いに異なる着地点であることを補足しておく。
本研究は前者の言い方に賛成である。

後者の「機械操作係」という表現は、本研究にとっては好ましくない。
すこし後の知能科学の歴史からみれば、機械操作係の考え方は、生物と機械とを統一した制御・通信モデルでとらえる\textbf{サイバネティクス}\cite{wiener:1948}」と類同している。
サイバネティクスは、その後認知科学で「人間の心」のモデル化する「情報処理モデル」として輸入された。
情報処理モデルは、心と身体とを分離して扱っており、本研究はまさにそこを問題視しているのであった。
また。
サイバネティクスと情報処理の「フィードバック」の考え方は、「環(Krais)」のかたちが意味する「部分の相互関係で全体が維持されると同時に、全体のなかでこそ部分は存続できる」という関係性とは異なる。
したがって、身体知の思想を推し進める本研究からすると、機械操作係という表現は好ましくない。
ヴァイツゼッカーもこの後者の点に批判的である。
\footnote{ユクスキュルの息子で医師であったトゥーレ・ユクスキュルもまた、父ヤーコプの環世界の考え方を人間へと敷衍した「状況環」という考え方を展開している。
ちなみに、状況環はヴァイツゼッカーに近い考え方ではあるが、トゥーレはヴァイツゼカーを引用しなかった。
これには、感情的な事情があるらしいとのことである。
}。
\subection{「表情」なき世界:離人症}

読者のなかには、「そこに建っているビルには表情を感得できない。あれはただの無機質な物体にすぎないじゃないか」という疑問をもった者もいるかもしれない。
こうした指摘に対する反駁となりうる記述も、文献\cite{hiromatsu:1989}には書いてある。
\begin{quotation}  
  なるほど、現相のうちには、これというほどの感情価やこれというほどの即応価を帯びていないものもある。
  だが、その場合でも、表情価が端的に\ruby{零}{ゼロ}なのではない。よしんば零としか言いようのない“欠如態”の相にあるとしても、
  欠如態は(いわゆる“無色透明”が一種の色であるのと類比的に)それ自身、れっきとした価値態であることを忘れてはなるまい。\\
  (『表情』, p.79)
\end{quotation}

このようなかたちで廣松は、感得される表情現相は
「人物や動物の顔面表情や身体的挙措表情には限られない。
原基的な相においては(中略)、一切の現相が\ruby{悉}{ことごと}く表情性を帯びて
\footnote{鋭い読者は「Xが表情性を帯びる」という表現方式は表情にふさわしくないのでは?
と思ったかもしれない。
それは正しい。
しかしその部分こそ、「表情に対して語彙が貧困」という廣松が指摘する問題でもあろう。
実は廣松は、「表現の便宜上、以下では事物が表情性を帯びた相で現前するかのように記す方式をも辞せないようにしよう(『表情』, p.10)」と
断りをいれたうえで「Xが表情性を帯びる」という書き方をしている。
}
感得される」と説明する。

では反対に、もし私たちが、ほんとうに、「表情」をまったく感得できないのだとするならば、どうなるのか?
私たちの体験の前にくりひろがる環世界は、どのようなものになるはずなのだろうか?

離人症という精神疾患がある。
離人症患者は、次のような体験をする。
\cite{nakamura:1979}によれば、離人症と診断された24歳のある女性は次のように語ったという。
\begin{quote}
  音楽を聞いても、いろいろの音が耳の中にはいりこんでくるだけだし、
  絵を見ていても、いろいろの色や形が眼の中にはいり込んでくるだけ。
  何の内容もないし、何の意味もない。(\cite{nakamura:1979}, p.47)
\end{quote}
また、42歳のある女性\footnote{診断は未確定とのこと}の症例では、患者からの手紙のなかで次のように綴った。
\begin{quote}
  暑い寒いという温度の高低はわかりますが、暑い寒いといった感じはどうもピンと来ません。
・・・本当にただ視聴覚に訴え、肉体的に感じることだけで、精神的な感じの方は相変わらず何も感じることができません。  
(\cite{nakamura:1979}, p.47)
\end{quote}

中村によれば、私たちの「共通感覚」が喪失しているのである。
共通感覚とは、体性感覚を中心として諸知覚が統合された感覚であり、
私たちが生きるうえでの基本的な感受性・常識の基盤となっていると中村はいう。
共通感覚を失えば、その当人の環世界からは「表情」が失われてしまう、と著者は考える。
それが上記の事例である。
なんとも殺風景的である。
「そこのビルに表情がない」と考えるひとは、
\subsection{立ち現れ一元論 by 大森荘蔵}
\label{sec:horon_tathiaraware}
大森荘蔵による一連の哲学「立ち現れ一元論」の象徴的な文言をいくつか載せておく。


% \todo{岩肌のやつがあったよなああ。}

たとえば私たちは、色や形を「世界」に属する性質として、感情を「私」に属する性質として描きがちである。
大森はそれすら否定してみせる。象徴的な言明をいくつか引いておこう。
\begin{quote}
  心という袋をひっくり返しにして、風景の立ち現れに吐き出す。
\end{quote}

\begin{quote}
  一本のネクタイの色はさまざまに見える。朝日の中で、木陰の中で、夕闇の中で、蛍光灯の下で、その色合いは微妙に変わる。 また、黄疸の人、色盲の人、呉服屋さんにはまた別様に見えよう。 これらの十人十色が全て「心に映じた」色であるというのであれば、ネクタイの客観的な色は一体何色であればいいのか。それは、カメレオンの本当の色は何かというのと同じように意味をなさない問いであろう。(\cite{ohmori:1976}, p.107)
\end{quote}
\begin{quote}
  一本の樹木もネクタイの色と同様、陽炎の向こうで、霧の中で、安物の窓ガラスの向こうで、二日酔いの人の目に、近視の人、老眼の人の目に、形を変えて見える。 この時、その樹の客観的な形とは正常な状況で正常な人に見える形だという人は、単に一つの「標準形」を指定しただけである。 それはカメレオンの「標準色」として緑を、ネクタイの「標準色」として、売り場の店員に見える色を(売手市場の場合だが)、指定するのと変わりはない。(\cite{ohmori:1976}, p.108)
\end{quote}
\begin{quote}
  (※気分や「心地」が我々の「心の内」にあるとしか言えないという考えに対して、)
しかしはたして、例えば恐ろしさは、すっぽり「心の内」に抱かれているのだろうか。歯医者と、あのピカピカ光る拷問器具をこわがるとき、恐ろしいのは、これらの道具と拷問者である。恐ろしさは、それらの人と事物に、いわば「附着」しているのである。 それを引き剥がして、一方に、怖くもなんともない歯医者と道具、そして今一方に、純粋結晶のように取り出された、純粋の恐怖(恐怖のエッセンス、恐怖のエキス)、 そして、この純粋恐怖だけが、私の「心の内」にある。しかし、もしそうなら、私は一体何が恐ろしいのだろう?(\cite{ohmori:1976}, p.116)
\end{quote}

このように大森は、「立ち現れ」の一元論を展開している。
「外なる物、内なる心」という「二段構え」の構図のなかに、「表情」はない。
「表情」は立ち現れてくるものである。

ほかにも大森は、以下のようにも書いている。
\begin{quotation}
    他人の「胃が痛い」という発言をその人の「胃痛」を構成する振舞の一部として受け取る。
    この発言以外に「胃痛」を構成する振舞は多々ある。
    身をよじる振舞、ものを食べられないという振舞、冷汗という振舞、
    ある種の表情という振舞、動作の不活性という振舞等を、あげればきりがない。
    これら無数の振舞とならんで「胃が痛い」という発言の振舞が「彼は胃痛」という情景を構成しているのである。
    「胃が痛い」という発言はこの「彼は胃痛」という情景の「報告」ではなく、その情景の一部なのである。(\cite{ohmori:1971}, p.26)
  
\end{quotation}
% \part*{付録}
% 目次にも表示させる
\addcontentsline{toc}{part}{付録}
% \setcounter{chapter}{0} % 章番号をリセット
\chapter{補論}
% \chapter{物語についての付録}
% \section{二元論・心脳問題・現象学をめぐる本研究の位置づけ}
% 本研究は、意識のありよう、すなわち主観を扱った。
% 途中、神経科学の知見を拠りどころにすることもあった。
% ホムンクルス問題や、

% たしかに脳は、意識の座であることは疑いようがない。
% 脳がなければ、意識現象は成り立たない。

% 池谷のいうように、
% \begin{itemize}
%   \item なぜ、(脳神経の)電気・化学的現象が、「感じ」を生むのか?
%   \item なぜ、すべては(脳神経の)電気・化学的現象であるにもかかわらず、「みる感じ」「きく感じ」「においの感じ」「感情」は、それぞれ違った「感じ」として感じられるのか?
% \end{itemize}

% あまりに深淵な問題である。
% 現代科学では、意識の統合情報理論(IIT)や--など、
% 期待の高まるアプローチはある。
% しかしいまだ、意識現象は謎に包まれている。

% ゆえに、本研究で深く立ち入れないのだが、少なくとも、

% 本章の議論に対して、次のようなツッコミが浮かぶかもしれない。

% 「この研究では、物と心が未分な『一元的なみえ』が原基的であるということの根拠に、脳神経科学、生態心理学など、経験科学の知見を援用していますよね?
% でも、それらの経験科学は『物』と『心』という二元論あるいはその一方のみ、という考えにもとづきます。
% つまり、「一元的である」ということを主張するために、二元的に舞い戻ってしまいませんか?あるいは混乱します。」

% 本節では補足的に、この疑問へ回答しておく。

% まず再確認だが、本研究は「一人称視点からのみえ」を扱うものであり、
% 一人称視点からのみえが「表情という一元的な構図」で描ける、ということが著者の主張の根幹である。
% ここに「なぜ、一人称視点からのみえが一元的なのか?」というツッコミはありうる。
% それにこたえるときに、諸経験科学の知見を援用している。

% 本研究では、二種類の二元論のあいだに線引きしている。
% 否定しているのは、「実物対象@外部世界ーその表象@内部世界」という二元論である。
% 受け入れているのは、「物質現象ー意識現象」という二元論である。
% 一人称的なみえを「意識現象」とみなしたとき、それを成り立たせるしくみとしての「物質現象」がある、ということだ。
% そのしくみが、諸経験科学で記述されてきた、脳身体環境の因果的相互作用である。
% それも、80s情報処理モデルだとダメだが、身体性認知科学のモデルを、根拠にしているのである。

% たしかに、究極的につきつめるなら、この2種類の二元論には違いはなくなる。
% じじつ、大森荘蔵の一連の仕事は、いっさいがっさいの物-心という二元構造を排し、
% すべての「立ち現れ」という一元へと帰そうとしている。

% それら経験科学社が、そのような研究をできたのも、まずは立ち現れ・表情・質なるものを感得したからであろう。
% あたかも、ポランニーが言ったように、カエルの分類学的研究をするとき、
% まずは、分類体系や基準をもたぬままに、「それ」を「カエルだ」とわからなければ、そもそもカエルの研究はできない。
% これは明晰的な知というより、それをささえる暗黙知にほかならない、ということである。

% それを明晰的な理性でもって反省し・分解することで、

\section{環世界概念の別解釈について}
\label{sec:horon_kansekai}

ユクスキュルは環世界の考え方によって、「生物は固有な世界を生きている主体である」ということを「機械操作係である」という言い方でも書いている。
2つの言い方はいずれも、「生物が客観的世界に組み込まれた機械である(生理学ではそのように生物を記述する)」ことを批判的に飛び越えようとしている。
しかし、本研究からすればこの2つは互いに異なる着地点であることを補足しておく。
本研究は前者の言い方に賛成である。

後者の「機械操作係」という表現は、本研究にとっては好ましくない。
すこし後の知能科学の歴史からみれば、機械操作係の考え方は、生物と機械とを統一した制御・通信モデルでとらえる\textbf{サイバネティクス}\cite{wiener:1948}」と類同している。
サイバネティクスは、その後認知科学で「人間の心」のモデル化する「情報処理モデル」として輸入された。
情報処理モデルは、心と身体とを分離して扱っており、本研究はまさにそこを問題視しているのであった。
また。
サイバネティクスと情報処理の「フィードバック」の考え方は、「環(Krais)」のかたちが意味する「部分の相互関係で全体が維持されると同時に、全体のなかでこそ部分は存続できる」という関係性とは異なる。
したがって、身体知の思想を推し進める本研究からすると、機械操作係という表現は好ましくない。
ヴァイツゼッカーもこの後者の点に批判的である。
\footnote{ユクスキュルの息子で医師であったトゥーレ・ユクスキュルもまた、父ヤーコプの環世界の考え方を人間へと敷衍した「状況環」という考え方を展開している。
ちなみに、状況環はヴァイツゼッカーに近い考え方ではあるが、トゥーレはヴァイツゼカーを引用しなかった。
これには、感情的な事情があるらしいとのことである。
}。
\section{「表情」なき世界:離人症}

読者のなかには、「そこに建っているビルには表情を感得できない。あれはただの無機質な物体にすぎないじゃないか」という疑問をもった者もいるかもしれない。
こうした指摘に対する反駁となりうる記述も、文献\cite{hiromatsu:1989}には書いてある。
\begin{quotation}  
  なるほど、現相のうちには、これというほどの感情価やこれというほどの即応価を帯びていないものもある。
  だが、その場合でも、表情価が端的に\ruby{零}{ゼロ}なのではない。よしんば零としか言いようのない“欠如態”の相にあるとしても、
  欠如態は(いわゆる“無色透明”が一種の色であるのと類比的に)それ自身、れっきとした価値態であることを忘れてはなるまい。\\
  (『表情』, p.79)
\end{quotation}

このようなかたちで廣松は、感得される表情現相は
「人物や動物の顔面表情や身体的挙措表情には限られない。
原基的な相においては(中略)、一切の現相が\ruby{悉}{ことごと}く表情性を帯びて
\footnote{鋭い読者は「Xが表情性を帯びる」という表現方式は表情にふさわしくないのでは?
と思ったかもしれない。
それは正しい。
しかしその部分こそ、「表情に対して語彙が貧困」という廣松が指摘する問題でもあろう。
実は廣松は、「表現の便宜上、以下では事物が表情性を帯びた相で現前するかのように記す方式をも辞せないようにしよう(『表情』, p.10)」と
断りをいれたうえで「Xが表情性を帯びる」という書き方をしている。
}
感得される」と説明する。

では反対に、もし私たちが、ほんとうに、「表情」をまったく感得できないのだとするならば、どうなるのか?
私たちの体験の前にくりひろがる環世界は、どのようなものになるはずなのだろうか?

離人症という精神疾患がある。
離人症患者は、次のような体験をする。
\cite{nakamura:1979}によれば、離人症と診断された24歳のある女性は次のように語ったという。
\begin{quote}
  音楽を聞いても、いろいろの音が耳の中にはいりこんでくるだけだし、
  絵を見ていても、いろいろの色や形が眼の中にはいり込んでくるだけ。
  何の内容もないし、何の意味もない。(\cite{nakamura:1979}, p.47)
\end{quote}
また、42歳のある女性\footnote{診断は未確定とのこと}の症例では、患者からの手紙のなかで次のように綴った。
\begin{quote}
  暑い寒いという温度の高低はわかりますが、暑い寒いといった感じはどうもピンと来ません。
・・・本当にただ視聴覚に訴え、肉体的に感じることだけで、精神的な感じの方は相変わらず何も感じることができません。  
(\cite{nakamura:1979}, p.47)
\end{quote}

中村によれば、私たちの「共通感覚」が喪失しているのである。
共通感覚とは、体性感覚を中心として諸知覚が統合された感覚であり、
私たちが生きるうえでの基本的な感受性・常識の基盤となっていると中村はいう。
共通感覚を失えば、その当人の環世界からは「表情」が失われてしまう、と著者は考える。
それが上記の事例である。
なんとも殺風景的である。
「そこのビルに表情がない」と考えるひとは、
\section{立ち現れ一元論 by 大森荘蔵}
\label{sec:horon_tathiaraware}
大森荘蔵による一連の哲学「立ち現れ一元論」の象徴的な文言をいくつか載せておく。


% \todo{岩肌のやつがあったよなああ。}

たとえば私たちは、色や形を「世界」に属する性質として、感情を「私」に属する性質として描きがちである。
大森はそれすら否定してみせる。象徴的な言明をいくつか引いておこう。
\begin{quote}
  心という袋をひっくり返しにして、風景の立ち現れに吐き出す。
\end{quote}

\begin{quote}
  一本のネクタイの色はさまざまに見える。朝日の中で、木陰の中で、夕闇の中で、蛍光灯の下で、その色合いは微妙に変わる。 また、黄疸の人、色盲の人、呉服屋さんにはまた別様に見えよう。 これらの十人十色が全て「心に映じた」色であるというのであれば、ネクタイの客観的な色は一体何色であればいいのか。それは、カメレオンの本当の色は何かというのと同じように意味をなさない問いであろう。(\cite{ohmori:1976}, p.107)
\end{quote}
\begin{quote}
  一本の樹木もネクタイの色と同様、陽炎の向こうで、霧の中で、安物の窓ガラスの向こうで、二日酔いの人の目に、近視の人、老眼の人の目に、形を変えて見える。 この時、その樹の客観的な形とは正常な状況で正常な人に見える形だという人は、単に一つの「標準形」を指定しただけである。 それはカメレオンの「標準色」として緑を、ネクタイの「標準色」として、売り場の店員に見える色を(売手市場の場合だが)、指定するのと変わりはない。(\cite{ohmori:1976}, p.108)
\end{quote}
\begin{quote}
  (※気分や「心地」が我々の「心の内」にあるとしか言えないという考えに対して、)
しかしはたして、例えば恐ろしさは、すっぽり「心の内」に抱かれているのだろうか。歯医者と、あのピカピカ光る拷問器具をこわがるとき、恐ろしいのは、これらの道具と拷問者である。恐ろしさは、それらの人と事物に、いわば「附着」しているのである。 それを引き剥がして、一方に、怖くもなんともない歯医者と道具、そして今一方に、純粋結晶のように取り出された、純粋の恐怖(恐怖のエッセンス、恐怖のエキス)、 そして、この純粋恐怖だけが、私の「心の内」にある。しかし、もしそうなら、私は一体何が恐ろしいのだろう?(\cite{ohmori:1976}, p.116)
\end{quote}

このように大森は、「立ち現れ」の一元論を展開している。
「外なる物、内なる心」という「二段構え」の構図のなかに、「表情」はない。
「表情」は立ち現れてくるものである。

ほかにも大森は、以下のようにも書いている。
\begin{quotation}
    他人の「胃が痛い」という発言をその人の「胃痛」を構成する振舞の一部として受け取る。
    この発言以外に「胃痛」を構成する振舞は多々ある。
    身をよじる振舞、ものを食べられないという振舞、冷汗という振舞、
    ある種の表情という振舞、動作の不活性という振舞等を、あげればきりがない。
    これら無数の振舞とならんで「胃が痛い」という発言の振舞が「彼は胃痛」という情景を構成しているのである。
    「胃が痛い」という発言はこの「彼は胃痛」という情景の「報告」ではなく、その情景の一部なのである。(\cite{ohmori:1971}, p.26)
  
\end{quotation}

% \section{身体の「表情」あれこれ}
% 「身体」にも表情があるのであり、
% 実践家は、鑑賞対象となる動く身体の「表情」を感得すればよい。
% これが、\ref{sec:学びにおける、動く身体を鑑賞するという体験}節で投げかけた問いへの著者のアンサーである。

% 日常の身振り手振りそぶりは、身体の表情の典型例である。
% 「ーー」と言いながらもモジモジしていたり、
% (発話内容と身体表情が「矛盾」しているが、私たちは身体表情のほうを「真」であると受けとるわけである!)。
% 腕組みしてガードしたり、
% マジシャンは華麗な仕草で、観ている私たちの注意を違うほうへいざない、私たちの認識枠を操る。
% 観ている私たちは、目の前のマジシャンのふるまいに、「タネとなる部分の仕草」を見破ることができない。
% タネの仕草とは異なる表情をみてしまう。

% 実践家の身体運動も、この線でとらえなおしたい。
% ボクシングの試合を観戦しているとき、
% その選手の殴り姿にはすでに、相手を殴る相手を「ブン殴る」という強い闘志がこもっている。
% 選手が相手選手へ抱く「誠意」や、
% 競馬では「前進気勢」ということばもある。

% 観ている私たちは、つい力む。じっとしてはいられなくなる。
% これを即応価とみるならば、上述した「運動共感」が強く生じるケースだろう。
% 格闘技全般において、相手に対して「自らの気配を殺す」ことをする。

% 運動主体(観測対象の)強い「意志」でもない、別種の即応価もある。
% もっとプリミティヴな、「うごめき」としてなにかを感じるのかもしれない。

% たとえば、羽生結弦のスケーティングに、
% ラグビーニュージーランド代表オールブラックスのハカに、
% その全体に、
% これも表情である。


% ほかにも、たとえば、バキシリーズから、宮本武蔵の動きが。


% 表情論のまとめからいえば、
% \begin{itemize}
%   \item 「見慣れた事物的姿」を保留すること
%   \item 鑑賞主体との関係性に帰着すること
%   \item 動きのなかにこそ現れるようにすること
% \end{itemize}

% たしかにそう考えてみれば、

% \section{実践者のことば}
% 奥平(2020)は,棒高跳選手である自身を対象とした一人称研究を行い,自分なりの身体運用の原理を追究した.
% 彼の学びに,曖昧模糊とした問いの重要性を如実に示す雄弁な例が見られる.
% 一例を示すと,彼は「会陰をキュッと持ち上げ,『おえっ』とえずくようにしてその持ち上げた意識を百会につなぎ止めておく」という体感的な表現を言葉に残している.これらの<感触>は,日々の練習での体感に意識を向けて言語に残したり,身体スキルに関する様々な文献を読み考えたりする中で,彼の意識に強く印象付けられた言語表現である.

%  国内トップレベルの100m走選手であった土江(2004)が走りのフォーム改善を図った事例報告にも,体感的な問いを垣間見ることができる。
% 土江は,自身の走りについて「接地時間を短くするため,はじくようにバネのような動き(p.15)」という意識から,「接地中に重心を浮かび上がらないように,むしろやや下向きにすべり落とすようにスライドさせる(p.16)」という意識に変えることで(これらは(ibid.)の一部に過ぎない),フォームを改善した.
% (ibid.)に掲載される改善前後の走りの連続写真を見ると,確実にフォームが変容していることをTは理解できる.
% そして,このフォーム改善後に迎えたシーズンで,土江は自己ベスト記録を更新し,オリンピック代表に選出されたのだ(ibid.).


% \section*{謝辞}
本研究は、さまざまな方の支えのうえに成り立っています。
すべての方に感謝申し上げます。

\cite{horiuchi:2016a}

% 下のサイトにいろんなスタイルあり。どれかを選ぶ。
% https://mathlandscape.com/latex-bibstyles/
% \nocite{*}
\bibliographystyle{jplain}%日本語連番形式
% \nocite{*}
\bibliography{reference}

\end{document}